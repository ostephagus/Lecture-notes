\documentclass[../Main.tex]{subfiles}

\begin{document}
\section{Definition and Basic Properties}
\begin{definition}{Legendre transform}
    The \underline{Legendre transform} (LT) of $f : D(f) \mapsto \R$ is:
    \begin{equation}
        f^*(\vec{p}) = \sup_{\vec{x} \in D(f)} \left[\vec{p} \cdot \vec{x} - f(\vec{x})\right]
        \label{eqnLT}
    \end{equation}
\end{definition}
\begin{proposition}
    The LT is always convex.
    \label{propLTConvex}
\end{proposition}
\begin{proof}
    Let $\vec{p}, \vec{q} \in D(f^*)$, let $t \in (0, 1)$. Then we need to show that:
    \begin{align*}
        &\sup_{x \in D(f)} \left\{\left[(1 - t)\vec{p} + t\vec{q}\right] \cdot \vec{x} - f(\vec{x})\right\} \\
        &= \sup_{x \in D(f)} \left\{(1-t)\left[\vec{p} \cdot \vec{x} - f(\vec{x})\right] + t \left[\vec{q} \cdot \vec{x} - f(\vec{x})\right]\right\} \\
        &\leq (1-t) \sup_{x \in D(f)} \left[\vec{p} \cdot \vec{x} - f(\vec{x})\right] + t \sup_{x \in D(f)} \left[\vec{q} \cdot \vec{x} - f(\vec{x})\right]
    \end{align*}
    And so since the RHS is finite then the LHS is finite and $(1-t)\vec{p} + t\vec{q} \in D(f)$. We have therefore that the domain is oonvex and equation~\ref{eqnConvexityDef} holds, so the LT is convex.
\end{proof}
\begin{proposition}
    If $f$ is convex then so is $F_{\vec{p}}(\vec{x}) = f(\vec{x}) - \vec{p} \cdot \vec{x}$.
\end{proposition}
\begin{proof}
    Left as an exercise.
\end{proof}
\begin{corollary}
    If $f$ is convex and differentiable then any stationary point of $\vec{p} \cdot \vec{x} - f(\vec{x})$ is a global maximum occurring at $\vec{x}(\vec{p})$ given by solving:
    \begin{equation}
        \nabla f(\vec{x}) = \vec{p}
        \label{eqnLegendreMax}
    \end{equation}

    The Legendre Transform is then $f^*(\vec{p}) = \vec{p} \cdot \vec{x}(\vec{p}) = f(\vec{x}(\vec{p}))$.
\end{corollary}
\begin{remark}
    If $f$ is strictly convex, then the solution of equation~\ref{eqnLegendreMax} is unique.
\end{remark}
\begin{examples}{
        Consider functions of a single variable.
    }
    \item $f(x) = \frac{1}{2} ax^2$, where $a > 0$. Then this is strictly convex, and solving equation~\ref{eqnLegendreMax} gives $ax = p$. Then the Legendre transform is:
        \begin{equation*}
            f^*(p) = \frac{p^2}{a} - \frac{p^2}{2a} = \frac{p^2}{2a}
        \end{equation*}
        here the domain of $f^*$ is $\R$.
    \item Let $f(v) = -\sqrt{1 - v^2}$. The domain is $v \in (-1, 1)$. This is convex. Then solving equation~\ref{eqnLegendreMax} gives $v = \frac{p}{\sqrt{1 + p^2}}$. This is defined for any $p \in \R$. The Legendre transform is:
        \begin{equation*}
            f^*(p) = \frac{p^2}{\sqrt{1 + p^2}} + \frac{1}{\sqrt{1 + p^2}} = \sqrt{1 + p^2}
        \end{equation*}
    \item $f(x) = cx, c > 0$. Note that this function is convex, but not strictly convex. Then we cannot take the required supremum if $p \neq c$. Therefore the domain of $f^*$ is $\{c\}$, where $f^*(c) = 0$. \label{exLineLT}
\end{examples}
\begin{theorem}
    If $f$ is convex and twice continuously-differentiable then $f^{**}(x) = f(x)$. For this class of functions, the LT is self-inverse.
\end{theorem}
\begin{proof}
    To determine $f^{**}$ we must consider:
    \begin{equation*}
        \vec{p}(\vec{x}) \text{ such that } \nabla f^*(\vec{p}(\vec{x})) = \vec{x}
    \end{equation*}
    Note that $f^*(\vec{p}) = \vec{p} \cdot \vec{x}(\vec{p}) - f(\vec{x}(\vec{p}))$, differentiating with respect to $\vec{p}$ gives:
    \begin{align*}
        \frac{\partial f^*}{\partial p_i} &= x_i + p_j \frac{\partial x_j}{\partial p_i} - \frac{\partial f}{\partial x_j} \frac{\partial x_j}{\partial p_i} \\
        &= x_i + \left(p_j - \nabla_i f\right) \frac{\partial x_j}{\partial p_i}
    \end{align*}
    Note that, by equation~\ref{eqnLegendreMax}, the bracketed term is zero and so $\nabla f^*(\vec{p}) = \vec{x}(\vec{p})$. Then calculating the second Legendre transform:
    \begin{align*}
        f^{**}(\vec{x}) &= \vec{x} \cdot \vec{p}(\vec{x}) - f^*(\vec{p}(\vec{x})) \\
        &= \vec{x} \cdot \vec{p}(\vec{x}) - \left[f(\vec{x}) \cdot \vec{x} - f(\vec{x}(\vec{p}(\vec{x})))\right] \\
        &= f(\vec{x}(\vec{p}(\vec{x}))) = f(\vec{x})
    \end{align*}
\end{proof}
\begin{remark}
    Convexity is essential, since we need equation~\ref{eqnLegendreMax} to hold. The differentiability assumptions are stronger than is actually needed.
\end{remark}
\begin{example}[Revisiting example~\ref{exLineLT}]
    Calculating $f^{**}(x) = \sup_{p \in \{c\}} \left(xp - f^*(p)\right) = cx$, as required.
\end{example}
\section{An Application in Thermodynamics}
Consider a system of many molecules. We consider the macroscopic dynamics (length scales large compared to the separation of the molecules). In thermal equilibrium, macroscopic dynamics are described by a few quantities: total energy ($E$), the volume of the system ($V$), temperature of the system ($T$) and pressure ($P$). An \textit{isolated system} is one that is not interacting with anything else.

In equilibrium, the macroscopic dynamics of an isolated system is fully specified by $E$ and $V$. However, there exists some enormous number $\Omega(E, V)$ of microscopic configurations of the individual molecules that appear the same on macroscopic scales. Define the entropy of the system $S(E, V) = h_B \log{\Omega(E, V)}$, where $h_B$ is the Boltzmann constant.

If the energy increases, while the volume is unchanged, the the entropy will increase. This is because there are more ways to partition the energy amongst the various molecules. Therefore $S$ is strictly increaseing with $E$, so can be inverted to define $E(S, V)$. Now we have the temperature and pressure:
\begin{align}
    T &= \left(\frac{\partial E}{\partial S}\right)_v > 0 \label{eqnTempDef} \\
    P &= -\left(\frac{\partial E}{\partial V}\right)_S \label{eqnPressDef}
\end{align}
We also have the relation:
\begin{equation}
    dE = T~dS - P~dV
    \label{eqnThermoRelation}
\end{equation}
We can use the Legendre transform to derive equation~\ref{eqnTempDef}. Consider a LT of $E(S, V)$ with respect to $S$ at fixed $V$ with parameter $T$. Let this be $-F$:
\begin{equation*}
    -F(T, V) = \sup_S \left\{TS - E(S, V)\right\}
\end{equation*}
Note that, for a stable system, $E$ is convex and so we can use equation~\ref{eqnLegendreMax}. This recovers equation~\ref{eqnTempDef}. We have also that:
\begin{equation}
    F(T, V) = E - TS
    \label{eqnHFreeEnergy}
\end{equation}
and this $F$ is the \underline{Helmholtz free energy}. We can also write:
\begin{equation*}
    dF = -S~dT - P~dV
\end{equation*}
Taking instead the LT with respect to $T$ and keep $S$ fixed, with parameter $-P$, we derive equation~\ref{eqnPressDef}:
\begin{equation*}
    -H(S, P) = \sup_V \left\{-PV - E(S, V)\right\}
\end{equation*}
Then $E$ is convex as a function of $V$ for stable systems, so equation~\ref{eqnLegendreMax} gives equation~\ref{eqnPressDef} as well as:
\begin{equation*}
    H(S, P) = PV + E
\end{equation*}
Then $H$ is the \underline{enthalpy} of the system. We obtain:
\begin{equation}
    dH = T~dS + V~dP 
    \label{eqnEnthalphyDiff}
\end{equation}
\end{document}