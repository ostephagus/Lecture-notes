\documentclass[../Main.tex]{subfiles}

\begin{document}
\section{Configuration Space}
Consider a system of $N$ particles in $3$ dimensions. Use the conventions from IA Dynamics and Relativity.

Rather than considering the positions of the individual particles, we can combine them into a $3N$-dimensional vector. This is known as the \underline{configuration space} of the system.

It may also be convenient to use non-Cartesian coordinates\\$\vec{q} = (q_1, q_2, \cdots, q_N)^T$. There is a simple way to determine the equations of motion if we know the kinetic energy $T$ and potential energy $V$ in terms of the position in configuration space $\vec{q}(t)$ and its time derivative $\dot{\vec{q}}(t)$.

\section{The Lagrangian}
\subsection{In General Coordinates}
Define the \underline{Lagrangian}:
\begin{equation}
    L(\vec{q}, \dot{\vec{q}}, t) = T - V
    \label{eqnLagrangian}
\end{equation}
Then also define the \underline{action} of a path $\vec{q}(t)$ in configuration space:
\begin{equation}
    I[\vec{q}] = \int_{t_A}^{t_B} L(\vec{q}(t), \dot{\vec{q}}(t), t)dt
    \label{eqnAction}
\end{equation}
Then by dimensional analysis this has dimension $ML^2T^{-1}$, which is notably the same as Planck's constant in quantum mechanics.

Then the \underline{Principle of Least Action} states that the actual path from the start point $\vec{q_A}$ to the end point $\vec{q_B}$ in configuration space extremises the action.

Since $\vec{q_A}$ and $\vec{q_B}$ are fixed, we have the Euler-Lagrange equations:
\begin{equation*}
    \frac{\partial L}{\partial q_i} - \frac{d}{dt} \frac{\partial L}{\partial \dot{q}_i} = 0.
\end{equation*}
\begin{proposition}
    If $L$ has no explicit dependence on $t$ (i.e. the corresponding derivative is zero), then there exists a first integral:
    \begin{equation*}
        L = \sum_{i = 1}^{3N} \dot{q}_i \frac{\partial L}{\partial \dot{q}_i} = -E
    \end{equation*}
    where $E$ is a constant.
\end{proposition}
\begin{remark}
    In fact, we see that $E$ is the total energy of the system.
\end{remark}
\begin{proof}
    \begin{align*}
        \frac{dL}{dt} &= \frac{d}{dt} L(\vec{q}(t), \dvec{q}(t),t) \\
        &= \frac{\partial L}{\partial t} + \sum_{i = 1}^{3N} \left(\frac{\partial L}{\partial q_i} \dot{q}_i + \frac{\partial L}{\partial \dot{q}_i} \ddot{q}_i\right) \\
        &= \frac{\partial L}{\partial t} + \sum_{i = 1}^{3N} \dot{q}_i \left(\frac{\partial L}{\partial q_i} - \frac{d}{dt} \frac{\partial L}{\partial \dot{q}_i}\right) + \frac{d}{dt} \sum_{i = 1}^{3N} \dot{q_i} \frac{\partial L}{\partial \dot{q}_i} \\
        &= \frac{\partial L}{\partial t}  + \frac{d}{dt} \sum_{i = 1}^{3N} \dot{q_i} \frac{\partial L}{\partial \dot{q}_i} \text{ by Euler-Lagrange}
    \end{align*}
    And therefore we rearrange the equation:
    \begin{align*}
        \frac{d}{dt} \left(L - \sum_{i = 1}^{3N} \dot{q}_i \frac{\partial L}{\partial \dot{q}_i}\right) &= \frac{\partial L}{\partial t} \\ 
        &= 0
    \end{align*}
\end{proof}
\begin{example}
    Consider a single particle moving under the influence of a potential $V = V(\vec{x})$. Then:
    \begin{align*}
        -E &= \frac{1}{2} m|\dvec{x}|^2 - V - \sum_{i = 1}^3 \dot{x}_i m \dot{x}_i \\
        &= -\frac{1}{2} m|\dvec{x}|^2 - V
    \end{align*}
    so we have verified that indeed $E$ is the total energy.
\end{example}
\subsection{Example: In Polar Coordinates}
Consider a particle in 2 dimensions, and use polar coordinates $(r, \phi)$. We have that the kinetic energy is:
\begin{equation*}
    T = \frac{1}{2} m \dot{x}^2 = \frac{1}{2} m (\dot{r}^2 + r^2 \dot{\phi}^2)
\end{equation*}
and assume that the potential depends only on $r$: $V = V(r)$.

The Lagrangian becomes $L = \frac12 m (\dot{r}^2 + r^2 \dot{\phi}^2) - V(r)$. Then Euler-Lagrange equation for $\phi(t)$ is:
\begin{equation*}
    \frac{\partial L}{\partial \phi} - \frac{d}{dt} \frac{\partial L}{\partial \dot{\phi}} = 0
\end{equation*}
Then note that $L$ has no explicit $\phi$ dependence, so we have a first integral:
\begin{equation*}
    \frac{dl}{d\dot{\phi}} = \text{const}
\end{equation*}
Evaluating this, and cancelling mass (which we assume to be constant), we find $r^2 \dot{\phi} = h$, where $h$ is constant. This is the specific angular momentum which is conserved.

$L$ also has no explicit $t$ dependence, so using the above we have a first integral:
\begin{equation*}
    E = \dot{r} \frac{\partial L}{\partial \dot{r}} + \dot{\phi} \frac{\partial L}{\partial \dot{\phi}} - L = \frac{1}{2} m(\dot{r}^2 + r^2 \dot{\phi}^2) + V(r)
\end{equation*}
Finally, note that this contains $\dot{\phi}$, however we can eliminate this as in Dynamics and Relativity using the conserved quantity $h$.
\section{The Hamiltonian}
\subsection{In General Coordinates}
\begin{definition}{Hamiltonian}
    Assume that the Lagrangian is a convex function of $\dot{\vec{q}}$. Then the\\\underline{Hamiltonian} is the Legendre transform with respect to $\dot{\vec{q}}$.
\end{definition}
Taking this Legendre Transform:
\begin{align}
    H(\vec{q}, \vec{p}, t) &= \sup_{\dot{\vec{q}}} \left[\vec{p} \cdot \dot{\vec{q}} - L(\vec{q}, \dot{\vec{q}}, t)\right] \label{eqnHamiltonian} \\
    &= \left[\vec{p} \cdot \dot{\vec{q}} - L(\vec{q}, \dot{\vec{q}}, t)\right]_{\dot{\vec{q}} = \dot{\vec{q}}(\vec{q}, \vec{p}, t)} \nonumber
\end{align}
where $\dot{\vec{q}}(\vec{q}, \vec{p}, t)$ is defined to be the local extremum (note by convexity that this exists and is the global extremum), where $\frac{\partial}{\partial \dot{q}_i}$ of the square bracket term is zero.

We also determine $\vec{p}$ as follows:
\begin{equation}
    p_i = \frac{\partial L}{\partial \dot{q}_i}
    \label{eqnHamiltonParameter}
\end{equation}
\begin{example}
    Consider a single particle, $N = 1$, $\vec{q} = \vec{x}$. Then the Lagrangian is $L = \frac{1}{2} m |\dvec{x}|^2 - V(\vec{x}, t)$.
    
    Then the Hamiltonian parameter $p_i = \frac{\partial L}{\partial \dot{x}_i} = m \dot{x}_i$, or equivalently, $\vec{p} = m\dvec{x}$. This is the momentum.

    Then the Hamiltonian is $H = \vec{p} \cdot \frac{\vec{p}}{m} - \left(\frac{|\vec{p}^2|}{2m} - V\right)$ which, after tidying, is the total energy.
\end{example}
In general, $p_i$ is called the \underline{conjugate momentum} to $q_i$. In polar coordinates, for example, $p_i$ is the angular momentum.

Given that $H$ is equal to the total energy, $H$ will be conserved if $\frac{\partial L}{\partial t} = 0$ (as seen earlier).
\subsection{Hamiltonian Equations of Motion}
Consider:
\begin{align*}
    \left(\frac{\partial H}{\partial p_i}\right)_{\vec{q}, t} &= \dot{q}_i + \sum_j p_j \frac{\partial \dot{q}_i}{\partial p_i} - \sum_j \frac{\partial L}{\partial \dot{q}_i} \frac{\partial \dot{q}_i}{\partial p_i} \\
    &= \dot{q}_i \text{ by equation~\ref{eqnHamiltonParameter}}
\end{align*}
\begin{align*}
    \left(\frac{\partial H}{\partial q_i}\right)_{\vec{p}, t} &= \sum_j p_j \frac{\partial \dot{q}_j}{\partial q_i} - \frac{\partial L}{\partial q_i} - \sum_j \frac{\partial L}{\partial \dot{q}_j} \frac{\partial \dot{q_j}}{\partial q_i} \\
    &= -\frac{\partial L}{\partial q_i} \text{ by equation~\ref{eqnHamiltonParameter}} \\
    &= -\frac{d}{dt} \frac{\partial L}{\partial q_i} \text{ using the Euler-Lagrange equations} \\
    &= -\dot{p}_i \text{ again by equation~\ref{eqnHamiltonParameter}}
\end{align*}
and now the Euler-Lagrange equations imply:
\begin{align}
    \dot{q}_i &= \frac{\partial H}{\partial p_i} \label{eqnHamilton1} \\
    \dot{p}_i &= -\frac{\partial H}{\partial q_i} \label{eqnHamilton2}
\end{align}
and these are known as \underline{Hamilton's Equations}. These are first-order equations in $6N$-dimensional phase space with coordinates $(\vec{q}, \vec{p})$.

We can also obtain Hamilton's Equations as the Euler-Lagrange equations for the functional of $\vec{q} and \vec{p}$:
\begin{equation*}
    I[\vec{q}, \vec{p}] = \int_{t_a}^{t_b} \left(\dvec{q} \cdot \vec{p} - H(\vec{q}, \vec{p}, t)\right) dt
\end{equation*}
with boundary conditions required that fix $\vec{q}$ at the endpoints.

The time derivative of the Hamiltonian is:
\begin{align*}
    \frac{dH}{dt} &= \frac{d}{dt} H(\vec{q}(t), \vec{p}(t), t) \\
    &= \sum_i \left(\frac{\partial H}{\partial q_i} \dot{q}_i + \frac{\partial H}{\partial p_i} \dot{p}_i\right) + \frac{\partial H}{\partial t} \\
    &= \frac{\partial H}{\partial t}
\end{align*}
Therefore if there is no explicit dependence on $t$ in the Hamiltonian, then it must be constant (despite implicit $t$ dependence from $\vec{p}$ and $\vec{q}$).
\section{Symmetry and Noether's Theorem}
We would like to show that continuous symmetries of the action are equivalent to first-integrals of the Euler-Lagrange equations (which also leads to conserved quantities).
\subsection{General Symmetries of the Action}
Consider a curve $\vec{q}(t)$ in configuration space, where $t \in [t_A, t_B]$, and at the endpoints $\vec{q}(t_A) = \vec{q_A}$, and $\vec{q}(t_B) = \vec{q_B}$.

Define a change of variables in configuration space:
\begin{align*}
    \vec{q}^* = \vec{Q}(\vec{q}(t), \dot{\vec{q}}(t), t) \\
    t^* = T(\vec{q}(t), \dot{\vec{q}}(t), t)
\end{align*}
and require that $t \mapsto t^*$ is invertible. Then we can define a new curve $\vec{q}^*(t^*)$ in phase space, where $t^* \in [t_A^*, t_B^*]$ and at the endpoints $\vec{q}^*(t_A^*) = \vec{q_A}^*, \vec{q}^*(t_B^*) = \vec{q_B}^* $. We say that such a map is a \underline{symmetry of the action} if, for all curves in configuration space:
\begin{equation}
    \int_{t_A^*}^{t_B^*} L(\vec{q}^*(t^*), \frac{d\vec{q}^*}{dt^*}(t^*), t^*) dt^* = \int_{t_A}^{t_B} L(\vec{q}(t), \frac{d\vec{q}}{dt}(t), t) dt
    \label{eqnActionSymmetry}
\end{equation}
\subsection{Time Translational Symmetry}
Consider a Lagrangian with no explicit $t$ dependence, $L = L(\vec{q}, \dvec{q})$. Consider the following transformation:
\begin{align*}
    \vec{Q} &= \vec{q} \\
    T &= t + s
\end{align*}
Therefore $\vec{q}^*(t^*) = \vec{q}(t) = \vec{q}(t^* - s)$. Limits of integration also shift by $s$. Then the LHS of equation~\ref{eqnActionSymmetry} is:
\begin{align*}
    \int_{t_A + s}^{s_b + s} L(\vec{q}(t^* - s), \frac{d\vec{q}}{dt} (t^* - s)) dt^* &= \int_{t_A}^{t_B} L(\vec{q}(t), \frac{d^{}\vec{q}}{dt^{}}(t)) dt \\
    &= \text{ RHS of equation~\ref{eqnActionSymmetry}}
\end{align*}
\subsection{Parameterised Transformations of the Action (*)}
Some of the content of this subsection is \textbf{non-examinable}.

Consider now a 1-parameter family of transformations. We let $\vec{q}^* = \vec{Q}(\vec{q}, \dvec{q}, t, s)$ where $s$ is some parameter. Also, $t^* = T(\vec{q}, \dvec{q}, t, s)$ and let $s = 0$ give the identity transform on $\vec{q}^*$ and $t^*$. For small $s$ this gives:
\begin{align*}
    \vec{q}^* &= \vec{q} + s \left(\frac{\partial \vec{Q}}{\partial s}\right)_{s = 0} + \cdots \\
    t^* &= t + s \left(\frac{\partial T}{\partial s}\right)_{s = 0} + \cdots \\
\end{align*}
This statement and its proof are \textbf{non-examinable}.
\begin{theorem}[Noether's Theorem]
    If the action has a 1-parameter family of symmetries then any solution of the Euler-Lagrange equations has a conserved quantity:
    \begin{equation}
        \sum_{i = 1}^{3N} \frac{\partial L}{\partial \dot{q}_i} \left(\frac{\partial Q_i}{\partial s}\right)_{s = 0} + \left(L - \sum_{i = 1}^{3N} \dot{q}_i \frac{\partial L}{\partial \dot{q}_i}\right) \left(\frac{\partial T}{\partial s}\right)_{s = 0} = \text{ const}
        \label{eqnNoetherThmConservedQ}
    \end{equation}
    \label{thmNoether}
\end{theorem}
\begin{proof}
    \begin{equation*}
        \frac{d t^*}{dt} = 1 + s \frac{d}{dt}\left(\frac{\partial T}{\partial S}\right)_{s = 0} + O(s^2)
    \end{equation*}
    We assume that the action has a symmetry, so:
    \begin{align*}
        0 &= \int_{t_A^*}^{t_B^*} L(\vec{q}^*(t^*), \frac{d \vec{q}^*}{dt^*} (t^*), t^*) dt^* - \int_{t_A}^{t_B} L(\vec{q}(t), \frac{d \vec{q}}{dt} (t), t) dt \\
        &= I^* - I
    \end{align*}
    Then change variables in $I^*$ from $t^*$ to $t$:
    \begin{equation*}
        I^* = \int_{t_A}^{t_B} L\left(\vec{q}^*(t^*(t)), \left(\frac{dt^*}{dt}\right)^{-1} \frac{d}{dt} \left[\vec{q}^*(t^*(t))\right], t^*(t)\right) \frac{dt^*}{dt} dt
    \end{equation*}
    then the middle term of $L$ is:
    \begin{align*}
        & \left(1 + s \frac{d}{dt} \left(\frac{d^{}T}{ds^{}}\right)_{s = 0} + O(s^2)\right)^{-1} \frac{d}{dt} \left(\vec{q} + s\left(\frac{d^{}\vec{Q}}{ds^{}}\right)_{s = 0} + O(s^2)\right) \\
        &= \frac{d^{}\vec{q}}{dt^{}} - s\frac{d}{dt} \left(\frac{d^{}T}{ds^{}}\right)_{s = 0} \frac{d\vec{q}}{dt} + s \frac{d}{dt}\left(\frac{d\vec{q}}{ds}\right)_{s = 0} + \cdots
    \end{align*}
    And so we substitute this back into $I^*$:
    \begin{align*}
        I^*&= \int_{t_A}^{t_B} L \left(\vec{q} + s\left(\frac{d\vec{q}}{ds}\right)_{s = 0} + \cdots, \right.\\
        & \left. \frac{d^{}\vec{q}}{dt^{}} - s\frac{d}{dt} \left(\frac{d^{}T}{ds^{}}\right)_{s = 0} \frac{d\vec{q}}{dt} + s \frac{d}{dt}\left(\frac{d\vec{q}}{ds}\right)_{s = 0} + \cdots, t + s \left(\frac{dt}{ds}\right)_{s = 0} + \cdots\right) \\
        &\times \left(1 + s \left(\frac{dt}{ds}\right)_{s = 0} + \cdots\right) dt \\
        &= \int_{t_A}^{t_B} \left\{L\left(\vec{q}(t), \frac{d\vec{q}}{dt}, t\right) + s \frac{\partial^{}L}{\partial q_i^{}} \left(\frac{\partial^{}q_i}{\partial s^{}}\right)_{S = 0} \right.\\
        &+ s \frac{\partial^{}L}{\partial \dot{q}_i^{}} \left[\frac{d}{dt} \frac{\partial^{}Q_i}{\partial s^{}} - \dot{q}_i \frac{d}{dt} \left(\frac{\partial^{}T}{\partial s^{}}\right)_{s = 0}\right] + s \frac{\partial^{}L}{\partial t^{}} \left(\frac{\partial^{}T}{\partial s^{}}\right)_{s = 0} \\
        &\left.+ sL\frac{d}{dt}\left(\frac{\partial^{}T}{\partial s^{}}\right)_{s = 0} + O(S^2) \right\} dt \\
        &+ I + \int_{t_A}^{t_B} \left\{ \left(\frac{\partial^{}Q_i}{\partial s^{}}\right)_{s = 0} \left[\frac{\partial^{}L}{\partial q_i^{}} - \frac{d}{dt} \left(\frac{\partial^{}L}{\partial q_i^{}}\right)\right] \right. \\
        &+ \frac{d}{dt} \left(\left(\frac{\partial^{}Q_i}{\partial s^{}}\right)_{s = 0} \frac{\partial^{}L}{\partial q_i^{}}\right) + \left(L - \dot{q}_i \frac{\partial^{}L}{\partial \dot{q}_i^{}}\right)\frac{d}{dt}\left(\frac{\partial^{}T}{\partial s^{}}\right)_{s = 0} \\
        &\left. + \frac{\partial^{}L}{\partial t^{}}\left(\frac{\partial^{}T}{\partial s^{}}\right)_{s = 0} \right\}dt + O(s^2) \\
        &= I + s \left[\left(\frac{\partial^{}Q_i}{\partial s^{}}\right)_{s = 0} \frac{\partial^{}L}{\partial \dot{q}_i^{}} + \left(L - \dot{q}_i \frac{\partial^{}L}{\partial \dot{q}_i^{}}\right) \left(\frac{\partial^{}T}{\partial s^{}}\right)_{s = 0}\right]_{t_A}^{t_B} \\
        &- \int_{t_A}^{t_B} \left\{-\frac{d}{dt}\left(L - \dot{q}_i \frac{\partial^{}L}{\partial \dot{q}_i^{}}\right) + \frac{\partial L}{\partial t}\right\} \left(\frac{\partial^{}T}{\partial s^{}}\right)_{s = 0} dt + O(s^2) \\
        &= I + s \left[\left(\frac{\partial^{}Q_i}{\partial s^{}}\right)_{s = 0} \frac{\partial^{}L}{\partial \dot{q}_i^{}} + \left(L - \dot{q}_i \frac{\partial^{}L}{\partial \dot{q}_i^{}}\right) \left(\frac{\partial^{}T}{\partial s^{}}\right)_{s = 0}\right]_{t_A}^{t_B} + O(s^2) \\
    \end{align*}
    but we know that $I^* = I$ for all $s$, and therefore the boundary term must be 0. That is, it takes the same value at $t_A$ and $t_B$. Since these endpoints are arbitrary, we have that the required expression is constant.
\end{proof}
\begin{example}[Time translational symmetry by theorem~\ref{thmNoether}]
    Assume again that $L$ has no explicit $t$ dependence, and use the same form of time translations as above. Then Noether's Theorem gives:
    \begin{equation*}
        L - \dot{q}_i \frac{\partial L}{\partial \dot{q}_i} = -E
    \end{equation*}
    is constant. Therefore, the invariance of the laws of physics under time translations imply conservation of energy.
\end{example}
\begin{example}[Space translational symmetry in one coordinate]
    Suppose that $L$ does not depend on a coordinate $q_1$. Define a translation:
    \begin{align*}
        \vec{q}^* &= (q_1 + s, q_2, \cdots, q_{3N})^T \\
        t^* &= t
    \end{align*}
    where $s$ is a parameter. Then by theorem~\ref{thmNoether}:
    \begin{equation*}
        \left(\frac{\partial Q_1}{\partial s}\right)_{s = 0} \frac{\partial L}{\partial \dot{q}_i} = \frac{\partial L}{\partial \dot{q}_1} = p_1
    \end{equation*}
    so spatial translation symmetry gives rise to conservation of conjugate momentum.
\end{example}
\begin{example}[Space translational symmetry in Euclidean space]
    Assume $L$ depends only on the relative position between particles. Consider Euclidean space, and define a transformation, for some constant vector $\vec{a}$ and parameter $s$:
    \begin{align*}
        \vec{q}^* &= (\vec{x_1} + s\vec{a}, \vec{x_2} + s\vec{a}, \cdots, \vec{x_N} + s\vec{a})^T
        t^* &= t
    \end{align*}
    then Noether's theorem gives:
    \begin{equation*}
        \sum_{i=1}^{3N} \frac{\partial L}{\partial \dot{q}_i} \left(\frac{\partial Q_i}{\partial s}\right)_{s = 0} = \vec{a} \cdot \vec{P}
    \end{equation*}
    is conserved. But since $\vec{a}$ was arbitrary, we must have that $\vec{P}$ is conserved.

    Therefore, symmetry of the laws of physics under spatial translation gives rise to conservation of momentum.
\end{example}
\begin{example}[Rotational symmetry of Cartesian coordinates]
    In Euclidean space, consider a rotation $R$:
    \begin{align*}
        \vec{q}^* &= (R\vec{x_1}, R\vec{x_2}, \cdots, R\vec{x_N})^T
        t^* &= t
    \end{align*}
    And let $s$ be the angle of rotation. Assume that $L$ is invariant under rotations.

    Eventually, this gives that angular momentum must be conserved. So, the invariance of the laws of physics under rotation gives rise to conservation of angular momentum.
\end{example}
\begin{remark}
    The following forms of Noether's Theorem are required for the exam. The more general statement is not.
\end{remark}
\begin{theorem}[Noether's Theorem for trivial $T$]
    %TODO: Write out the less general forms
\end{theorem}
\begin{theorem}[Noether's Theorem for trivial $Q$]
    
\end{theorem}
\end{document}