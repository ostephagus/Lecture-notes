\documentclass[../Main.tex]{subfiles}

\begin{document}
We now consider constrained optimisation for arbitrary functions, that are not necessarily convex. This makes the problem much harder, so we introduce Lagrange's Method as a final hope for solving such a problem.
\section{Setting Up the Method}
Suppose we have the following problem:

Minimise $f(\vec{x})$, subject to:
\begin{align*}
    h(\vec{x}) &= \vec{b} \\
    \vec{x} &\in \Xset
\end{align*}
where $h : \R^n \mapsto \R^m$, and so $\vec{b} \in \R^m$.

We define the \underline{Lagrangian} as:
\begin{equation*}
    L(\vec{x}, \vec{\lambda}) = f(\vec{x}) - \vec{\lambda}^T (h(\vec{x}) - \vec{b})
\end{equation*}
where $\lambda \in \R^m$ is some vector. Instead of solving the original problem, we now solve:

Minimise $L(\vec{x}, \vec{\lambda})$ such that $\vec{x} \in \Xset$.
\begin{theorem}[Lagrange Sufficiency Theorem]
    Let $\vec{x}^* \in \Xset$ and $h(\vec{x}^*) = \vec{b}$.

    Let $\vec{\lambda^*} \in \R^m$ such that:
    \begin{equation*}
        L(\vec{x}^*, \vec{\lambda}^*) = \min_{\vec{x} \in \Xset} L(\vec{x}, \vec{\lambda^*})
    \end{equation*}
    Then $\vec{x}^*$ is an optimal solution of the original problem.
\end{theorem}
\begin{proof}
    Define $\Xset(\vec{b}) = \subsetselect{\vec{x} \in \Xset}{h(\vec{x}) = b}$.

    We have that $f(\vec{x}^*) \geq \min_{\vec{x} \in \Xset(\vec{b})} f(\vec{x})$. It is then sufficient to show that $f(\vec{x}^*)$ is less than or equal to this quantity.

    Observe that:
    \begin{align*}
        \min_{\vec{x} \in \Xset(\vec{b})} &= \min_{\vec{x} \in \Xset(\vec{b})} f(\vec{x}) - \vec{\lambda^*} \cdot \left(h(\vec{x}) - \vec{b}\right) \\
        &\geq \min_{\vec{x} \in \Xset} f(\vec{x}) - \vec{\lambda^*} \cdot \left(h(\vec{x}) - \vec{b}\right) \\
        &= f(\vec{x}^*) - \vec{\lambda}^* \cdot \left(h(\vec{x}^*) - \vec{b}\right) \\
        &= f(\vec{x}^*)
    \end{align*}
\end{proof}
\section{The Algorithm}
Consider a problem:

Minimise $f(\vec{x})$ subject to:
\begin{align*}
    h(\vec{x}) &\leq b \\
    \vec{x} &\in \Xset
\end{align*}

We perform the following steps:
\begin{enumerate}
    \item Add a slack variable $\vec{s}$, and consider a new formulation:
        Minimise $f(\vec{x})$ subject to:
        \begin{align*}
            h(\vec{x}) &+ \vec{s} = b \\
            x &\in \Xset, \vec{s} \geq 0
        \end{align*}
    \item Set the Lagrangian:
        \begin{align*}
            L(\vec{x}, \vec{s}, \vec{\lambda}) &= f(\vec{x}) - \vec{\lambda} \cdot (h(\vec{x}) + \vec{s} - \vec{b}) \\
            &= f(\vec{x}) - \vec{\lambda} \cdot (h(\vec{x}) - \vec{b}) - \vec{\lambda} \cdot \vec{s}
        \end{align*}
    \item Define the set:
        \begin{equation*}
            \Lambda = \subsetselect{\vec{\lambda} \in \R^m}{\min_{x \in \Xset(\vec{s})} L(\vec{x}, \vec{s}, \vec{\lambda} > -\infty)}
        \end{equation*}
    \item For each $\vec{\lambda} \in \Lambda$, find $\vec{x}^*(\vec{\lambda})$ and $\vec{s}^*(\vec{\lambda})$ such that the Lagrangian is minimal.
    \item Find $\vec{\lambda}^* \in \Lambda$ such that $(\vec{x}^*(\vec{\lambda}^*), \vec{s}^*(\vec{\lambda}^*))$ is feasible.
\end{enumerate}
\begin{remarks}
    \item Note that the Lagrangian, as defined in step 2, is separable in $\vec{x}$ and $\vec{s}$. Therefore, we minimise with respect to each in turn.
    \item We observe that any component of $\vec{\lambda}$ must be non-positive, since any positive value would require that component of $\vec{s}$ is infinite.
    \item For any $i$, $\lambda_i s_i = 0$. This is \underline{complementary slackness}.
\end{remarks}
\begin{example}
    Minimise $f(\vec{x}) = -x_1 - x_2 + x_3$, subject to:
    \begin{align*}
        x_1^2 + x_2^2 &= 4 \\
        x_1 + x_2 + x_3 &= 1
    \end{align*}
    Geometrically, this is a cylinder intersected with a plane, which means that the feasible set is \textit{nice}, and we will be able to find a minimum.

    Then the lagrangian is:
    \begin{align*}
        L(x_1, x_2, x_3, \lambda_1, \lambda_2) &= -x_1 - x_2 + x_3 - \lambda_1(x_1^2 + x_2^2 - 4) \\
        &- \lambda_2(x_1 + x_2 + x_3 - 1) \\
        &= (-x_1 - \lambda_1 x_1^2 - \lambda_2 x_1) + (-x_2 - \lambda_1 x_2^2 - \lambda_2 x_2) \\
        &+(x_3 - \lambda_2 x_3) + 4\lambda_1 + \lambda_2 \\
        &= \left(-(1 + \lambda_2) x_1 - \lambda_1 x_1^2\right) + \left(-(1 + \lambda_2)x_2 - \lambda_1 x_2^2\right)\\
        &+ \left(1 - \lambda_2\right)x_3 + 4\lambda_1 + \lambda_2
    \end{align*}
    Then we note that the minimum over $\vec{x}$ of the Lagrangian must not be negative infinity, so we require $\lambda_2 = 1$ and $\lambda_1 < 0$ (by considering the specific problem).
    
    Therefore the set $\Lambda = \subsetselect{(\lambda_1, \lambda_2) \in \R^2}{\lambda_1 < 0, \lambda_2 = 1}$.
    Given any $\vec{\lambda}$ in this set, we find that the minimum is:
    \begin{equation*}
        x_1^*(\vec{\lambda}) = -\frac{1}{\lambda_1}, x_2^*(\vec{\lambda}) = -\frac{1}{\lambda_1}
    \end{equation*}
    Now we need to find a choice of $\vec{\lambda}$ such that $\vec{x}^*$ is feasible. The correct choice is $\lambda_1 = -\frac{1}{\sqrt{2}}$, and:
    \begin{equation*}
        x_1^* = \sqrt{2}, x_2^* = \sqrt{2}, x_3^* = 1 - 2\sqrt{2}
    \end{equation*}
\end{example}
\section{Duality}
\subsection{Defining the Problem}
Consider:
\begin{align}
    \text{Minimise } &f(\vec{x}) \nonumber\\
    \text{Such that } &h(\vec{x}) = \vec{b} \label{eqnPrimalProb} \\
    &x \in \Xset \nonumber
\end{align}
The problem in \ref{eqnPrimalProb} known as the \underline{primal problem}.
\begin{definition}{Dual objective function}
    Given a primal problem such as \ref{eqnPrimalProb}, let $L(\vec{x}, \vec{\lambda})$ be the lagrangian and:
    \begin{equation*}
        \Lambda = \subsetselect{\vec{\lambda}}{\min_{\vec{x} \in \Xset} L(\vec{x}, \vec{\lambda})}
    \end{equation*}
    define the \underline{dual objective function}:
    \begin{align*}
        g : \Lambda &\mapsto \R \\
        \vec{\lambda} &\mapsto \min_{\vec{x} \in \Xset} L(\vec{x}, \vec{\lambda})
    \end{align*}
\end{definition}
Then, given this definition, the \underline{dual problem} is:
\begin{align}
    \text{Maximise } &g(\vec{\lambda}) \nonumber\\
    \text{Subject to } &\vec{\lambda} \in \Lambda \label{eqnDualProb}
\end{align}
From the example sheet, the maximum of a set of convex functions is also a convex function.
\subsection{Properties of Duality}
\begin{theorem}[Weak duality]
    If $\vec{x} \in \Xset(\vec{b})$, $\vec{\lambda} \in \Lambda$, then:
    \begin{equation*}
        f(\vec{x}) \geq g(\vec{\lambda})
    \end{equation*}
    and in particular:
    \begin{equation*}
        \min_{\vec{x} \in \Xset} f(\vec{x}) \geq \max_{\vec{\lambda} \in \Lambda} g(\vec{\lambda})
    \end{equation*}
\end{theorem}
\begin{remarks}
    \item Solving the dual problem gives a lower bound for the optimal solution of the minimisation problem.
    \item If we have equality in the final equation, we say that strong duality holds.
\end{remarks}
\begin{proof}
    For any $\vec{\lambda} \in \Lambda$ and $\vec{x} \in \Xset(\vec{b})$,
    \begin{align*}
        f(\vec{x}) &\geq \min_{\vec{x}' \in \Xset(\vec{b})} f(\vec{x}') \\
        &= \min_{\vec{x}' \in \Xset(\vec{b})} f(\vec{x}') - \vec{\lambda}^T (h(\vec{x}') - \vec{b}) \\
        &\geq \min_{\vec{x}' \in \Xset} f(\vec{x}') - \vec{\lambda}^T (h(\vec{x}') - \vec{b}) \\
        &= g(\vec{\lambda})
    \end{align*}
\end{proof}
\begin{proposition}
    The two following statements are equivalent:
    \begin{itemize}
        \item Strong duality holds
        \item The Lagrange method works
    \end{itemize}
\end{proposition}
\begin{proof}
    \begin{proofdirection}{$\Rightarrow$}{Suppose the Lagrange method works}
        Then we can find $\vec{\lambda}^*$ and $\vec{x}^*$ such that:
        \begin{align*}
            g(\vec{\lambda}^*) &= L(\vec{x}^*, \vec{\lambda}^*) \\
            &= f(\vec{x}^*)
        \end{align*}
        and we have strong duality.
    \end{proofdirection}
    \begin{proofdirection}{$\Leftarrow$}{Suppose strong duality holds}
        Then we can find $\vec{x}^*$ and $\vec{\lambda}^*$ such that:
        \begin{equation*}
            f(\vec{x}^*) = g(\vec{\lambda}^*) \\
        \end{equation*}
        then using $\vec{\lambda}^*$ in Lagrange's method will work.
    \end{proofdirection}
\end{proof}
\subsection{Conditions for the Lagrange Method}
We then want to find the class of problems for which strong duality holds, and the Lagrange method works.
\begin{definition}{Supporting hyperplane}
    A function $\phi : \R^m \mapsto \R$ has a \underline{supporting hyperplane} at $\vec{b} \in \R^m$ if there exists $\lambda \in \R^m$ such that:'
    \begin{equation}
        \phi(\vec{c}) \geq \phi(\vec{b}) + \vec{\lambda}^T (\vec{c} - \vec{b})
        \label{eqnSupportHyperPln}
    \end{equation}
    for all $\vec{c} \in \R^m$.
\end{definition}
\begin{remarks}
    \item If $\phi$ is differentiable at $\vec{b}$ and admits a supporting hyperplane, then the slope of the hyperplane then $\vec{\lambda} = \nabla \phi(\vec{b})$.
    \item Convex functions have supporting hyperplanes at all points by the first-order condition for convexity.
\end{remarks}
\begin{definition}{Value function}
    Given a primal problem of the form in \ref{eqnPrimalProb}, define the \underline{value function} $\phi : \R^m \mapsto \R$ by:
    \begin{equation*}
        \phi(\vec{c}) = \min_{h(\vec{x}) = \vec{c}, x \in \Xset} f(\vec{x})
    \end{equation*}
    or alternatively the minimum of $f(\vec{x})$ over $\vec{x} \in \Xset(\vec{c})$.
\end{definition}
\begin{theorem}
    The primal problem in \ref{eqnPrimalProb} satisfies strong duality if its value function $\phi$ has a supporting hyperplane at $\vec{b}$.

    Moreover, if strong duality holds, so for some $\vec{\lambda}$ we have $g(\vec{\lambda}) = \phi(\vec{b})$, then there exists a supporting hyperplane to $\phi$ at $\vec{b}$ with slope $\vec{\lambda}$.
\end{theorem}
\begin{remarks}
    \item We now have 3 equivalences: Lagrange method works, strong duality holds, supporting hyperplane exists.
    \item If we have a series of function $f, f_1, f_2, \cdots, f_m$, all of which are convex, and the problem is to minimise $f(\vec{x})$ subject to $f_i(\vec{x}) \leq b_i$, then $\phi(\vec{b})$ is convex. Therefore, we have that the Lagrange method will work for these types of problems (convex objective, convex constraints). This also applies to linear programming.
\end{remarks}
\begin{proof}
    Suppose that a supporting hyperplane to $\phi$ exists at $\vec{b}$ with slope $\vec{\lambda}$:
    \begin{equation*}
        \phi(\vec{c}) \geq \phi(\vec{b}) + \vec{\lambda}^T (\vec{c} - \vec{b})
    \end{equation*}
    then:
    \begin{align*}
        g(\vec{\lambda}) &= \min_{\vec{x} \in \Xset} f(\vec{x}) - \vec{\lambda}^T (h(\vec{x}) - \vec{b}) \\
        &= \min_{\vec{c}} \min_{\vec{x} \in \Xset(\vec{c})} f(\vec{x}) - \lambda^T (h(\vec{x}) - \vec{c}) - \lambda^T (\vec{c} - \vec{b}) \\
        &= \min_{\vec{c}} \phi(\vec{c}) - \vec{\lambda}^T (\vec{c} - \vec{b}) \\
        &\geq \phi(\vec{b})
    \end{align*}
    However, by weak duality, $g(\vec{\lambda}) \leq \phi(\vec{b})$, so they must be equal and strong duality exists.

    Now, for the converse, suppose strong duality holds. That is,
    \begin{equation*}
        \exists \vec{\lambda} \in \R^m \text{ such that } g(\vec{\lambda}) = \phi(\vec{b})
    \end{equation*}
    Then:
    \begin{align*}
        \phi(\vec{b}) &= g(\vec{\lambda}) = \min_{x \in \Xset} f(\vec{x}) - \vec{\lambda}^T (h(\vec{x}) - \vec{b}) \\
        &= \min_{\vec{x} \in \Xset} f(\vec{x}) - \vec{\lambda}^T (h(\vec{x}) - \vec{c}) - \lambda^T (\vec{c} - \vec{b}) \\
        &\leq \phi(\vec{c}) - \vec{\lambda}^T (\vec{c} - \vec{b})
    \end{align*}
\end{proof}
\begin{example}[Shadow prices]
    %TODO: Finish
\end{example}
\begin{theorem}
    If $\phi$ is differentiable at $\vec{b}$ and has a supporting hyperplane with slope $\vec{\lambda}$, then:
    \begin{equation*}
        \vec{\lambda} = \nabla \phi(\vec{b})
    \end{equation*}
\end{theorem}
\end{document}