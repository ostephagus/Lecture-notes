\documentclass[../Main.tex]{subfiles}

\begin{document}
\section{Defining Continuous Probability}
Previously we defined the probability space $\left(\Omega, \sigalg, P\right)$, where $X$ is a random variable $X : \Omega \mapsto \R$ with a notion of order: $\forall x \in \R \{X \leq x\} = \subsetselect{\omega}{X(\omega) \leq x} \in \sigalg$.
\begin{definition}{Probability distribution function}
    The \underline{probability distribution function} is defined to be:
    \begin{align*}
        F : \R &\mapsto [0, 1] \\
        x &\mapsto P(X \leq x)
    \end{align*}
\end{definition}
\begin{propositions}{
        Suppose that $F$ is a probability distribution function as above defined.
        \label{propsPDFProps}
    }
    \item If $x \leq y$, then $F(x) \leq F(y)$. \label{propPDFIncreasing}
    \item For any real numbers $a < b$, $P(a < X \leq b) = F(b) - F(a)$. \label{propPDFSubtract}
    \item $F$ is right-continuous, and left limits always exist. \label{propPDFContinuity}
    \item $\lim_{y \to x^-} F(y) = P(X < x)$.\label{propPDFStrictLessThan}
    \item $\lim_{x \to \infty} F(x) = 1$ and $\lim_{x \to -\infty} F(x) = 0$. \label{propPDFLimits}
\end{propositions}
\begin{proof}
    \begin{enumerate}
        \item Simply note that $\{X \leq x\} \subseteq \{X \leq y\}$.
        \item \begin{align*}
            P(A < X \leq B) &= P(X \leq b, X > a) \\
            &= P(X \leq b) - P(X \leq b, X \leq a) \\
            &= P(X \leq b) - P(X \leq a) \\
            &= F(b) - F(a)
        \end{align*}
        \item Let $x_n$ be a decreasing sequence converging to $x$ as $n \to \infty$.
            Define $A_n = \{x < X \leq x_n\}$. Note that $A_n$ is a decreasing sequence ($A_{n + 1} \subseteq A_n$).\par
            Note that for a decreasing sequence,
            \begin{equation*}
                P(A_n) \to P\left(\bigcap_{n \in \N} A_n\right)
            \end{equation*}
            and in this case the infinite intersection is the empty set, so $P(A_n) \to 0$.\par
            Therefore $F(x_n) - F(x) \to 0$, and so $F$ is right-continuous.\par
            To show that left limits exist, we can bound them from above (since $F$ is increasing) by their limit point:
            \begin{equation*}
                \lim_{y \to x^-} F(y) \leq F(x)
            \end{equation*}
        \item $\lim_{y \to x^-} F(y) = \lim_{n \to \infty} F(x - \frac{1}{n})$ (we can choose any increasing sequence converging to $x$).\par
            Define also $B_n = \{X \leq x - \frac{1}{n}\}$. $B_n$ is increasing, and so $P(B_n)$ tends to the probability of the union. The probability of the union is exactly equal to $\{X < x\}$.
        \item Proof by the previous case.
    \end{enumerate}
\end{proof}
\begin{definition}{Continuous random variable}
    A random variable $X$ is \underline{continuous} if its probability distribution function is continuous. That is,
    \begin{equation*}
        \lim_{y \to x^-} F(y) = \lim_{y \to x^+} F(x)
    \end{equation*}
\end{definition}
Immediately this tells us that $P(X < x) = P(X \leq x)$, or $P(X = x) = 0$.\par
$F$ may not always be differentiable. However, in this course, we will assume the differentiability of $F$ on the whole of $\R$.
\begin{definition}{Probability density function}
    Given that, for a random variable $X$, the probability distribution function $F$ is differentiable everywhere, the \underline{probability density function} $f$ is defined to be the derivative $F'$.
\end{definition}
\begin{propositions}{
        Consider a random variable $X$, with probability distribution function $F$ and probability density function $f$.
        \label{propsPdensFuncProps}
    }
    \item $f(x) \geq 0$. \label{propPdensFuncNonNegative}
    \item $\int_{-\infty}^\infty f(x) dx = 1$ \label{propPdensFuncIntegralOne}
    \item $F(x) = \int_{-\infty}^x f(y) dy$. \label{propPdensFuncIntegratePDF}
\end{propositions}
More generally, for any subset of the real numbers $A$,
\begin{equation*}
    P(X \in A) = \int_A f(x) dx
\end{equation*}
\section{Simple Continuous Distributions}
\subsection{Uniform Distribution}
\begin{definition}{Uniform distribution}
    Given a real interval $[a, b]$, the \underline{uniform distribution} on that interval is $U([a, b])$. Its probability density function is: 
    \begin{equation*}
        f(x) =
        \begin{cases}
            \frac{1}{b-a} & x \in [a, b] \\
            0 & \text{otherwise}
        \end{cases}
    \end{equation*}
    The probability $P(X \leq x)$ is:
    \begin{equation*}
        P(X \leq x) = \int_{-\infty}^x f(y) dy = \frac{x - a}{b - a}
    \end{equation*}
\end{definition}
\subsection{Exponential Distribution}
\begin{definition}{Exponential distribution}
    Let $\lambda > 0$ be a real number. Then the \underline{exponential distribution} is $\text{Exp}(\lambda)$, with probability density function:
    \begin{equation*}
        f(x) =
        \begin{cases}
            \lambda e^{-\lambda x} & x > 0 \\
            0 & \text{otherwise}
        \end{cases}
    \end{equation*}
\end{definition}
\begin{proposition}
    Let $T \sim Exp(\lambda)$, let $T_n = \lfloor nT \rfloor$.\par
    Then $T_n$ is a geometric distribution, and
    \begin{equation*}
        \frac{T_n}{n} \to T \text{ as } n \to \infty
    \end{equation*}
    \label{propExpLimitOfGeom}
\end{proposition}
\begin{proof}
    For any $k \in \N$,
    \begin{align*}
        P(T_n \geq k) &= P(\lfloor nT \rfloor \geq k) \\
        &= P(nt \geq k) \\
        &= P(T \geq \frac{k}{n}) \\
        &= e^{-\frac{\lambda k}{n}} \\
        &= \left(e^{-\frac{\lambda}{n}}\right)^k
    \end{align*}
    Note that this is a geometric random variable with success parameter $p_n = 1 - e^{-\frac{\lambda}{n}}$, as required, and this tends to $\frac{\lambda}{n}$ when $n$ is large.\par
    Therefore $\frac{T_n}{n} \to T$ as $n \to \infty$.
\end{proof}
\begin{proposition}[Memoryless property]
    If $T$ is an exponential random variable with parameter $\lambda$,
    \begin{equation*}
        P(T \geq t + s | T \geq s) = P(T \geq t)
    \end{equation*}
    \label{propExpMemoryless}
\end{proposition}
\begin{proof}
    \begin{align*}
        P(T \geq t + s | T \geq s) &= \frac{P(T \geq t + s, T \geq s)}{P(T \geq s)} \\
        &= \frac{P(T \geq t + s)}{P(T \geq s)} \\
        &= \frac{e^{-\lambda(t + s)}}{e^{-\lambda s}} \\
        &= e^{-\lambda t}
    \end{align*}
\end{proof}
\begin{theorem}[Memoryless property unique to exponentials]
    Let $T$ be a positive random variable with a density $g$, not identically equal to $0$ or $\infty$. Then $T$ has the memoryless property only if $T$ is an exponential random variable.
    \label{thmExpMemorylessUnique}
\end{theorem}
\begin{proof}
    $T$ has the memoryless property, so $g(t + s) = g(t) g(s)$.\par
    If $t > 0$ and $m \in \N$,
    \begin{equation*}
        g(mt) = \left(g(t)\right)^m
    \end{equation*}
    Then setting $m = 1$ gives $g(m) = g(1)^m$.\par
    Now set $\lambda = -\log(P(T \geq 1))$, so $P(T \geq 1) = e^{-\lambda}$.\par
    Therefore, for any $m \in \N$, $g(m) = e^{-\lambda m}$. Let also $n \in \N$:
    \begin{equation*}
        g(\frac{m}{n}) = e^{-\lambda \frac{m}{n}}
    \end{equation*}
    Therefore we have the property that $g(r) = r^{-\lambda r}$ for any $r \in \Q_+$.\par
    Let $t$ be a positive real number. Then let $s$ and $r \in \Q_+$ such that $s < t < r$ and $|s - r| \leq \epsilon$. Since $g$ is a decreasing function,
    \begin{equation*}
        g(r) \leq g(t) \leq g(s) \implies e^{-\lambda r} \leq g(t) \leq e^{-\lambda s}
    \end{equation*}
    Then taking the limit as $\epsilon$ goes to zero gives the required result for positive real values also.
\end{proof}
\section{Expectation and Variance}
\subsection{Expectation}
\begin{definition}{Continuous expectation}
    The \underline{expectation} of a non-negative continuous random variable $X$, with a probability density function $f$, is:
    \begin{equation*}
        E[X] = \int_0^\infty xf(x) dx
    \end{equation*}
\end{definition}
\begin{remarks}
    \item We can define the expectation for random variables that take negative values in the same way as in the discrete case, using $X_+$ and $X_-$. This gives:
        \begin{equation*}
            E[X] = \int_{-\infty}^\infty xf(x) dx
        \end{equation*}
    \item Expectation is still a linear function, i.e.
        \begin{equation*}
            E\left[\sum_{i = 1}^n a_i X_i\right] = \sum_{i = 1}^n a_i E[X_i]
        \end{equation*}
\end{remarks}
\begin{lemma}
    Given $X \geq 0$, the expectation is:
    \begin{equation*}
        E[X] = \int_0^\infty P(X \geq x)
    \end{equation*}
\end{lemma}
\begin{proof}
    \begin{align*}
        E[X] &= \int_0^\infty x f(x) dx \\
        &= \int_0^\infty \left(\int_0^x 1 dy\right)f(x) dx \\
        &= \int_0^\infty \left(\int_0^y f(x) dx\right) dy \\
        &= \int_0^\infty P(X \geq y) dy
    \end{align*}
\end{proof}
\subsection{Variance}
\begin{definition}{Variance}
    Given a continuous random variable $X$, the \underline{variance} is:
    \begin{equation*}
        \Var(X) = E\left[(X - E[X])^2\right] = E\left[X^2\right] - \left(E[X]\right)^2
    \end{equation*}
\end{definition}
\begin{remark}
    Note that this definition needs no modification from the discrete case.
\end{remark}
\begin{example}[Expectation and variance of uniform distribution]
    Let $X \sim U([a, b])$. Then the expectation is:
    \begin{align*}
        E[X] &= \int_{-\infty}^\infty xf(x) dx \\
        &= \int_a^b \frac{x}{b-a} dx \\
        &= \frac{b^2-a^2}{2(b-a)} \\
        &= \frac{a + b}{2}
    \end{align*}
    That is, the expectation of a uniform distribution is its midpoint, as we would expect.\par
    The variance of a uniform distribution is $\frac{1}{12}(b - a)^2$.
\end{example}
\begin{example}[Expectation and variance of exponential distribution]
    Let $X \sim Exp(\lambda)$. Then its expectation is:
    \begin{align*}
        E[X] &= \int_0^\infty xf(x) \\
        &= \int_0^\infty x \lambda e^{-\lambda x} \\
        &= \frac{1}{\lambda}
    \end{align*}
    As we would expect, given the result of proposition~\ref{propExpLimitOfGeom}.\par
    Note also that its variance is $\frac{1}{\lambda^2}$.
\end{example}
\section{The Normal Distribution}
\begin{definition}{Normal distribution}
    Given real numbers $\mu$ and $\sigma$, with $\sigma > 0$, define the \underline{normal distribution} with parameters $\mu$ and $\sigma$, $N(\mu, \sigma^2)$. Let the probability density function be:
    \begin{equation*}
        f(x) = \frac{1}{\sqrt{2\pi \sigma^2}} e^{-\frac{(x - \mu)^2}{2\sigma^2}}
    \end{equation*}
\end{definition}
\begin{proposition}
    $f(x)$, as above defined, is a valid density function.
\end{proposition}
\begin{proof}
    We need to show that $I = \int_{-\infty}^\infty f(x) dx = 1$.
    First change variables to $u = \frac{x - \mu}{\sigma}$, so $dx = \sigma du$:
    \begin{align*}
        \int_{-\infty}^\infty \frac{1}{\sigma\sqrt{2\pi}} e^{-\frac{(x - \mu)^2}{2\sigma^2}} &= \int_{-\infty}^\infty \frac{1}{\sigma\sqrt{2\pi}}e^{-\frac{u^2}{2}} \sigma du \\
        &= \int_{-\infty}^\infty \frac{1}{\sqrt{2\pi}}e^{-\frac{u^2}{2}} du
    \end{align*}
    Then consider $I^2$ and use polar coordinates:
    \begin{align*}
        I^2 &= \frac{1}{2\pi} \int_{x = -\infty}^\infty \int_{y = -\infty}^\infty e^{-\frac{x^2}{2}} e^{-\frac{y^2}{2}} dx dy \\
        &= \frac{1}{2\pi} \int_{x = -\infty}^\infty \int_{y = -\infty}^\infty e^{-\frac{x^2 + y^2}{2}} dx dy \\
        &= \frac{1}{2\pi} \int_{r = 0}^\infty \int_{\phi = 0}^{2\pi} e^{-\frac{r^2}{2}} r dr d\phi \\
        &= \frac{1}{2\pi} \int_{r = 0}^\infty \int_{\phi = 0}^{2\pi} e^{-\frac{r^2}{2}} r dr d\phi \\
        &= \int_{r = 0}^\infty \frac{d}{dx}\left(-e^{-\frac{r^2}{2}}\right) dr \\
        &= 1
    \end{align*}
    Therefore, since $f(x)$ is a positive function, we must have the positive root:
    \begin{equation*}
        I = 1
    \end{equation*}
\end{proof}
\begin{propositions}{
        Let $X \sim N(\mu, \sigma^2)$.
        \label{propsNormalMeanVar}
    }
    \item $E[X] = \mu$ \label{propNormalMean}
    \item $\Var(X) = \sigma^2$ \label{propNormalVar}
\end{propositions}
\begin{proof}
    \begin{enumerate}
        \item Expectation given by:
            \begin{align*}
                E[X] &= \int_{-\infty}^\infty xf(x) dx \\
                &= \int_{-\infty}^\infty x \frac{1}{\sigma \sqrt{2\pi}} e^{-\frac{(x - \mu)^2}{2\sigma^2}} dx \\
                &= \int_{-\infty}^\infty \frac{x - \mu}{\sigma \sqrt{2\pi}} e^{-\frac{(x - \mu)^2}{2\sigma^2}} dx + \int_{-\infty}^\infty \frac{\mu}{\sigma\sqrt{2\pi}} e^{-\frac{(x - \mu)^2}{2\sigma^2}} dx \\
                &= 0 + \mu
            \end{align*}
        \item Variance given by:
            \begin{align*}
                \Var(X) &= E[(X - \mu)^2] \\
                &= \int_{-\infty}^\infty \frac{(x - \mu)^2}{\sqrt{2\pi \sigma^2}} e^{-\frac{(x - \mu)^2}{2\sigma^2}} dx
            \end{align*}
            Change variables to $u = \frac{x - \mu}{\sigma}$:
            \begin{align*}
                \Var(X) &= \int_{-\infty}^\infty \frac{\sigma^2u^2}{\sqrt{2\pi}} e^{-\frac{u}{2}}du \\
                &= \sigma^2 \int_{-\infty}^\infty \frac{u^2}{\sqrt{2\pi}} e^{-\frac{u}{2}} du
            \end{align*}
            Which is equal to $\sigma^2$ (integration by parts).
    \end{enumerate}
\end{proof}
\subsection{Transforming a Distribution}
\begin{theorem}
    Let $X$ have density $f$. Let $g$ be a continuous function which is strictly monotone and has inverse $g^{-1}$ which is differentiable.\par
    Then $g(X)$ has a density given by:
    \begin{equation*}
        f(g^{-1}(x)) \left|\frac{d(g^{-1})}{dx}\right|
    \end{equation*}
\end{theorem}
\begin{proof}
    \begin{case}{$g$ is strictly increasing.}
        \begin{align*}
            P(g(X) \geq x) &= P(X \leq g^{-1}(x)) \\
            &= F(g^{-1}(x))
        \end{align*}
        Then denote by $f_{g(X)}$ the density function of $g(X)$.
        \begin{align*}
            f_{g(X)} &= \frac{d}{dx} P(g(X) \leq x) \\
            &= f(g^{-1}(x)) \frac{d(g^{-1}(x))}{dx}
        \end{align*}
        Note also that since $g$ is increasing, the derivative of its inverse (which is also increasing) is positive.
    \end{case}
    \begin{case}{$g$ is strictly decreasing.}
        Proceed as above, but note:
        \begin{equation*}
            P(g(X) \geq x) = -f(g^{-1}(x)) \frac{d(g^{-1}(x))}{dx}
        \end{equation*}
        and that its derivative is negative, but since we use absolute value:
        \begin{equation*}
            f_{g(X)} = f(g^{-1}(x)) \left|\frac{d(g^{-1})}{dx}\right|
        \end{equation*}
    \end{case}
\end{proof}
\begin{example}[Transforming the normal distribution]
    Suppose $X \sim N(\mu, \sigma^2)$. Let $a, b$ be real numbers and $a \neq 0$. Define $g(x) = ax + b$, and $g(X) = Y$. Then we consider the density function of $Y$.\par
    First note that $g^{-1} = \frac{x - b}{a}$, and that its derivative is $\frac{1}{a}$. Then $f_{Y}$ is:
    \begin{align*}
        f_Y(y) &= f_X (g^{-1}(y)) \left|\frac{d(g^{-1}(y))}{dy}\right| \\
        &= \frac{1}{\sigma\sqrt{2\pi}} \exp{\left(-\frac{\left(\frac{y - b}{a} - \mu\right)^2}{2\sigma^2}\right)} \frac{1}{a} \\
        &= \frac{1}{a\sigma\sqrt{2\pi}} \exp{\left(-\frac{y - (a\mu + b)}{2(a\sigma)^2}\right)}
    \end{align*}
    Which is a distribution $N(a\mu + b, (a\sigma)^2)$.\par
    We can therefore find any normal distribution by transforming the standard normal: suppose that $X \sim N(\mu, \sigma^2)$. Then $\frac{X - \mu}{\sigma}$ is a standard normal random variable.
\end{example}
\subsection{The Standard Normal}
\begin{definition}{Standard normal distribution}
    If $X \sim N(0, 1)$, $X$ has the \underline{standard normal distribution}. Its probability density function is:
    \begin{equation*}
        \frac{1}{\sqrt{2\pi}} e^{-\frac{x^2}{2}}
    \end{equation*}
\end{definition}
\begin{example}[More on the standard normal]
    Define:
    \begin{align*}
        \Phi(x) &= \int_{-\infty}^x \frac{1}{\sqrt{2\pi}} e^{-\frac{u^2}{2}} du = P(N(0, 1) \leq x) \\
        \phi(x) &= \Phi'(x) = \frac{1}{\sqrt{2\pi}} e^{-\frac{x^2}{2}}
    \end{align*}
    Note now that $\phi$ is even: $\phi(x) = \phi(-x)$.
    %TODO: Continue
\end{example}
The ability to transform any normal distribution to a standard normal is very useful, only one table of probabilities is needed (that of the standard normal) to be able to calculate any normal distribution probability.
\subsection{The Median of a Distribution}
\begin{definition}{Median}
    Let $X$ be a continuous random variable. Then the \underline{median} of $X$, $m$, is the number that satisfies:
    \begin{equation*}
        P(X \leq m) = P(X \geq m)
    \end{equation*}
\end{definition}
\begin{example}[Median of $X$]
    We can transform any normal distribution to the standard normal, so consider only the standard normal. As above calculated, $\Phi(0) = \frac{1}{2}$, and so the median is 0. Transforming back to the general normal distribution, this gives that the median is $\mu$.
\end{example}
\section{Multivariate Density Functions}
\begin{definition}{Multivariate density function}
    Let $X$ be a continuous random variable. Let $(X_i)_{i=1}^n$ be continuous random variables. Then $X$ has \underline{density function} $f$ if:
    \begin{equation*}
        P(X_1 \leq x_1, \cdots, X_n \leq x_n) = \int_{-\infty}^{x_1} \cdots \int_{-\infty}^{x_n} f(y_1, \cdots, y_n) dy_1 \cdots dy_n
    \end{equation*}
\end{definition}
Note that we can further generalise this: for a set $B \subseteq R^n$,
\begin{equation*}
    P((X_1, \cdots, X_n) \in B) = \int_B f(y_1, \cdots, y_n) dy_1 \cdots dy_n
\end{equation*}
\subsection{Independence of Random Variables}
\begin{definition}{Independence}
    If $X_1, \cdots, X_n$ are random variables, we say that they are \underline{independent} if for any $x_1, \cdots, x_n$:
    \begin{equation*}
        P(X_1 \leq x_1, \cdots, X_n \leq x_n) = \prod_{i=1}^{n} P(X_i \leq x_i)
    \end{equation*}
\end{definition}
\begin{theorem}
    Let $X = (X_1, \cdots, X_n)$ have density $f$.\par
    Suppose that $X_i$ are independent with densities $f_i$, $i = 1, \cdots, n$. Then:
    \begin{equation}
        f(x_1, \cdots, x_n) = \prod_i f_i(x_i)
        \label{eqnDensityFuncProduct}
    \end{equation}
    Conversely, suppose that $f$ factorises as in equation~\ref{eqnDensityFuncProduct} for some non-negative functions $f_i$. Then $X_i$ are independent and have densities proportional to the $f_i$.
    \label{thmDensityFactorising}
\end{theorem}
\begin{proof}
    \begin{proofdirection}{$\Rightarrow$}{Suppose that $X_i$ are independent with densities $f_i$}
        \begin{align*}
            P(X_1 \leq x_1, &\cdots, X_n \leq x_n) = \prod_{i = 1}^n P(X_i \leq x_i) \\
            &= \prod_{i = 1}^n \int_{-\infty}^{x_i} f_i(y_i) dy_i \\
            &= \int_{-\infty}^{x_1} \int_{-\infty}^{x_2} \cdots \int_{-\infty}^{x_n} f_1(y_1) \cdots f_1(y_n) dy_1 dy_2 \cdots dy_n
        \end{align*}
        Which, by differentiation $n$ times with respect to $y_i$ for each $i$, gives the result.
    \end{proofdirection}
    \begin{proofdirection}{$\Leftarrow$}{Suppose that $f$ factorises as above.}
        Fix $i$ and defined $B_j = \R \forall j \neq i$. Let $B_i \subseteq \R$.
        \begin{align*}
            P(X_i \in B_i) &= P(X_i \in B_i, X_j \in B_j~\forall j \neq i) \\
            &= \int_{B_i} \int_{B_1} \cdots_{\text{excl. } i} \int_{B_n} f_1(x_1) \cdots f_n(x_n) dx_1 \cdots dx_n \\
            &= \int_{B_i} f_i(x_i) dx_i \prod_{j \neq i} \int_{-\infty}^\infty f_j(x_j) dx_j
        \end{align*}
        However, we know that $f$ is a density. Therefore,
        \begin{equation*}
            \int_{-\infty}^\infty \cdots \int_{-\infty}^\infty f_1(x_1) \cdots f_n(x_n) dx_1 \cdots dx_n = 1
        \end{equation*}
        Therefore, we have that:
        \begin{equation*}
            \implies \int_{-\infty}^\infty f_i(x_i) dx_i \left(\prod_{j \neq i} \int_{-\infty}^\infty f_j(x_j) dx_j\right) = 1
        \end{equation*}
        So we can get the required probability:
        \begin{equation*}
            P(X_i \in B_i) = \frac{\int_{B_i} f_i(x_i) dx_i}{\int_{-\infty}^\infty f_i(x_i) dx_i}
        \end{equation*}
        And therefore the density of $X_i$ is:
        \begin{equation*}
            \frac{f_i}{\int_{-\infty}^\infty f_i(x) dx}
        \end{equation*}
        Then consider the probability with this new density:
        \begin{align*}
            P(X_1 \leq x_1)&\cdots P(X_n \leq x_n) = \prod_{i=1}^{n} \frac{\int_{-\infty}^{x_i} f_i(y_i) dy_i}{\int_{-\infty}^\infty f_i(x_i) dx_i} \\
            &= \frac{\int_{y_1 = -\infty}^{x_1} \cdots \int_{y_n = -\infty}^{x_n} f_1(y_1) \cdots f_n(y_n) dy_1 \cdots dy_n}{\int_{y_1 = -\infty}^{\infty} \cdots \int_{y_n = -\infty}^{\infty} f_1(y_1) \cdots f_n(y_n) dy_1 \cdots dy_n} \\
            &= \frac{\int_{y_1 = -\infty}^{x_1} \cdots \int_{y_n = -\infty}^{x_n} f(y_1, \cdots, y_n) dy_1 \cdots dy_n}{\int_{y_1 = -\infty}^{\infty} \cdots \int_{y_n = -\infty}^{\infty} f(y_1, \cdots y_n) dy_1 \cdots dy_n} \\
            &=\frac{P(X_1 \leq x_1, \cdots, X_n \leq x_n)}{1}
        \end{align*}
    \end{proofdirection}
\end{proof}
Therefore, suppose that $X = (X_1, \cdots, X_n)$ has density $f$.
\begin{align*}
    P(X_1 &\leq x) = P(X_1 \leq x, X_2 \leq \infty, \cdots, X_n \leq \infty) \\
    &= \int_{x_1 = -\infty}^x \int_{x_2 = -\infty}^\infty \cdots \int_{x_n = -\infty}^\infty f(x_1, x_2, \cdots, x_n) dx_1 \cdots dx_n \\
    &= \int_{x_1 = -\infty}^x \left(\int_{x_2 = -\infty}^\infty \cdots \int_{x_n = -\infty}^\infty f(x_1, \cdots, x_n) dx_2 \cdots dx_n\right)dx_1 \\
\end{align*}
Therefore the density of $X_1$ is:
\begin{equation*}
    f_{X_1}(x) = \int_{x_2 = -\infty}^\infty \cdots \int_{x_n = -\infty}^\infty f(x, x_2, \cdots, x_n) dx_2 \cdots dx_n
\end{equation*}
by differentiating. This is the \underline{marginal density} of $X_1$.
\subsection{Summing Random Variables}
Consider random variables $X$ and $Y$ with densities $f_X$ and $f_Y$, which are independent. We want to consider $f_{X + Y}$.\par
In the discrete case, we had the convolution:
\begin{equation*}
    P(X + Y = z) = \sum_x P(X = x)P(Y = z - x)
\end{equation*}
Therefore in the continuous case, we can do a similar thing:
\begin{proposition}[Convolutions in the continuous case]
    For random variables $X$ and $Y$ as above defined,
    \begin{equation*}
        f_{X + Y}(z) = \int_{-\infty}^\infty f_X(x) f_Y(z - x) dx
    \end{equation*}
    \label{propSumContRVs}
\end{proposition}
\begin{proof}
    Let the joint density of $X$ and $Y$ be $f_{X, Y}$.
    \begin{align*}
        P(X + Y \leq z) &= \int_{\{x + y \leq z\}} f_{X, Y} (x, y) dx dy \\
        &= \int_{\{x + y \leq z\}} f_X(x) f_Y(y) dx dy \\
        &= \int_{x = -\infty}^\infty \int_{y = -\infty}^{z - x} f_X(x) F_Y(y) dx dy \\
        &= \int_{-\infty}^\infty f_X(x) dx \int_{-\infty}^{z - x} f_Y(y) dy \\
        &= \int_{-\infty}^\infty f_X(x) dx \int_{-\infty}^z f_Y(y - x) dy \\
        &= \int_{-\infty}^z \left(\int_{-\infty}^\infty f_X(x) f_Y(y-x)dx\right)dy
    \end{align*}
    And therefore we have the required result.
\end{proof}
This is important enough that we use a special notation:
\begin{definition}{Convolution}
    Given two independent densities $f$ and $g$, the \underline{convolution} is defined to be:
    \begin{equation}
        f \star g(x) = \int_{-\infty}^\infty f(x - y) g(y) dy
        \label{eqnContConvolution}
    \end{equation}
\end{definition}
\subsection{Conditional Density and Expectation}
\begin{definition}{Conditional density}
    Let $X$ and $Y$ be random variables with joint density $f_{X, Y}$ and marginal densities $f_X$ and $f_Y$. Then the \underline{conditional density} of $X$ given $Y = y$ is:
    \begin{equation*}
        F_{X | Y}(x | y) = \frac{f_{X, Y}(x, y)}{F_Y(y)}
    \end{equation*}
\end{definition}
This is as would be expected given the continuous case.\par
We have also the Law of Total Probability:
\begin{equation*}
    f_X(x) = \int_{-\infty}^\infty f_{X | Y}(x | y) f_Y(y) dy
\end{equation*}
\begin{definition}{Conditional expectation}
    Given $X$ and $Y$ random variables, with densities as above defined, the \underline{conditional expectation} of $X$ given $Y = y$ is:
    \begin{equation*}
        g(y) = \int_{-\infty}^\infty xf_{X | Y} (x | y) dx
    \end{equation*}
    Then applying this function to the random variable $Y$ gives:
    \begin{equation*}
        E[X | Y] = g(Y)
    \end{equation*}
    Which, as in the discrete case, is a random variable depending on $Y$.
\end{definition}
\begin{theorem}
    Let $X$ be a random variable with values in $D \subseteq \R^d$ with density $f$. Let $g$ be a bijection from $D$ to $g(D)$ with a continuous derivative on $D$ and $\det{(g'(x) \neq 0)}~\forall x \in D$.\par
    Then the random variable $Y = g(X)$ has density given by:
    \begin{equation*}
        f_Y(y) = f_X(x) |J|
    \end{equation*}
    where $x = g^{-1}(y)$ and $J$ is the Jacobian of the transform $g$.
    \label{thmTransformRV}
\end{theorem}
This theorem will be used without proof.
\begin{example}
    Let $X$ and $Y$ be independent random variables with standard normal distribution.\par
    Consider now a change of variables from $(X, Y)$ to $(R, \Theta)$:
    \begin{align*}
        f_{R, \Theta}(r, \theta) &= f_{X, Y}(r \cos(\theta), r\sin(\theta)) |J| \\
        &= f_{X, Y}(r \cos(\theta), r\sin(\theta)) r \\
        &= rf_X(r\cos(\theta)) f_Y(r \cos(\theta)) \\
        &= r\frac{1}{\sqrt{2\pi}} e^{-\frac{r^2 \cos^2(\theta)}{2}} \frac{1}{\sqrt{2\pi}} e^{-\frac{r^2 \sin^2(\theta)}{2}} \\
        &= \frac{1}{2\pi} r e^{-\frac{r^2}{2}}
    \end{align*}
    Which, notably, is independent of $\theta$.\par
    We have that:
    \begin{equation*}
        f_{R, \Theta} = \left(\frac{1}{2\pi}\right) \times \left(re^{-\frac{r^2}{2}}\right)
    \end{equation*}
    So the density function factorises into a function of $r$, the second bracket, and a ``function'' of $\theta$ (first bracket, which is independent of $\theta$).

    Therefore, by theorem~\ref{thmDensityFactorising}, we can extract the individual densities:
    \begin{align*}
        f_R(r) &= re^{-\frac{r^2}{2}} \\
        F_\Theta(\theta) &= \frac{1}{2\pi}
    \end{align*}
\end{example}
\subsection{Order Statistics}
Suppose $X_1, \cdots, X_n$ are identically distributed independent random variables. Suppose they have combined distribution function $F$ and combined density $f$.

We consider putting them in increasing order:
\begin{equation*}
    X_{(1)} \leq X_{(2)} \leq \cdots \leq X_{(n)}
\end{equation*}
And set $Y_i = X_{(i)}$. Here $Y_i$ are called the order statistics of the random sample. We want to calculate the density of the $Y_i$:
\begin{align*}
    P(Y_1 \leq x) &= P(\min_{i \in \{1, \cdots, n\}} X_i \leq x) \\
    &= 1 - P(\min_{i \in \{1, \cdots, n\}} X_i > x) \\ 
    &= 1 - P(X_1 > x)^n \text{ since all must be greater than } x \\
    &= 1 - (1 - F(X))^n
\end{align*}
So the density is:
\begin{equation*}
    f_{Y_1}(x) = n(1 - F(x))^{n-1} f(x)
\end{equation*}
by differentiation.

We can also consider $Y_n$:
\begin{align*}
    P(Y_n \leq x) &= P(\max_{i \in \{1, \cdots, n\}} X_i \leq x) \\
    &= P(X_1 \leq x)^n \\
    &= (F(x))^n
\end{align*}
So by differentiation:
\begin{equation*}
    f_{Y_n}(x) = n(F(x))^{n-1} f(x)
\end{equation*}
Then we consider the combined density of all the $Y_i$.
\begin{align*}
    &P(Y_1 \leq x_1, \cdots, Y_n \leq x_n) \\
    &= n! P(X_1 \leq x_1, \cdots, X_n \leq x_n, X_1 < X_2 < \cdots < X_n) \\
    &= n! \int \cdots \int I_{u_1 \leq x_1, \cdots, u_n \leq x_n, u_1 \leq \cdots \leq u_n} \times \\
    & f(u_1) f(u_2) \cdots f(u_n) du_1 \cdots du_n \\
    &= n! \int_{u_1 = -\infty}^{x_1} \int_{u_2 = u_1}^{x_2} \cdots \int_{u_n = u_{n - 1}}^{x_n} f(u_1) \cdots f(u_n) du_1 \cdots du_n
\end{align*}
So by differentiating with respect to every $x_i$:
\begin{equation*}
    f_{Y_1, \cdots, Y_n}(x_1, \cdots, x_n) =
    \begin{cases}
        n! f(x_1) f(x_2) \cdots f(x_n) & \text{if } x_i \text{ are in order} \\
        0 & \text{otherwise}
    \end{cases}
\end{equation*}
\begin{example}
    Let $X \sim \text{Exp}(\lambda)$, and let $Y \sim \text{Exp}(\mu)$. Let them be independent.

    Let $Z = \min\{X, Y\}$.
    \begin{align*}
        P(Z \leq z) = P(\min\{X, Y\} \leq z) \\
        &= 1 - P(\min\{X, Y\} > z) \\
        &= 1 - e^{-\lambda z} e^{-\mu z} \\
        &= 1 - e^{-(\lambda + \mu) z}
    \end{align*}
    So $Z \sim \text{Exp}(\lambda + \mu)$. Applying this logic inductively, if $X_i$ are independent exponential random variables with parameters $\lambda_i$,
    \begin{equation*}
        \min\{X_1, \cdots, X_n\} \sim \text{Exp}\left(\sum_{i = 1}^n \lambda_i\right)
    \end{equation*}
\end{example}
\begin{example}
    Let $X_i, i \in \{1, \cdots, n\}$ be independent and identically distributed exponential random variables with parameter $\lambda$. Let $Y_i$ be their order statistics. Set $Z_1 = Y_1$, $Z_i = Y_i - Y_{i-1}, i \geq 2$. Then consider the joint density of the $Z_i$.

    Define the vectors $\vec{Z}$ and $\vec{Y}$ which are related by the matrix $A$:
    \begin{equation*}
        \begin{pmatrix}Z_1 \\ \vdots \\ Z_n\end{pmatrix} = A \begin{pmatrix}Y_1 \\ \vdots \\ Y_n\end{pmatrix}
    \end{equation*}
    where $A$ is:
    \begin{equation*}
        A = 
        \begin{pmatrix}
            1&0&0&\cdots&0 \\
            -1&1&0&\cdots&0 \\
            0&-1&1& & \vdots \\
            \vdots& &\ddots&\ddots&0 \\
            0 &\cdots&0&-1&1
        \end{pmatrix}
    \end{equation*}
    and has determinant 1.
    \begin{align*}
        f_{Z_1, \cdots, Z_n}(z_1, \cdots, z_n) &= f_{Y_1, \cdots, Y_n}(y_1, \cdots, y_n) |J| \\
        &= n! f(y_1) \cdots f(y_n) \\
        &= n! \lambda e^{-\lambda y_1} \cdots \lambda e^{-\lambda y_n} \\
        &= \prod_{i = 1}^n (n - i + 1) \lambda e^{-\lambda(n - i + 1)z_i}
    \end{align*}
    So $Z_i$ are independent with $Z_i \sim \text{Exp}(\lambda(n - i + 1))$.
\end{example}
\section{Moment Generating Functions}
\begin{definition}{Moment generating function}
    Let $X$ be a random variable with density $f$. Then the \\\underline{moment generating function} is defined to be:
    \begin{equation*}
        m(\theta) = E[e^{\theta X}] = \int_{-\infty}^\infty e^{\theta x} f(x) dx
    \end{equation*}
    whenever this integral is finite. $m(0) = 1$.
\end{definition}
The following theorems will be used without proof:
\begin{theorem}
    The moment generating function uniquely determines the distribution of a random variable provided it is defined for an open interval of values of $\theta$.
    \label{thmMGFDetermines}
\end{theorem}
\begin{theorem}
    Suppose the moment generating function is defined for an open interval of values of $\theta$. Then we can get the moments of $\theta$:
    \begin{equation*}
        E[X^r] = m^{(r)}(0)
    \end{equation*}
    that is, the $r$th derivative evaluated at 0.
\end{theorem}
\subsection{Gamma Distribution}
\begin{definition}{Gamma distribution}
    A random variable $X$ has the \underline{gamma distribution} with parameter $n \in \N$ if its density function is:
    \begin{equation*}
        f(x) = \frac{e^{-\lambda x} \lambda^n x^{n-1}}{(n-1)!}, x > 0
    \end{equation*}
\end{definition}
\begin{proposition}
    The gamma function, as above defined, has a valid density.
    \label{propGammaDensityValid}
\end{proposition}
\begin{proof}
    Define:
    \begin{align*}
        I_n &= \int_0^\infty f(x) dx \\
        &= \int_0^\infty \frac{\left(\lambda e^{-\lambda x}\right)\lambda^{x-1} x^{n-1}}{(n-1)!} \\
        &= \int_0^\infty \frac{e^{-\lambda x}\lambda^{n-1} x^{n-2}}{(n-2)!} \text{ by integration by parts} \\
        &= I_{n-1} = \cdots = I_1 \\
        &= \int_0^\infty \lambda e^{-\lambda x} dx \\
        &= 1
    \end{align*}
\end{proof}
If $X$ has density $f$, $X \sim \Gamma(n, \lambda)$, we can find the moment generating function:
\begin{align*}
    m(\theta) &= E[e^{\theta X}] \\
    &= \int_0^\infty e^{\theta x} e^{-\lambda x} \frac{\lambda^n x^{n-1}}{(n-1)!} \\
    &= \int_0^\infty e^{-(\lambda - \theta)x} \frac{(\lambda - \theta)^n x^{n-1}}{(n-1)!} dx \frac{\lambda^n}{(\lambda - \theta)^n} \\
    &= \left(\frac{\lambda}{\lambda - \theta}\right)^n
\end{align*}
\begin{proposition}
    If $X_1, \cdots, X_n$ are independent, then:
    \begin{equation*}
        m(\theta) = E[e^{\theta (X_1 + \cdots + X_n)}] = \prod_{i=1}^n E[e^{\theta X_i}]
    \end{equation*}
    \label{propMGFFactorise}
\end{proposition}
Then we can consider two independent gamma distributions: \\$X \sim \Gamma(n, \lambda)$ and $Y \sim \Gamma(m, \lambda)$. Then the moment generating function of $X+Y$ is:
\begin{align*}
    E[e^{\theta X}] E[e^{\theta Y}] &= \left(\frac{\lambda}{\lambda - \theta}\right)^n \left(\frac{\lambda}{\lambda - \theta}\right)^m \\
    &= \left(\frac{\lambda}{\lambda - \theta}\right)^{n+m}
\end{align*}
And therefore $X + Y \sim \Gamma(n+m, \lambda)$.

Then we can consider a sum of identically distributed independent exponential random variables. Let $X_i \sim \text{Exp}(\lambda), i \in \{1, \cdots, n\}$. Note also that $\Gamma(1, \lambda)$ is the same as the exponential distribution. Therefore applying the above result for two gamma distributions inductively:
\begin{equation*}
    \sum_{i = 1}^n X_i \sim \Gamma(n, \lambda)
\end{equation*}
We can also define: $\Gamma(\alpha, \lambda)$ for any $\alpha \in \R^+$. The only problem so far with our definition is that we have a factorial in the denominator. We replace the $(n-1)!$ with:
\begin{equation*}
    \Gamma(\alpha) = \int_0^\infty e^{-x} x^{\alpha - 1}
\end{equation*}
Then the density function $\Gamma(\alpha, \lambda)$ is:
\begin{equation*}
    f(x) = \frac{e^{-\lambda x}\lambda^\alpha x^{\alpha - 1}}{\Gamma(\alpha)}
\end{equation*}
\subsection{The Normal Distribution}
If $X \sim N(\mu, \sigma^2)$, then $X$ has density:
\begin{equation*}
    f(x) = \frac{1}{\sigma\sqrt{2\pi}} \exp\left(-\frac{(x - \mu)^2}{2\sigma^2}\right)
\end{equation*}
Then $m(\theta) = E[e^{\theta X}]$ is given by:
\begin{equation*}
    \int_{-\infty}^\infty e^{\theta x} \frac{1}{\sigma\sqrt{2\pi}} \exp\left(-\frac{(x - \mu)^2}{2\sigma^2}\right) dx
\end{equation*}
Then consider the exponent:
\begin{align*}
    &\frac{2\sigma^2\theta x}{2\sigma^2} - \frac{x^2}{2\sigma^2} + \frac{2x\mu}{2\sigma^2} - \frac{\mu^2}{2\sigma^2} \\
    &= \frac{1}{2\sigma^2}\left(-x^2 + 2x(\theta \sigma^2 + \mu) - \mu^2 + (\theta \sigma^2 + \mu)^2 - (\theta \sigma^2 + \mu)^2\right) \\
    &= \frac{1}{2\sigma^2}\left(-(x - (\mu + \theta \sigma^2))^2 - \mu^2 + \theta^2 \sigma^4 + 2\theta \sigma^2 \mu + \mu^2\right) \\
    &= -\frac{1}{2\sigma^2}(x - (\mu + \theta \sigma^2))^2 + \frac{\theta^2 \sigma^2}{2} + \theta \mu
\end{align*}
Then substituting this back into the original formula:
\begin{align*}
    m(\theta) &= \int_{-\infty}^\infty \frac{1}{\sigma\sqrt{2\pi}} \exp\left(-\frac{(x - (\mu + \theta \sigma^2))^2}{2\sigma^2}\right)dx \\
    &\times \exp\left(\frac{\theta^2 \sigma^2}{2} + \theta \mu\right) \\
    &= \exp\left(\frac{\theta^2 \sigma^2}{2} + \theta \mu\right) \text{ by recognising normal dist.}
\end{align*}
Now we have a moment generating function, we can consider the sum of two independent normal distributions. Consider $X \sim N(\mu, \sigma^2)$ and $Y \sim N(\nu, \tau^2)$.
\begin{align*}
    E[e^{\theta(X + Y)}] &= E[e^{\theta X}] E[e^{\theta Y}] \\
    &= \exp\left(\theta \mu + \frac{\theta^2 \sigma^2}{2}\right) \exp\left(\theta \nu + \frac{\theta^2 \tau^2}{2}\right) \\
    &= \exp\left(\theta (\mu + \nu) + \frac{\theta^2 (\sigma^2 + \tau^2)}{2}\right)
\end{align*}
So $X + Y \sim N(\mu + \nu, \sigma^2 + \tau^2)$.
\subsection{Cauchy Distribution}
Define the \underline{Cauchy Distribution} with density:
\begin{equation*}
    f(x) = \frac{1}{\pi (1 + x^2)}
\end{equation*}
Then the moment generating function is given by:
\begin{equation*}
    m(\theta) = \int_{-\infty}^\infty \frac{e^{\theta x}}{\pi(1 + x^2)}dx
\end{equation*}
However, this is infinite for all non-zero $\theta$. Furthermore, $X, 2X, \cdots$ all have the same moment generating function, but they cannot have the same distribution. Therefore, this provides a counterexample to theorem~\ref{thmMGFDetermines} in the case where $m(\theta)$ is not finite on an open interval of values of $\theta$.
\subsection{Multivariate Moment Generating Functions} %TODO: Figure out what needs to be vector bold.
\begin{definition}{Multivariate moment generating function}
    Let $X = (X_1, X_2, \cdots, X_n) \in \R^n$ be a random variable. Then the \underline{moment generating function} of $X$ is defined to be:
    \begin{align*}
        m(\theta) &= E[e^{\theta^T X}] = E[e^{\sum_{i=1}^n \theta_i X_i}] \\
        \theta &= (\theta_1, \cdots, \theta_n)^T
    \end{align*}
\end{definition}
\begin{theorem}
    If the moment generating function is finite for an open set of values of $\theta$, then it uniquely determines the distribution.

    In this case,
    \begin{equation*}
        \frac{\partial^r m}{\partial \theta_i^r} \vert_{\theta = 0} = E[X_i^r]
    \end{equation*}
    and also:
    \begin{equation*}
        \frac{\partial^{r + s}m}{\partial x_i^r \partial x_j^s} \vert_{\theta = 0} = E[X_i^r X_j^s]
    \end{equation*}
    \label{thmMGFMultiVarDetermines}
\end{theorem}
\begin{proposition}
    The moment generating function factorises:
    \begin{equation*}
        m(\theta) = \prod_{i = 1}^n E[e^{\theta_i X_i}]
    \end{equation*}
    if and only if the $X_i$ are independent.
    \label{propMGFMultiVarFactorise}
\end{proposition}
\subsection{Multidimensional Normal Random Variables}
Recall that a random variable in $\R$, $X$, is called Gaussian or Normal in $\R$ if it can be written as $X = \mu + \sigma Z$, where $Z \sim N(0, 1)$.
\begin{definition}{Gaussian Vector}
    Let $\vec{X} = (X_1, \cdots, X_n)^T$ with values in $R^n$. $\vec{X}$ is a \underline{Gaussian Vector} if for all $\vec{u} \in \R^n$:
    \begin{equation*}
        \vec{u}^T \vec{X} = \sum_{i=1}^n u_i X_i \text{ is a Gaussian random variable in } \R.
    \end{equation*}
\end{definition}
\begin{proposition}
    Let $A$ be an $m \times n$ matrix and $\vec{b} \in \R^m$. Then let $\vec{X} = (X_1, \cdots, X_n)^T$ be a Gaussian vector. Then $A\vec{X} + \vec{b}$ is a Gaussian vector in $R^m$.
    \label{propGaussVecLinearMap}
\end{proposition}
\begin{proof}
    Let $\vec{u} = (u_1, \cdots, u_m)^T$. Then:
    \begin{equation*}
        \vec{u}^T (AX + \vec{b}) = (\vec{u}^T A)\vec{X} + \vec{u}^T \vec{b}
    \end{equation*}
    Then set $\vec{v} = A^T \vec{u}$.
    \begin{equation*}
        \vec{u}^T(A\vec{X} + \vec{b}) = \vec{v}^T \vec{X} + \sum_{i = 1}^m u_i b_i
    \end{equation*}
    So, since $\vec{X}$ is Gaussian, so is $\vec{v}^T \vec{X}$ and so $A\vec{X} + \vec{b}$ must also be Gaussian.
\end{proof}
Let $\vec{X} = (X_1, \cdots X_n)$ in $\R^n$ be Gaussian. Set $\vec{\mu} = E[\vec{X}]$, which is a vector defined such that $\mu_i = E[X_i]$. Define also the variance of the vector to be $E[(\vec{X} - \vec{\mu})(\vec{X} - \vec{\mu})^T]$. This gives an $n \times n$ matrix, where the entry $(Var(X))_{i, j} = \Cov(X_i, X_j)$. Note that the elements on the diagonal are therefore $\Var(X_i)$, and that this matrix is symmetric.
\begin{equation*}
    \begin{pmatrix}
        \Var(X_1) & \Cov(X_1, X_2) & \Cov(X_1, X_3) & \cdots & \Cov(X_1, X_n) \\
        \Cov(X_1, X_2) & \Var(X_2) & \Cov(X_2, X_3) & \cdots & \Cov(X_2, X_n) \\
        \Cov(X_1, X_3) & \Cov(X_2, X_3) & \Var(X_3) & & \vdots \\
        \vdots & \vdots & & \ddots & \Cov(X_{n-1}, X_n) \\
        \Cov(X_1, X_n) & \Cov(X_2, X_n) & \cdots & & \Var(X_n) \\
    \end{pmatrix}
\end{equation*}
\begin{align*}
    \Var(\vec{u}^T \vec{X}) &= \Var\left(\sum_{i=1}^n u_i X_i\right) \\
    &= \sum_{i, j=1}^n u_i \Cov(X_i, X_j) x_j \\
    &= \vec{u}^T V \vec{u}
\end{align*}
\begin{proposition}
    $V = \Var(X)$ is a non-negative definite matrix. That is, for any vector $\vec{u}$, $\vec{u}^T V \vec{u} \geq 0$.
    \label{propVarNonNegDefinite}
\end{proposition}
\begin{proof}
    Simply note that variance is always non-negative, so:
    \begin{align*}
        \vec{u}^T V \vec{u} = \Var(\vec{u}^T \vec{X}) \geq 0
    \end{align*}
\end{proof}
We can then consider the Moment Generating Function for $\vec{X}$.

\begin{align*}
    m(\vec{\lambda}) &= E[e^{\vec{\lambda}^T \vec{X}}] \\
    \vec{\lambda}^T \vec{X} &\sim N(\vec{\lambda}^T \vec{\mu}, \vec{\lambda}^T V \vec{\lambda}) \\
    \therefore m(\vec{\lambda}) &= \text{exp}\left({\vec{\lambda}^T \vec{\mu} + \frac{\vec{\lambda^T} V \vec{\lambda}}{2}}\right)
\end{align*}
Therefore, since the moment generating function uniquely characterises the distribution, we have that the Gaussian vector must be parameterised fully by its mean vector, $\vec{\mu}$, and its variance matrix $V$.

So far, we have not seen an example of a Gaussian vector. We would like to construct a Gaussian vector that is similar to the standard normal, and then we would like to transform it (as we do with the standard normal) to general Gaussian vectors.

\begin{proposition}
    Let $Z_1, \cdots, Z_n$ be independent standard normal random variables. Then $\vec{Z} = (Z_1, \cdots, Z_n)^T$ is a Gaussian vector.
    \label{propStdVectorIsGaussian}
\end{proposition}
\begin{proof}
    For any constant vector $\vec{u}$, the moment generating function of $\vec{u}^T \vec{Z}$ is:
    \begin{align*}
        m(\vec{\lambda}) &= E\left[e^{\vec{\lambda} \vec{u}^T \vec{Z}}\right] \\
        &= E\left[\text{exp}\left(\vec{\lambda} \sum_{i = 1}^n u_i Z_i\right)\right] \\
        &= \prod_{i=1}^{n} E[e^{\vec{\lambda} u_i Z_i}] \\
        &= \prod_{i = 1}^n \text{exp}\left({\frac{|\vec{\lambda}|^2 u_i^2}{2}}\right) \\
        &= \text{exp}\left({\frac{|\vec{\lambda}|^2 |\vec{u}|^2}{2}}\right)
    \end{align*}
    And therefore $\vec{u}^T \vec{Z} \sim N(0, |\vec{u}|^2)$.

    Then the variance of $\vec{Z}$ is the identity matrix (since the variables $Z_i$ are independent, and all have variance 1), and so $\vec{Z} \sim N(\vec{0}, I)$ where $I$ is the $n \times n$ identity matrix.
\end{proof}
Now we have shown that a vector of standard normal random variables is a Gaussian vector. We would like to construct a Gaussian vector with mean $\vec{\mu}$ and variance matrix $V$.

First, we need to define the square root of a matrix to be able to consider the standard deviation matrix rather than the variance matrix.
\begin{definition}{Square root of a Matrix}
    Given a non-negative definite real symmetric matrix $V = U^T D U$ where $U$ is an orthogonal matrix and $D$ is a diagonal matrix with all positive elements.

    Then the \underline{square root} of $V$ is:
    \begin{equation*}
        \sqrt{V} = U^T \sqrt{D} U
    \end{equation*}
    Where $\sqrt{D}$ is the matrix formed by taking the positive square root of each element of $D$.
\end{definition}
\begin{theorem}
    Let $\vec{\mu} \in \R^n$ and let $V$ be a non-negative definite matrix.

    Let $\vec{Z}$ be a vector of independent standard normal random variables.

    Then the vector $\vec{X} = \vec{\mu} + \sqrt{V} \vec{Z}$ is a Gaussian vector with mean $\vec{\mu}$ and variance $V$.
    \label{thmGeneralGaussVec}
\end{theorem}
\begin{theorem}
    We model the proof on the case of 1 dimension, where we defined $X = \mu + \sigma Z$. Therefore, we define $\sigma = \sqrt{V}$ and consider $\vec{X} = \vec{\mu} + \sigma \vec{Z}$.

    $\vec{X}$ must be a Gaussian random variable since it is a linear transformation of the Gaussian vector $\vec{Z}$, which is a Gaussian vector by proposition~\ref{propGaussVecLinearMap}.

    The mean is simply $\vec{\mu}$, since the mean of $\vec{Z}$ is $\vec{0}$. The variance is:
    \begin{align*}
        \Var(\vec{X}) &= E[(\vec{X} - \vec{\mu})(\vec{X} - \vec{\mu})^T] \\
        &= E[\sigma \vec{Z} \vec{Z}^T \sigma^T] \\
        &= \sigma E[\vec{Z} \vec{Z}^T] \sigma^T \\
        &= \sigma I \sigma^T \\
        &= \sigma \sigma = V
    \end{align*}
\end{theorem}
Then we consider the density of $\vec{X}$. We have that in the one-dimensional case,
\begin{equation*}
    f_X(x) = \frac{1}{\sigma\sqrt{2\pi}} \text{exp}\left(-\frac{(x - \mu)^2}{2\sigma^2}\right)
\end{equation*}
\begin{theorem}
    For an n-dimensional Gaussian vector with mean $\vec{\mu}$ and variance matrix $V$, the density function is:
    \begin{equation*}
        f_{\vec{X}}(\vec{x}) = \frac{1}{\sqrt{(2\pi)^n} \det{V}} \text{exp} \left(-\frac{(\vec{x} - \vec{\mu})^T V^{-1} (\vec{x} - \vec{\mu})}{2}\right)
    \end{equation*}
\end{theorem}
\begin{proof}
    \begin{case}{$V$ is positive definite (no zero eigenvalues)}
        We can write $\vec{X} = \vec{\mu} + \sigma \vec{Z}$, where $\vec{Z}$ was as above defined. Then consider a change of variables $f_{\vec{X}}(\vec{x}) \mapsto f_{\vec{Z}}(\vec{z})$. From theorem~\ref{thmTransformRV}, this is defined by:
        \begin{equation*}
            f_{\vec{X}}(\vec{x}) = F_{\vec{Z}}(\vec{z}) |J|
        \end{equation*}
        The Jacobian is $\det{\sigma^{-1}}$, and this is $\frac{1}{\sqrt{\det{V}}}$.

        Note also that $\vec{z} = \sigma^{-1} (\vec{x} - \vec{\mu})$.
        \begin{align*}
            f_{\vec{X}}(\vec{x}) &= \prod_{i=1}^{n} \exp\left({-\frac{z_i^2}{2}}{\sqrt{2\pi}} \frac{1}{\sqrt{\det{V}}}\right) \\
            &= \frac{1}{\sqrt{(2\pi)^n \det{V}}} \exp\left({-\frac{|\vec{z}|^2}{2}}\right) \\
            &= \frac{1}{\sqrt{(2\pi)^n \det{V}}} \exp\left({-\frac{\vec{z}^T \vec{z}}{2}}\right) \\
            &= \frac{1}{\sqrt{(2\pi)^n \det{V}}} \exp\left({-\frac{(\vec{x} - \vec{\mu})^T (\sigma^{-1})^T \sigma^{-1} (\vec{x} - \vec{\mu})}{2}}\right) \\
            &= \frac{1}{\sqrt{(2\pi)^n \det{V}}} \exp\left({-\frac{(\vec{x} - \vec{\mu})^T (\sigma \sigma)^{-1} (\vec{x} - \vec{\mu})}{2}}\right) \\
            &= \frac{1}{\sqrt{(2\pi)^n \det{V}}} \exp\left({-\frac{(\vec{x} - \vec{\mu})^T V^{-1} (\vec{x} - \vec{\mu})}{2}}\right)
        \end{align*}
    \end{case}
    \begin{case}{$V$ has some zero eigenvalues}
        By an orthogonal change of basis, we can assume that:
        \begin{equation*}
            V =
            \begin{pmatrix}
                U & 0 \\
                0 & 0
            \end{pmatrix}
        \end{equation*}
        Where $U$ is positive definite, and an $m \times m$ matrix.

        Write also that $\vec{\mu} = \begin{pmatrix}\vec{\lambda} \\ \vec{\nu}\end{pmatrix}$ where $\vec{\lambda} \in \R^m, \vec{\nu} \in \R^{n-m}$

        Then we can now write $\vec{X} = \begin{pmatrix} \vec{Y} \\ \vec{\nu}\end{pmatrix}$ where $Y \sim N(\vec{\lambda}, U)$ and $\vec{Y}$ has density:
        \begin{equation*}
            F_{\vec{Y}}(\vec{y}) = \frac{1}{\sqrt{(2\pi)^n} \det{U}} \exp\left(-\frac{(\vec{y} -\lambda)^T U^{-1} (\vec{y} - \vec{\lambda})}{2}\right)
        \end{equation*}
        by case 1.
    \end{case}
\end{proof}

\begin{proposition}
    If $X_1, X_2, \cdots, X_n$ are independent, then $V$ is a diagonal matrix.
    \label{propDiagonalVariance}
\end{proposition}
\begin{proof}
    Simply note that the non-diagonal elements of $V$ are covariances, which are zero if the variables are independent.
\end{proof}
\begin{lemma}
    For a Gaussian vector $\vec{X}$, with positive-definite variance matrix $V$, if $V$ is diagonal then the $X_i$ are independent.
    \label{lemGaussVecIndepIfZeroCov}
\end{lemma}
\begin{proof}[by factorising the density function]
    Let $V = \text{diag}(\lambda_1, \lambda_2, \cdots, \lambda_n)$, and note that these are all positive. Then the density function is:
    \begin{align*}
        f_{\vec{X}}(\vec{x}) &= \frac{1}{\sqrt{(2\pi)^n} \det{V}} \text{exp} \left(-\frac{(\vec{x} - \vec{\mu})^T V^{-1} (\vec{x} - \vec{\mu})}{2}\right) \\
        f_{\vec{X}}(\vec{x}) &= \frac{1}{\sqrt{(2\pi)^n} \det{V}} \text{exp} \left(-\frac{1}{2} \sum_{i = 1}^n \frac{(x_i - \mu_i)^2}{2\lambda_i}\right) \\
        \prod_{i = 1}^n f_{\vec{X}}(\vec{x}) &= \frac{1}{\sqrt{(2\pi)^n} \det{V}} \text{exp} \left(-\frac{(x_i - \mu_i)^2}{4\lambda_i}\right)
    \end{align*}
    So the density function factorises, and we have that the individual elements are normal random variables, $X_i \sim N(\mu_i, \lambda_i)$ as expected.
\end{proof}
\begin{proof}[by moment generating functions]
    We find the moment generating function of $\vec{X}$:
    \begin{align*}
        m(\vec{\theta}) &= E[\exp(\vec{\lambda}^T \vec{X})] \\
        \vec{\theta}^T\vec{X} &\sim N(\vec{\theta}^T \mu \vec{\theta}^T V \vec{\theta}) \\
        m(\vec{\theta}) &= \exp\left(\vec{\theta}^T \mu + \frac{\vec{\theta}^T V \vec{\theta^T}}{2}\right) \\
        &= \exp\left(\sum_i \theta_i \mu_i + \sum_i \frac{\theta_i^2 \lambda_i}{2}\right)
    \end{align*}
    So $m(\vec{\theta})$ is a product of the required individual random variables.
\end{proof}
\begin{remark}
    Note that this is not true for any vector of random variables. In general, a diagonal variance matrix does not imply independence, as we have seen in the 1-dimensional case with $\Cov(X, Y) = 0 \nRightarrow X \perp Y$.
\end{remark}
\begin{definition}{Correlation}
    For two random variables $X$ and $Y$, the \underline{correlation}, $\Corr(X, Y)$, is defined to be:
    \begin{equation*}
        \Corr(X, Y) = \frac{\Cov(X, Y)}{\sqrt{\Var(X)\Var(Y)}}
    \end{equation*}
\end{definition}
\begin{example}[Bivariate Gaussian]
    Let $\vec{X} = (X_1, X_2)^T$ be a Gaussian vector in $\R^2$. Define $\mu_k = E[X_k]$, and $\sigma_k^2 = \Var(X_k)$

    Let $\rho = \Corr(X_1, X_2)$. Then we want to show that $\rho \in [-1, 1]$. We can prove this by the Cauchy-Schwarz inequality:
    \begin{align*}
        \Corr(X_1, X_2) &= \frac{E[(X_1 - \mu_1)(X_2 - \mu_2)]}{\sqrt{E[(X_1 - \mu_1)^2]E[(X_2 - \mu_2)^2]}} \\
        |\Corr(X_1, X_2)| &\leq \frac{\sqrt{E[(X_1 - \mu_1)^2] E[(X_2 - \mu_2)^2]}}{\sqrt{E[(X_1 - \mu_1)^2] E[(X_2 - \mu_2)^2]}} \\
        &= 1
    \end{align*}
    Then the variance matrix is:
    \begin{equation*}
        V =
        \begin{pmatrix}
            \sigma_1^2 & \rho \sigma_1 \sigma_2 \\
            \rho \sigma_1 \sigma_2 & \sigma_2^2
        \end{pmatrix}
    \end{equation*}
    Then we shall show that for any $\sigma_1, \sigma_2 > 0$ and any $\rho \in [-1, 1]$ then the matrix $V$ as above defined is a non-negative definite matrix. It is sufficient to show that for any $\vec{u} \in \R^2$, $\vec{u}^T V \vec{u}$. Then:
    \begin{align*}
        \vec{u}^T V \vec{u} &= (1 - \rho)\left(\sigma_1^2 u_1^2 + \sigma_2^2 u_2^2\right) + \rho(\sigma_1 u_1 + \sigma_2 u_2)^2 \\
        &= (1 + \rho)(\sigma_1^2 u_1^2 + \sigma_2^2 u_2^2) - \rho(\sigma_1 u_1 - \sigma_2 u_2)^2
    \end{align*}
    We use the correct expression for positive or negative $\rho$. If $\rho \in [-1, 0]$, we use the second equality (which is certainly non-negative), and if $\rho \in (0, 1]$ we use the first equality, which is certainly non-negative.

    Note that if $\rho = 0$, then $V$ becomes a diagonal matrix.

    Now consider $E[X_2 | X_1]$. Consider $a \in \R$. Write $X_2 = (X_2 - a X_1) + aX_1$
    \begin{align*}
        \Cov(X_2 - aX_1, X_1) &= cov(X_2, X_1) - a\Var(X_1) \\
        &= \rho \sigma_1 \sigma_2 - a\sigma_1^2
    \end{align*}
    So now choose $\rho \sigma_1 \sigma_2 - a\sigma_1^2 = 0$. That is, $a = \frac{\rho \sigma_2}{\sigma_1}$.

    Now let $Y = X_2 - aX_1$. Then we have shown that $\Cov(Y, X_1) = 0$ for this choice of $a$. Then we consider the vector $(X_1, Y)^T$.
    We can write $(X_1, Y)$ as a linear map:
    \begin{equation*}
        \begin{pmatrix}X_1 \\ Y\end{pmatrix} = \begin{pmatrix}1 & 0 \\ -a & 1\end{pmatrix}\begin{pmatrix}X_1 \\ X_2\end{pmatrix}
    \end{equation*}
    and therefore this is a Gaussian vector with elementwise covariance 0. By lemma~\ref{lemGaussVecIndepIfZeroCov}, $X_1$ and $Y$ are independent.

    Then now we can take the expectation:
    \begin{align*}
        E[X_2 | X_1] &= E[X_2 - aX_1 | X_1] + E[aX_1 | X_1] \\
        &= E[X_2 - aX_1] + aX_1 \text{ by independence}
    \end{align*}
    And so we have shown:
    \begin{equation}
        E[X_2 | X_1] = E[X_2] + a(X_1 - E[X_1])
        \label{eqnCondExpecGaussVars}
    \end{equation}
\end{example}
\section{Convergence of Random Variables}
\begin{definition}{Convergence in Probability}
    A sequence of random variables $X_n, n \in \N$ \underline{converges in probability} to a random variable $X$, and we write $X_n \probconverge X$ as $n \to \infty$ if:
    \begin{equation*}
        P(|X_n - X| > \epsilon) \to 0 \text{ as } n \to \infty
    \end{equation*}
\end{definition}
\subsection{Weak Law of Large Numbers}
\begin{theorem}[Weak law of large numbers]
    Let $(X_n)$ be independent and identically distributed random variables with finite mean $\mu$. Set $S_n = \sum_{i = 1}^n X_i$. Then:
    \begin{equation*}
        \frac{S_n}{n} \probconverge \mu \text{ as } n \to \infty
    \end{equation*}
    \label{thmWeakLawLargeNums}
\end{theorem}
\begin{proof}[assuming finite variance]
    We need to show that given any $\epsilon > 0$,
    \begin{equation*}
        P\left(\left|\frac{S_n}{n} - \mu\right| > \epsilon\right) \to 0
    \end{equation*}
    \begin{align*}
        P\left(\left|\frac{S_n}{n} - \mu\right| > \epsilon\right) &= P(|S_n - n \mu| > n\epsilon) \\
        &\leq \frac{\Var(S_n)}{n^2 \epsilon^2} \text{ by Chebyshev's inequality}
    \end{align*}
    Then the variance of $S_n$ is $n \sigma^2$:
    \begin{align*}
        P\left(\left|\frac{S_n}{n} - \mu\right| > \epsilon\right) &\leq \frac{n\sigma^2}{n^2\epsilon^2} \\
        &\to 0 \text{ as } n \to \infty \text{ since all terms are finite.}
    \end{align*}
\end{proof}
\subsection{Convergence Almost Surely}
\begin{definition}{Almost certain convergence}
    A sequence of random variables $X_n, n \in \N$ \underline{converges almost surely} to a random variable $X$, and we write $X_n \to X$ a.s. as $n \to \infty$ if:
    \begin{equation*}
        P\left(\lim_{n \to \infty} X_n = X\right) = 1
    \end{equation*}
\end{definition}
\begin{remark}
    We say \textit{almost} surely because there could exist an event in the probability space where this convergence does not hold, but it must have probability 1. If there were no such events, we could drop the \textit{almost} and say that $X_n \to X$ surely.
\end{remark}
\begin{lemma}
    If $X_n \to 0$ almost surely, then $X_n \probconverge 0$.
    \label{lemASConvergeImpliesProbCoverge}
\end{lemma}
\begin{proof}
    We want to show that $\forall \epsilon > 0$, $P(|X_n| > \epsilon) \to 0$ as $n \to \infty$ or equivalently $P(|X_n| \leq \epsilon) \to 1$.
    \begin{equation*}
        P(|X_n| \leq \epsilon) \geq P\left(\bigcap_{m = n}^\infty {|X_m| \geq \epsilon}\right)
    \end{equation*}
    Then define $A_n = {|X_m| \leq \epsilon}$. Then $A_n \subseteq A_{n + 1}$, and the union is:
    \begin{equation*}
        \bigcup_{n = 1}^\infty = \left\{|X_m| \leq \epsilon \text{ for all } m \text{ sufficiently large}\right\}
    \end{equation*}
    \begin{align*}
        \therefore \lim_{n \to \infty} P(|X_n| \leq \epsilon) &\lim_{n \to \infty} P(A_n) \\
        &= P\left(\cup_n A_n\right) \\
        \geq P(\forall \epsilon > 0, |X_m| \leq \epsilon \text{ for } m \text{ sufficiently large}) \\
        &= P(\lim_{n \to \infty} X_n = 0) \\
        &= 1 \text{ by assumption}
    \end{align*}
\end{proof}
\begin{remark}
    This shows that convergence almost surely is a stronger statement that convergence in probability.
\end{remark}
\begin{theorem}[Strong law of large numbers]
    Let $(X_n)_{n \in \N}$ be and independent and identically distributed sequence of random variables, with finite mean $\mu$. Set $S_n = X_1 + \cdots + X_n$. Then
    \begin{equation*}
        \frac{S_n}{n} \to \mu \text{ as } n \to \infty \text{ almost surely.}
    \end{equation*}
    \label{thmStrongLawLargeNums}
\end{theorem}
\begin{proof}[Non-examinable]
    Assume that $X_1$ has finite variance and that $E[X_1^4] < \infty$.

    Set $Y_i = X_i = \mu$. Then $E[Y_i] = 0$ and:
    \begin{equation*}
        E[Y_i^4] = E[(X_i - \mu)^4] \leq 2^4 \left(E[X_i^4] + \mu^4\right)
    \end{equation*}
    Which is finite by assumption, so $E[Y_i^4] < \infty$.
    Then re-define $S_n = \sum_{i = 1}^n Y_i$, so we need to show that this tends to $0$.
    Consider the expression:
    \begin{equation*}
        \sum_{i = 1}^\infty \left(\frac{S_n}{n}\right)
    \end{equation*}
    We want to show it is finite with probability 1.
    \begin{align*}
        E[S_n^4] &= E\left[\left(\sum_{i = 1}^n Y_i\right)^4\right] \\
        &= E\left[\sum_{i = 1}^n Y_i^4\right] + \choose{4}{2} \sum_{1 \leq i < j \leq n} E[Y_i^2 Y_j^2] + R
    \end{align*}
    Where $R$ is a set of terms of the forms:
    \begin{equation*}
        Y_i^3 Y_j,~Y_i^2 Y_j Y_k,~Y_i Y_j Y_k Y_l
    \end{equation*}
    Where all the indices are distinct. However, such terms include a 1st power of $Y_i$:
    \begin{equation*}
        E[Y_i^3 Y_j] = E[Y_i^3] E[Y_j] = 0
    \end{equation*}
    by independence, so we need only consider the terms external to $R$.
    \begin{align*}
        E[Y_i^2] E[Y_j^2] &= \left(E[Y_i^2]\right)^2 \text{ by identical distribution} \\
        &\leq E[Y_i^4] \text{ by Jensen's Inequality.}
    \end{align*}
    And therefore:
    \begin{align*}
        E[S_n^4] &= n E[Y_i^4] + 6 \frac{n(n-1)}{2} E[Y_i^4] \\
        &\leq 3n^2 E[Y_1^4]
    \end{align*}
    \begin{equation*}
        E\left[\frac{S_n^4}{n^4}\right] \leq \frac{3}{n^2} E[Y_1^4]
    \end{equation*}
    Which, when summed, is finite. Therefore we have that the individual terms, $\frac{S_n}{n}$, must tend to 0 by the $n$th term test.
\end{proof}
\begin{remark}
    This statement can be intuitively considered as the fact that we can estimate the outcome of the first $n$ random variables: $S_n \approx n\mu$.
\end{remark}
\subsection{Central Limit Theorem}
The Central Limit Theorem provides an answer to the question of how $S_n$ fluctuates around $n\mu$.
\begin{align*}
    \Var\left(\frac{S_n}{n} - \mu\right) &= \Var\left(\frac{S_n}{n}\right) \\
    &= \frac{\sigma^2}{n}
\end{align*}
And therefore the random variable:
\begin{equation*}
    \frac{S_n - n\mu}{\sigma\sqrt{n}}
\end{equation*}
has variance 1 and mean 0.

Initially, suppose that $X_i$ are independent and identically distributed with $X_i \sim N(\mu, \sigma^2)$.
Therefore the new random variable in this case has standard normal distribution:
\begin{equation}
    \frac{S_n - n\mu}{\sigma\sqrt{n}} \sim N(0, 1).
    \label{eqnSnIsStdNormal}
\end{equation}
We will show that, in fact, $N(0, 1)$ is universal, in the sense that no matter what the distribution of $X_i$ is, for $n$ sufficiently large, equation~\ref{eqnSnIsStdNormal} will hold.
For this, we need to define a new notion of convergence:
\begin{definition}{Convergence in Distribution}
    A sequence of random variables $(X_n)_{n \in \N}$ \underline{converges in distribution} to a random variable $X$ as $n \to \infty$, and we write $X_n \distconverge X$ as $n \to \infty$, if:
    \begin{equation*}
        F_{X_n}(x) \to F_X(x)~\forall x \in \R \text{ that are continuity points of } F_X
    \end{equation*}
    as $n \to \infty$.
\end{definition}
\begin{theorem}[Continuity property of moment generating functions]
    Suppose that $(X_n)$ are random variables with moment generating functions $m_n(\theta)$; suppose also that $X$ is a random variable with moment generating function $m(\theta)$. Assume that $m(\theta)$ is finite for some $\theta \neq 0$. Then if $m_n(\theta) \to m(\theta)$ as $n \to \infty$, for all $\theta \in \R$, then:
    \begin{equation*}
        X_n \distconverge X \text{ as } n \to \infty
    \end{equation*}
    \label{thmMGFContinuity}
\end{theorem}
This theorem will be used without proof.
\begin{theorem}[Central Limit Theorem]
    Suppose that $(X_n)_{n \in \N}$ are independent and identically distributed random variables with finite mean $\mu$ and finite variance $\sigma^2$. Define $S_n = X_1 + \cdots + X_n$. Then:
    \begin{equation*}
        \frac{S_n - n\mu}{\sigma\sqrt{n}} \distconverge Z
    \end{equation*}
    as $n \to \infty$ where $Z \sim N(0, 1)$.

    In other words, for all real $x$,
    \begin{equation*}
        P\left(\frac{S_n - n\mu}{\sigma\sqrt{n}}\leq x\right) \to \int_{-\infty}^\infty \frac{e^\frac{-y^2}{2}}{\sqrt{2\pi}} dy = \Phi(x)
    \end{equation*}
    \label{thmCentralLimit}
\end{theorem}
\begin{proof}
    Consider $Y_i = \frac{X_i - \mu}{\sigma}$. Then $E[Y_i] = 0$ and $\Var(Y_i) = 1$. Therefore it suffices to prove the theorem for the case $X_i$ has mean 0 and variance 1.
    In this case, we are required to prove that:
    \begin{equation*}
        \frac{S_n}{\sqrt{n}} \distconverge Z \sim N(0, 1)
    \end{equation*}
    Also assume that $\exists \delta > 0$ such that $E[e^{\delta X_i}]$ and $E[e^{-\delta X_i}]$ are finite, and that the moment generating function is defined.
    
    Let $m(\theta) = E[e^{\theta X_i}]$ be the moment generating function of $X_i$. Therefore, using theorem~\ref{thmMGFContinuity}, it suffices to show that:
    \begin{equation*}
        E\left[\exp\left(\theta \frac{S_n}{\sqrt{n}}\right)\right] \to E\left[e^{\theta Z}\right] = \exp\left(\frac{\theta^2}{2}\right)
    \end{equation*}
    Therefore:
    \begin{align*}
        E\left[\exp\left(\theta\frac{S_n}{\sqrt{n}}\right)\right] &= \left(E\left[\exp\left(\theta \frac{X_i}{\sqrt{n}}\right)\right]\right)^n \\
        &= \left(m\left(\frac{\theta}{\sqrt{n}}\right)\right)^n
    \end{align*}
    So we want to show that:
    \begin{equation*}
        \left(m\left(\frac{\theta}{\sqrt{n}}\right)\right) \to e^\frac{\theta^2}{2}
    \end{equation*}
    as $n \to \infty$.
    \begin{align*}
        m\left(\frac{\theta}{\sqrt{n}}\right) &= E\left[\exp\left(\frac{\theta X_i}{\sqrt{n}}\right)\right] \\
        &= E\left[1 + \frac{\theta^2X_i^2}{2n} + \sum_{k \geq 3} \frac{\theta^k X_i^k}{\left(\sqrt{n}\right)^k k!}\right] \\
        &= 1 + \frac{\theta^2}{2n} + E\left[\sum_{k \geq 3} \frac{\theta^k X_i^k}{\left(\sqrt{n}\right)^k k!}\right]
    \end{align*}
    \begin{subproof}{$\left|E\left[\sum_{k \geq 3} \sigma^k X_i^k / (k!)\right]\right| = o(|\theta|^2)$ as $\theta \to 0$.}
        Let $|\theta| \leq \frac{\delta}{2}$.
        \begin{align*}
            \left|E\left[\sum_{k \geq 3} \frac{\theta^k X_i^k}{k!}\right]\right| &\leq E\left[\sum_{k \geq 3} \frac{|\theta^k X_i^k|}{k!}\right] \\
            &= E\left[|\theta X_i|^3 \sum_{k \geq 0} \frac{|\theta X_i|^k}{(k + 3)!}\right] \\
            &\leq E\left[|\theta X_i|^3 \sum_{k \geq 0} \frac{|\theta X_i|^k}{k!}\right] \\
            &= E\left[|\theta X_i|^3 \exp\left(|\theta X_i|\right)\right] \\
            &= E\left[|\theta X_i|^3 \exp\left(\frac{\delta}{2}|X_i|\right)\right] \\
        \end{align*}
        Now note that:
        \begin{align*}
            |\theta X_1|^3 &= |\theta|^3 \frac{|\frac{\delta}{2} X_i|^3}{3!} \left(\frac{2}{\delta}\right)^3 \times 3! \\
            &\leq 3! \left(\frac{2}{\delta}\right)^3 |\theta|^3 e^{\frac{\delta}{2}|X_i|}
        \end{align*}
        Then substituting back, and setting $C$ to be the constant on the left of this expression:
        \begin{align*}
            E\left[|\theta X_i|^3 \exp\left(\frac{\delta}{2}|X_i|\right)\right] &= C\times E\left[|\theta|^3 e^{\delta |X_1|}\right] \\
            &= C |\theta|^3 E[E^{\delta |X_1|}] \\
            &\leq C |\theta|^3 \left(E[E^{\delta X_1}] + E[E^{-\delta X_1}]\right) \\
        \end{align*}
        which is finite by assumption. Therefore, we have the required result.
    \end{subproof}
    Now we are done, because $m\left(\frac{\theta}{\sqrt{n}}\right) = 1 + \frac{\theta}{2n} + o\left(\frac{\theta^2}{n}\right)$ as $n \to \infty$, so:
    \begin{equation*}
        \left(m\left(\frac{\theta}{\sqrt{n}}\right)\right) \to e^{\theta^2}{2}
    \end{equation*}
    as $n \to \infty$.
\end{proof}
\begin{remark}
    We can rearrange the result to get that:
    \begin{equation*}
        S_n \approx n\mu + \sigma \sqrt{n} Z \sim N(\mu n, \sigma^2 n)
    \end{equation*}
\end{remark}
\begin{example}
    Suppose that $S_n \sim B(n, p)$. Therefore, the $X_i$ are Bernoulli random variables with parameter $p$.

    By theorem~\ref{thmCentralLimit}, we must have that:
    \begin{equation*}
        \frac{S_n - np}{\sqrt{np(1-p)}} \to N(0, 1)
    \end{equation*}
    Therefore in the limit:
    \begin{equation*}
        S_n \sim N(np, np(1-p))
    \end{equation*}
\end{example}
\begin{example}
    Now suppose that $S_n \sim B\left(n, \frac{\lambda}{n}\right)$. Then this converges to a Poisson distribution $Po(\lambda)$ as proven earlier. This is an example of a case where theorem~\ref{thmCentralLimit} does not apply, because the variance is a function of $n$. However, we can apply the central limit theorem to a Poisson random variable:
    
    \begin{align*}
        S_n &\sim Po(\lambda) \\
        &= X_1 + X_2 + \cdots + X_n, X_i \sim Po(1)
    \end{align*}
    And therefore:
    \begin{equation*}
        \frac{S_n - n}{\sqrt{n}} \to N(0, 1)
    \end{equation*}
    or, $S_n \approx N(n, n)$.
\end{example}
\subsection{Sampling Error via the Central Limit Theorem}
Suppose that we hold a referendum and that every individual votes Yes with probability $p$, and No with probability $1-p$. We want to estimate $p$ using a sample of size $N$ sufficiently large. Let $X_i$ be $1$ if the vote from the $i$th individual is Yes, and $0$ otherwise.

Set $S_n = \sum_{i = 1}^N X_i$. Define:
\begin{equation*}
    \hat{p}_n = \frac{S_n}{n} \to p \text{ almost surely}
\end{equation*}
We want to estimate $p$ with an accuracy of $\pm 4\%$, with probability at least $0.99$. By theorem~\ref{thmCentralLimit}, for large enough $N$ we have:
\begin{equation*}
    S_N \approx Np + \sqrt{Np(1-p)} Z, Z \sim N(0, 1)
\end{equation*}
And therefore:
\begin{equation*}
    \hat{p}_N = \frac{S_N}{N} \approx p + \frac{\sqrt{p(1-p)}}{\sqrt{N}} Z
\end{equation*}
And therefore we need $N$ large enough so that:
\begin{equation*}
    P\left(|\hat{p}_N - p| \geq \frac{1}{25}\right) \leq \frac{1}{100}
\end{equation*}
\begin{align*}
    |\hat{p}_N - p| \approx \frac{\sqrt{p(1-p)}}{\sqrt{n}} \\
    \therefore P(|Z| \geq z) &= 2(1 - \Phi(z))
\end{align*}
From tables of values we can get that $z = 2.58$ gives $P(|Z| \leq z) = 0.01$.

And therefore we take $N$ such that:
\begin{equation*}
    \frac{0.04\sqrt{N}}{\sqrt{p(1-p)}} \geq 2.58
\end{equation*}
The denominator is maximised (and the fraction is minimised) when $p = 0.5$. In this case, we require:
\begin{equation*}
    N \geq 1040.
\end{equation*}
\section{Useful Results in Probability}
\subsection{Simulation of Random Variables}
Suppose $X$ is a random variable with probability distribution function $F$. We know how to simulate a uniform distribution, $U([0, 1])$. Therefore we want to find some transformation to simulate the random variable $X$.

Let $U \sim U([0, 1])$. Assume that $F$ is a bijection. Therefore take $X = F^{-1}(U)$. Therefore:
\begin{align*}
    P(X \leq x) &= P(F^{-1}(U) \leq x) \\
    &= P(U \leq F(x)) \\
    &= F(x) \text{ since } U \text{ is uniform.}
\end{align*}
However, if $F$ is not a bijection we need a \underline{generalised inverse}. Define:
\begin{equation*}
    G(u) = \inf\subsetselect{x \in \R}{u \leq F(x)}
\end{equation*}
Then $G : (0, 1) \mapsto \R$. Now we need to ensure that $G(u) \leq x \Leftrightarrow u \leq F(x)$.

The forward direction is true by definition of the infimum.

Now consider the reverse direction. If $u \leq F(x)$, suppose $G(u) > x$. This implies $u > F(x)$, reaching a contradiction since $F$ is increasing. Therefore, define $X = G(U)$. As before, we have the required:
\begin{equation*}
    P(X \leq x) = P(U \leq F(x)) = F(x)
\end{equation*}
\subsection{Box-Muller Transform}
We now want to generate $X, Y$ independent standard normal random variables.

Earlier, we saw that transforming $X, Y$ to polar coordinates, $R$ has density $re^{-\frac{r^2}{2}}$ and $\theta \sim U([0, 2\pi])$. We then want to invert this transformation.

Let $U$ and $V$ be independent random variables with distribution $U([0, 1])$. Define $\theta = 2\pi U$, which is the required $\theta$. If then we set $R = \sqrt{-2\log(V)}$ then:
\begin{align*}
    P(R \geq r) &= P(\sqrt{-2\log(V)} \geq r) \\
    &= P(V \leq e^\frac{-r^2}{2})
\end{align*}
And therefore differentiating gives the required density for $R$. Now setting $X = R\cos(\theta)$ and $Y = R\sin(\theta)$ gives independent standard normal random variables.
\subsection{Rejection Sampling}
Suppose we have a random variable $X$ with density $f(x) = \frac{I_{X \in A}}{|A|}$ where $A$ is a subset of the unit cube in $n$ dimensions: $A \subseteq [0, 1]^d$.

Let $(\vec{U_n})$ be independent and identically distributed $d$-dimensional uniform random variables. Define:
\begin{equation*}
    \left(U_{k, n}~|~k \in \{1, \cdots, d\}, n\in\N\right) \text{ to be } U([0, 1])
\end{equation*}
And therefore we have independent and identically distributed uniform vectors $\vec{U_n}$. Define:
\begin{equation*}
    N = \min\subsetselect{n}{U_n \in A} \text{ and } X = U_N
\end{equation*}
Let $X \in [0, 1]^d$.
\begin{align*}
    P(X \in B) &= P(U_n \in B) \\
    &= \sum_{n = 1}^\infty P(U_N \in B, N = n) \\
    &= \sum_{n = 1}^\infty P(U_n \in B, U_n \in A, U_1 \notin A, \cdots, U_{n-1} \notin A) \\
    &= \sum_{n = 1}^\infty P(U_n \in A \cap B) \left(P(U_1 \notin A)\right)^{n-1} \\
    &= \sum_{n = 1}^\infty |A \cap B| \left(1 - |A|\right)^{n-1} \\
    &= \frac{|A \cap B|}{|A|}
\end{align*}
\end{document}