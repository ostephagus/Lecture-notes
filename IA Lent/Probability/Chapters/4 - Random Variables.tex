\documentclass[../Main.tex]{subfiles}

\begin{document}
\section{Conditional Expectation}
\begin{definition}{Conditional expectation on an event}
    Let $X$ be a random variable, $B$ an event, and $P(B) > 0$. Then we define the \underline{conditional expectation}:
    \begin{equation}
        E[X | B] = \frac{E[X \times I_B]}{P(B)}
        \label{eqnConditionalExpec}
    \end{equation}
\end{definition}
\begin{lemma}[Law of Total Expectation]
    Let $X$ be a random variable and $(\Omega_n)$ be a partition of $\Omega$. Suppose that $P(\Omega_n) > 0$ for all $n$. Then:
    \begin{equation*}
        E[X] = \sum_n E[X | \Omega_n] \times P(\Omega_n)
    \end{equation*}
\end{lemma}
\begin{proof}
    Let $X = XI_\Omega$:
    \begin{align*}
        X &= XI_{\cup_n \Omega_n} \\
        &= \sum_n XI_{\Omega_n} \text{ as the } \Omega_n \text{ are disjoint}
    \end{align*}
    Taking expectations:
    \begin{align*}
        E[X] &= \sum_n E[XI_{\Omega_n}] \\
        &= \sum_n E[X | \Omega_n] P(\Omega_n) \text{ by rearranging equation~\ref{eqnConditionalExpec}}
    \end{align*}
\end{proof}
\begin{definition}{Joint distribution}
    Let $X_1, \cdots, X_n$ be discrete random variables. Their \underline{joint distribution} is defined to be:
    \begin{equation*}
        P(X_1 = x_1, \cdots, X_n = x_n)~\forall x_1, \cdots, x_n
    \end{equation*}
    That is, a single distribution that encompasses all the random variables.
\end{definition}
\begin{definition}{Marginal distribution}
    Given a joint distribution, the \underline{marginal distribution} is the distribution of only one of the random variables. It is obtained by summing probabilities over all the values of the other variables:
    \begin{equation*}
        P(X_1 = x_1) = \sum_{x_2, \cdots, x_n} P(X_1 = x_2, X_2 = x_2, \cdots, X_n = x_n)
    \end{equation*}
\end{definition}
\begin{definition}{Conditional distribution}
    Let $X$ and $Y$ be two random variables. Then the \underline{conditional distribution} of $X$ given $Y = y$ is defined to be:
    \begin{equation*}
        P(X = x | Y = y) = \frac{P(X = x, Y = y)}{P(Y = y)}~\forall x
    \end{equation*}
\end{definition}
By the law of total probability,
\begin{equation*}
    P(X = x) = \sum_y P(X = x | Y = y) P(Y = y)
\end{equation*}
\begin{proposition}[Convolutions]
    Consider two independent discrete random variables $X$ and $Y$. Then the probability distribution given by their sum is a \underline{convolution}:
    \begin{align*}
        P(X + Y = z) &= \sum_y P(Y = y) P(X = z - y) \\
        \text{ or } P(X + Y = z) &= \sum_x P(X = x) P(Y = z - x)
    \end{align*}
\end{proposition}
\begin{proof}
    \begin{align*}
        P(X + Y = z) &= \sum_x P(X + Y = z, X = x) \\
        &= \sum_x P(Y = z - x, X = x) \\
        &= \sum_x P(X = x) P(Y = z - x) \text{ by independence}
    \end{align*}
    And the second formula is obtained by summing over $y$ instead of $x$.
\end{proof}
\begin{example}
    Let $X \sim Po(\lambda)$ and $Y \sim Po(\mu)$. Let them be independent.\par
    Then consider $P(X + Y = z)$.\par
    \begin{align*}
        P(X + Y = n) &= \sum_{k = 0}^n P(X + y = n, X = k) \\
        &= \sum_{k = 0}^n P(X = k) P(Y = n-k) \\
        &= \sum_{k = 0}^n e^{-\lambda} \frac{\lambda^k}{k!} e^{-\mu} \frac{\mu^{n-k}}{(n-k)!} \\
        &= \frac{e^{-(\lambda + \mu)}}{n!} \sum_{k = 0}^n \choose{n}{k} \mu^{n-k} \lambda^k \\
        &= \frac{e^{-(\lambda + \mu)}}{n!} (\lambda + \mu)^n
    \end{align*}
    And so $X + y \sim Po(\lambda + \mu)$.\par
    In general, adding two random variables is not so easy.
\end{example}
\begin{definition}{Conditional distribution}
    Let $X$ and $Y$ be random variables. The \underline{conditional distribution} of $X$ given $Y = y$ is:
    \begin{equation*}
        P(X = x | Y = y) = \frac{P(X = x, Y = y)}{P(Y = y)}
    \end{equation*}
\end{definition}
Consider again $X$ and $Y$, and the event $Y = y$. Then the conditional expectation of $X$ given $Y = y$ is:
\begin{align*}
    E[X | Y = y] &= \frac{E[X I_{Y = y}]}{P(Y = y)} \\
    &= \sum_x \frac{xP(X = x, Y = y)}{P(Y = y)} \\
    &= \sum_x x P(X = x | Y = y)
\end{align*}    
Note that $E[X | Y = y]$ is only a function of $y$. Let this function be $g(y)$
We now want to define the conditional expectation given a random variable:
\begin{definition}{Conditional expectation given a random variable}
    Let $X$ and $Y$ be random variables. The \underline{conditional expectation} of $X$ given $Y$ is:
    \begin{equation*}
        E[X | Y] = g(Y)
    \end{equation*}
    where $g$ was defined above.
\end{definition}
Note that $E[X | Y]$ is a random variable, not a number. This is because it is a function of $Y$. We can alternatively write this as:
\begin{align*}
    E[X | Y] &= g(Y) \times 1 \\
    &= g(Y) \sum_y I_{Y = y} \\
    &= \sum_y g(Y) I_{Y = y} \\
    &= \sum_y g(y) I_{Y = y}
\end{align*}
\begin{example}
    \label{exCoinTossCondExpec}
    Consider tossing a coin, probability $p$ of heads, $n$ times independently. Set $X_i = I_{\text{toss } i \text{ is a head}}$.\par
    Let $Y_n$ be the total number of heads, $Y_n = X_1 + \cdots + X_n$.\par
    Then consider $E[X_1 | Y_n]$:
    \begin{align*}
        E[X_1 | Y_n] &= \sum_y g(y) I_{Y_n = y} \\
        \text{where } g(y) &= E[X_1 | Y_n = y], y \in \{1, \cdots, n\} \\
        &= 0 \times P(X_1 = 0 | Y_n = y) + 1 \times P(X_1 = 1 | Y_n = y) \\
        &= P(X_1 = 1 | Y_n = y) \\
        &= \frac{P(X_1 = 1, Y_n = y)}{P(Y_n = y)} \\
        &= \frac{p \choose{n-1}{y-1} p^{y-1} (1-p)^{n-y}}{\choose{n}{y} p^y (1-p)^{n-y}} \\
        &= \frac{y}{n}.
    \end{align*}
    So $g(y) = \frac{y}{n}$ and therefore $E[X_1 | Y_n] = g(Y_n) = \frac{Y_n}{n}$.
\end{example}
\begin{propositions}[Properties of conditional expectation] {
        Let $X$ and $Y$ be random variables. Let $c$ be a real number.
        \label{propsCondExpecProperties}
    }
    \item $E[cX | Y] = cE[X | Y]$, and $E[c | Y] = c$. \label{propCondExpecConstant}
    \item If $X_1, \cdots, X_n$ are random variables,
        \begin{equation*}
            E\left[\sum_{i=1}^n X_i | Y\right] = \sum_{i=1}^n E[X_i | Y]
        \end{equation*}
        \label{propCondExpecSum}
    \item $E[E[X | Y]] = E[X]$ \label{propCondExpecTwice}
    \item If $X$ and $Y$ are independent, $E[X | Y] = E[X]$. \label{propCondExpecIndep}
    \item If $Z$ is also a random variable, independent of $Y$, then:
        \begin{equation*}
            E\left[E[X | Y] | Z\right] = E[X]
        \end{equation*}
        \label{propCondExpecSecondIndepRV}
    \item Let $h : \R \mapsto \R$. Then $E[h(Y) X | Y] = h(Y) E[X | Y]$. \label{propCondExpecFactorOutFunc}
    \item $E[E[X | Y] | Y] = E[X | Y]$ and $E[X | X] = X$. \label{propCondExpecTwiceGiven}
\end{propositions}
\begin{proof}[Of \ref{propCondExpecTwice} to \ref{propCondExpecTwiceGiven}]
    \begin{enumerate}
        \setcounter{enumi}{2} % Start at 3
        \item The expectation of a conditional expectation is $E[X]$:
            \begin{equation*}
                E[X | Y] = \sum_y I_{Y = y} g(y)
            \end{equation*}
            Where $g(y) = E[X | Y = y]$. Then:
            \begin{align*}
                E[E[X | Y]] &= \sum_y g(y) P(Y = y) \\
                &= \sum_y E[X | Y = y] P(Y = y) \\
                &= \sum_y E[X \times I_{Y = y}] \\
                &= E[X]
            \end{align*}
        \item Expectation given an independent variable is $E[X]$:
            \begin{align*}
                E[X | Y] &= g(Y) = \sum_y g(y) I_{Y = y} \\
                &= \sum_y E[X | Y = y] I_{Y = y} \\
                &= \sum_y \left(\sum_x x P(X = x, Y = y)\right) I_{Y = y} \\
                &= \sum_y \left(\sum_x x P(X = x)\right) I_{Y = y} \\
                &= \sum_y E[X] I_{Y = y} \\
                &= E[X]
            \end{align*}
        \item Conditional expectation given a second independent random variable is $E[X]$:
            The claim is equivalent to showing that $g(Y) = E[X | Y]$ is independent of $Z$:
            \begin{align*}
                P(g(Y) = w, Z = z) &= \sum_{y \in g^{-1}(\{w\})} P(Y = y, Z = z) \\
                &= \sum_{y \in g^{-1}(\{w\})}
            \end{align*}
            And this is independent of $Z$, so by proposition~\ref{propCondExpecIndep}, we have that:
            \begin{equation*}
                E[E[X | Y] | Z] = E[E[X | Y]] = E[X] \text{ by proposition~\ref{propCondExpecTwice}}
            \end{equation*}
        \item \begin{align*}
                E[h(Y) X | Y] &= \sum_y I_{Y = y} E[h(Y) X | Y = y] \\
                &= \sum_y I_{Y = y} E[h(y) X | Y = y] \\
                &= \sum_y I_{Y = y} h(y) E[X | Y = y] \\
                &= h(Y) E[X | Y]
            \end{align*}
        \item This follows immediately from \ref{propCondExpecFactorOutFunc}, factoring out the function of $Y$, $E[X | Y]$, in the first case, and the function of $X$, $X$, in the second case.
    \end{enumerate}
\end{proof}
\begin{example}[Example~\ref{exCoinTossCondExpec} revisited]
    Consider again example~\ref{exCoinTossCondExpec}. We can now reach the result much more easily.\par
    By symmetry, $E[X_i | Y_n]$ must all be the same:
    \begin{align*}
        E[X_1 + X_2 + \cdots + X_n | Y_n] &= n E[X_1 | Y_n] \\
        E[Y_n | Y_n] &= n E[X_1 | Y_n] \\
        \therefore E[X_1 | Y_n] &= \frac{Y_n}{n}
    \end{align*}
\end{example}
\section{Random Walks}
\begin{definition}{Stochastic process}
    A stochastic process is a sequence of random variables, that are not necessarily independent.    
\end{definition}
Stochastic is simply another word for random.
\begin{definition}{Random walk}
    A \underline{random walk} is a random process that can be expressed by:
    \begin{equation*}
        X_n = x + Y_1 + \cdots + Y_n
    \end{equation*}
    Where $x$ is a number, $Y_i$ are independent and identically distributed random variables. $Y_i$ are known as the \underline{steps} of the random walk.
\end{definition}
Consider a simple random walk, wherein $Y_i$ is defined, with parameter $p$, by:
\begin{equation*}
    Y_i =
    \begin{cases}
        1 & \text{ probability } p \\
        -1 & \text{ probability } 1-p
    \end{cases}
\end{equation*}
\subsection{Extended Example: Gambler's Ruin}
A gambler plays a game where they can double their money, with probability $p$, or lose their money, with probability $1-p$. They start with £$x$ and each time bets £1. They stop when either they run out of money, or when they reach £$a$.\par
This can be modelled using a random walk.\par
First consider the probability that the gambler reaches £$a$ before reaching $0$, given their start point of £$x$. Let:
\begin{equation*}
    h(x) = P((X_n) \text{ reaches } a \text{ before reaching } 0~|~X_0 = x)
\end{equation*}
Denote $P(A | X_0 = x)$ as $P_x(A)$ for some event $A$.\par
Note that $h(a) = 1$, and $h(0) = 0$.\par
Also, for convenience, let $W$ be the event that $X$ hits $a$ before $0$ (a win).\par
Split the probability based on the value of $Y_1$:
\begin{align*}
    h(x) &= P_x(W) \\
    &= P_x(W | Y_1 = 1) + P_x(W | Y_1 = -1)\\
    &= P_{x+1}(W)P(Y_1 = 1)+ P_{x-1}(W)P(Y_1 = -1) \\
    &= h(x + 1)p + h(x - 1)(1 - p)
\end{align*}
Which gives a difference equation.\par
Consider the case $p = \frac{1}{2}$. This is a simple symmetric random walk.
\begin{align*}
    h(x) &= \frac{1}{2} h(x + 1) + \frac{1}{2} h(x - 1) \\
    h(x) - h(x - 1) &= h(x + 1) - h(x) \\
    h(x) &= \sum_{i=1}^x (h(i) - h(i - 1)) \\
    &= cx \text{ for some } c
\end{align*}
Then by boundary conditions, $c = \frac{1}{a}$ and $h(x) = \frac{x}{a}$.\par
Now consider the general case. Try a solution of the form $h(x) = \lambda^x$ in the difference equation:
\begin{align*}
    &\lambda^x = p \lambda^{x + 1} + (1 - p) \lambda^{x - 1} \\
    &\implies \lambda = p\lambda^2 + (1 - p) \\
    &\implies \lambda = 1 \text{ or } \frac{1 - p}{p}
\end{align*}
Reject the solution $\lambda = 1$ and apply the boundary conditions:
\begin{equation*}
    h(x) = \frac{\left(\frac{1-p}{p}\right)^x - 1}{\left(\frac{1-p}{p}\right)^a - 1}
\end{equation*}
\subsection{Time to Absorption}
\begin{definition}{Time to absorption}
    In a bounded random walk, with bounds $0$ and $a$, define the \\\underline{time to absorption} as:
    \begin{equation*}
        T = \min{\subsetselect{n \geq 0}{X_n \in \{0, a\}}}
    \end{equation*}
\end{definition}
Then we want to find, for a simple random walk, $E_x[T] = E[T | X_0 = x]$. Let this be $k(x)$.
\begin{align*}
    E_x[T] &= E_x[T | Y_1 = 1]P_x(Y_1 = 1) + E_x[T | Y_1 = -1]P_x(Y_1 = -1) \\
    &= E_x[T | Y_1 = 1] \times p + E_x[T | Y_1 = -1] \times (1-p) \\
    &= p(1 + k(x + 1)) + (1 - p)(1 + k(x - 1)) \\
    &= 1 + pk(x + 1) + (1-p)k(x - 1)
\end{align*}
This gives another difference equation. We have the boundary conditions $k(0) = k(a) = 0$.\par
Consider first the case $p = \frac{1}{2}$. Then the difference equation is:
\begin{equation*}
    k(x) = 1 + \frac{1}{2}k(x + 1) + \frac{1}{2}k(x - 1)
\end{equation*}
Then consider a solution of the form $Ax^2$. This gives the result that $k(x) = x(a - x)$.\par
Then consider $p \neq \frac{1}{2}$. Try a solution of the form $Cx$, and this gives the result that:
\begin{equation*}
    k(x) = \frac{1}{1 - 2p} x - \frac{1 - p}{1 - 2p}\frac{\left(\frac{1-p}{p}\right)^x - 1}{\left(\frac{1-p}{p}\right)^a - 1}
\end{equation*}
\begin{remark}
    Note the key idea here: each step, we can consider the following steps as a new random walk, starting from the new position. This is the defining idea of a Markov Chain. See the course IB Markov Chains.
\end{remark}
\section{Probability Generating Functions}
We have already seen the probability mass function for a random variable, the function that takes a possible value of the random variable as input, and returns as output its probability.
\begin{definition}{Probability generating function}
    For a random variable $X$, the \underline{probability generating function} is defined to be:
    \begin{equation*}
        p(z) = \sum_{r = 0}^\infty p_r z^r = E[z^X], |z| \leq 1
    \end{equation*}
    where $p_r = P(X = r)$. Note that this displays absolute convergence for any $|z| \leq 1$.
\end{definition}
\begin{theorem}
    The probability generating function uniquely determines the distribution of $X$.
\end{theorem}
\begin{proof}
    Suppose $(p_r)$ are two probability distributions with the same PGFs:
    \begin{equation}
        \forall |z| \leq 1, \sum_{r = 0}^\infty p_r z^r = \sum_{r = 0}^\infty q_r z^r
        \label{eqnEqualPGFs}
    \end{equation}
    \induction{$r = 0$}{
        Set $z = 0$. Then equation~\ref{eqnEqualPGFs} gives $p_0 = q_0$.
    }{$r \leq k$}{}
    {$r = k + 1$}{
        We have, by assumption, that
        \begin{equation*}
            \sum_{r = k + 1}^{\infty} p_r z^r = \sum_{r = k + 1}^{\infty} q_r z^r
        \end{equation*}
        Therefore divide by $z^{n+1}$ and take the limit as $z \to 0$. This gives $p_{k + 1} = q_{k + 1}$.
    }
\end{proof}
\begin{example}
    Let $X \sim B(n, p)$.\par
    Then the probability generating function is:
    \begin{align*}
        p(z) &= \sum_{r = 0}^n z^r \choose{n}{r} p^r (1-p)^{n-r} \\
        &= (zp + 1 - p)^n
    \end{align*}
\end{example}
\begin{proposition}
    Suppose that $(X_i)$ are a set of independent random variables, with PGFs $q_i(z)$.\par
    Then the PGF of their sum is:
    \begin{equation*}
        q(z) = \prod_i q_i(z)
    \end{equation*}
\end{proposition}
\begin{proof}
    \begin{align*}
        q(z) &= E[z^{X_1 + X_2 + \cdots + X_n}] \\
        &= E[z^{X_1} \times \cdots \times z^{X_n}] \\
        &= E[z^{X_1}] \times \cdots \times E[z^{X_n}] \text{ by independence} \\
        &= q_1(z) \times \cdots \times q_n(z)
    \end{align*}
\end{proof}
\begin{example}[Geometric distribution]
    Consider $X \sim Geo(p)$. Now consider $E[z^X]$:
    \begin{align*}
        E[z^X] &= \sum_{r=0}^\infty z^r (1-p)^r p \\
        &= \frac{p}{1-z(1-p)}
    \end{align*}
\end{example}
\begin{example}[Poisson distribution]
    Consider $X \sim Po(\lambda)$.
    \begin{align*}
        E[z^X] &= \sum_{k=0}^\infty z^k e^{-\lambda} \frac{\lambda^k}{k!} \\
        &= e^{-\lambda} e^{\lambda z} \\
        &= e^{\lambda (z-1)}
    \end{align*}
\end{example}
\begin{theorem}
    Given a random variable $X$, and probability generating function $p(z)$,
    \begin{equation*}
        \lim_{z \to 1-} p'(z) = E[X]
    \end{equation*}
\end{theorem}
\begin{proof}
    \begin{case}{$X$ has finite expectation.}
        Let $0 < z < 1$.
        \begin{align*}
            p'(z) &= \sum_{r=1}^\infty r p_r z^{r-1} \\
            ^\leq \sum_{r=1}^\infty rp_r = E[X]
        \end{align*}
        Note also that $p'(z)$ is an increasing function in $z$\par
        Therefore $\lim_{z \to 1^-} \leq E[X]$.\par
        Let $N > 0$. For any $\epsilon > 0$, let $N$ be such that:
        \begin{equation*}
            \sum_{r=1}^N rp_r \geq E[X] - \epsilon
        \end{equation*}
        Also $p'(z) \geq \sum_{r=1}^N rp_r ^{n-1}$\par
        \begin{equation*}
            \lim_{z \to 1^-} p'(z) \geq \sum_{r=1}^N rp_r \geq E[X] - \epsilon
        \end{equation*}
        Since this holds for any $\epsilon$,
        \begin{equation*}
            \lim_{z \to 1^-} p'(z) = E[X]
        \end{equation*}
    \end{case}
    \begin{case}{$X$ has infinite expectation.}
        In this case, for any $M$ there exists $N$ such that:
        \begin{equation*}
            \sum_{r=1}^N \geq M
        \end{equation*}
        And therefore, by the above,
        \begin{align*}
            \lim_{z \to 1^-} p'(z) &\geq \lim_{z \to 1^-} \sum_{r=1}^N rp_r z^{r-1} \\
            &=\sum_{r=1}^N rp_r \geq M 
        \end{align*}
        And therefore $\lim_{z \to 1-} = \infty$.
    \end{case}
\end{proof}
\begin{theorem}
    \begin{equation*}
        \lim_{z \to 1^-} p''(z) = E[X(X-1)]
    \end{equation*}
\end{theorem}
The proof is as above. More generally,
\begin{equation}
    \lim_{z \to 1^-} p^{(k)}(z) = E[X(X-1)(X-2) \cdots(X-k+1)]
    \label{eqnPGFDerivative}
\end{equation}
And therefore $\Var{(X)} = \lim_{z \to 1^-} \left(p''(z) + p'(z) - (p'(z))^2\right)$.\par
Note also that we can recover $p_n$:
\begin{equation}
    P(X = n) = \frac{1}{n!} \frac{d^n p(z)}{dz^n} |_{z = 0}
    \label{eqnPGFProb}
\end{equation}
\begin{proposition}
    Let $X_1, \cdots, X_n$ be independent identically distributed random variables. Let $N$ be an independent random variable with values in $\N$. Define:
    \begin{align*}
        S_n &= \sum_{i=1}^n X_i \\
        S_N &= \sum_{i=1}^N X_i \\
        S_n(\omega) &= \sum_{i=1}^{N(\omega)} X_i(\omega)
    \end{align*}
    Let $q$ be the probability generating function of $N$, $q(z) = E[z^N]$; let $p$ be the probability generating function of $X_i$ (note that these all have the same distribution).
    Then the probability generating function of $S_N$, $r(z)$ is:
    \begin{equation*}
        r(z) = qp(z)
    \end{equation*}
\end{proposition}
\begin{proof}
    \begin{align*}
        r(z) &= E[z^{S_N}] = \sum_{n=0}^\infty E[z^{S_N} I_{N=n}] \\
        &= \sum_{n=0}^\infty E[z^{X_1 + X_2 + \cdots + X_n} I_{N=n}] \\
        &= \sum_{n=0}^\infty E[z^{X_1 + X_2 + \cdots + X_n}]P(N=n) \\
        &= \sum_{n=0}^\infty \left(p(z)\right)^n P(N=n) \\
        &= E[p(z)^N] \\
        &= q(p(z))
    \end{align*}
    So $r(z) = qp(z)$.
\end{proof}
We can provide an alternate proof:
\begin{proof}
    \begin{align*}
        r(z) &= E[z^{S_n}] \\
        &= E[E[z^{S_n} | N]]
    \end{align*}
    Now consider $E[Z^{S_n} | N] = g(N)$:
    \begin{align*}
        g(n) &= E[z^{S_N}|N=n] \\
        &= E[z^{S_n}|N=n] \\
        &= E[z^{S_N}]
    \end{align*}
    by independence. Therefore, we get the expectation:
    \begin{align*}
        E[z^{S_N} | N] &= g(N) = (p(z))^n \\
        r(z) &= q(p(z))
    \end{align*}
\end{proof}
Now we have the probability generating function $r(z)$, we can evaluate probabilities:
\begin{align*}
    E[S_N] &= \lim_{z \to 1^-} r'(z) \\
    &= \lim_{z \to 1^-} q'(p(z)) p'(z) \\
    &= E[N] E[X_i]
\end{align*}
We can also get variance:
\begin{equation*}
    \Var{(S_N)} = E[N] \Var{(X_i)} + \Var{(N)} \left(E[X_i]\right)^2
\end{equation*}
\section{Branching Processes}
Consider a model of population growth. Let $X_n$ be the number of individuals in each generation. Let $X_0 = 1$. Then each individual independently produces a random number of offspring, with distribution $g_k$. Therefore, $P(X_1 = k) = g_k$, since there is one individual in generation $0$.\par
Define:
\begin{equation*}
    \left(Y_{k, n}~|~k \geq 1, n \geq 0\right)
\end{equation*}
Then $Y_{k, n}$ is the number of offspring that the $k$th individual in generation $n$ produces. All of these are independent, and with distribution $g_k$. Then the relation between generations is:
\begin{equation*}
    X_{n + 1} = Y_{1, n} + \cdots + Y_{{X_n}, n}
\end{equation*}
Note that if $X_n = 0$, all subsequent generations have $X_m = 0$ (the population dies out).
\begin{theorem}
    \begin{equation*}
        E[X_n] = \left(E[X_1]\right)^n
    \end{equation*}
    \label{thmBranchingExpectation}
\end{theorem}
\begin{proof}
    \begin{align*}
        E[X_{n + 1}] &= E[E[X_{n + 1} | X_n]] \\
        E[X_{n + 1} | X_n = m] &= E[Y_{1, n} + \cdots + Y_{{X_n}, m} | X_n = m] \\
        &= E[Y_{1, n} + \cdots + Y_{m, n}] \\
        &= mE[X_1]
    \end{align*}
    Then
    \begin{align*}
        E[X_{n + 1} | X_n] &= X_n E[X_1] \\
        \therefore E[X_{n + 1}] &= E[X_n E[X_1]] = E[X_1] E[X_n] \\
        \therefore E[X_{n + 1}] &= \left(E[X_1]\right)^{n + 1} \text{ by chaining the formula}
    \end{align*}
\end{proof}
\begin{theorem}
    Let $G(z) = E[z^{X_1}]$ and $G_n(z) = E[z^{X_n}]$.\par
    Then $G_{n + 1}(z) = G(G_n(z))$.
    \label{thmBranchingGenFunc}
\end{theorem}
\begin{proof}
    \begin{align*}
        G_{n + 1}(z) &= E[z^{X_{n + 1}}] \\
        &= E[E[z^{X_{n+1}} | X_n]]
    \end{align*}
    \begin{align*}
        E[z^{X_{n + 1}} | Z_n = m] &= E[z^{Y_{1, n} + \cdots + Y_{X_n, 1}} | X_n = m] \\
        &= E[z^{Y_1 + \cdots + Y_m} | X_n = m] \\
        &= \left(E[z^{X_1}]\right)^m
    \end{align*}
    and therefore $E[Z^{X_{n + 1}} | X_n] = \left(G(z)\right)^{X_n}$
    \begin{equation*}
        G_{n + 1}(z) = E[\left(G(z)\right)^{X_n}] = G_n(G(z))
    \end{equation*}
\end{proof}
Now consider the probability that the population goes extinct. Set:
\begin{equation*}
    q = P(X_n = 0 \text{ for some } n \geq 1)
\end{equation*}
and $q_n = P(X_n = 0)$. Therefore, $A_n = \{X_n = 0\}$ is a subset of $A_{n + 1}$. Now we have an increasing family of events, and $q_n \to q$ as $n \to \infty$.
\begin{proposition}
    $q_{n + 1} = G(q_n)$, where $G$ is as defined before: $G(z) = E[z^{X_1}]$.
    \label{propExtinctionProb}
\end{proposition}
\begin{proof}[by conditioning on $X_n$]
    \begin{align*}
        q_{n + 1} &= P(X_{n + 1} = 0) \\
        &= G_{n + 1}(0) \text{ by expanding out the expectation function} \\
        &= G(G_n(0)) \text{ by theorem~\ref{thmBranchingGenFunc}}
    \end{align*}
\end{proof}
\begin{proof}[by conditioning on $X_1$]
    We condition on $X_1 = m$. Let $X_n^{(1)}, \cdots, X_n^{(m)}$ be independent and identically distributed branching processes with the same offspring distribution as $X_1$. We can do this because the offspring from the second generation can be thought of as another independent branching process as before.\par
    Note also that $X_{n + 1} = X_n^{(1)} + \cdots + X_n^{(m)}$ given $X_1 = m$\par
    \begin{align*}
        q_{n + 1} &= P(X_{n + 1} = 0) \\
        &= \sum_m P(X_{n + 1} = 0 | X_1 = m)P(X_1 = m) \\
        &= \sum_m P(X_1 = m) \left(P(X_n^{(1)} = 0)\right)^m \text{ by independence} \\
        &= \sum_m P(X_1 = m) \left(P(X_n = 0)\right)^m \\
        &= G(P(X_n = 0)) \\
        &= G(q_n)
    \end{align*}
\end{proof}
Now we have that $q_n \to q$ and $q_{n + 1} = G(q_n)$, so by the continuity of $G$, the limit probability $q$ must satisfy $q = G(q)$.\par
We can reason by theorem~\ref{thmBranchingExpectation}:
\begin{align*}
    &\text{If } E[X_1] < 1 \implies E[X_n] \to 0 \implies q = 1 \\
    &\text{If } E[X_1] > 1 \implies E[X_n] \to \infty \implies q < 1 \\
    &\text{If } E[X_1] = 1 \implies E[X_n] = 1
\end{align*}
\begin{theorem}
    Suppose $P(X_1 = 1) < 1$. Then the extinction probability is the minimal non-negative solution to $x = G(x)$. Moreover, $q < 1$ if and only if $E[X_1] > 1$.
    \label{thmExtinctionProb}
\end{theorem}
\begin{proof}
    We know that $q = G(q)$. Let $t$ be the smallest solution of $x = G(x)$. We will show that $t = q$, by showing $q \leq t$.\par
    \induction{$n = 0$}{$q_0 = 0 \leq t$ as required.}
    {$n = k$}{Suppose $q_k \leq t$}
    {$n = k + 1$}{
        Note that $G(z) = E[z^{X_1}]$ is an increasing function.
        \begin{equation*}
            q_{n + 1} = G(q_n) \leq G(t) = t
        \end{equation*}
    }
    Now we have that $q_n \leq t~\forall n$, so by taking the limit, $q \leq t$. But, since $t$ is the smallest possible solution to $x = G(x)$, and $q$ is a solution, $t = q$.\par
    It now remains to prove that $q < 1 \Leftrightarrow E[X_1] > 1$.
    \begin{case}{$g_0 + g_1 = 1$}
        Then $P(X_1 \leq 1) = 1$ and $E[X_1] = g_1$. Therefore, $G(z) = g_0 + g_1 z = 1 - E[X_1] + E[X_1] z$. Then solving $G(z) = z$:
        \begin{equation*}
            1 - E[X_1] = \left(1 - E[X_1]\right)z \implies z = 1
        \end{equation*}
        Then $E[X_1] < 1$ since $P(X_1 = 1) < 1$.
    \end{case}
    \begin{case}{$g_0 + g_1 < 1$}
        Define $H(z) = G(z) - z$. Then certainly $H(1) = 0$, and we will prove that there can exist at most one more in $[0, 1)$.\par
        Consider $H''(z)$:
        \begin{equation*}
            H''(z) = \sum_{r=2}^\infty r(r-1) g_r z^{r-2}
        \end{equation*}
        This is strictly positive by assumption of the case ($g_0 + g_1 < 1$). Therefore $H'(z)$ is strictly increasing on $[0, 1)$, and this implies $H$ can have at most $1$ other root in the range. To see this, suppose there exist two (or more) other roots in the range. Then $H'$ would have to be equal to zero at two points (between each root) by Rolle's Theorem, which is a contradiction.\par
        Consider now the case that there is no second solution, $q = 1$.
        \begin{equation*}
            \lim_{z \to 1^-} H'(z) = \lim_{z \to 1^-} G'(z) - 1
        \end{equation*}
        Note that $H(0) = G(0) = g_0 \geq 0$, and $H(1) = 0$, so this implies that $H(z) \geq 0$ for all $z \in [0, 1)$.\par
        Then we can consider the limit definition of $H'$:
        \begin{equation*}
            \lim_{z \to 1^-} H'(z) = \lim_{z \to 1^-} \frac{H(z) - H(1)}{z - 1} \leq 0
        \end{equation*}
        But $\lim_{z \to 1^-} H'(z) = E[X] - 1$, so indeed $E[x] \leq 1$.\par
        Consider the case that there does exist a second solution, $r < 1$.\par
        Now $H(r) = 0$, and $H(1) = 0$, so there exists some $w \in (r, 1)$ such that $H'(w) = 0$ by Rolle's Theorem.\par
        We have that $H'$ is strictly increasing, so for any $z > w$, $H'(z) > H'(w) = 0$.\par
        Therefore $\lim_{z \to 1^-} H'(z) > 0$, which means $E[X] - 1 > 0$, and $E[X] > 1$ as required.
    \end{case}
\end{proof}
\end{document}