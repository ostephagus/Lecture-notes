\documentclass[../Main.tex]{subfiles}

\begin{document}
\section{Probability Spaces}
\subsection{Fundamental Definitions}
\begin{definition}{$\sigma$-algebra}
    Suppose $\Omega$ is a set and $\sigalg$ a collection of subsets.\par
    We call $\sigalg$ a \underline{$\sigma$-algebra} if:
    \begin{enumerate}
        \item $\Omega \in \sigalg$
        \item If $A \in \sigalg$, then $A^c \in \sigalg$ % What is a complement?
        \item For any countable collection $(A_n)_{n \in \N}$ of sets is $\sigalg$, if all $A_n$ are in $\sigalg$ then the union must be in $\sigalg$.
    \end{enumerate}
\end{definition}
\begin{definition}{Probability measure}
    A function $P : \sigalg \mapsto [0, 1]$ is called a \underline{probability measure} if:
    \begin{enumerate}
        \item $P(\Omega) = 1$
        \item For any countable collection of disjoint sets $(A_n)_{n \in \N}$ in $\sigalg$ then the probability of the union is:
            \begin{equation*}
                P(\bigcup_{n \in \N} A_n) = \sum_{n \in \N} P(A_n)
            \end{equation*}
    \end{enumerate}
    And we say $P(A)$ is the probability of $A$.
\end{definition}
\begin{definition}{Probability space}
    Then the triplet $(\Omega, \sigalg, P)$ is a \underline{probability space}.\par
    The elements of $\Omega$ are \underline{outcomes} and the elements of $\sigalg$ are called \underline{events}.
\end{definition}
Important properties:
\begin{enumerate}
    \item $P(A^c) = 1 -P(A)$ for any $A \in \sigalg$
    \item $P(\emptyset) = 0$
    \item If $A \subseteq B$ then $P(A) \leq P(B)$
    \item $P(A \cup B) = P(A) + P(B) - P(A \cap B)$.
\end{enumerate}
\subsection{Examples of Probaility Spaces}
\begin{example}[Rolling a fair die]
    $\Omega = \{1, 2, \cdots, 6\}, \sigalg = \powerset(\Omega)$\par % This is power set - how to distinguish?
        Then $P(\{\omega\}) = \frac{1}{6}$, and so $P(A) = \frac{|A|}{6}$.
\end{example}
\begin{remark}
    When $\Omega$ is countable we take $\sigalg$ to be all the subsets of $\Omega$.
\end{remark}
\begin{example}[Uniform distribution]
    Let $\Omega$ be a finite set $\Omega = \{\omega_1, \cdots, \omega_n\}$.\par
    Then let $\sigalg$ be all subsets.\par
    Let $P(A) = \frac{|A|}{|\Omega|}$.
\end{example}
\begin{example}[Picking balls from a bag]
    Suppose we have $n$ balls labelled $1$ to $n$ in a bag indistinguishable when in the bag. Pick $k$ balls at random, without replacement.\par
    Then $\Omega = \subsetselect{A \in \{1, \cdots, n\}}{|A| = k}$.\par
    Then $|Omega| = \begin{pmatrix}n \\ k\end{pmatrix}$.\par
    $P(\omega) = \frac{1}{nCk}$.
\end{example}
\begin{example}[Deck of cards]
    Consider a well-shuffled deck of cards.\par
    Then $\Omega = \{\text{all permutations of 52 elements}\}$. It has size $52!$.\par
    Then consider $P(\text{two top cards are aces}) = \frac{4 \times 3 \times 50!}{52!}$.
\end{example}
\begin{example}{Largest digit}
    Consider a string of n base-10 digits chosen randomly.\par
    $\Omega = \{0, \cdots, 9\}^n$, $|\Omega| = 10^n$.\par
    Then let $A_k=\{\text{no digit exceeds } k\}$.\par
    \begin{equation*}
        P(A_k) = \frac{|A_k|}{|\Omega|} = \frac{(k + 1)^n}{10^n}
    \end{equation*}
    Also consider $B_k = \{\text{largest digit is } k\}$. Note that $B_k = A_k \backslash A_{k - 1}$.
\end{example}
\begin{example}[Birthday problem]
    Consider $n$ people and assume each person is equally likely to have been born on each day (excluding 29th February). We calculate the probability that two people share the same birthday.\par
    $\Omega = \{1, \cdots, 365\}^n$, $\sigalg$ is all subsets.\par
    Then the probability of any outcome $\omega$ is $\frac{1}{365^n}$.\par
    Let $A = \{\text{at least two people share a birthday}\}$. It is easier to consider $A^c$: all birthdays are distinct.\par
    \begin{equation*}
        P(A^c) = \frac{|A^c|}{|\Omega|} = \frac{365 \times 364 \times \cdots \times (365-n + 1)}{365^n}
    \end{equation*}
    And $P(A) = 1 - P(A^c)$.
\end{example}
\section{Combinatorial Analysis}
\subsection{Examples of Combinatorics}
Suppose $\Omega$ is a finite set, and $|\Omega| = n$. Then consider a partition of $\Omega$ into $k$ disjoint subsets. How many ways are there to do this?\par
Let $M$ be the number of ways.\par
\begin{align*}
    M &= \choose{n}{n_1} \choose{n - n_1}{n_2} \cdots \choose{n - n_1 - \cdots n_{k-1}}{n_k} \\
    &= \frac{n!}{n_1! n_2! \cdots n_k!} \\
    &= \choose{n}{n_1, \cdots, n_k}
\end{align*}
We can also consider the number of possible functions between finite sets:
\begin{definition}{Increasing function}
    A function $f$ is \underline{increasing} if:
    \begin{equation*}
        x < y \implies f(x) < f(y)
    \end{equation*}
\end{definition}
\begin{definition}{Non-decreasing function}
    A function $f$ is \underline{non-decreasing} if:
    \begin{equation*}
        x < y \implies f(x) \leq f(y)
    \end{equation*}
\end{definition}
Now consider the set of increasing functions $f : \{1, \cdots, k\} \mapsto \{1, \cdots, \}$.\par
We can uniquely determine each function by its image, which is a subset of $\{1, \cdots, n\}$ of size $k$. Therefore there are $\choose{n}{k}$ such functions.\par
Now we consider a similar set of non-decreasing functions $f : \{1, \cdots, k\} \mapsto \{1, \cdots, k\}$. We can put this in bijection with a set of increasing functions: define a map $\phi$:
\begin{equation*}
    \phi(f(x)) = g(x) = f(x) + x - 1
\end{equation*}
So now the set of all such functions $g$ is the set of functions $g : \{1, \cdots, k\} \mapsto \{1, \cdots, n + k - 1\}$. And by above, the number of non-decreasing functions is:
\begin{equation*}
    \choose{n + k - 1}{k}
\end{equation*}
\subsection{Stirling's Formula}
Notation: write $a_n \sim b_n$ as $n \to \infty$ if $\frac{a_n}{b_n} \to 1$ as $n \to \infty$.
\begin{theorem}[Stirling's Formula]
    \begin{equation}
        n! \sim n^n \sqrt{2\pi n} e^{-n}
        \label{eqnStirlingFormula}
    \end{equation}
\end{theorem}
A weaker statement is: $\log{n!} \sim n \log{n}$.
\begin{proof}[of the weaker statement]
    Define $\lfloor x \rfloor$ as the integer part of $x$.\par
    \begin{equation*}
        \log{\lfloor x \rfloor} \leq \log{x} \log{\lfloor x + 1 \rfloor}
    \end{equation*}
    Integrating from $1$ to $n$:
    \begin{align*}
        \sum_{k=1}^{n-1} \log{k} &\leq \int_{1}^{n} \log{x} dx \leq \sum{k=1}^{n} \log{k} \\
        \log{(n-1)!} &\leq n \log{n} - n + 1 \leq \log{n!}
    \end{align*}
    Rearranging these terms gives:
    \begin{align*}
        n \log{n} - n + 1 &\leq \log{n!} \leq (n + 1) \log{(n + 1)} - (n+1) + 1 \\
        1 + \frac{1 - n}{n\log{n}} &\leq \frac{\log{n!}}{n\log{n}} \leq \frac{(n+1)\log{(n+1)} - n}{n\log{n}}
    \end{align*}
    And both sides go to $1$ when $n \to \infty$, so by sandwiching the middle term it must also go to $1$, and $\log{n!} \sim n \log{n}$.
\end{proof}
\begin{proof}[of the theorem]
    Let $f$ be a twice-differentiable function, $a < b$ be real numbers. Then by integration by parts,
    \begin{equation}
        \int_{a}^{b} f(x) dx = \frac{f(a) + f(b)}{2} (b - a) - \frac{1}{2} \int_a^b (x - a)(b - x)f''(x)dx
        \label{eqnDoubleParts}
    \end{equation}
    Now let $f(x) = \log{x}, a = k, b = k + 1$ in equation~\ref{eqnDoubleParts}
    \begin{equation*}
        \int_k^{k+1} \log{x} dx - \frac{\log{k} + \log{k + 1}}{2} + \frac{1}{2} \int_k^{k + 1} \frac{(x - k)(k + 1 - x)}{x^2}
    \end{equation*}
    And let the final term be $a_k$:
    \begin{equation*}
        a_k = \frac{1}{2} \int_0^1 \frac{x(1-x)}{(x + k)^2}
    \end{equation*}
    Then take the sum over $k = 1, \cdots, n-1$:
    \begin{align*}
        \int_1^n \log{x} dx &= \frac{\log{(n - 1)!} + \log{n!}}{2} + \sum_{k=1}^{n} a_k \\
        n \log{n} - n + 1 &= \log{n!} - \frac{\log{n}}{2} + \sum_{k=1}^{n} a_k \\
        \log{n!} &= n \log{n} - n + 1 + \frac{\log{n}}{2} - \sum_{k=1}^{n} a_k \\
    \end{align*}
    Taking exponentials:
    \begin{equation*}
        n! = n^n e^{-n} \sqrt{n} \exp{\left(1 - \sum_{k=1}^{n} a_k\right)}
    \end{equation*}
    Then consider $a_k$:
    \begin{align*}
        a_k &= \frac{1}{2}\int_0^1 \frac{x(1-x)}{(x-k)^2} dx \\
        &\leq \frac{1}{2} \int_0^1 \frac{x(1-x)}{k^2} dx \\
        &= \frac{1}{12k^2}
    \end{align*}
    So $\sum_{k=1}^{\infty} a_k$ is finite.
    \begin{equation*}
        \text{Let } A = \exp{\left(1 - \sum_{k=1}^{\infty} a_k\right)}
    \end{equation*}
    Then $A > 0$.
    \begin{equation*}
        n! = n^n e^{-n} \sqrt{n} A \exp{\left(\sum_{k=1}^{\infty} a_k\right)}
    \end{equation*}
    Then, since the term in the exponential goes to $0$ in the limit, $n! \sim n^n e^{-n} \sqrt{n} A$.
    \par
    Now we need to evaluate $A$.\par
    Consider the quantity$B$:
    \begin{equation*}
        B = 2^{-2n} \choose{2n}{n}
    \end{equation*}
    Then we evaluate this in two different ways (in terms of $A$ and using integration) to find the value of $n$.
    \begin{align*}
        B &= 2^{-2n} \frac{(2n)!}{(n!)(n!)} \\
        &\sim \frac{2^{-2n} (2n)^{2n} \sqrt{2n} e^{-2n} A}{(n^n e^{-n} \sqrt{n} A)^2} \text{ by replacing factorials.} \\
        &= \frac{\sqrt{2}}{A \sqrt{n}} \text{ by much cancellation.}
    \end{align*}
    Then define, for any $n \in \N$,
    \begin{equation*}
        I_n \int_0^{\frac{\pi}{2}} (\cos{\theta})^n d\theta
    \end{equation*}
    Note that $I_0 = \frac{\pi}{2}$ and $I_1 = 1$.\par
    Also, by integration by parts we have a reduction formula:
    \begin{equation*}
        I_n = \frac{n-1}{n} I_{n-2}
    \end{equation*}
    \begin{align*}
        I_{2n} &= \frac{2n - 1}{2n} I_{2n-2} \\
        &= \frac{(2n-1)(2n-3)}{2n(2n-2)} I_{2n-4} \\
        &= \frac{(2n-1)(2n-3) \cdots(3)(1)}{2n(2n-2) \cdots (4)(2)} I_0 \\
        &= \frac{(2n)!}{2^{2n}(n!)^2} \frac{\pi}{2} \\
        &= 2^{-2n} \choose{2n}{n} \frac{\pi}{2}
    \end{align*}
    Simiarly,
    \begin{align*}
        I_{2n+1} &= \frac{1}{2n+1}\left(2^{-2n} \choose{2n}{n}\right)^{-1} I_1 \text{ by the same method} \\
        &= \frac{1}{2n+1}\left(2^{-2n} \choose{2n}{n}\right)^{-1}
    \end{align*}
    Now from the reduction formula, $\frac{I_{2n}}{I_{2n-2}} \to 1$ as $n \to \infty$.\par
    Also, since $\cos$ is positive on the interval being integrated, $I_n$ is positive and must therefore be decreasing.\par
    Therefore, $\frac{I_{2n}}{I_{2n+1}} \to 1$ by sandwiching between the odd sequence and even sequence.\par
    And this ratio can be evaluated:
    \begin{equation*}
        \frac{I_{2n}}{I_{2n+1}} = \left(2^{-2n} \choose{2n}{n}\right)^2 \frac{\pi}{2} (2n+1)
    \end{equation*}
    And so as $n \to \infty$:
    \begin{align*}
        \left(2^{-2n} \choose{2n}{n}\right)^2 &\sim \frac{2}{\pi(2n+1)} \\
        &\sim \frac{1}{\pi n}
    \end{align*}
    And this is $B^2$. Now we can equate (as $n \to \infty$):
    \begin{align*}
        \frac{\sqrt{2}}{A \sqrt{n}} &\sim \frac{1}{\sqrt{\pi n}} \\
        \frac{\sqrt{2}}{A} &\sim \frac{1}{\sqrt{\pi}} \\
        A &\sim \sqrt{2\pi}
    \end{align*}
    And since this is no longer in terms of $n$, $A = \sqrt{2\pi}$
\end{proof}
\end{document}