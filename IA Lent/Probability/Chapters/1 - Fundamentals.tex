\documentclass[../Main.tex]{subfiles}

\begin{document}
\section{Probability Spaces}
\subsection{Fundamental Definitions}
\begin{definition}{Complement of a subset}
    If $A$ is a subset of another set $\Omega$, then the \underline{complement} of $A$ is defined to be:
    \begin{equation*}
        A^C = \Omega \backslash A
    \end{equation*}
\end{definition}
\begin{definition}{$\sigma$-algebra}
    Suppose $\Omega$ is a set and $\sigalg$ a collection of subsets.\par
    We call $\sigalg$ a \underline{$\sigma$-algebra} if:
    \begin{enumerate}
        \item $\Omega \in \sigalg$
        \item If $A \in \sigalg$, then $A^c \in \sigalg$
        \item For any countable collection $(A_n)_{n \in \N}$ of sets is $\sigalg$, if all $A_n$ are in $\sigalg$ then the union must be in $\sigalg$.
    \end{enumerate}
\end{definition}
\begin{definition}{Probability measure}
    A function $P : \sigalg \mapsto [0, 1]$ is called a \underline{probability measure} if:
    \begin{enumerate}
        \item $P(\Omega) = 1$
        \item For any countable collection of disjoint sets $(A_n)_{n \in \N}$ in $\sigalg$ then the probability of the union is:
            \begin{equation*}
                P(\bigcup_{n \in \N} A_n) = \sum_{n \in \N} P(A_n)
            \end{equation*}
    \end{enumerate}
    And we say $P(A)$ is the probability of $A$.
\end{definition}
\begin{definition}{Probability space}
    Then the triplet $(\Omega, \sigalg, P)$ is a \underline{probability space}.\par
    The elements of $\Omega$ are \underline{outcomes} and the elements of $\sigalg$ are called \underline{events}.
\end{definition}
\subsection{Properties of a Probability Space}
\begin{propositions}[Properties of a probability measure] {
        Let $(\Omega, \sigalg, P)$ be a probability space. Let $A$ and $B$ be events.
        \label{propsProbMeasure}
    }
    \item $P(A^c) = 1 - P(A)$ for any $A \in \sigalg$ \label{propComplmentProb}
    \item $P(\emptyset) = 0$ \label{propPropEmpty}
    \item If $A \subseteq B$ then $P(A) \leq P(B)$ \label{propSubsetProbs}
    \item $P(A \cup B) = P(A) + P(B) - P(A \cap B)$. \label{propUnionProb}
\end{propositions}
\begin{proposition}[Countable subaddivity of P]
    If $(A_n)_{n \geq 1}$ is a countable collection if $\sigalg$, then:
    \begin{align*}
        P\left(\bigcup_{n \in \N} A_n\right) \leq \sum_{n \in \N} P(A_n)
    \end{align*}
    \label{propUnionProbInequality}
\end{proposition}
\begin{proof}
    Define $B_1 = A_1$, and for $n \geq 2$
    \begin{equation*}
        B_n = A_n \backslash (A_1 \cup \cdots \cup A_{n-1})
    \end{equation*}
    Then the sets $B_n$ are disjoint and in $\sigalg$. Also, the union of all $B_n$ is equal to the union of all $A_n$.
    \begin{align*}
        P\left(\bigcup_{n \in \N} A_n\right) &= P\left(\bigcup_{n \in \N} B_n \right) \\
        &= \sum_{n \in \N} P(B_n).
    \end{align*}
    But $B_n \subseteq A_n$ for any $n$. This means $P(A_n) \geq P(B_n)$. So:
    \begin{equation*}
        P\left(\bigcup_{n \in \N} A_n\right) = \sum_{n \in \N} P(B_n) \leq \sum_{n \in \N} P(A_n)
    \end{equation*}
\end{proof}
\begin{proposition}[Continuity of Probability measures]
    Suppose $(A_i)_{i \geq 1}$ is an increasing family of events in $\sigalg$, and:
    \begin{equation*}
        A_i \subseteq A_{i+1}
    \end{equation*}
    Therefore $P(A_i) \leq P(A_{i+1})$ for all $i$.\par
    Then $P(A_i) \to P\left(\bigcup_{n \in \N} A_n\right)$ as $i \to \infty$.
    \label{propProbIncreasing}
\end{proposition}
\begin{proof}
    Define $B_i$ as in proof of proposition~\ref{propUnionProbInequality}.
    \begin{align*}
        P(A_n) &= P\left(\bigcup_{k=1}^n B_k\right) \\
        &= \sum_{k=1}^{n} P(B_k)
    \end{align*}
    So as $n \to \infty$, along with the second property of a probability measure,
    \begin{equation*}
        P\left(\bigcup_{n=1}^\infty B_n \right) = \sum_{n=1}^{\infty} P(B_n)
    \end{equation*}
    And since the union of $A_n$ is the union of $B_n$,
    \begin{equation*}
        P\left(\bigcup_{n \in \N} A_n \right) = \sum_{k=1}^{\infty} P(B_k)
    \end{equation*}
\end{proof}
\begin{proposition}
    Let $A_i$ be a sequence of events in $\sigalg$ for any natural $i$. Let also $A_i \supseteq A_{i+1}$\par
    Then $P(A_i) \to P\left(\cap_{n \in \N} A_n\right)$ as $i \to \infty$.
    \label{propProbDecreasing}
\end{proposition}
\begin{proof}
    Take complements and then use proposition~\ref{propProbIncreasing}.
\end{proof}
Proposition~\ref{propUnionProb}: for events $A$ and $B$ in $\sigalg$,
\begin{equation*}
    P(A \cup B) = P(A) + P(B) - P(A \cap B)
\end{equation*}
This can be extended to $n$ different events.
\begin{theorem}[Inclusion-Exclusion formula]
    Let $A_1, \cdots, A_n$ be events. Then
    \begin{align*}
        P\left(\cup_{i = 1}^n A_i\right) &= \sum_{i = 1}^n P(A) - \sum_{i \leq i_1 < i_2 \leq n} P(A_{i_1} \cap P(A_{i_2})) \cdots \\
        &+ (-1)^{n+1} P(A_1 \cap \cdots \cap A_n) 
    \end{align*}
    Or,
    \begin{equation}
        \sum_{k=1}^{n} (-1)^{k+1} \sum_{1 \leq i_1 < \cdots < i_k \leq n} P(A_{i_1} \cap \cdots \cap A_{i_k})
        \label{eqnInclusionExclusion}
    \end{equation}
    \label{thmInclusionExclusion}
\end{theorem}
\begin{proof}
    \induction{$n=2$}{In this case the statement is identical to the starting formula.}
    {$n = k - 1$}{}
    {$n = k$}{
        %TODO: finish.
    }
\end{proof}
\begin{propositions}[Bonferroni inequalities]{
        The Bonferroni Inequalities are:
        \label{propsBonferroni}
    }
    \item $P(A \cup B) \leq P(A) + P(B)$
    \item For events $A_i$,
        \begin{equation*}
            P\left(\cup_{i = 1}^n\right)
            \begin{cases}
                \leq \sum_{k=1}^r (-1)^{k+1} \sum_{1 \leq i_1 < \cdots < i_n \leq n} P(A_{i_1} \cap \cdots \cap A_{i_k}) & r \text{ odd} \\
                \geq \sum_{k=1}^r (-1)^{k+1} \sum_{1 \leq i_1 < \cdots < i_n \leq n} P(A_{i_1} \cap \cdots \cap A_{i_k}) & r \text{ even}
            \end{cases}
        \end{equation*}
\end{propositions}
\begin{proof}
    \induction{$n = 2$}{This is the first Bonferroni inequality, proven by considering proposition~\ref{propUnionProbInequality}}
    {$n = k - 1$}{}
    {$n = k$}{
        \begin{case}{$r$ is odd}
            %TODO: finish
        \end{case}
        \begin{case}{$r$ is even}
        \end{case}
    }
\end{proof}
Now consider a set $\Omega$ such that the probability measure is:
\begin{equation*}
    P(A) = \frac{|A|}{|\Omega|}
\end{equation*}
Then using the inclusion-exclusion formula:
\begin{equation}
    |A_1 \cup \cdots \cup A_n| = \sum_{k=1}^{n} (-1)^{k+1} \sum_{1 \leq i_1 < \cdots < i_n \leq n} |A_{i_1} \cap \cdots \cap A_{i_k}|
    \label{eqnInclusionExclusionFiniteSets}
\end{equation}
\subsection{Examples of Probaility Spaces}
\begin{example}[Rolling a fair die]
    $\Omega = \{1, 2, \cdots, 6\}, \sigalg = \powerset(\Omega)$\par % This is power set - how to distinguish?
        Then $P(\{\omega\}) = \frac{1}{6}$, and so $P(A) = \frac{|A|}{6}$.
\end{example}
\begin{remark}
    When $\Omega$ is countable we take $\sigalg$ to be all the subsets of $\Omega$.
\end{remark}
\begin{example}[Uniform distribution]
    Let $\Omega$ be a finite set $\Omega = \{\omega_1, \cdots, \omega_n\}$.\par
    Then let $\sigalg$ be all subsets.\par
    Let $P(A) = \frac{|A|}{|\Omega|}$.
\end{example}
\begin{example}[Picking balls from a bag]
    Suppose we have $n$ balls labelled $1$ to $n$ in a bag indistinguishable when in the bag. Pick $k$ balls at random, without replacement.\par
    Then $\Omega = \subsetselect{A \in \{1, \cdots, n\}}{|A| = k}$.\par
    Then $|Omega| = \begin{pmatrix}n \\ k\end{pmatrix}$.\par
    $P(\omega) = \frac{1}{nCk}$.
\end{example}
\begin{example}[Deck of cards]
    Consider a well-shuffled deck of cards.\par
    Then $\Omega = \{\text{all permutations of 52 elements}\}$. It has size $52!$.\par
    Then consider $P(\text{two top cards are aces}) = \frac{4 \times 3 \times 50!}{52!}$.
\end{example}
\begin{example}{Largest digit}
    Consider a string of n base-10 digits chosen randomly.\par
    $\Omega = \{0, \cdots, 9\}^n$, $|\Omega| = 10^n$.\par
    Then let $A_k=\{\text{no digit exceeds } k\}$.\par
    \begin{equation*}
        P(A_k) = \frac{|A_k|}{|\Omega|} = \frac{(k + 1)^n}{10^n}
    \end{equation*}
    Also consider $B_k = \{\text{largest digit is } k\}$. Note that $B_k = A_k \backslash A_{k - 1}$.
\end{example}
\begin{example}[Birthday problem]
    Consider $n$ people and assume each person is equally likely to have been born on each day (excluding 29th February). We calculate the probability that two people share the same birthday.\par
    $\Omega = \{1, \cdots, 365\}^n$, $\sigalg$ is all subsets.\par
    Then the probability of any outcome $\omega$ is $\frac{1}{365^n}$.\par
    Let $A = \{\text{at least two people share a birthday}\}$. It is easier to consider $A^c$: all birthdays are distinct.\par
    \begin{equation*}
        P(A^c) = \frac{|A^c|}{|\Omega|} = \frac{365 \times 364 \times \cdots \times (365-n + 1)}{365^n}
    \end{equation*}
    And $P(A) = 1 - P(A^c)$.
\end{example}
\begin{example}[Counting the number of surjections]
    Let $\Omega = \{f : \{1, \cdots, n\} \mapsto \{1, \cdots, m\}\}$. \\
    Let $A = \subsetselect{f \in \Omega}{f \text{ is a surjection}}$.\par
    Define:
    \begin{equation*}
        A_i = \subsetselect{f \in \Omega}{i \notin \{f(1), \cdots, f(n)\}}
    \end{equation*}
    Therefore $A = \cap_{i=1}^m A_i^C$.\par
    And so the size of $A$ is:
    \begin{equation*}
        |A| = \left|\cap_{i=1}^m A_i^C\right| = |\Omega| - \left|\cup_{i=1}^m A_i\right|
    \end{equation*}
    Note that $|\Omega| = m^n$.
\end{example}
\begin{example}[Number of derangements]
    A derangement is a permutation that has no fixed points.\par
    Let $\Omega = \{\text{set of permutations of } \{1, \cdots, n\}\}$.\par
    And let $A$ be the set of derangements: $A = \subsetselect{f \in \Omega}{f(i) \neq i~\forall i}$.
    Now we pick a permutation at random. What is the probability that it is a derangement?\par
    For this, we need to consider $|A|$.\par
    Let $A_i$ be a sequence of events: $A_i = \subsetselect{f \in \Omega}{f(i) = i}$. This is such that:
    \begin{equation*}
        A = \bigcap_{i=1}^n A_i^C = \left(\bigcup_{i=1}^n A_i\right)^C
    \end{equation*}
    \begin{align*}
        P(A) &= 1- P\left(\bigcup_{i=1}^n A_i\right) \\
        &= 1 = \sum_{k=1}^{n} (-1)^{k+1} \sum_{1 \leq i_1 < \cdots < i_k \leq n} P(A_{i_1} \cap \cdots \cap A_{i_k})
    \end{align*}
    Now we consider this intersection:
    \begin{align*}
        P(A_{i_1} \cap \cdots \cap A_{i_k}) &= \frac{|A_{i_1} \cap \cdots \cap A_{i_k}}{|\Omega} \\
        &= \frac{(n-k)!}{n!}
    \end{align*}
    Then our probability is:
    \begin{align*}
        P(A) &= 1 - \sum_{k=1}^n (-1)^{k+1} \choose{n}{k} \frac{(n-k)!}{n!} \\
        &= 1 - \sum_{k=1}^n (-1)^{k+1} \frac{1}{k!}
    \end{align*}
    And as $n \to \infty$, this tends to $e^{-1}$.
\end{example}
\section{Combinatorial Analysis}
\subsection{Examples of Combinatorics}
Suppose $\Omega$ is a finite set, and $|\Omega| = n$. Then consider a partition of $\Omega$ into $k$ disjoint subsets. How many ways are there to do this?\par
Let $M$ be the number of ways.\par
\begin{align*}
    M &= \choose{n}{n_1} \choose{n - n_1}{n_2} \cdots \choose{n - n_1 - \cdots n_{k-1}}{n_k} \\
    &= \frac{n!}{n_1! n_2! \cdots n_k!} \\
    &= \choose{n}{n_1, \cdots, n_k}
\end{align*}
We can also consider the number of possible functions between finite sets:
\begin{definition}{Increasing function}
    A function $f$ is \underline{increasing} if:
    \begin{equation*}
        x < y \implies f(x) < f(y)
    \end{equation*}
\end{definition}
\begin{definition}{Non-decreasing function}
    A function $f$ is \underline{non-decreasing} if:
    \begin{equation*}
        x < y \implies f(x) \leq f(y)
    \end{equation*}
\end{definition}
Now consider the set of increasing functions $f : \{1, \cdots, k\} \mapsto \{1, \cdots, \}$.\par
We can uniquely determine each function by its image, which is a subset of $\{1, \cdots, n\}$ of size $k$. Therefore there are $\choose{n}{k}$ such functions.\par
Now we consider a similar set of non-decreasing functions $f : \{1, \cdots, k\} \mapsto \{1, \cdots, k\}$. We can put this in bijection with a set of increasing functions: define a map $\phi$:
\begin{equation*}
    \phi(f(x)) = g(x) = f(x) + x - 1
\end{equation*}
So now the set of all such functions $g$ is the set of functions $g : \{1, \cdots, k\} \mapsto \{1, \cdots, n + k - 1\}$. And by above, the number of non-decreasing functions is:
\begin{equation*}
    \choose{n + k - 1}{k}
\end{equation*}
\subsection{Stirling's Formula}
Notation: write $a_n \sim b_n$ as $n \to \infty$ if $\frac{a_n}{b_n} \to 1$ as $n \to \infty$.
\begin{theorem}[Stirling's Formula]
    \begin{equation}
        n! \sim n^n \sqrt{2\pi n} e^{-n}
        \label{eqnStirlingFormula}
    \end{equation}
\end{theorem}
A weaker statement is: $\log{n!} \sim n \log{n}$.
\begin{proof}[of the weaker statement]
    Define $\lfloor x \rfloor$ as the integer part of $x$.\par
    \begin{equation*}
        \log{\lfloor x \rfloor} \leq \log{x} \log{\lfloor x + 1 \rfloor}
    \end{equation*}
    Integrating from $1$ to $n$:
    \begin{align*}
        \sum_{k=1}^{n-1} \log{k} &\leq \int_{1}^{n} \log{x} dx \leq \sum{k=1}^{n} \log{k} \\
        \log{(n-1)!} &\leq n \log{n} - n + 1 \leq \log{n!}
    \end{align*}
    Rearranging these terms gives:
    \begin{align*}
        n \log{n} - n + 1 &\leq \log{n!} \leq (n + 1) \log{(n + 1)} - (n+1) + 1 \\
        1 + \frac{1 - n}{n\log{n}} &\leq \frac{\log{n!}}{n\log{n}} \leq \frac{(n+1)\log{(n+1)} - n}{n\log{n}}
    \end{align*}
    And both sides go to $1$ when $n \to \infty$, so by sandwiching the middle term it must also go to $1$, and $\log{n!} \sim n \log{n}$.
\end{proof}
\begin{proof}[of the theorem]
    Let $f$ be a twice-differentiable function, $a < b$ be real numbers. Then by integration by parts,
    \begin{equation}
        \int_{a}^{b} f(x) dx = \frac{f(a) + f(b)}{2} (b - a) - \frac{1}{2} \int_a^b (x - a)(b - x)f''(x)dx
        \label{eqnDoubleParts}
    \end{equation}
    Now let $f(x) = \log{x}, a = k, b = k + 1$ in equation~\ref{eqnDoubleParts}
    \begin{equation*}
        \int_k^{k+1} \log{x} dx - \frac{\log{k} + \log{k + 1}}{2} + \frac{1}{2} \int_k^{k + 1} \frac{(x - k)(k + 1 - x)}{x^2}
    \end{equation*}
    And let the final term be $a_k$:
    \begin{equation*}
        a_k = \frac{1}{2} \int_0^1 \frac{x(1-x)}{(x + k)^2}
    \end{equation*}
    Then take the sum over $k = 1, \cdots, n-1$:
    \begin{align*}
        \int_1^n \log{x} dx &= \frac{\log{(n - 1)!} + \log{n!}}{2} + \sum_{k=1}^{n} a_k \\
        n \log{n} - n + 1 &= \log{n!} - \frac{\log{n}}{2} + \sum_{k=1}^{n} a_k \\
        \log{n!} &= n \log{n} - n + 1 + \frac{\log{n}}{2} - \sum_{k=1}^{n} a_k \\
    \end{align*}
    Taking exponentials:
    \begin{equation*}
        n! = n^n e^{-n} \sqrt{n} \exp{\left(1 - \sum_{k=1}^{n} a_k\right)}
    \end{equation*}
    Then consider $a_k$:
    \begin{align*}
        a_k &= \frac{1}{2}\int_0^1 \frac{x(1-x)}{(x-k)^2} dx \\
        &\leq \frac{1}{2} \int_0^1 \frac{x(1-x)}{k^2} dx \\
        &= \frac{1}{12k^2}
    \end{align*}
    So $\sum_{k=1}^{\infty} a_k$ is finite.
    \begin{equation*}
        \text{Let } A = \exp{\left(1 - \sum_{k=1}^{\infty} a_k\right)}
    \end{equation*}
    Then $A > 0$.
    \begin{equation*}
        n! = n^n e^{-n} \sqrt{n} A \exp{\left(\sum_{k=1}^{\infty} a_k\right)}
    \end{equation*}
    Then, since the term in the exponential goes to $0$ in the limit, $n! \sim n^n e^{-n} \sqrt{n} A$.
    \par
    Now we need to evaluate $A$.\par
    Consider the quantity$B$:
    \begin{equation*}
        B = 2^{-2n} \choose{2n}{n}
    \end{equation*}
    Then we evaluate this in two different ways (in terms of $A$ and using integration) to find the value of $n$.
    \begin{align*}
        B &= 2^{-2n} \frac{(2n)!}{(n!)(n!)} \\
        &\sim \frac{2^{-2n} (2n)^{2n} \sqrt{2n} e^{-2n} A}{(n^n e^{-n} \sqrt{n} A)^2} \text{ by replacing factorials.} \\
        &= \frac{\sqrt{2}}{A \sqrt{n}} \text{ by much cancellation.}
    \end{align*}
    Then define, for any $n \in \N$,
    \begin{equation*}
        I_n \int_0^{\frac{\pi}{2}} (\cos{\theta})^n d\theta
    \end{equation*}
    Note that $I_0 = \frac{\pi}{2}$ and $I_1 = 1$.\par
    Also, by integration by parts we have a reduction formula:
    \begin{equation*}
        I_n = \frac{n-1}{n} I_{n-2}
    \end{equation*}
    \begin{align*}
        I_{2n} &= \frac{2n - 1}{2n} I_{2n-2} \\
        &= \frac{(2n-1)(2n-3)}{2n(2n-2)} I_{2n-4} \\
        &= \frac{(2n-1)(2n-3) \cdots(3)(1)}{2n(2n-2) \cdots (4)(2)} I_0 \\
        &= \frac{(2n)!}{2^{2n}(n!)^2} \frac{\pi}{2} \\
        &= 2^{-2n} \choose{2n}{n} \frac{\pi}{2}
    \end{align*}
    Simiarly,
    \begin{align*}
        I_{2n+1} &= \frac{1}{2n+1}\left(2^{-2n} \choose{2n}{n}\right)^{-1} I_1 \text{ by the same method} \\
        &= \frac{1}{2n+1}\left(2^{-2n} \choose{2n}{n}\right)^{-1}
    \end{align*}
    Now from the reduction formula, $\frac{I_{2n}}{I_{2n-2}} \to 1$ as $n \to \infty$.\par
    Also, since $\cos$ is positive on the interval being integrated, $I_n$ is positive and must therefore be decreasing.\par
    Therefore, $\frac{I_{2n}}{I_{2n+1}} \to 1$ by sandwiching between the odd sequence and even sequence.\par
    And this ratio can be evaluated:
    \begin{equation*}
        \frac{I_{2n}}{I_{2n+1}} = \left(2^{-2n} \choose{2n}{n}\right)^2 \frac{\pi}{2} (2n+1)
    \end{equation*}
    And so as $n \to \infty$:
    \begin{align*}
        \left(2^{-2n} \choose{2n}{n}\right)^2 &\sim \frac{2}{\pi(2n+1)} \\
        &\sim \frac{1}{\pi n}
    \end{align*}
    And this is $B^2$. Now we can equate (as $n \to \infty$):
    \begin{align*}
        \frac{\sqrt{2}}{A \sqrt{n}} &\sim \frac{1}{\sqrt{\pi n}} \\
        \frac{\sqrt{2}}{A} &\sim \frac{1}{\sqrt{\pi}} \\
        A &\sim \sqrt{2\pi}
    \end{align*}
    And since this is no longer in terms of $n$, $A = \sqrt{2\pi}$
\end{proof}
\section{Independence of Events}
\subsection{Definition of Independence}
\begin{definition}{Independence of events}
    Let $A, B \in \sigalg$. They are \underline{independent} if:
    \begin{equation*}
        P(A \cap B) = P(A) \times P(B)
    \end{equation*}
    In this case, we denote independence by $A \perp B$.
\end{definition}
Let $A_i$ be a countable collection of events. They are independent if for all distinct indices $i_1, \cdots, i_k$, and for all $k \leq n$:
\begin{align*}
    P(A_{i_1} \cap \cdots \cap A_{i_k}) = \prod_{j=1}^{k} P(A_{i_j})
\end{align*}
\begin{warning}
    Pairwise independence does not imply independence. Independence is a stronger statement, as any possible sub-collection of $A_i$ must be independent.
\end{warning}
\begin{example}[Counter-example of pairwise independence]
    Consider tossing a fair coin twice. Denote heads with 1 and tails with 0.
    \begin{equation*}
        \Omega = \{(0, 0), (0, 1), (1, 0), (1, 1)\}
    \end{equation*}
    And for any $\omega \in \Omega, P(\omega) = \frac{1}{4}$.\par
    Let $A, B$ and $C$ be pairwise independent. Define:
    \begin{align*}
        A &= \{(0, 0), (0, 1)\} \\
        B &= \{(0, 0), (1, 0)\} \\
        C &= \{(0, 1), (1, 0)\}
    \end{align*}
    Then the probability of all of these events is $\frac{1}{2}$. Also, these events all have exactly one element in common (pairwise), so the probability of any pairwise intersection is $\frac{1}{4}$, so the elements are pairwise independent. However,
    \begin{align*}
        P(A \cap B \cap C) &= P(\emptyset) = 0 \\
        P(A)P(B)P(C) &= \frac{1}{8} \neq 0
    \end{align*}
\end{example}
\begin{proposition}
    If $A$ is independent of $B$, then $A \perp B^C$.
    \label{propIndependentComplement}
\end{proposition}
\begin{proof}
    \begin{align*}
        P(A \cap B^C) &= P(A) - P(A \cap B) \\
        &= P(A) - P(A)P(B) \text{ by independence} \\
        &= P(A) (1 - P(B)) \\
        &= P(A) P(B^C)
    \end{align*}
\end{proof}
\subsection{Conditional Probability}
We want to define the probability of an event $A$ \textit{given} another event $B$ occurs.\par
We require that, if $A$ and $B$ are independent, the probability of $A$ given $B$ should be $P(A)$. Thus we define:
\begin{definition}{Conditional probability}
    Let $A$ and $B$ be events, and $P(B) > 0$. The \underline{probability of $A$ given $B$} is:
    \begin{equation*}
        P(A | B) = \frac{P(A \cap B)}{P(B)}
    \end{equation*}
\end{definition}
\begin{remark}
    As required, if $A \perp B$ then $P(A | B) = P(A)$.
\end{remark}
\begin{proposition}
    Supose that $(A_n)$ are disjoint events in $\sigalg$, and let $B$ be an event such that $P(B) > 0$.
    \begin{equation*}
        P\left(\bigcup_{n \in \N} A_n | B\right) = \sum_{n \in \N} P(A_n | B)
    \end{equation*}
    \label{propConditionalDisjointProb}
\end{proposition}
\begin{proof}
    \begin{align*}
        P\left(\bigcup_{n \in \N} A_n | B\right) &= \frac{P\left(\bigcup_{n \in \N} A_n \cup B\right)}{P(B)} \\
        &= \frac{P\left(\bigcup_{n \in \N} (A_n | B)\right)}{P(B)} \\
        &= \sum_{n \in \N} \frac{P(A_n \cap B)}{P(B)} \\
        &= \sum_{n \in \N} P(A_n | B)
    \end{align*}
\end{proof}
\begin{theorem}[Law of Total Probability]
    Suppose $(B_n)_{n \in \N}$ is a disjoint collection of events whose union is $\Omega$ and that $P(B_n) > 0$.\par
    Let $A \in \sigalg$. Then:
    \begin{equation*}
        P(A) = \sum_{n \in \N} P(A | B_n) P(B_n)
    \end{equation*}
    \label{thmTotalProbability}
\end{theorem}
\begin{proof}
    \begin{align*}
        A &= A \cap \Omega \\
        &= A \cap \left(\bigcup_{n \in \N}B_n\right)
    \end{align*}
    And then:
    \begin{align*}
        P(A) &= P\left(\bigcup_{n \in \N} (A \cap B_n) \right) \\
        &= \sum_{n \in \N}P(A \cap B_n) \\
        &= \sum_{n \in \N} P(A | B_n) P(B_n)
    \end{align*}
\end{proof}
\subsection{Bayes' Formula}
\begin{theorem}[Bayes' Formula]
    Let $(B_n)_{n \in \N}$ be a collection of disjoint subsets, and let their union be $\Omega$.
    \begin{equation*}
        P(B_n | A) = \frac{P(A | B_n) P(B_n)}{\sum_{k \in \N} P(A | B_k) P(B_k)}
    \end{equation*}
    \label{thmBayesFormula}
\end{theorem}
\begin{proof}
    \begin{align*}
        P(B_n | A) &= \frac{P(B_n) | A}{P(A)} \\
        &= \frac{P(A | B_n)P(B_n)}{\sum_{k \in \N} P(A | B_k) P(B_k)}
    \end{align*}
    by theorem~\ref{thmTotalProbability}
\end{proof}
\begin{example}[False positives]
    Suppose that a condition $A$ affects $0.1 \%$ of the population, and a medical test that gives a positive test result for $98\%$ of those affected with the condition, and a positive test result for $1\%$ of those who do not have the condition (false positives).\par
    Then we want to know the probability that a person suffers from $A$ given that they test positive.\par
    Let $A$ be the event that the individual suffers from the disease, let $T$ be the event that the individual tested positive.\par
    Then we are given: $P(A) = 0.001, P(T | A) = 0.98, P(T | A^C) = 0.001$.\par
    We want to find $P(A | T)$.\par
    By theorem~\ref{thmBayesFormula}:
    \begin{align*}
        P(A | T) &= \frac{P(T | A)P(A)}{P(T | A)P(A) + P(T | A^C)(P(A^C))} \\
        &= 0.09
    \end{align*}
    This is $9\%$, much lower than the intuition given that the test is fairly accurate. The reason is $P(T | A^C)$ ($1\%$) is much higher than $P(A)$ ($0.1\%$).
\end{example}
Bayes' Formula also tells us how to include extra knowledge:
\begin{example}
        Suppose there are 2 children. Assume there is equal likelihood for all outcomes (boy or girl and birthday).
        \begin{enumerate}
        \item First, suppose there is at least 1 boy. What then is the probability both children are boys?\par
        Define the events:
        \begin{align*}
            BB &= \{\text{both children are boys}\} \\
            BG &= \{\text{elder is a boy, younger is a girl}\} \\
            GB &= \{\text{elder is a girl, younger is a boy}\} \\
        \end{align*}
        We then want $P(BB | BB \cup BG \cup GB)$:
        \begin{align*}
            \frac{P(BB)}{P(BB \cup BG \cup GB)} &= \frac{P(BB)}{P(BB) + P(BG) + P(GB)} \\
            &= \frac{1/4}{1/4 + 1/4 + 1/4} = \frac{1}{3}
        \end{align*}
        \item Instead, suppose that the elder child is a boy.
        Then $P(BB | BB \cup BG) = \frac{1}{2}$, which makes sense: this is the probability that the younger is a boy.
        \item Suppose now that there is a boy born on a thirsday. Define:
            \begin{align*}
                TT &= \{\text{Both boys born on Thursday}\} \\
                TN &= \{\text{Elder is boy on Thursday, younger is boy not on Thursday}\} \\
                NT &= \{\text{Elder is boy not on Thursday, younger is boy on Thursday}\} \\
                GT &= \{\text{Elder is girl, younger is boy on Thursday}\} \\
            \end{align*}
            and define the events $TG, GN$ and $NG$ similarly.\par
            Then the probability of 2 boys given the informaion one is born on a thursday is:
            \begin{align*}
                &P(TT \cup TN \cup NT | TT \cup TN \cup NT \cup GT \cup TG) \\
                &= \frac{P(TT \cup TN \cup NT)}{P(TT \cup TN \cup NT \cup GT \cup TG)}
            \end{align*}
            And working each probability out, this is $\frac{13}{27}$.
    \end{enumerate}
\end{example}
\subsection{Simpson's Paradox: Extended Example}
Suppose all 100 Cambridge applicants to a course come from either the North or the South. Suppose also that they were educated at either a state school or an independent school.\par
Then consider the date in figure~\ref{figApplicationData}.\par
\begin{figure}[ht]
    \centering
    \begin{tabular}{c|c|c|c}
        All Applicants& Admitted & Rejected & Proportion \\
        \hline
        State & 25 & 25 & 50\% \\
        Independent & 28 & 22 & 56\%
    \end{tabular}
    \par
    Then consider the data from each region:\par
    \begin{tabular}{c|c|c|c}
        The North & Admitted & Rejected & Proportion \\
        \hline
        State & 15 & 22 & 41\% \\
        Independent & 5 & 8 & 38\% \\
        \hline
    \end{tabular}
    \begin{tabular}{c|c|c|c}
        The South & Admitted & Rejected & Proportion \\
        \hline
        State & 10 & 3 & 77\% \\
        Independent & 23 & 14 & 62\%
    \end{tabular}
    \caption{Sample application data split by region and schooling}
    \label{figApplicationData}
\end{figure}
We have that the acceptance rate from independent schools is lower in both North and South, but overall is higher than in state schools.\par
This phenomenon is called \underline{confounding statistics}. It arises when combining data from disparate populations. Let:
\begin{align*}
    A &= \{\text{Individual is admitted}\} \\
    B &= \{\text{Individual is from North}\} \\
    B^C &= \{\text{Individual is from South}\} \\
    C &= \{\text{Individual is from State school}\} \\
    C^C &= \{\text{Individual is from Independent school}\}
\end{align*}
Now, from the second table we have $P(A | B \cap C) > P(A | B \cap C^C)$,\\
and from the third table we have $P(A | B^C \cap C) > P(A | B^C \cap C^C)$\\
However, from the first table we have $P(A | C) > P(A | C^C)$.\par
For this to occur,
\begin{align*}
    P(A | C) &= P(A \cap B | C) + P(A \cap B^C | C) \\
    &= \frac{P(A \cap B \cap C)}{P(C)} + \frac{P(A \cap B^C \cap C)}{P(C)} \\
    &= \frac{P(A | B \cap C)P(B \cap C)}{P(C)} + \frac{P(A | B^C \cap C)P(B^C \cap C)}{P(C)} \\
    &= P(A | B \cap C)P(B | C) + P(A | B^C \cap C)P(B^C | C) \\
    &> P(A | B \cap C^C)P(B | C) + P(A | B^C \cap C^C) P(B^C | C)
\end{align*} 
Then if $P(B|C) = P(B | C^C)$:
\begin{align*}
    P(A | C) &> P(A | B \cap C^C) P(B | C^C) + P(A | B^C \cap C^C) P(B^C | C^C) \\
    &= P(A | C^C)
\end{align*}
Which is not true.\par %TODO: go over, annotate, give intuition.
This pseudo-paradox is known as Simpson's Paradox. It comes from the fact that the inequalities:
\begin{equation*}
    \frac{a_1}{a_2} > \frac{b_1}{b_2} \text{ and } \frac{c_1}{c_2} > \frac{d_1}{d_2}
\end{equation*}
do not always imply that:
\begin{equation*}
    \frac{a_1 + c_1}{a_2 + c_2} > \frac{b_1 + d_1}{b_2 + d_2}
\end{equation*}
\end{document}