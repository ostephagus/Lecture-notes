\documentclass[../Main.tex]{subfiles}

% Redef for \mathbb{P}, \mathcal{F}

\begin{document}
\section{Fundamental Definitions}
\begin{definition}{$\sigma$-algebra}
    Suppose $\Omega$ is a set and $\mathcal{F}$ a collection of subsets.\par
    We call $\mathcal{F}$ a \underline{$\sigma$-algebra} if:
    \begin{enumerate}
        \item $\Omega \in \mathcal{F}$
        \item If $A \in \mathcal{F}$, then $A^c \in \mathcal{F}$ % What is a complement?
        \item For any countable collection $(A_n)_{n \in \N}$ of sets is $\mathcal{F}$, if all $A_n$ are in $\mathcal{F}$ then the union must be in $\mathcal{F}$.
    \end{enumerate}
\end{definition}
\begin{definition}{Probability measure}
    A function $\mathbb{P} : \mathcal{F} \mapsto [0, 1]$ is called a \underline{probability measure} if:
    \begin{enumerate}
        \item $\mathbb{P}(\Omega) = 1$
        \item For any countable collection of disjoint sets $(A_n)_{n \in \N}$ in $\mathcal{F}$ then the probability of the union is:
            \begin{equation*}
                \mathbb{P}(\bigcup_{n \in \N} A_n) = \sum_{n \in \N} \mathbb{P}(A_n)
            \end{equation*}
    \end{enumerate}
    And we say $\mathbb{P}(A)$ is the probability of $A$.
\end{definition}
\begin{definition}{Probability space}
    Then the triplet $(\Omega, \mathcal{F}, \mathbb{P})$ is a \underline{probability space}.\par
    The elements of $\Omega$ are \underline{outcomes} and the elements of $\mathcal{F}$ are called \underline{events}.
\end{definition}
Important properties:
\begin{enumerate}
    \item $\mathbb{P}(A^c) = 1 -\mathbb{P}(A)$ for any $A \in \mathcal{F}$
    \item $\mathbb{P}(\emptyset) = 0$
    \item If $A \subseteq B$ then $\mathbb{P}(A) \leq \mathbb{P}(B)$
    \item $\mathbb{P}(A \cup B) = \mathbb{P}(A) + \mathbb{P}(B) - \mathbb{P}(A \cap B)$.
\end{enumerate}
\section{Examples of Probaility Spaces}
\begin{example}[Rolling a fair die]
    $\Omega = \{1, 2, \cdots, 6\}, \mathcal{F} = \mathbb{P}(\Omega)$\par % This is power set - how to distinguish?
        Then $\mathbb{P}(\{\omega\}) = \frac{1}{6}$, and so $\mathbb{P}(A) = \frac{|A|}{6}$.
\end{example}
\begin{remark}
    When $\Omega$ is countable we take $\mathcal{F}$ to be all the subsets of $\Omega$.
\end{remark}
\begin{example}[Uniform distribution]
    Let $\Omega$ be a finite set $\Omega = \{\omega_1, \cdots, \omega_n\}$.\par
    Then let $\mathcal{F}$ be all subsets.\par
    Let $\mathbb{P}(A) = \frac{|A|}{|\Omega|}$.
\end{example}
\begin{example}[Picking balls from a bag]
    Suppose we have $n$ balls labelled $1$ to $n$ in a bag indistinguishable when in the bag. Pick $k$ balls at random, without replacement.\par
    Then $\Omega = \subsetselect{A \in \{1, \cdots, n\}}{|A| = k}$.\par
    Then $|Omega| = \begin{pmatrix}n \\ k\end{pmatrix}$.\par
    $\mathbb{P}(\omega) = \frac{1}{nCk}$.
\end{example}
\begin{example}[Deck of cards]
    Consider a well-shuffled deck of cards.\par
    Then $\Omega = \{\text{all permutations of 52 elements}\}$. It has size $52!$.\par
    Then consider $\mathbb{P}(\text{two top cards are aces}) = \frac{4 \times 3 \times 50!}{52!}$.
\end{example}
\begin{example}{Largest digit}
    Consider a string of n base-10 digits chosen randomly.\par
    $\Omega = \{0, \cdots, 9\}^n$, $|\Omega| = 10^n$.\par
    Then let $A_k=\{\text{no digit exceeds } k\}$.\par
    \begin{equation*}
        \mathbb{P}(A_k) = \frac{|A_k|}{|\Omega|} = \frac{(k + 1)^n}{10^n}
    \end{equation*}
    Also consider $B_k = \{\text{largest digit is } k\}$. Note that $B_k = A_k \backslash A_{k - 1}$.
\end{example}
\begin{example}[Birthday problem]
    Consider $n$ people and assume each person is equally likely to have been born on each day (excluding 29th February). We calculate the probability that two people share the same birthday.\par
    $\Omega = \{1, \cdots, 365\}^n$, $\mathcal{F}$ is all subsets.\par
    Then the probability of any outcome $\omega$ is $\frac{1}{365^n}$.\par
    Let $A = \{\text{at least two people share a birthday}\}$. It is easier to consider $A^c$: all birthdays are distinct.\par
    \begin{equation*}
        \mathbb{P}(A^c) = \frac{|A^c|}{|\Omega|} = \frac{365 \times 364 \times \cdots \times (365-n + 1)}{365^n}
    \end{equation*}
    And $\mathbb{P}(A) = 1 - \mathbb{P}(A^c)$.
\end{example}
\end{document}