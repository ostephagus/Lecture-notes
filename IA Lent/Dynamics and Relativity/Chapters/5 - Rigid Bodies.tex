\documentclass[../Main.tex]{subfiles}

\begin{document}
A rigid body problem is a type of $N$-body problem that is tractable, since the distance between particles is fixed (due to very strong molecular forces). The only motions a rigid body can undergo are translations of the centre of mass and rotations.
\section{Rotational Motion}
\subsection{Angular Velocity}
In 3 dimensions, rotations are described by a vector:
\begin{definition}{Angular velocity}
    The \underline{angular velocity} is defined by:
    \begin{equation*}
        \vec{\omega} = \omega \uvec{n}
    \end{equation*}
    such that $\omega = \dot{\theta}$, the angular speed of rotation, and $\uvec{n}$ is the axis of rotation.
\end{definition}
The \textit{right-hand rule} is used: $\theta$ increases looking down the axis from positive to negative.\par
The key equation is:
\begin{equation}
    \dvec{x} = \vec{\omega} \times \vec{x}
    \label{eqnRotationalMotion}
\end{equation}
\begin{figure}[ht]
    \centering
    \begin{tikzpicture}[scale=2]
        \tdplotsetmaincoords{0}{0}
        \tdplotsetrotatedcoords{90}{40}{0}
        \draw[->] (0, 0, 0) -- (0.9, 0.5, 0) node[below right] {$\vec{x}$};
        \draw[fill] (0.9, 0.5, 0) circle[radius=0.2mm];
        \draw[->] (0, 0, 0) -- (0, 1, 0) node[above] {$\vec{\omega}$};
        \draw[dashed] (0, -0.5, 0) -- (0, 0, 0);

        \tdplotdrawarc{(0, 0, 0)}{0.4}{29}{90}{}{}
        \node[anchor=center] at (0.1, 0.2) {\small$\phi$};

        \draw[dashed] (0, 0.5, 0) -- (0.9, 0.5, 0) node[pos=0.5, anchor=south] {$d$};
        \draw (0.75, 0.5, 0) -- (0.75, 0.5, -0.15) -- (0.9, 0.5, -0.15);

        \draw[->] (0.9, 0.5, 0) -- (0.9, 0.5, -1) node[right] {$\dvec{x}$};

        \tdplotdrawarc[tdplot_rotated_coords,->]{(-0.1, 0, 0)}{0.1}{30}{300}{left}{$\theta$}
    \end{tikzpicture}
    \caption{Diagram of vectors involved in rotational motion}
    \label{figRotationalMotion}
\end{figure}
Consider equation~\ref{eqnRotationalMotion} and figure~\ref{figRotationalMotion}. Here we see that $\dvec{x}$ is orthogonal to both $\vec{\omega}$ and $\vec{x}$, and $|\dvec{x}| = \omega |\vec{x}| \sin{\phi} = \omega d$. Therefore, $d = |\uvec{n}| |\vec{x}|$.\par
In addition to the angular velocity, a rotation must specify a point through which the axis passes. The vector $\vec{x}$ is relative to this point.
\subsection{Moment of Inertia as Mass}
Moment of inertia provides an analogue to mass for rotating bodies.\par
Kinetic energy of a particle that is rotating is given by:
\begin{align*}
    T &= \frac{1}{2}m\dvec{x} \cdot \dvec{x} \\
    &= \frac{1}{2} m \left(\vec{\omega} \times \vec{x}\right) \cdot \left(\vec{\omega} \times \vec{x}\right) \\
    &= \frac{1}{2} m \omega^2 d^2
\end{align*}
In a rigid body, all particles rotate with the same angular velocity. To check this is an allowed motion, ensure particle distances stay fixed:
\begin{align*}
    \frac{d}{dt}|x_i - x_j|^2 &= 2(\vec{x_i} - \vec{x_j}) \cdot (\dvec{x_i} - \dvec{x_j}) \\
    &= 2(\vec{x_i} - \vec{x_j}) \cdot \left[\vec{\omega} \times (\vec{x_i} - \vec{x_j})\right] \\
    &= 0
\end{align*}
Therefore this motion keeps particle distances fixed and is permitted.\par
\begin{definition}{Moment of Inertia}
    The \underline{Moment of Inertia}, $I$, of a rigid body is defined by:
    \begin{equation}
        I = \sum_{i=1}^{N} m_i d_i^2
        \label{eqnMomentOfInertia}
    \end{equation}
    It depends on the axis of rotation and the body.
\end{definition}
The kinetic energy of a rotating body is then:
\begin{align}
    T &= \frac{1}{2} \sum_{i=1}^{N} m_i \omega_i^2 d_i^2 \nonumber \\
    &= \frac{1}{2}I\omega^2 \label{eqnRotationalKE}
\end{align}
We would like to consider moment of inertia as mass for rotating bodies. We see this by relating angular momentum and angular velocity to moment of inertia.\par
Angular momentum of a rigid body is:
\begin{align*}
    \vec{L} &= \sum_{i=1}^{N} m_i \vec{x_i} \times \dvec{x_i} \\
    &= \sum_{i=1}^{N} m_i \vec{x_i} \times (\vec{\omega} \times \vec{x_i})
\end{align*}
Then consider the angular momentum along the axis of rotation:
\begin{align*}
    L &= \vec{L} \cdot \uvec{n} \\
    &= \sum_{i=1}^{N} m_i \left[\vec{x_i} \times \left(\omega \vec{n} \times \vec{x_i}\right)\right] \cdot \uvec{n} \\
    &= \omega \sum_{i=1}^{N} m_i \left(\uvec{n} \times \vec{x_i}\right) \cdot \left(\uvec{n} \times \vec{x_i}\right) \\
    &= \omega \sum_{i=1}^{N} m_i d_i^2
\end{align*}
And so:
\begin{equation}
    L = \omega I
    \label{eqnAngularMass}
\end{equation}
This solidifies the understanding of $I$ as angular mass. Recall that torque, $\vec{G}$, causes change in the angular momentum. If torque is also in the direction of rotation, then $\vec{G} = G \uvec{n}$ and so:
\begin{equation}
    G = I \dot{\omega} = I \ddot{\theta}
    \label{eqnTorqueEffect}
\end{equation}
\subsection{Calculating Moment of Inertia}
To calculate the moment of inertia for a given body, we use the fact that a sum over many closely spaced points in a rigid body can be approximated by an integral:
\begin{equation*}
    \sum_{i=1}^{N} m_i f(\vec{x_i}) \approx \int \rho(\vec{x})f(\vec{x}) d^3x
\end{equation*}
Where $\rho(\vec{x})$ is the scalar field that represents density of the body at a point.\par
So then we can define mass:
\begin{equation}
    M = \int \rho(\vec{x}) d^3x
    \label{eqnIntegralMass}
\end{equation}
and moment of inertia:
\begin{equation}
    I = \int \rho(\vec{x}) x_\perp^2 d^3x
    \label{eqnIntegralMomentInertia}
\end{equation}
Where $x_\perp$ is $d_i$, the perpendicular distance of $\vec{x}$ to the axis of rotation.
\begin{examples}{
        Consider different rigid bodies with constant density $\rho$.
    }
    \item Consider a ring with radius $a$. Then its mass is $2\pi \rho a$, and the moment of inertia is $I = 2 \pi \rho a^3$. Alternatively, $I = Ma^2$.
    \item Consider a rod with length $l$ rotating around one of its ends. Then the mass is $M = l \rho$, and the moment of inertia is:
        \begin{equation*}
            I = \rho \int_0^l x^2 dx = \rho \frac{l^3}{3}
        \end{equation*}
        Or, in terms of mass, this is $M\frac{l^2}{3}$.
    \item Consider a disc (filled in ring) with radius $a$. Its mass is then $M = \pi a^2 \rho$, and then its moment of inertia is:
        \begin{align*}
            I &= 2 \pi \rho \int_0^a r \times r^2 dr \\
            &= \frac{pi}{2} \rho a^4
        \end{align*}
        Which is $\frac{1}{2} M a^2$.
    \item Consider a solid sphere with radius $a$. Then its mass is $\frac{4}{3} \pi a^3 \rho$, and its moment of inertia is:
        \begin{align*}
            I &= 2 \pi \rho \int_{r=0}^a r^2 \int_{\theta = 0}^\pi \sin{\theta} \times r^2 \sin^2{\theta} dr d\theta \\
            &= 2 \pi \rho \int_{\theta = 0}^\pi \sin^3{\theta} d\theta \int_{r=0}^a r^4 dr \\
            &= \frac{2}{5} Ma^2
        \end{align*}
\end{examples}
\subsection{Theorems for Moment of Inertia}
\begin{lemma}[Perpendicular axis theorem]
    Consider a 2-dimensional body (a lamina). Let any point have coordinates $(x, y)$. Then the moments of inertia about the $x$ and $y$ axes are:
    \begin{align*}
        I_x &= \int \rho y^2 dx dy \\
        I_y &= \int \rho x^2 dx dy
    \end{align*}
    Then the moment of inertia through the $z$ axis is:
    \begin{equation}
        I_z = I_x + I_y
        \label{eqnPerpendicularAxis}
    \end{equation}
\end{lemma}
\begin{lemma}[Parallel axis theorem]
    Consider a generic 3-dimensional body. Let the moment of inertia about the centre of mass be $I_\text{COM}$. Consider a second, parallel axis with perpendicular distance $h$ from the line through the centre of mass. Then the moment of inertia through this new axis, $I_p$, is:
    \begin{equation}
        I_p = I_\text{COM} + Mh^2
    \end{equation}
\end{lemma}
\begin{remark}
    The moment of inertia in a given direction is minimised when it passes through the centre of mass, since the term $Mh^2$ is always positive.
\end{remark}
\section{Motion of Rigid Bodies}
As previously discussed, we will allow 2 motions: rotation and translation of the centre of mass. We describe this:
\begin{equation}
    \vec{x_i} = \vec{R}(t) + \vec{y_i}
    \label{eqnRigidBodyMotion}
\end{equation}
We consider the kinetic energy:
\begin{equation*}
    T = \frac{1}{2} M \dvec{R}^2 + \frac{1}{2} \sum_i m_i \dvec{y_i}^2
\end{equation*}
And by using equation~\ref{eqnRotationalKE},
\begin{equation}
    T = \frac{1}{2} M \dvec{R}^2 + \frac{1}{2} I_\text{COM}\omega^2
    \label{eqnRigidBodyKE}
\end{equation}
Recall also that external forces act as though the rigid body were a point particle at the centre of mass. Thus, if the force is conservative:
\begin{equation}
    E = T + V(\vec{R})
    \label{eqnTotalEnergyConservForce}
\end{equation}
This suggests that it is often easier to consider axes through the centre of mass. Note, however, that this is not always the case:
\begin{example}[Rigid pendulum]
    Consider a rigid pendulum of mass $M$, length $L$. Let $\theta(t)$ be the angle that the pendulum makes with the vertical.\par
    In the first method, consider the kinetic energy in relation to the fixed endpoint:
    \begin{equation*}
        T = \frac{1}{2} I\dot{\theta}^2
    \end{equation*}
    Note that the angular velocity is $\dot{\theta}$, and moment of inertia is $\frac{1}{2} ML^2$ as calculated earlier.\par
    In the second method, consider the kinetic energy in relation to the centre of mass.
    \begin{align*}
        T &= \frac{1}{2} M \dvec{R}^2 + \frac{1}{2} I_\text{COM} \omega^2 \\
        &= \frac{1}{2} \left(M \left(\frac{L}{2}\right)^2 + I_\text{COM}\right) \\
        &= \frac{1}{2} I \dot{\theta}^2 \text{ by recognising parallel axis theorem.}
    \end{align*}
    Note that here it was easier to consider an axis through the fixed point because there was no translational motion there.\par
    Then the total energy is:
    \begin{align*}
        E &= \frac{1}{2} I \dot{\theta}^2 - Mg \frac{L}{2} \cos{\theta} \\
        \frac{dE}{dt} &= \frac{1}{2} I \dot{\theta} \ddot{\theta} + Mg \frac{L}{2} \sin{\theta} \dot{\theta} = 0 \\
        I \ddot{\theta} &= -Mg \frac{L}{2} \sin{\theta}
    \end{align*}
\end{example}
\begin{figure}[ht]
    \centering
    \begin{tikzpicture}[scale=2]
        \draw (0, 0) -- (2, 0);
        \draw (0, 0) -- (30:2);
        \draw (0.5, 0) arc[start angle=0, end angle=30, radius=0.5];
        \node at (0.35, 0.1) {$\alpha$};
        
        \draw (30:1.5) -- ++(120:0.5)
            node[pos=0.5, anchor=west] {$a$}
            circle[radius=0.5]
            node (C) {};

        \draw[->] (28:1.7) -- ++(210:0.4)
            node[anchor=north, pos=0.5] {$x$};

        \draw[->] ($(C) + (120:0.6)$) arc[start angle=120, end angle=160, radius=0.6]
            node[anchor=south east, pos=0.4] {$\omega$};

        \draw[->] ($(C) + (210:0.6)$) -- ++(210:0.5)
            node[pos=0.3, anchor=south east] {$\dot{x} = v$};
    \end{tikzpicture}
    \caption{Diagram of a ball rolling down a slope}
    \label{figBallSlope}
\end{figure}
\begin{example}[Rolling ball]
    No-slip rolling occurs when the friction between the bottom of the ball and the ground is so strong that the relative velocity between the ball and the ground is 0. The ball experiences two kinds of motion: a rotational motion about the centre of mass, and a translational motion. Let the translational speed be $v$, the rotational speed be $\omega$, and the radius be $a$. If $v = a\omega$, then the point of contact always has velocity $0$, since the rotational and translational motions cancel out.\par
    In this case, the kinetic energy is:
    \begin{align*}
        T &= \frac{1}{2} M v^2 + \frac{1}{2} I\omega^2 \\
        &= \frac{1}{2} \left(\frac{I}{a^2} + M\right)v^2
    \end{align*}
    We see that rolling changes the effective mass of the ball. That is, a greater impulse is needed to start the ball moving. Furthermore, in this case the friction \textit{does} conserve energy because the friction does no work (there is no relative velocity at the point where friction acts). The only role of friction in this system is to impose the no-slip condition. Since it is infinitely strong, there can be no movement against it and it thusly does no work.\par
    Consider now figure~\ref{figBallSlope}, a ball rolling down a slope with angle $\alpha$ to the horizontal. We consider the conserved energy:
    \begin{equation*}
        E = \frac{1}{2}\left(\frac{I}{a^2} + M\right)\dot{x}^2 - Mgx \sin{\alpha}
    \end{equation*}
    Note that this is not quite the energy, there is an extra constant to do with the height of the centre of mass above the slope, but we can disregard this since it is constant for the motion.\par
    We can get the equation of motion by differentiating this energy:
    \begin{equation*}
        \left(\frac{I}{a^2} + M\right)\ddot{x} - Mg\sin{\alpha} = 0
    \end{equation*}
\end{example}
\section{Normal forces and Dry Friction}
Rigid bodies do not fall through each other. This is due to normal forces. These originate from complicated microscopic dynamics, but can be modelled as a force normal to the surface of contact between two rigid bodies.
\subsection{Dry friction}
\begin{figure}[ht]
    \centering
    \begin{tikzpicture}[scale=2]
        \draw (0, 0) -- (2, 0);
        \draw (0, 0) -- (30:2);
        \draw (0.5, 0) arc[start angle=0, end angle=30, radius=0.5];
        \node at (0.35, 0.1) {$\theta$};

        \coordinate (C) at ($(30:1.5) + (120:0.2)$);
        \draw[fill] (C) circle[radius=0.2mm];

        % Rectangle, centre C, x length 0.6, y length 0.4.
        \draw ($(C) + (300:0.2) + (30:0.3)$)
        -- ++(120:0.4) -- ++(210:0.6) -- ++(300:0.4) -- cycle;

        \draw[->] (C) -- +(30:0.6) node[right] {$F_f$};
        \draw[->] (C) -- +(120:0.6) node[above] {$F_n = -mg\cos{\theta}$};
        \draw[->] (C) -- +(210:0.6) node[left] {$mg\sin{\theta}$};
        \draw[->] (C) -- +(300:0.6) node[right] {$mg\cos{\theta}$};
        \draw[->,dotted] (C) -- +(270:0.4) node[below] {$mg$};
    \end{tikzpicture}
    \caption{Force diagram of a box on a slope}
    \label{figBoxSlopeForces}
\end{figure}
In figure~\ref{figBoxSlopeForces}, we see the normal force, $F_n$ acting perpendicular to the surface. However, \underline{dry friction} also acts on the box. This force acts parallel to the surface, and counteracts any motion in this direction. It takes the minimum value required to stop the box's motion, but the maximum value is:
\begin{equation}
    F_f \leq \mu F_N
\end{equation}
Here $\mu$ is a dimensionless constant, the coefficient of friction. It is often slightly below $1$.\par
\subsection{Bouncing under Normal Forces}
Consider a ball moving freely and bouncing against a wall. See figure~\ref{figBallBounceWall}.
\begin{figure}[ht]
    \centering
    \begin{tikzpicture}[scale=2]
        \tikzset{
            patharrow/.pic={
                \draw (-0.1, -0.1) -- (0, 0) -- (-0.1, 0.1);
            }
        }
        \begin{scope}[every node/.style={minimum size=1cm, draw, circle}]
            \node (A) at (150:1) {};            
            \node (B) at (0, 0) {};            
            \node (C) at (30:1) {};            
        \end{scope}
        \node[above=0.1cm of A] {Ball};
        
        \draw (A) -- (B)
            pic[pos=0.5, rotate=-30] {patharrow}
            (B) -- (C)
            pic[pos=0.5, rotate=30] {patharrow};
        \draw [dashed] (-0.7, 0) edge (B)
            (B) edge(0.7, 0);

        \draw (-0.5, 0) arc[start angle=180, end angle=150, radius=0.5] 
            node[pos=0.3, anchor=west] {$\alpha$};
        \draw (0.5, 0) arc[start angle=0, end angle=30, radius=0.5] 
            node[pos=0.3, anchor=east] {$\beta$};
        \draw (-1, -0.25) -- (1, -0.25)
            node[pos=0.5, anchor=north] {Wall};

        \draw[->] (0, -0.25) -- ++(0, 0.75) node[above] {$F_N$};
    \end{tikzpicture}
    \caption{Ball bouncing against a wall}
    \label{figBallBounceWall}
\end{figure}
The normal force on impact has no effect on the momentum in the $x$ direction, so the speed of the ball in the $x$ direction stays the same.\par
Consider the momentum in the $y$ direction. Let $p_y$ be the momentum of the ball before the collision. Let $q_y$ be the momentum of the ball after the collision, and $Q_y$ be the momentum of the wall after the collision.\par
Then conservation of momentum implies:
\begin{equation}
    p_y = q_y + Q_y
    \label{eqnBounceCOM}
\end{equation}
In an elastic collision, energy is conserved so:
\begin{equation}
    \frac{p_y^2}{2m} = \frac{q_y^2}{2m} + \frac{Q_y^2}{2M}
    \label{eqnBounceCOE}
\end{equation}
Where $m$ is the mass of the ball, and $M$ is the mass of the wall.\par
Substituting equation~\ref{eqnBounceCOM} into \ref{eqnBounceCOE}:
\begin{align*}
    \frac{1}{2m}\left(q_y^2 + 2q_y Q_y + Q_y^2\right) &= \frac{q_y^2}{2m} + \frac{Q_y^2}{2M} \\
    \frac{1}{2m}\left(2q_yQ_y + Q_y^2\right) &= \frac{Q_y^2}{2M}
\end{align*}
Take the limit $M >> m$:
\begin{equation*}
    \frac{1}{2m} \left(2q_yQ_y + Q_y^2\right) = 0
\end{equation*}
This then gives $Q_y = 0$ or $-2q_y$. The first solution is unphysical (the wall and ball must interact), so take $Q_y = -2q_y$. That is, $p_y = -q_y$.\par
So in the case of elastic collision, the ball bounces and has the same speed as before. Therefore, $\alpha = \beta$ (see figure~\ref{figBallBounceWall})
\end{document}