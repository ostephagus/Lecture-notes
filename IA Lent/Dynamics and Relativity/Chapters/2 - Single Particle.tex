\documentclass[../Main.tex]{subfiles}

\begin{document}
\section{Motion in One Dimension}
Often, we can reduce problems to motion in one dimension. In one dimension, Newton's Second Law is:
\begin{equation}
    m\ddot{x} = F_x
    \label{eqnNewtonSecondOneDim}
\end{equation}
If $F_x$ is velocity independent, then it can always be written as a gradient of a potential:
\begin{equation}
    V(x) = -\int_{x_0}^{x} F_x(u) du
    \label{eqnPotentialOneDim}
\end{equation}
Then we can conserve an energy $E = \frac{1}{2} m \dot{x}^2 + V(x)$. Solving this as a first-order differential equation:
\begin{equation}
    t - t_0 = \pm \int_{x_0}^x \frac{du}{\sqrt{\frac{2}{m}\left(E - V(u)\right)}}
    \label{eqnTimeGivenEnergy}
\end{equation}
Such a formula is only easy to derive for 1 dimension. In more dimensions something like this may not even be possible to find.
\end{document}