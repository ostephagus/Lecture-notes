\documentclass[../Main.tex]{subfiles}

\begin{document}
A rigid body problem is a type of $N$-body problem that is tractable, since the distance between particles is fixed (due to very strong molecular forces). The only motions a rigid body can undergo are translations of the centre of mass and rotations.
\section{Rotational Motion}
\subsection{Angular Velocity}
In 3 dimensions, rotations are described by a vector:
\begin{definition}{Angular velocity}
    The \underline{angular velocity} is defined by:
    \begin{equation*}
        \vec{\omega} = \omega \uvec{n}
    \end{equation*}
    such that $\omega = \dot{\theta}$, the angular speed of rotation, and $\uvec{n}$ is the axis of rotation.
\end{definition}
The \textit{right hand rule} is used: $\theta$ increases looking down the axis from positive to negative.\par
The key equation is:
\begin{equation}
    \dvec{x} = \vec{\omega} \times \vec{x}
    \label{eqnRotationalMotion}
\end{equation}
\begin{figure}[ht]
    \centering
    \begin{tikzpicture}[scale=2]
        \tdplotsetmaincoords{0}{0};
        \tdplotsetrotatedcoords{90}{40}{0};
        \draw[->] (0, 0, 0) -- (0.9, 0.5, 0) node[below right] {$\vec{x}$};
        \draw[fill] (0.9, 0.5, 0) circle[radius=0.2mm];
        \draw[->] (0, 0, 0) -- (0, 1, 0) node[above] {$\vec{\omega}$};
        \draw[dashed] (0, -0.5, 0) -- (0, 0, 0);

        \tdplotdrawarc{(0, 0, 0)}{0.4}{29}{90}{}{};
        \node[anchor=center] at (0.1, 0.2) {\small$\phi$};

        \draw[dashed] (0, 0.5, 0) -- (0.9, 0.5, 0) node[pos=0.5, anchor=south] {$d$};
        \draw (0.75, 0.5, 0) -- (0.75, 0.5, -0.15) -- (0.9, 0.5, -0.15);

        \draw[->] (0.9, 0.5, 0) -- (0.9, 0.5, -1) node[right] {$\dvec{x}$};

        \tdplotdrawarc[tdplot_rotated_coords,->]{(-0.1, 0, 0)}{0.1}{30}{300}{left}{$\theta$};
    \end{tikzpicture}
    \caption{Diagram of vectors involved in rotational motion}
    \label{figRotationalMotion}
\end{figure}
Consider equation~\ref{eqnRotationalMotion} and figure~\ref{figRotationalMotion}. Here we see that $\dvec{x}$ is orthogonal to both $\vec{\omega}$ and $\vec{x}$, and $|\dvec{x}| = \omega |\vec{x}| \sin{\phi} = \omega d$. Therefore, $d = |\uvec{n}| |\vec{x}|$.\par
In addition to the angular velocity, a rotation must specify a point through which the axis passes. The vector $\vec{x}$ is relative to this point.
\subsection{Moment of Inertia as Mass}
Moment of inertia provides an analogue to mass for rotating bodies.\par
Kinetic energy of a particle that is rotating is given by:
\begin{align*}
    T &= \frac{1}{2}m\dvec{x} \cdot \dvec{x} \\
    &= \frac{1}{2} m \left(\vec{\omega} \times \vec{x}\right) \cdot \left(\vec{\omega} \times \vec{x}\right) \\
    &= \frac{1}{2} m \omega^2 d^2
\end{align*}
In a rigid body, all particles rotate with the same angular velocity. To check this is an allowed motion, ensure particle distances stay fixed:
\begin{align*}
    \frac{d}{dt}|x_i - x_j|^2 &= 2(\vec{x_i} - \vec{x_j}) \cdot (\dvec{x_i} - \dvec{x_j}) \\
    &= 2(\vec{x_i} - \vec{x_j}) \cdot \left[\vec{\omega} \times (\vec{x_i} - \vec{x_j})\right] \\
    &= 0
\end{align*}
Therefore this motion keeps particle distances fixed and is permitted.\par
\begin{definition}{Moment of Inertia}
    The \underline{Moment of Inertia}, $I$, of a rigid body is defined by:
    \begin{equation}
        I = \sum_{i=1}^{N} m_i d_i^2
        \label{eqnMomentOfInertia}
    \end{equation}
    It depends on the axis of rotation and the body.
\end{definition}
The kinetic energy of a rotating body is then:
\begin{align}
    T &= \frac{1}{2} \sum_{i=1}^{N} m_i \omega_i^2 d_i^2 \nonumber \\
    &= \frac{1}{2}I\omega^2 \label{eqnRotationalKE}
\end{align}
We would like to consider moment of inertia as mass for rotating bodies. We see this by relating angular momentum and angular velocity to moment of inertia.\par
Angular momentum of a rigid body is:
\begin{align*}
    \vec{L} &= \sum_{i=1}^{N} m_i \vec{x_i} \times \dvec{x_i} \\
    &= \sum_{i=1}^{N} m_i \vec{x_i} \times (\vec{\omega} \times \vec{x_i})
\end{align*}
Then consider the angular momentum along the axis of rotation:
\begin{align*}
    L &= \vec{L} \cdot \uvec{n} \\
    &= \sum_{i=1}^{N} m_i \left[\vec{x_i} \times \left(\omega \vec{n} \times \vec{x_i}\right)\right] \cdot \uvec{n} \\
    &= \omega \sum_{i=1}^{N} m_i \left(\uvec{n} \times \vec{x_i}\right) \cdot \left(\uvec{n} \times \vec{x_i}\right) \\
    &= \omega \sum_{i=1}^{N} m_i d_i^2
\end{align*}
And so:
\begin{equation}
    L = \omega I
    \label{eqnAngularMass}
\end{equation}
This solidifies the understanding of $I$ as angular mass. Recall that torque, $\vec{G}$, causes change in the angular momentum. If torque is also in the direction of rotation, then $\vec{G} = G \uvec{n}$ and so:
\begin{equation}
    G = I \dot{\omega} = I \ddot{\theta}
    \label{eqnTorqueEffect}
\end{equation}
\subsection{Calculating Moment of Inertia}
To calculate the moment of inertia for a given body, we use the fact that a sum over many closely spaced points in a rigid body can be approximated by an integral:
\begin{equation*}
    \sum_{i=1}^{N} m_i f(\vec{x_i}) \approx \int \rho(\vec{x})f(\vec{x}) d^3x
\end{equation*}
Where $\rho(\vec{x})$ is the scalar field that represents density of the body at a point.\par
So then we can define mass:
\begin{equation}
    M = \int \rho(\vec{x}) d^3x
    \label{eqnIntegralMass}
\end{equation}
and moment of inertia:
\begin{equation}
    I = \int \rho(\vec{x}) x_\perp^2 d^3x
    \label{eqnIntegralMomentInertia}
\end{equation}
Where $x_\perp$ is $d_i$, the perpendicular distance of $\vec{x}$ to the axis of rotation. %TODO: Check
\end{document}