\documentclass[../Main.tex]{subfiles}
\begin{document}
\section{Definitions}
Three dimensional space can be endowed with a Cartesian reference frame (origin and axes) such that points in space can be labelled:
\begin{equation*}
    \vec{x} = \begin{pmatrix}x_1 \\ x_2 \\ x_3\end{pmatrix}
\end{equation*}
Time can be labelled with respect to an arbitrary reference time by a real number $t$.\par
\begin{definition}{Point particle}
    A \underline{point particle} is an idealised object that is completely determined by its position at a given time $\vec{x}(t)$.
\end{definition}
We can model various real-world objects: we can model an electron, tennis ball, or planet (given appropriate scale).
\begin{definition}{Velocity}
    The \underline{velocity} is the vector:
    \begin{equation*}
        \vec{v} \equiv \frac{d\vec{x}}{dt} = \dot{\vec{x}}
    \end{equation*}
    The velocity is tangent to the trajectory.
\end{definition}
In Cartesian coordinates, we differentiate each component with respect to time.
\begin{definition}{Acceleration}
    The \underline{acceleration} is $\vec{a} = \dot{\vec{v}} = \ddot{\vec{x}}$. The second derivative of the position vector.
\end{definition}
Consider a free particle that does not experience any forces. The position of this particle is $\vec{x}(t)$. This particle may be at rest in a frame $S$ (a specific choice of origin and axes), but moving in a complicated way with respect to another frame.
\subsection{Law of Inertia}
\begin{definition}{Inertial reference frame}
    A frame $S$ is \underline{inertial} if a free particle has a constant velocity. For this particle, $\ddot{\vec{x}} = \vec{0}$.
\end{definition}
The law of inertia states that if a particle is not acted on by any forces, there exists a reference frame in which it moves at constant velocity.
\subsection{Galilean Relativity Principle}
\begin{definition}{Galilean transformation}
    A \underline{Galilean transformation} transforms coordinate vectors $\vec{x}$:
    \begin{equation}
        \vec{x}' = R\vec{x} + \vec{k} + \vec{w}t
        \label{eqnGalileanTransform}
    \end{equation}
    Where $R$ is an orthogonal matrix, and $\vec{k}$ and $\vec{w}$ are constant vectors. $\vec{w}$ defines a constant velocity, and is known as a boost.
\end{definition}
Then the Galilean Relativity Principle states that a frame related to an inertial reference frame by a Galilean transformation is also an inertial reference frame. All laws of physics are the same in both frames.\par
The Galilean transformations form the Galilean group, often supplemented by time translations $t' = t + t_0$.\par
Galilean relativity implies that the laws of physics make reference to no special point, direction, time or velocity. If these are made reference to, they must all be relative.\par
However, acceleration is not relative: if a body accelerates in one inertial reference frame, it does so in all others with the same magnitude (but directions may be different).
\section{Forces}
Interactions between particles are described by forces.
\subsection{Newton's Second Law}
\begin{definition}{Momentum}
    Momentum is equal to the product of mass and velocity:
    \begin{equation*}
        \vec{p} = m \dot{\vec{x}}
    \end{equation*}
    Where $m$ is the inertial mass.
\end{definition}
Mass is an additional property of point particles. It may be a function of time, though we will consider constant mass unless otherwise stated.\par
Then Newton's Second Law states that, in an inertial frame,
\begin{equation}
    \dot{\vec{p}} = \vec{F}.
    \label{eqnNewtonII}
\end{equation}
That is, the change in momentum is equal to the force.\par
The force $\vec{F}$ depends on the interactions but can only depend on $\vec{x}$ and $\dot{\vec{x}}$. This means that Newton's Second Law is a second-order differential equation for the vector $\vec{x}(t)$. This means, given the position and velocity at some time for all particles in a system, $\vec{x}(t)$ is uniquely determined for all times.
\end{document}