\documentclass[../Main.tex]{subfiles}
\begin{document}
\section{Reference Frames}
\subsection{Key Definitions}
Three dimensional space can be endowed with a Cartesian reference frame (origin and axes) such that points in space can be labelled:
\begin{equation*}
    \vec{x} = \begin{pmatrix}x_1 \\ x_2 \\ x_3\end{pmatrix}
\end{equation*}
Time can be labelled with respect to an arbitrary reference time by a real number $t$.\par
\begin{definition}{Point particle}
    A \underline{point particle} is an idealised object that is completely determined by its position at a given time $\vec{x}(t)$.
\end{definition}
We can model various real-world objects: we can model an electron, tennis ball, or planet (given appropriate scale).
\begin{definition}{Velocity}
    The \underline{velocity} is the vector:
    \begin{equation*}
        \vec{v} \equiv \frac{d\vec{x}}{dt} = \dot{\vec{x}}
    \end{equation*}
    The velocity is tangent to the trajectory.
\end{definition}
In Cartesian coordinates, we differentiate each component with respect to time.
\begin{definition}{Acceleration}
    The \underline{acceleration} is $\vec{a} = \dot{\vec{v}} = \ddot{\vec{x}}$. The second derivative of the position vector.
\end{definition}
Consider a free particle that does not experience any forces. The position of this particle is $\vec{x}(t)$. This particle may be at rest in a frame $S$ (a specific choice of origin and axes), but moving in a complicated way with respect to another frame.
\subsection{Law of Inertia}
\begin{definition}{Inertial reference frame}
    A frame $S$ is \underline{inertial} if a free particle has a constant velocity. For this particle, $\ddot{\vec{x}} = \vec{0}$.
\end{definition}
The law of inertia states that if a particle is not acted on by any forces, there exists a reference frame in which it moves at constant velocity.
\subsection{Galilean Relativity Principle}
\begin{definition}{Galilean transformation}
    A \underline{Galilean transformation} transforms coordinate vectors $\vec{x}$:
    \begin{equation}
        \vec{x}' = R\vec{x} + \vec{k} + \vec{w}t
        \label{eqnGalileanTransform}
    \end{equation}
    Where $R$ is an orthogonal matrix, and $\vec{k}$ and $\vec{w}$ are constant vectors. $\vec{w}$ defines a constant velocity, and is known as a boost.
\end{definition}
Then the Galilean Relativity Principle states that a frame related to an inertial reference frame by a Galilean transformation is also an inertial reference frame. All laws of physics are the same in both frames.\par
The Galilean transformations form the Galilean group, often supplemented by time translations $t' = t + t_0$.\par
Galilean relativity implies that the laws of physics make reference to no special point, direction, time or velocity. If these are made reference to, they must all be relative.\par
However, acceleration is not relative: if a body accelerates in one inertial reference frame, it does so in all others with the same magnitude (but directions may be different).
\section{Forces}
Interactions between particles are described by forces.
\subsection{Newton's Second Law}
\begin{definition}{Momentum}
    Momentum is equal to the product of mass and velocity:
    \begin{equation*}
        \vec{p} = m \dot{\vec{x}}
    \end{equation*}
    Where $m$ is the inertial mass.
\end{definition}
Mass is an additional property of point particles. It may be a function of time, though we will consider constant mass unless otherwise stated.\par
Then Newton's Second Law states that, in an inertial frame,
\begin{equation}
    \dot{\vec{p}} = \vec{F}.
    \label{eqnNewtonII}
\end{equation}
That is, the change in momentum is equal to the force.\par
The force $\vec{F}$ depends on the interactions but can only depend on $\vec{x}$ and $\dot{\vec{x}}$. This means that Newton's Second Law is a second-order differential equation for the vector $\vec{x}(t)$. So, given the position and velocity at some time for all particles in a system, $\vec{x}(t)$ is uniquely determined for all times.
\subsection{Conservative Forces and Gravity}
\begin{definition}{Conservative force}
    A force is \underline{conservative} if it can be written in the form:
    \begin{equation}
        \vec{F} = -\nabla V
        \label{eqnConservativeForce}
    \end{equation}
    For some potential $V(\vec{x})$, which is a scalar (with vector input).
\end{definition}
For the rest of this subsection, we consider gravity as an example of a force.\par
The gravitational potential energy if a particle with mass $m$ at $\vec{x}$ due to another particle with mass $M$ at $\vec{x_0}$ is:
\begin{equation}
    V = \frac{-GMm}{|\vec{x} - \vec{x_0}|}
    \label{eqnGravitationalPotential}
\end{equation}
Then we need $\nabla V$, so we need $\nabla |\vec{x} - \vec{x_0}$:
\begin{align*}
    \partial_i (|\vec{x} - \vec{x_0}|^2) &= 2 |\vec{x} - \vec{x_0}| \partial_i |\vec{x} - \vec{x_0}| \\
    \partial_i (\vec{x} - \vec{x_0})_j (\vec{x} - \vec{x_0})_j &= 2 |\vec{x} - \vec{x_0}| \partial_i |\vec{x} - \vec{x_0}| \\
    2(\vec{x} - \vec{x_0})_j \partial_i (\vec{x} - \vec{x_0})_j &= 2 |\vec{x} - \vec{x_0}| \partial_i |\vec{x} - \vec{x_0}| \\
    (\vec{x} - \vec{x_0})_j \delta_{ij} &= |\vec{x} - \vec{x_0}| \partial_i |\vec{x} - \vec{x_0}| \\
    (\vec{x} - \vec{x_0})_i &= |\vec{x} - \vec{x_0}| \partial_i |\vec{x} - \vec{x_0}|
\end{align*}
Then rearranging for the required term:
\begin{equation}
    \nabla |\vec{x} - \vec{x_0}| = \frac{\vec{x} - \vec{x_0}}{|\vec{x} - \vec{x_0}}
    \label{eqnDelSizeOfVec}
\end{equation}
Then using this result:
\begin{equation}
    \vec{F} = -\nabla V = -GMm \frac{\vec{x} - \vec{x_0}}{|\vec{x} - \vec{x_0}|^3}
    \label{eqnGravityTwoPoints}
\end{equation}
Or, if $\vec{r} = \vec{x} - \vec{x_0}$,
\begin{equation}
    \vec{F} = -\frac{GMm}{r^2} \uvec{r}
    \label{eqnGravitySeparation}
\end{equation}
We often write $V = m \Phi$, where $\Phi = -frac{GM}{|\vec{x} - \vec{x_0}}$ is the gravitational potential.
We can then consider an example:
\begin{example}[Motion on the surface of the earth]
    Near the surface of the earth, take $\vec{x_0}$ to be the centre of the earth, and $|\vec{x}| = R + z$, where $R$ is the radius of the earth, $R >> z$.
    \begin{align*}
        \Phi(R + z) &= -\frac{GM}{R + z} \\
        &= -\frac{GM}{R}(1 - \frac{z}{r} + \frac{z^2}{R^2} + O(z^3)) \\
        &\approx \text{ constant } + \frac{GMz}{R^2}
    \end{align*}
    So the force is $\vec{F} = -mg\uvec{z}$.\par
    This leads to the simplest possible motion due to a force:
    \begin{equation*}
        m\ddvec{x} = m \vec{g}
    \end{equation*}
    Then, taking the $z$ component:
    \begin{align*}
        \ddot{z} &= -g \\
        \dot{z} &= v_0 - gt \\
        z &= z_0 + v_0 t - \frac{1}{2}gt^2
    \end{align*}
\end{example}
\subsection{Energy}
Conservative forces have an associated energy that is conserved:
\begin{equation}
    E = \frac{1}{2} m \dvec{x} \cdot \dvec{x} + V(\vec{x})
    \label{eqnEnergy}
\end{equation}
Checking it is constant:
\begin{align*}
    \frac{dE}{dt} &= m \dot{x}_i \ddot{x}_j + \partial_i V \dot{x}_i \\
    &= \dot{x}_i \left(m\ddot{x}_i + \partial_i V\right) \\
    &= 0 \text{ by equation of gravity.}
\end{align*}
\begin{example}[Escape velocity]
    Escape velocity is the speed at which an object must be projected up such that it never falls back down.\par
    At the moment the object is thrown, the energy is:
    \begin{equation*}
        E_0 = \frac{1}{2} m v_0^2 - \frac{GMm}{R}
    \end{equation*}
    To not fall back, the object must reach $R \rightarrow \infty$ without velocity falling to 0.
    \begin{equation*}
        E_\infty = \frac{1}{2} m v_\infty^2 > 0
    \end{equation*}
    So we require $E_0 > 0$:
    \begin{align*}
        v_0^2 &> \frac{2GM}{R} \\
        v_0 &> \sqrt{\frac{2GM}{R}}
    \end{align*}
\end{example}
The mass $m$ cancelled because the gravitational mass, that appears in the force, is the same as the inertial mass that appears in the Second Law. The fact that these two are the same is called the Principle of Equivalence.\par
It is useful to write energy as $E = T + V$, where $T$ is the kinetic energy, and $V$ is the potential energy.\par
\begin{definition}{Work done} % TODO: Come back to this after Vector Calculus
    The \underline{work done}, $W$, is defined to be:
    \begin{equation}
        W = \int_{C} \vec{F} \cdot \vec{dx}
    \end{equation}
    It is a line integral.
\end{definition}
Conserivative forces have the property that work done by the force as a particle moves along a trajectory depends only on the endpoints of the trajectory, not the specific path. There are two ways to see this.
\begin{align*}
    W &= \int_C \vec{F} \cdot \vec{dx} \\
    &= \int_{t_1}^{t_2} dt \vec{F} \cdot \dvec{x} \\
    &= m \int_{t_1}^{t_2} dt \ddvec{x} \cdot \dvec{x} \\
    &= \frac{m}{2} \int_{t_1}^{t_2} dt \frac{d}{dt}(\dvec{x}^2) \\
    &= T(t_2) - T(t_1) \\
    &= V(\vec{x}(t_1)) - V(\vec{x}(t_2)) \text{ by conservation of energy}
\end{align*}
Which only depends on the endpoints.\par
Note that the quantity $\vec{F} \cdot \dvec{x}$ is the power, $P$. It is the rate of change of doing work.\par
The second way is using a property of line integrals and gradients.
\begin{align*}
    W &= \int_C \vec{F} \cdot \vec{dx} \\
    &= -\int_C \nabla V dx \\
    &= V(\vec{x_1}) - V(\vec{x_2})
\end{align*}
\subsection{Electromagnetic Forces}
Forces that depend on velocity typically do not have a conserved energy. One important exception is the Lorentz force that an electromagnetic field exerts on a charged particle.
\begin{definition}{Lorentz force}
    The force experienced by a particle with charge $q$ and at position $\vec{x}$ is the Lorentz force:
    \begin{equation}
        \vec{F} = q\left(\vec{E}(\vec{x}) + \dvec{x} \times \vec{B}(\vec{x})\right)
        \label{eqnLorentzForce}
    \end{equation}
    Here $\vec{E}$ is the electric field, $\vec{B}$ is the magnetic field. We assume static fields (that do not depend on time).
\end{definition}
Also, the electric field is given by:
\begin{equation}
    \vec{E}(\vec{x}) = -\nabla \phi(\vec{x})
    \label{eqnElectricField}
\end{equation}
And $\phi$ is the electric potential.\par
In the electric field, the conserved energy is $E = \frac{1}{2} m\dvec{x}^2 + q\phi(\vec{x})$. Checking the field is conservative:
\begin{align*}
    \frac{dE}{dt} &= m\dvec{x} \cdot \ddvec{x} + q \nabla\phi \cdot \vec{x} \\
    &= \dvec{x} \cdot (\vec{F} - q\vec{E}(\vec{x})) \\
    &= \dvec{x} \cdot (q\vec{x} \times \vec{B}(\vec{x})) = 0
\end{align*}
The velocity-dependent force is orthogonal to the trajectory of the particle, and so does no work on the particle, and therefore energy is conserved.\par
The electric force is similar to gravitational force. The potential at $\vec{x}$ due to a particle of charge $Q$ at $\vec{x_0}$ is:
\begin{equation}
    \phi(\vec{x}) = \frac{Q}{4\pi \epsilon_0} \frac{1}{|\vec{x} - \vec{x_0}}
    \label{eqnElectricPotential}
\end{equation}
And therefore like gravity we have an inverse-square law, this is called the Coulomb Force.\par
One important difference between the gravitational force and the Coulomb force is that mass is always positive, so gravity is attractive. Charges can be positive or negative, and so the electric force can be attractive or repulsive.\par
Magnetic forces are different. A charged particle in a magnetic field obeys:
\begin{equation}
    m\ddvec{x} = q\dvec{x} \times \vec{B} \\
    \label{eqnMagneticForce}
\end{equation}
This is a vector differential equation. We solve for constant magnetic field. Align the coordinate system such that $\vec{B} = B\uvec{z}$. Also write $\vec{x} = (x, y, z)^T$.\par
Then solve equation~\ref{eqnMagneticForce} using this coordinate system:
\begin{align*}
    m\ddot{x} &= qB\dot{y} \\
    m\ddot{y} &= -qB\dot{x} \\
    m\ddot{z} &= 0
\end{align*}
Parameterising $m, q, B$ as $\omega = \frac{qB}{m}$ fully describes the system. $\omega$ is called the cyclotron frequency.\par
Solving the equation in $z$:
\begin{equation*}
    z = z_0 + v_z t
\end{equation*}
So we have constant-velocity motion in the $z$ direction.\par
Then we have a system of 2 equations to solve ($x$ and $y$). We can eliminate a variable:
\begin{align*}
    m \frac{d\ddot{x}}{dt} &= qb\ddot{y} \\
    &= -\omega^2 \dot{x} \text{ by substituting for } \ddot{y} \\
    \dot{x} &= \tilde{A}\sin{(\omega t + \phi)} \\
    x = x_0 + A \cos{(\omega t + \phi)}
\end{align*}
And we can solve for $y$:
\begin{equation*}
    y = y_0 - A \sin{(\omega t + \phi)}
\end{equation*}
And this motion is circular in the x-y plane, with constant velocity in the $z$ direction.\par
We can solve this in a much easier way using complex numbers.\par
Let $\xi = x + iy$. Now consider:
\begin{align*}
    \ddot{\xi} &= \ddot{x} + i\ddot{y} \\
    &= -i\omega \dot{\xi} \\
    \xi &= c_1 e^{-i\omega t} + c_2
\end{align*}
Then relabelling the constants as: $c_1 = Ae^{-\phi}, c_2 = x_0 + i y_0$, we get the solutions as above.
\end{document}