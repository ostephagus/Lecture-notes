\documentclass[../Main.tex]{subfiles}

\begin{document}
\section{Basic Postulates of Relativity}
Maxwell's Equations, formulated by Maxwell in 1862, predict the existence of electromagnetic waves that travel at the speed of light. This is $c = 299792458 m s^{-1}$ (note that this number is precise, because it defines the metre).\par
A theory such as Maxwell's equations with a preferred velocity cannot be Galilean invariant. In principle this is permissible, sound waves travel through the air around $300 ms^{-1}$, but this is relative to the rest frame of the air. Thus, it was assumed that light must travel through a medium, this proposed medium was known as the Luminiferous Ether. However, experiments to demonstrate the existence of the Luminiferous Ether (Michelson \& Morley, 1881) showed that light travels at the same speed regardless of how fast the observer (on Earth) is moving through the Ether.\par
In 1905, Einstein postulated that there was no Ether. He provided two postulates:
\begin{enumerate}
    \item The laws of physics are the same in all inertial reference frames. This is the principle of relativity that Galileo postulated before.
    \item The speed of light in a vacuum is the same in all inertial reference frames. This is clearly not compatible with Galilean Relativity.
\end{enumerate}
\section{Lorentz Transformations}
We will derive the Lorentz transformations to replace Galilean transformations.\par
\subsection{1-Dimensional Space}
We will first consider 1 spatial dimension. An inertial frame $S$ has coordinates $(x, t)$. Consider a second frame $S'$ that moves at a speed $v$ relative to $S$ and has coordinates $(x', t')$.\par
According to Galileo, $x' = x - vt$ and $t' = t$.\par
This will not work, as the speed of light will not be constant in both frames. Instead consider a general transformation:
\begin{align*}
    x' &= f(x, t) \\
    t' &= g(x, t) \\
\end{align*}
Postulate 1 requires that in both frames $S$ and $S'$, a particle experiencing no forces must move at a constant velocity (Law of Inertia). So for a free particle in $S$, with $x = A + Bt$, the particle in $S'$ must be moving with $x' = A' + B't'$. Therefore the transformation $(f, g)$ must map lines in the $(x, t)$ plane to lines in the $(x', t')$ plane. Therefore $f$ and $g$ must be linear:
\begin{align*}
    x' &= ax + bt \\
    t' &= cx + dt
\end{align*}
Note that $a, b, c, d$ could depend on $v$ but must not depend on $x, t$. Note also that we have chosen the frames to have a common origin. We could just as well have added a constant term to both of the above equations.\par
The frame $S'$ is moving at speed $v$ in the frame $S$, but obviously must be at rest with respect to itself, $x' = 0$. Therefore the line $x = vt$ must be mapped to the line $x' = 0$. Therefore we can get the function $f$ up to scaling:
\begin{equation*}
    x' = \gamma_v (x - vt)
\end{equation*}
We can also relate in the other direction: $S$ moves at speed $-v$ in relation to $S'$, so $x = \gamma_{-v} (x' + vt)$.
\begin{proposition}
    In the above, $\gamma_v = \gamma_{-v}$
\end{proposition}
If we assume this to be true, we have that:
\begin{align}
    x' &= \gamma(x - vt) \\
    t' &= \frac{1-\gamma^2}{\gamma v} x + \gamma t
\end{align}
We cannot give a rigorous proof. We give 3 arguments:
\begin{enumerate}
    \item There is no preferred direction of space. Therefore $\gamma_v$ should be a function of $|v|$ only: $\gamma_v = \gamma_{-v}$.
    \item Consider frames $\tilde{S}$ and $\tilde{S'}$ where the $x$ axis is reflected. That is, $\tilde{x} = -x, \tilde{x'} = -x'$. Then if $S$ moves at speed $v$ with respect to $S'$, $\tilde{S}$ moves at speed $-v$ with respect to $\tilde{S'}$. Therefore
    \begin{align*}
        \tilde{x'} &= \gamma_{-v}(\tilde{x} + vt) \\
        -x' &= \gamma_{-v} (-x + vt) \\
        x' &= \gamma_{-v} (x - vt) \\
        \gamma_v &= \gamma_{-v}
    \end{align*}
    \item If a particle increases to speed $v$ relative to $S$, and then changes its speed by $-v$, it returns to rest is $S$. Consider the $x$ coordinate after these two changes in speed, assuming $\gamma_v = \gamma_{-v} = \gamma$:
        \begin{align*}
            x'' &= \gamma_{-v} (x' + vt') \\
            &= \gamma \left(\gamma (x - vt) + v\left(\gamma t + \frac{1 - \gamma^2}{\gamma v}\right)\right) \\
            &= x
        \end{align*}
\end{enumerate}
\end{document}