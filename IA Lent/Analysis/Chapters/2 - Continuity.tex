\documentclass[../Main.tex]{subfiles}

\begin{document}
\section{Definitions and Examples}
\subsection{Definitions}
\begin{definition}{Continuity at a point}
    Suppose that $E$ is a non-empty subset of the complex numbers. Let $f: E \mapsto \C$ be any function.\par
    Then $f$ is \underline{continuous at the point} $a \in E$ if, given $\epsilon > 0$, there exists $\delta(\epsilon) > 0$ such that:
    \begin{equation}
        |z - a| < \delta \implies |f(z) - f(a)| < \epsilon
        \label{eqnContinuityDefinition}
    \end{equation}
\end{definition}
Equation~\ref{eqnContinuityDefinition} roughly says that points close together in the domain are mapped onto points that are close together in the image.
\begin{definition}{Continuity on a set}
    A function $f$ as defined above is \underline{continuous on the set} $E$ if $f$ is continuous at every point in $E$.
\end{definition}
\begin{theorem}[Sequential characterisation of continuity]
    A function $f : E \mapsto \C$ is continuous at $a \in E$ if and only if $f(z_n) \to f(a)$ for all sequences $z_n$ where $z_n \to a$, and each point $z_n \in E$.
    \label{thmContinuitySequenceDef}
\end{theorem}
Informally, this says that a function is convergent if it commutes with sequence limits:
\begin{equation}
    f(\lim_{n \to \infty}z_n) = \lim_{n \to \infty}f(z_n)
    \label{eqnLimitCommutativity}
\end{equation}
\begin{proof}
    \begin{proofdirection}{$\Rightarrow$}{Suppose that $f$ is continuous at $a$}
        Let $z_n$ be a sequence satisfying the required properties. Let $\epsilon > 0$, then there exists $\delta(\epsilon) > 0$ such that if $z \in E$,
        \begin{equation*}
            |z - a| < \delta \implies |f(z) - f(a)| < \epsilon
        \end{equation*}
        But since $z_n \to a$, there exists $N(\delta)$ such that:
        \begin{align*}
            n \geq N &\implies |z_n - a| < \delta \\
            &\implies |f(z_n) - f(a)| < \epsilon
        \end{align*}
    \end{proofdirection}
    \begin{proofdirection}{$\Leftarrow$}{Suppose, on the contrary, that $f$ is not continuous at $a$.}
        Therefore there exists some number $\epsilon_0 > 0$ such that for all $\delta(\epsilon_0)$ there exists a $z \in E$ with $|z - a| < \delta$ and $|f(z) - f(a)| \geq \epsilon_0$.\par
        Then consider this with $\delta = 1, \frac{1}{2}, \frac{1}{3}, \cdots, \frac{1}{n}, \cdots$. Then for each $n$ there exists a $z_n$ with $|z_n - a| < \frac{1}{n}$ and:
        \begin{equation*}
            |f(z_n) - f(a)| \geq \epsilon_0
        \end{equation*}
        Then $z_n \to a$ but $f(z_n) \not\to f(a)$.
    \end{proofdirection}
\end{proof}
This is very useful. We can get lots of extra results about convergence with little extra work.
\begin{propositions}{
        Suppose that $f$ and $g$ are functions $E \mapsto \C$, and are continuous at a point $a \in E$. Then the following are continuous at $a$:
        \label{propsContinuityProperties}
    }
    \item $f(z) + g(z)$ \label{propSumContinuity}
    \item $f(z)g(z)$ \label{propProductContinuity}
    \item $\lambda f(z)$ for complex coefficient $\lambda$ \label{propScalingContinuity}
    \item $\frac{1}{f(z)}$ as long as $f(z)$ non-zero \label{propReciprocalContinuity}
\end{propositions}
\begin{proof}
    All proofs by proposition~\ref{propsSequenceFacts} and theorem~\ref{thmContinuitySequenceDef}
\end{proof}
\begin{theorem}[Composition of continuous functions]
    Consider two subsets $A$ and $B$ of the complex numbers. Let $f : A \mapsto B$, and $g : B \mapsto \C$.\par
    Suppose that $f$ is continuous at $a \in A$, and $g$ is continuous at $f(a) \in B$. Then the composition $g \circ f$ is continuous at $a$.
    \label{thmCompositionContinuous}
\end{theorem}
\begin{proof}
    Suppose that $z_n$ is any sequence with $z_n \in A$, and $z_n \to a$. Then for each $n$, $f(z_n) \in B$, and $f(z_n) \to f(a)$ by continuity of $f$. Therefore this is a sequence in $B$ converging to $f(a)$ and so by the continuity of $g$, the sequence $g(f(z_n))$ must converge to $g(f(a))$.\par
    Now we have that any sequence $z_n$ in $A$ converges to $g(f(a))$ in $\C$.
\end{proof}
\subsection{Examples}
\begin{examples}{\label{exContinuityExamples}}
    \item $f(z) = z$ is continuous at all $z \in \C$.
    \item Any polynomial is continuous at any point by the above example and proposition~\ref{propsContinuityProperties}.
    \item $f(z) = |z|$ is continuous at all points (consider the estimate \\ $\big ||z| - |w|\big | \leq |z - w|$)
    \item The function $f(x) = 0$ if $x \leq 0$, and $f(x) = 1$ if $x > 0$, is not continuous at $x = 0$. \label{exStepDiscontinuity}
    \item Consider $f : \R \mapsto \R$:
        \begin{equation*}
            f(x) =
            \begin{cases}
                \sin{(1/x)} & x \neq 0 \\
                0 & x = 0
            \end{cases}
        \end{equation*}
        Assume that $\sin$ is continuous everywhere, so if $x \neq 0$ then proposition~\ref{propReciprocalContinuity} gives that $f$ is continuous for non-zero $x$. However, $f$ is discontinuous at $x = 0$. To see this, let $x_n = \frac{1}{(2n + 1/2)\pi}$. Then for all $n$, $f(x_n) = 1$, but $x_n \to 0$ and $f(0) = 0$.
        \label{exSinOneOverXDiscontinuity}
    \item Consider a similar function $f : \R \mapsto \R$:
        \begin{equation*}
            f(x) = 
            \begin{cases}
                x \sin{(1/x)} & x \neq 0 \\
                0 & x = 0
            \end{cases}
        \end{equation*}
        We show that this function is, in fact, continuous. Consider any sequence $x_n \to 0$. Estimate $|f(x_n)| = |x_n| |\sin{(1/x_n)}| \leq |x_n|$. Note that this does not hold for $x_n = 0$, but no matter since we define $f(0) = 0$. Therefore the sequence converges by theorem~\ref{thmContinuitySequenceDef}
        \label{exXSinOneOverXContinuity}
    \item Consider the Dirichlet Function $f : \R \mapsto \R$:
        \begin{equation*}
            f(x) = 
            \begin{cases}
                1 & x \in \Q \\
                0 & x \notin \Q
            \end{cases}
        \end{equation*}
        Then this function is discontinuous at every point. If $x$ is irrational, then we consider the sequence of finite decimal expansions $x_n$ that converges to $x$. However, $f(x_n) = 1$, and $f(x) = 0$. Similarly, if $x$ is rational, then take any sequence of irrational numbers converging to $x$. \label{exDirichletDiscontinuity}
\end{examples}
Example \ref{exStepDiscontinuity} is the classic discontinuous function, but discontinuous functions can be more complex, as shown in example \ref{exSinOneOverXDiscontinuity}. We can also have continuous functions that are not very ``well-behaved'', as shown in example~\ref{exXSinOneOverXContinuity}
\section{Limit of a Function}
Suppose that $f : E \mapsto \C$. Then we want to define what we mean by
\begin{equation*}
    \lim_{z \to a} f(x)
\end{equation*}
even if $a \notin E$.\par
For example, we would like to define $\lim_{z \to 0} \frac{\sin{z}}{z}$.\par
This concept does not always make sense. For example, if a function is defined on the interval $[1, 2]$ and at the point $0$, it does not really make sense to consider $\lim_{x \to 1/2}f(x)$ because there are no points nearby, nor does it make sense to consider $\lim_{x \to 0} f(x)$, by the same logic.
\begin{definition}{Limit point}
    Suppose that $E \subseteq \C$ and $a \in \C$. Then $a$ is a \underline{limit point} of $E$ if, for any $\delta > 0$, there exists some $z \in E$ such that:
    \begin{equation*}
        0 < |z - a| < \delta
    \end{equation*}
\end{definition}
Intuitively, this requires that there be a point of $E$ inside any size disc around $a$. If $a$ belongs to $E$ but is not a limit point, it is an \underline{isolated point}.
\begin{remark}
    $a$ is a limit point of $E$ if and only if there exists a sequence $(z_n)$ such that $z_n \in E$, $z_n \neq a$, but $z_n \to a$.
\end{remark}
\begin{definition}{Limit}
    Suppose that $f: E \subseteq C \mapsto \C$. Let $a$ be a limit point of $E$. Then $l$ is the \underline{limit} as $z \to a$ of $f(a)$:
    \begin{equation*}
        \lim_{z \to a} f(z) = l
    \end{equation*}
    if for all $\epsilon > 0$, there exists some $\delta > 0$ such that for all $z \in E$ which lie within $\delta$ of $a$ ($0 < |z - a| < \delta$), we have that:
    \begin{equation*}
        |f(z) - l| < \epsilon
    \end{equation*}
\end{definition}
We also have an equivalent definition in terms of sequences:
\begin{lemma}
    If $f, E$ and $a$ are as above, then $\lim_{z \to a} f(x) = l$ if and only if $f(z_n) \to l$ for all sequences $z_n \in E$, $z_n \neq a$ and $z_n \to a$.
    \label{lemLimitSequenceDef}
\end{lemma}
The proof is almost identical to that of theorem~\ref{thmContinuitySequenceDef}.\par
\begin{remark}
    Combining the sequence definitions of continuity and of the limit, if $a \in E$ and is a limit point of $E \backslash \{a\}$, then $f$ is continuous if and only if $\lim_{z \to a} f(z) = f(a)$. Also, if $a$ is an isolated point, then continuity is immediate (and also not very useful).
\end{remark}
\begin{propositions}{
        Let $f$, $a$, $E$ be defined as above Let $g$ also be a function $E \mapsto \C$.
        \label{propsLimitProperties}
    }
    \item The limit of a function at any point is unique, or not defined. \label{propLimitUnique}
    \item $\lim_{z \to a} (f(z) + g(z)) = \lim_{z \to a}(f(z)) + \lim_{z \to a} (g(z))$ \label{propLimitSum}
    \item $\lim_{z \to a} (f(z) \times g(z)) = \lim_{z \to a}(f(z)) \times \lim_{z \to a} (g(z))$ \label{propLimitProduct}
    \item If $g(z)$ is non-zero and the limit is non-zero, then:
        \begin{equation*}
            \lim_{z \to a} \frac{f(z)}{g(z)} = \frac{\lim_{z \to a} f(z)}{\lim_{z \to a} g(z)}
        \end{equation*}
\end{propositions}
\section{Bounding functions}
\subsection{Intermediate Value Theorem}
The Intermediate Value Theorem informally captures the idea that if $f$ is continuous, it can be sketched without lifting the pen from the page.
\begin{theorem}[Intermediate value theorem]
    Suppose that $f : [a, b] \mapsto \R$ and is continuous on the whole interval. Without loss of generality, let $f(a) < f(b)$. Then for any $\eta$ with $f(a) < \eta < f(b)$, there exists at least one value $c$ with $a \leq c \leq b$ where $\eta = f(c)$.
    \label{thmIntermediateVal}
\end{theorem}
This theorem is aided by figure~\ref{figIntermediateValue}
\begin{figure}[ht]
    \centering
    \begin{tikzpicture}
        \draw[->] (-0.1, 0) -- (3, 0) node[right] {$x$};
        \draw[->] (0, -0.1) -- (0, 4) node[above] {$y$};
        \draw[domain=0.2:2.8, samples=50] plot (\x, {\x * \x * \x -4.2 * \x * \x + 4.9 * \x + 0.2}) node[right] {$f(x)$};
        \foreach \x / \xtext in {0.2/a, 2.8/b, 2.49045/c}
            \draw (\x, 0.1) -- (\x, -0.1) node[below] {$\xtext$};

        \foreach \y / \ytext in {1.02/$f(a)$,2.944/$f(b)$,1.8/$\eta$}
            \draw(0.1, \y) -- (-0.1, \y) node[left] {\ytext};

        \draw[dashed] (0, 1.8) -- (2.49045, 1.8) -- (2.49045, 0);

        \foreach \x/\y in {2.49045/1.8, 0.2/1.02, 2.8/2.944}
            \draw[fill] (\x, \y) circle[radius=0.5mm];
    \end{tikzpicture}    
    \caption{Illustration of the Intermediate Value Theorem}
    \label{figIntermediateValue}
\end{figure}
\begin{proof}
    Let $S = \subsetselect{x \in [a, b]}{f(x) < \eta}$. Note that $S$ is non-empty and bounded above by hypotheses, so there exists a supremum $c$.\par
    From the definition of a supremum, for each $n$ there exists some $x_n \in S$ with $c - \frac{1}{n} \leq x_n \leq c$. Now $x_n$ must converge to $c$. Since $f$ is continuous, $f(x_n) \to f(c)$, so since $x_n \in S$, $f(x_n) < \eta \implies f(c) \leq \eta$.\par
    Note we have $c \neq b$, so we can find a decreasing sequence $\tilde{x}_n = c + \frac{1}{n}$. Then for sufficiently large $n$, $\tilde{x}_n \in [a, b] \backslash S$, and this converges to $c$. Now $f(\tilde{x}_n) \to f(c)$ as above, and so $f(c) \geq \eta$.\par
    However, now $f(c) \leq \eta \leq f(c)$, so $\eta = f(c)$.
\end{proof}
\begin{remarks}
    \item By applying the theorem to $-f$, we can get the same result for $f(a) > f(b)$. The case $f(a) = f(b)$ is trivial, since $\eta = f(a) = f(b)$.
    \item This theorem can be used to prove the existence of roots of functions on a given interval (by the change-of-sign method).
\end{remarks}
\begin{example}[Existence of $n$th roots]
    Suppose $y > 0$, and let $N \in \N$.\par
    Define:
    \begin{align*}
        f : [0, 1+y] &\mapsto \R \\
        x &\mapsto x^N
    \end{align*}
    Since $(1+y)^N = 1 + Ny + \cdots \geq y$, we have:
    \begin{equation*}
        f(0) < y < f(1+y)
    \end{equation*}
    So there exists a $c \in (0, 1+y)$ such that $f(c) = y$ by theorem~\ref{thmIntermediateVal}.\par
    Thus $c^N = y$. That is, there exists a positive $N$th root of $y$.\par
    Also, since $f$ is strictly increasing, this $N$th root must be unique.
\end{example}
\subsection{Extreme Value Theorem}
\begin{lemma}[Bounds on continuous functions]
    Suppose that $f : [a, b] \mapsto \R$, and that $f$ is continuous. Then there exists some $K \geq 0$ such that $|f(x)| \leq K~\forall x \in [a, b]$.
    \label{lemContinuousBounded}
\end{lemma}
\begin{proof}
    Suppose, on the contrary, there is no such bound $K$. Therefore for each $n = \{1, 2, \cdots\}$, we can find $x_n \in [a, b]$ with $|f(x_n)| > n$.\par
    We have that the $x_n$ are bounded, so by theorem~\ref{thmBolzanoWeierstrass}, we can find a convergent subsequence $x_{n_j}$ with limit $x \in [a, b]$.\par
    However, $|f(x_{n_j})| > n_j \geq j$, so $f(x_{n_j})$ cannot converge. However, this contradicts theorem~\ref{thmContinuitySequenceDef}.\contradiction\par
    Therefore the lemma must be true.
\end{proof}
\begin{theorem}[Extreme value theorem]
    Suppose that $f : [a, b] \mapsto \R$ and $f$ is continuous. Then there exists a pair of numbers $y, Y \in [a, b]$ such that:
    \begin{equation*}
        f(y) \leq f(x) \leq f(Y)
    \end{equation*}
    for any $x \in [a, b]$.
    \label{thmExtremeValue}
\end{theorem}
\begin{proof}
    Let $A$ be the set of values that $f(x)$ takes on the interval. This is non-empty and bounded by lemma~\ref{lemContinuousBounded}, so has a supremum $M$.\par
    For sufficiently large $n$, there exists a $y_n \in A$ such that $M - \frac{1}{n} \leq y_n \leq M$.\par
    Then by the definition of $A$ there exists an $x_n \in [a, b]$ such that $f(x_n) = y_n$:
    \begin{equation*}
        M - \frac{1}{n} \leq f(x_n) \leq M
    \end{equation*}
    This therefore gives that $f(x_n) \to M$ as $n \to \infty$.\par
    By theorem~\ref{thmBolzanoWeierstrass}, we have a convergent subsequence $x_{n_j}$ which converges to $Y$ and which also has the property that $f(x_{n_j}) \to M$ as $j \to \infty$. Therefore by continuity of $f$, $f(Y) = M$.\par
    Applying the same to $-f$, we have that $-f$ is bounded above by some $-f(y)$, so $f$ is bounded below by $f(y)$.
\end{proof}
\begin{remark}
    It is crucial here that the domain is bounded and closed. $a$ and $b$ must be finite and both in the domain.
\end{remark}
\section{Inverse Functions}
\begin{definition}{Strictly increasing}
    A function $f : [a, b] \mapsto \R$ is \underline{strictly increasing} if:
    \begin{equation*}
        a \leq x_1 < x_2 \leq b \implies f(x_1) < f(x_2)
    \end{equation*}
\end{definition}
\begin{theorem}[Inverse function theorem I]
    Suppose that $f : [a, b] \mapsto \R$ is continuous and strictly increasing.\par
    Let $c = f(a), d = f(b)$. Then $f : [a, b] \mapsto [c, d]$ is a bijection and the inverse map $f^{-1} : [c, d] \mapsto [a, b]$ is continuous and strictly increasing.
    \label{thmInverseFunctionI}
\end{theorem}
\begin{proof}
    \begin{subproof}{$f$ is injective}
        Consider $x \neq y$. After relabelling, $x < y$ so by strictly increasing $f(x) < f(y) \implies f(x) \neq f(y)$.
    \end{subproof}
    \begin{subproof}{$f$ maps to $[c, d]$}
        If $a \leq x \leq b$ then $f(a) \leq f(x) \leq b$, so $c \leq f(x) \leq d$.
    \end{subproof}
    \begin{subproof}{$f$ is surjective}
        This follows immediately from the intermediate value theorem: if $\eta \in (c, d)$ then there exists $x \in [a, b]$ with $f(x) = \eta$, and the endpoints are hit by construction.
    \end{subproof}
    So now we have that $f$ is a bijection $[a, b] \mapsto [c, d]$, and so $f^{-1}$ is well-defined.
    \begin{subproof}{$f^{-1}$ is continuous}
        Consider $z \in (c, d)$, and let $y = f^{-1}(z)$. Let $\epsilon > 0$ be sufficiently small such that $(y - \epsilon, y + \epsilon)$ is contained within $[a, b]$.
        \begin{align*}
            a &\leq y - \epsilon < y < y + \epsilon \leq b \\
            \implies c &\leq f(y-\epsilon) < z < f(y + \epsilon) \leq d
        \end{align*}
        And since $f : (y - \epsilon, y + \epsilon) \mapsto (f(y - \epsilon), f(y + \epsilon))$ since $f$ is bijective,
        \begin{align*}
            |z' - z| < \delta &\implies z' \in (f(y - \epsilon), f(y + \epsilon)) \\
            &\implies f^{-1}(z') \in (y - \epsilon, y + \epsilon) \\
            &\implies |f^{-1}(z') - f^{-1}(z)| < \epsilon
        \end{align*}
    \end{subproof}
    \begin{subproof}{$f^{-1}$ is strictly increasing}
        Suppose that $f^{-1}$ is not strictly increasing:
        \begin{equation*}
            \exists c \leq z_1 < z_2 \leq d \text{ where } f^{-1}(z_1) \geq f^{-1}(z_2)
        \end{equation*}
        But then applying $f$ to both sides:
        \begin{equation*}
            f(f^{-1}(z_1)) \geq f(f^{-1}(z_2)) \implies z_1 \geq z_2
        \end{equation*}
        \contradiction
    \end{subproof}
\end{proof}
\end{document}