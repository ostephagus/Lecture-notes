\documentclass[../Main.tex]{subfiles}

\begin{document}
\section{Limits}
\subsection{Defining Limits}
\begin{definition}{Limit of a sequence}
    Consider a sequence of real numbers, $a_1, a_2, a_3, \cdots$ or $(a_n)_{n=1}^\infty$ and $a_n \in \R$.\par
    We say that the sequence \underline{tends to the limit a} as $n$ tends to infinity if, given $\epsilon > 0$ there exists a natural number $N$ such that
    \begin{equation}
        |a_n - a| < \epsilon~\forall n \geq N
        \label{eqnSequenceConvergence}
    \end{equation}
\end{definition}
\begin{remarks}
    \item $N$ depends on $\epsilon$
    \item The inequality can be non-strict without changing the definition
    \item Any fixed multiple of $\epsilon$ on the RHS does not change the definition (replace $\epsilon$ with a multiple to return to 1 on RHS)
    \item Here $a$ must be finite and real.
\end{remarks}
\subsection{Monotonic Sequences}
If $a_n \leq a_{n+1}~\forall n$, we say $(a_n)$ is increasing.\par
If $a_n \geq a_{n+1}~\forall n$, we say $(a_n)$ is decreasing.\par
Sequences can be strictly increasing or strictly decreasing if the inequalities above are strict.\par
If $(a_n)$ is either increasing or decreasing then it is \underline{monotone}.\par
\begin{proposition}[Fundamental axiom of real numbers]
    An increasing sequence of real numbers that is bounded above converges.
\end{proposition}
\begin{remarks}
    \item Equivalently, every decreasing sequence bounded below converges.
    \item Note that this is equivalent to the least upper bound axiom from IA Numbers and Sets.
    \item This is not true for the rational numbers. Consider the sequence of rational approximations to $\sqrt{2}$. This does not have a limit in the rational numbers.
\end{remarks}
\subsection{Facts About Sequences}
\begin{propositions}{
        \label{propsSequenceFacts}
    }
    \item If $a_n \rightarrow a$ and $a_n \rightarrow b$, then $a = b$. \label{propLimitUnique}
    \item If $a_n \rightarrow a$ and $n_1 < n_2 < n_3 < \cdots$ is a sequence of natural numbers, then $a_{n_j} \rightarrow a$ as $j \rightarrow \infty$. \label{propSubsequenceLimit}
    \item If $a_n = c~\forall n$, then $a_n \rightarrow c$.\label{propConstantLimit}
    \item If $a_n \rightarrow a$ and $b_n \rightarrow b$, $a_n + b_n \rightarrow a + b$.\label{propSumOfSequences}
    \item If $a_n \rightarrow a$ and $b_n \rightarrow b$, then $a_n b_n \rightarrow ab$.\label{propProductOfSequences}
    \item If $a_n \rightarrow a$, $a_n \neq 0, a \neq 0$, then $\frac{1}{a_n} \rightarrow \frac{1}{a}$ \label{propReciprocalSequence}
    \item If $a_n$ is bounded above by $A$, and $a_n \rightarrow a$, then $a \leq A$.\label{propBoundedLimit}
    \item If $a_n \rightarrow a$ and $c_n \rightarrow a$ as $n \rightarrow \infty$, and we have $b_n$ such that $a_n \leq b_n \leq c_n$, then $b_n \rightarrow a$. \label{propSequenceSandwich}
\end{propositions}
\begin{proof}
    \begin{enumerate}
        \item Limit is unique:
            For any $\epsilon > 0$, we can find $N_1(\epsilon)$ and $N_2(\epsilon)$ such that:
            \begin{equation*}
                n \geq N_1 \implies |a_n - a| < \epsilon
            \end{equation*}
            and
            \begin{equation*}
                n \geq N_2 \implies |a_n - b| < \epsilon
            \end{equation*}
            If $n \geq \max{\{N_1, N_2\}}$, then:
            \begin{align*}
                0 \leq |b - a| &= |b - a_n + a_n - a| \\
                &\leq |a_n - b| + |a_n - a| \text{ by triangle inequality} \\
                &\leq 2\epsilon
            \end{align*}
            And since $\epsilon$ was arbitrary, $|b - a| = 0$ and $a = b$. \proofend
        \item Subsequences converge to the same limit:
        Since $n_j < n_{j + 1}$, we must have that $n_{j + 1} \geq n_j + 1$, so by induction we must have that $n_j \geq j$.\par
        Then since $a_n \rightarrow a$, given $\epsilon > 0$ there exists $N(\epsilon)$ such that:
        \begin{equation*}
            n \geq N \implies |a_n - a| < \epsilon
        \end{equation*}
        So if $j \geq N(\epsilon)$ then $n_j \geq N(\epsilon)$ and therefore $|a_{n_j} - a| < \epsilon$. \proofend
        \setcounter{enumi}{4} % Next number is 5
        \item Product of sequences tends to product of limit:
        Since $a_n \rightarrow a$ and $b_n \rightarrow b$, for any $\epsilon > 0$, we can find $N_1(\epsilon)$ and $N_2(\epsilon)$ such that:
            \begin{equation*}
                n \geq N_1 \implies |a_n - a| < \epsilon
            \end{equation*}
            and
            \begin{equation*}
                n \geq N_2 \implies |b_n - b| < \epsilon
            \end{equation*}
            Then:
            \begin{align*}
                |a_n b_n - ab| &= |a_n b_n - a_n b + a_n b - ab| \\
                &\leq |a_n b_n - a_n b| + |a_n b - ab| \\
                &\leq |a_n| |b_n - b| + |b| |a_n - a|
            \end{align*}
            We are nearly there, but note that $a_n$ is not fixed. However, if, for example, $n \geq N_1(1)$, then $|a_n - a| < 1$, so $|a_n| \leq 1 + |a|$, which is fixed.\par
            Therefore, if $n \geq \max{\{N_1(\epsilon), N_1(1), N_2(\epsilon)\}}$, then $|a_n b_n - ab| < (1 + |a| + |b|)\epsilon$, which is a constant multiple so $a_n b_n \rightarrow ab$.
    \end{enumerate}
\end{proof}
\end{document}