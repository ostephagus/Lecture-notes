\documentclass[../Main.tex]{subfiles}

\begin{document}
\section{Defining Integration}
We are often introduced to integration as an inverse operation to differentiation. This is fine, if we know (or can guess) an antiderivative. However, this does not serve well as a definition when we can't. The idea is that the integral should be the area under a graph. We will consider the definite integral from $a$ to $b$ to be the signed area under the graph of $f(x)$. The problem with this, however, is to define such an area in terms of known concepts.\par
We first assume that $f$ is a bounded, real-valued function over the closed interval $[a, b]$. That is,
$f : [a, b] \mapsto \R$ and there exists some bound $K$ such that $|f(x)| \leq K$ for all $x$ in $[a, b]$.
\begin{definition}{Dissection}
    A \underline{Dissection}, $\dissec$, of $[a, b]$ is a finite subset of $[a, b]$ containing the endpoints $a, b$. We write $\dissec = \{x_0, x_1, \cdots, x_n\}$ with $a = x_0 < x_1 < \cdots < x_n < b$.
\end{definition}
Associated to such a dissection are the upper and lower sums given by:
\begin{equation}
    U(f, \dissec) = \sum_{j = 1}^n (x_j - x_{j - 1}) \sup_{x \in [x_{j-1}, x_j]} f(x)
    \label{eqnUpperSum}
\end{equation}
\begin{equation}
    L(f, \dissec) = \sum_{j = 1}^n (x_j - x_{j - 1}) \inf_{x \in [x_{j-1}, x_j]} f(x)
    \label{eqnLowerSum}
\end{equation}
Note that the suprema and infinima are defined here, since $f$ is bounded.\par
Clearly we have that $L(f, \dissec) \leq U(f, \dissec)$.
\begin{lemma}
    Suppose that $\dissec$ and $\dissec'$ are dissections. Suppose that $\dissec \subseteq \dissec'$. Then:
    \begin{equation*}
        L(f, \dissec) \leq L(f, \dissec') \leq U(f, \dissec') \leq U(f, \dissec)
    \end{equation*}
    \label{lemRefinement}
\end{lemma}
\begin{remark}
    In this case, where $\dissec \leq \dissec'$, we say that $\dissec'$ is a \underline{refinement} of $\dissec$.
\end{remark}
\begin{proof}
    Let $\dissec = \{x_0, \cdots, x_n\}$. First consider $\dissec' = \dissec \cup \{y\}$, where $y \notin\dissec$.\par
    Then $y \in (x_{r - 1}, x_r)$ for some $r \in \{1, \cdots, n\}$.\par
    We must have that:
    \begin{equation*}
        \sup_{x \in [x_{r - 1}, y]} f(x), \sum_{x \in [y, x_r]} f(x) \leq \sup_{x \in [x_{r - 1}, x_r]} f(x)
    \end{equation*}
    Therefore:
    \begin{align*}
        &(y - x_{r - 1}) \sup_{x \in [x_{r - 1}, y]}f(x) + (x_r - y) \sup_{x \in [y, x_r]} f(x) \\
        &\leq (y - x_{r - 1}) \sup_{x \in [x_{r - 1}, x_r]}f(x) + (x_r - y) \sup_{x \in [x_{r - 1}, x_r]} f(x) \\
        &\leq (x_r - x_{r - 1}) \sup_{x \in [x_{r - 1}, x_r]}f(x) \\
    \end{align*}
    That is, $U(f, \dissec') \leq U(f, \dissec)$.\par
    Similarly, we know that:
    \begin{equation*}
        \inf_{x \in [x_{r - 1}, y]} f(x), \inf_{x \in [y, x_r]} f(x) \geq \inf_{x \in [x_{r - 1}, x_r]} f(x)
    \end{equation*}
    and we can use the above idea to show that:
    \begin{equation*}
        L(f, \dissec) \leq L(f, \dissec')
    \end{equation*}
    Further, if there exists more than one point in $\dissec' \backslash \dissec$, we can use the result iteratively and add each point one-by-one.
\end{proof}
\begin{lemma}
    Suppose that $\dissec_1$ and $\dissec_2$ are two dissections. Then:
    \begin{equation*}
        L(f, \dissec_1) \leq U(f, \dissec_2)
    \end{equation*}
    \label{lemUpperSumAlwaysGreater}
\end{lemma}
\begin{proof}
    Consider $\dissec = \dissec_1 \cup \dissec_2$. Then this is a refinement of both $\dissec_1$ and $\dissec_2$. Using lemma~\ref{lemRefinement}:
    \begin{equation*}
        L(f, \dissec_1) \leq L(f, \dissec) \leq U(f, \dissec) \leq U(f, \dissec_2)
    \end{equation*}
\end{proof}
We have also the trivial dissection, $\dissec_0 = \{a, b\}$.
\begin{align*}
    L(f, \dissec_0) &= (b - a) \inf_{x \in [a, b]} f(x) \geq -(b - a)K \\
    U(f, \dissec_0) &= (b - a) \sup_{x \in [a, b]} f(x) \leq (b - a)K
\end{align*}
So we have that:
\begin{align*}
    \subsetselect{U(f, \dissec)}{\dissec \text{ is a dissection}} \text{ is bounded above.}\\
    \subsetselect{L(f, \dissec)}{\dissec \text{ is a dissection}} \text{ is bounded below.}
\end{align*}
Note also that, by considering the trivial dissection, neither set is empty. We can therefore take a supremum and infinimum:
\begin{definition}{Upper integral}
    The \underline{upper integral} of $f$ is:
    \begin{equation*}
        I^*(f) = \inf_{\dissec\text{ in set of dissections}} U(f, \dissec)
    \end{equation*}
\end{definition}
\begin{definition}{Lower integral}
    The \underline{lower integral} of $f$ is:
    \begin{equation*}
        I_*(f) = \sup_{\dissec\text{ in set of dissections}} L(f, \dissec)
    \end{equation*}
\end{definition}
Note that these are well-defined for any bounded function. We can also bound these in the following way by lemma~\ref{lemUpperSumAlwaysGreater}:
\begin{equation*}
    (b - a)\inf_{x \in [a, b]}f(x) \leq I_*(f) \leq I^*(f) \leq (b - a) \sup_{x \in [a, b]}f(x)
\end{equation*}
\begin{definition}{Integrable function}
    A bounded function $f : [a, b] \mapsto \R$ is \underline{Riemann integrable}, or just\\\underline{integrable}, if the lower integral of $f$ is equal to the upper integral of $f$. In this case, we define the integral to be this common limit:
    \begin{equation*}
        \int_a^b f(x) dx = I_*(f) = I^*(f)
    \end{equation*}
\end{definition}
\section{Examples of Integrable Functions}
\begin{example}
    For ease, define $\dissec_k = \{0, \frac{1}{k}, \frac{2}{k}, \cdots, 1\}$, the uniform dissection.
    $f : [0, 1] \mapsto \R, f(x) = x$.
    Then consider $U(f, \dissec_k)$:
    \begin{align*}
        U(f, \dissec_k) &= \sum_{j = 1}^k \left(\frac{j}{j} - \frac{j - 1}{k}\right) \sup_{x \in [\frac{j - 1}{k}, \frac{j}{k}]} f(x) \\
        &= \sum_{j = 1}^k \frac{1}{k} \frac{j}{k} \\
        &= \frac{1}{k^2} \frac{1}{2} k(k + 1) \\
        &= \frac{1}{2} + \frac{1}{2k}
    \end{align*}
    Similarly, the lower sum is:
    \begin{equation*}
        L(f, \dissec_k) = \frac{1}{2} - \frac{1}{2k}
    \end{equation*}
    We deduce that the upper integral is bounded above by the infinimum over $k$ of the upper sums. This is $\frac{1}{2}$. Further, the lower integral is bounded below by the supremum over $k$ of the lower sum. This is $\frac{1}{2}$. Now, by the inequalities already established, the upper and lower integrals must both be $\frac{1}{2}$.
\end{example}
\begin{example}
    Consider now the function:
    \begin{align*}
        f : [0, 1] &\mapsto \R \\
        x &\mapsto
        \begin{cases}
            1 & x \in \Q \\
            0 & x \notin \Q 
        \end{cases}
    \end{align*}
    $f$ is not integrable. For any $\dissec$, every upper sum gives $1$ (there is always a rational number between any two real numbers), and every lower sum gives $0$ (there is always an irrational number between two real numbers). Therefore, $I^* = 1$ and $I_* = 0$.
\end{example}
\section{Integrability of General Functions}
\begin{theorem}[Riemann's Integrability Criterion]
    A bounded function $f : [a, b] \mapsto \R$ is integrable if and only if for all $\epsilon > 0$, there exists a dissection $\dissec$ such that:
    \begin{equation*}
        U(f, \dissec) - L(f, \dissec) < \epsilon
    \end{equation*}
    \label{thmRiemannCriterion}
\end{theorem}
\begin{proof}
    \begin{proofdirection}{$\Rightarrow$}{Assume that there exist $U(f, \dissec) - L(f, \dissec) < \epsilon$}
        For any dissection $\dissec$, we have that:
        \begin{equation*}
            0 \leq I^*(f) - I_*(f) \leq U(f, \dissec) - L(f, \dissec)
        \end{equation*}
        Now we have the following bound:
        \begin{equation*}
            0 \leq I^*(f) - I_*(f) < \epsilon
        \end{equation*}
        And since $\epsilon$ is arbitrary, we must have that $I^*(f) = I_*(f)$, and $f$ is integrable.
    \end{proofdirection}
    \begin{proofdirection}{$\Leftarrow$}{Assume that $f$ is integrable.}
        Then for any $\epsilon$, there exist $\dissec_1, \dissec_2$ such that:
        \begin{align*}
            U(f, \dissec_1) &\leq I^*(f) + \frac{\epsilon}{2} \\
            L(f, \dissec_2) &\geq I_*(f) - \frac{\epsilon}{2}
        \end{align*}
        Therefore consider $\dissec = \dissec_1 \cup \dissec_2$. Therefore, by lemma~\ref{lemRefinement}
        \begin{align*}
            U(f, \dissec) &\leq U(f, \dissec_1) \\
            L(f, \dissec) &\geq L(f, \dissec_2)
        \end{align*}
        Which implies:
        \begin{align*}
            0 \leq U(f, \dissec) - L(f, \dissec) &\leq U(f, \dissec_1) - L(f, \dissec_2) \\
            &\leq I^*(f) - I_*(f) + \epsilon \\
            &= \epsilon
        \end{align*}
    \end{proofdirection}
\end{proof}
We can use this criterion to show that all monotone and all continuous functions are integrable.

\begin{remark}
    If $f : [a, b] \mapsto \R$ is monotone then it is bounded by its values at the endpoints. If $f$ is continuous, we can use theorem~\ref{thmExtremeValue} to show that it also is bounded.
\end{remark}
\begin{theorem}
    Suppose $f : [a, b] \mapsto \R$ is monotone. Then $f$ is integrable.
\end{theorem}
\begin{proof}
    If $f$ is decreasing, re-label $-f$ to be $f$. Therefore assume WLOG that $f$ is increasing.
    
    For any given dissection $\dissec$,
    \begin{align*}
        \sup_{x \in [x_{j-1}, x_j]} f(x) &= f(x_j) \\
        \inf_{x \in [x_{j-1}, x_j]} f(x) &= f(x_{j-1}) 
    \end{align*}
    Therefore:
    \begin{align*}
        U(f, \dissec) - L(f, \dissec) &= \sum_{j = 1}^n (x_j - x_{j - 1}) \sup_{x \in [x_{j-1}, x_j]} f(x) \\
        &- \sum_{j = 1}^n (x_j - x_{j - 1}) \inf_{x \in [x_{j-1}, x_j]} f(x) \\
        &= \sum_{j = 1}^n (x_j - x_{j - 1}) \left[f(x_j) - f(x_{j - 1})\right]
    \end{align*}
    Then choose the uniform dissection $\dissec_n$ into $n$ equal-length intervals. This gives:
    \begin{align*}
        U(f, &\dissec_n) - L(f, \dissec_n) \\
        &= \sum_{j = 1}^n \left(\frac{b - a}{n}\right) \left[f\left(a + \frac{j(b - a)}{n}\right) - f\left(a + \frac{(j - 1)(b - a)}{n}\right)\right] \\
        &= \frac{b - a}{n} \left[f(b) - f(a)\right] \text{ by telescoping series}
    \end{align*}
    Therefore, taking $n$ sufficiently large we have that this difference can be smaller than any given $\epsilon$. Therefore, by theorem~\ref{thmRiemannCriterion}, $f$ is integrable.
\end{proof}
\begin{theorem}
    Suppose $f : [a, b] \mapsto \R$ is continuous. Then $f$ is integrable.
\end{theorem}
\begin{proof}
    We prove the contrapositive. That is, if $f$ is not integrable then it must be discontinuous.

    Suppose $f$ is not integrable. Therefore, by theorem~\ref{thmRiemannCriterion}, there must exist some $\epsilon_0 > 0$ such that for any dissection $\dissec$, we have that:
    \begin{equation*}
        U(f, \dissec) - L(f, \dissec) > \epsilon_0
    \end{equation*}
    We also know that
    \begin{equation*}
        U(f, \dissec) - L(f, \dissec) = \sum_{j = 1}^n (x_j - x_{j - 1})\left[\sup_{x \in [x_{j-1}, x_j]} f(x) - \inf_{x \in [x_{j-1}, x_j]} f(x)\right]
    \end{equation*}
    It therefore follows that for any dissection there exists a $j$ with:
    \begin{equation*}
        \sup_{x \in [x_{j-1}, x_j]} f(x) - \inf_{x \in [x_{j-1}, x_j]} f(x) > \frac{\epsilon_0}{b - a}
    \end{equation*}
    since otherwise the sum cannot be greater than $\epsilon_0$.

    Now let $\dissec_n$ be the uniform dissection of $n$ equal-length intervals. And therefore we have a sub-interval with:
    \begin{equation*}
        \sup_{x \in [a + (j-1)(b-a)/n, a + j(b-1)/n]} f(x) - \inf_{x \in [a + (j-1)(b-a)/n, a + j(b-1)/n]} < \frac{\epsilon_0}{b - a}
    \end{equation*}
    And we deduce that there exist $y_n, z_n$ in the same interval (which gives $|y_n - z_n| \leq \frac{b - a}{n}$) such that:
    \begin{equation}
        f(y_n) - f(z_n) > \frac{\epsilon_0}{b - a}
        \label{eqnFOfSequenceFarApart}
    \end{equation}
    Note also that $(y_n)$ is a bounded sequence and by theorem~\ref{thmBolzanoWeierstrass}, we can extract a convergent subsequence $y_{n_k} \to \eta$ for $\eta \in [a, b]$.

    However, we also know that $|y_{n_k} - z_{n_k}| \leq \frac{b-a}{n_k} \to 0$ as $k \to \infty$, so we must also have that $z_{n_k} \to \eta$. However, the bound in equation~\ref{eqnFOfSequenceFarApart} implies that $f(y_n)$ and $f(z_n)$ do not tend to the same limit. Thus, by theorem~\ref{thmContinuitySequenceDef}, $f$ must be discontinuous at $\eta$.
\end{proof}
\begin{remark}
    With monotone and continuous functions, we have that a large class of functions are integrable. However, there do exist functions outside these classes that are integrable.
\end{remark}
\begin{example}[Thomae's Function]
    Define:
    \begin{align*}
        f : [0, 1] &\mapsto \R \\
        x &\mapsto
        \begin{cases}
            \frac{1}{q} & x = \frac{p}{q} \text{ where } p, q \text{ coprime} \\
            0 & x \notin \Q
        \end{cases}
    \end{align*}
    Then we show that $f$ is integrable. Clearly for any $\dissec, L(f, \dissec) = 0$, so it is sufficient to show that we can find a $\dissec$ with $U(f, \dissec) < \epsilon$.

    Given $\epsilon > 0$, choose $N \in \N$ such that $\frac{1}{N} < \frac{\epsilon}{2}$. Consider:
    \begin{equation*}
        P = \subsetselect{x \in [0, 1]}{f(x) \geq \frac{1}{N}} \subseteq \subsetselect{\frac{p, q}}{1 \leq p, q \leq N}
    \end{equation*}
    Then this is a finite set. Define $P = \{p_1, p_2, \cdots, p_m\}$ and order the $p_i$ to be increasing with $i$. Then choose a dissection $\dissec$ that satisfies the following properties:
    \begin{enumerate}
        \item Each $p_k$ is in some interval $(x_{j - 1}, x_j)$.
        \item For each $k$, the unique sub-interval containing $p_k$ has length at most $\frac{\epsilon}{2m}$.
    \end{enumerate}
    Write $I \in \dissec$ to mean $I = [x_{j-1}, x_j]$ for some $j$. Write also $|I| = (x_j - x_{j-1})$.

    Then the upper sum is:
    \begin{align*}
        U(f, \dissec) &= \sum_{I \in \dissec} |I| \sup_{x \in I} f(x) \\
        &= \sum_{I \in \dissec, I \cap P \neq \emptyset} |I| \sup_{x \in I} f(x) \sum_{I \in \dissec, I \cap P = \emptyset} |I| \sup_{x \in I} f(x) \\
        &\leq \sum_{I \in \dissec, I \cap P \neq \emptyset} \frac{\epsilon}{2M} + \sum_{I \in \dissec, I \cap P \neq \emptyset} \frac{|I|}{N} \\
        &\leq M \frac{\epsilon}{2M} + \frac{1}{N} < \epsilon
    \end{align*}
\end{example}
\begin{remark}
    We have seen that Thomae's Function is an integrable function that has discontinuities (it is continuous at every irrational, but discontinuous at every rational) and is not monotone.
\end{remark}
\section{Properties of the Integral}
\subsection{Properties of Suprema and Infinima}
\begin{propositions}{
        Suppose that $I = [a, b]$ and $f, g : I \mapsto \R$ are bounded. Then:
        \label{propsSupremumProps}
    }   
    \item If $f(x) \leq g(x)~\forall x \in I$, then
        \begin{equation*}
            \sup_{x \in I} f(x) \leq \sup_{x \in I} g(x) \text{ and } \inf_{x \in I} f(x) \leq \inf_{x \in I} g(x)
        \end{equation*}
        \label{propSupPreservesOrdering}
    \item $\sup_{x \in I} f(x) = -\inf_{x \in I} f(x)$. \label{propSupOfNegative}
    \item If $\lambda > 0$, then:
        \begin{equation*}
            \sup_{x \in I} \lambda f(x) = \lambda \sup_{x \in I} f(x) \text{ and } \sup_{x \in I} \lambda f(x) = \lambda \sup_{x \in I} f(x)
        \end{equation*}
        \label{propSupConstant}
    \item Summation:
        \begin{align*}
            \sup_{x \in I} (f(x) + g(x)) &\leq \sup_{x \in I} f(x) + \sup_{x \in I} g(x) \\
            \inf{x \in I} (f(x) + g(x)) &\geq \inf{x \in I} f(x) + \inf{x \in I} g(x) \\
        \end{align*}
        \label{propSupSum}
    \item Taking absolute values:
        \begin{equation*}
            \sup_{x \in I} |f(x)| - \inf_{x \in I} |f(x)| \leq \sup_{x \in I} f(x) - \inf_{x \in I} f(x)
        \end{equation*}
        \label{propSupAbs}
    \item Squaring:
        \begin{equation*}
            \sup_{x \in I} (f(x))^2 - \inf_{x \in I} (f(x))^2 \leq 2 \left(\sup_{x \in I} |f|\right) \left(\sup_{x \in I} f(x) - \inf_{x \in I} f(x) \right)
        \end{equation*}
        \label{propSupSquare}
\end{propositions}
\begin{proof}[of propositions~\ref{propSupSum} to \ref{propSupSquare}]
    \begin{enumerate}
        \setcounter{enumi}{3} % Start at 4
        \item For any $x \in I$,
            \begin{equation*}
                f(x) \leq \sup_{x \in I} f(x),~g(x) \leq \sup_{x \in I} g(x)
            \end{equation*}
            So:
            \begin{equation*}
                f(x) + g(x) \leq \sup_{x \in I} f(x) + \sup_{x \in I} g(x)
            \end{equation*}
            But this means we can take the supremum of the LHS:
            \begin{equation*}
                \sup_{x \in I} (f(x) + g(x)) \leq \sup_{x \in I} f(x) + \sup_{x \in I} g(x)
            \end{equation*}
            And the same method is used for infima. Alternatively, we can apply proposition~\ref{propSupOfNegative} to $-(f + g)$.
        \item We consider 3 cases:
            \begin{case}{$f(x)$ non-negative on $I$}
                Then $|f(x)| = f(x)$:
                \begin{equation*}
                    \sup_{x \in I} |f(x)| - \inf_{x \in I} |f(x)| = \sup_{x \in I} f(x) - \inf_{x \in I} f(x)
                \end{equation*}
            \end{case}
            \begin{case}{$f(x)$ non-positive on $I$}
                Then $|f(x)| = -f(x)$.
                \begin{align*}
                    \sup_{x \in I} |f(x)| - \inf_{x \in I} |f(x)| &= \sup_{x \in I} (-f(x)) - \inf_{x \in I} (-f(x)) \\
                    &= -\inf_{x \in I} f(x) + \sup_{x \in I} f(x)
               \end{align*}
            \end{case}
            \begin{case}{$f(x)$ takes values either side of 0}
                That is, $\inf_{x \in I} f(x) < 0 < \sup_{x \in I} f(x)$.
                \begin{align*}
                    \sup_{x \in I} |f(x)| - \inf_{x \in I} |f(x)| &\leq \sup_{x \in I} |f(x)| \text{ by negativity of inf.} \\
                    &= \max\{\sup_{x \in I} f(x), \sup_{x \in I} (-f(x))\} \\
                    &\leq \sup_{x \in I} f(x) + \sup_{x \in I} (-f(x)) \\
                    &= \sup_{x \in I} f(x) - \inf_{x \in I} f(x)
                \end{align*}
            \end{case}
        \item Consider difference of squares:
            \begin{align*}
                f(x)^2 - f(y)^2 &= (f(x) + f(y))(f(x) - f(y)) \\
                &\leq 2\sup_{x \in I} |f| \left(\sup_{x \in I} f(x) - \inf_{x \in I} f(x)\right)
            \end{align*}
            Then taking the supremum over $x$ and $y$ of the left-hand side gives:
            \begin{equation*}
                \sup_{x \in I}(f(x)^2) + \inf_{x \in I} (f(x)^2) \leq 2 \sup_{x \in I} |f| \left(\sup_{x \in I} f(x) - \inf_{x \in I} f(x)\right)
            \end{equation*}
    \end{enumerate}
\end{proof}
\subsection{Properties of Upper and Lower Sums}
These translate directly into results for upper and lower sums:
\begin{propositions}{
        Consider a dissection $\dissec$ and bounded functions $f, g : [a, b] \mapsto \R$.
        \label{propsUpperSumProps}
    }
    \item If $f(x) \leq g(x)$ for all $x$ in $[a, b]$ then:
        \begin{equation*}
            U(f, \dissec) \leq U(g, \dissec) \text{ and } L(f, \dissec) \leq L(g, \dissec)
        \end{equation*}
        \label{propupperSumPreservesOrdering}
    \item $L(-f, \dissec) = -U(f, \dissec)$. \label{propUpperSumNegative}
    \item If $\lambda \in \R, \lambda > 0$,
        \begin{equation*}
            U(\lambda f, \dissec) = \lambda U(f, \dissec) \text{ and } L(\lambda f, \dissec) = \lambda L(f, \dissec)
        \end{equation*}
        \label{propUpperSumConstant}
    \item Summation is such that:
        \begin{align*}
            U(f + g, \dissec) &\leq U(f, \dissec) + U(g, \dissec) \\
            U(f + g, \dissec) &\geq L(f, \dissec) + L(g, \dissec)
        \end{align*}
        \label{propUpperSumSum}
    \item Absolute values are such that:
        \begin{equation*}
            U(|f|, \dissec) - L(|f|, \dissec) \leq U(f, \dissec) - L(f, \dissec)
        \end{equation*}
        \label{propUpperSumAbs}
    \item Squaring is such that:
        \begin{equation*}
            U(f^2, \dissec) - L(f^2, \dissec) \leq 2\sup_{x \in [a, b]} |f| \left(U(f, \dissec) - L(f, \dissec)\right)
        \end{equation*}
        \label{propUpperSumSquare}
\end{propositions}
\begin{proof}
    Recall that:
    \begin{align*}
        U(f, \dissec) &= \sum_{i = 1}^n \left(x_i - x_{i - 1}\right) \sup_{x \in [x_{i - 1}, x_i]} f(x) \\
        L(f, \dissec) &= \sum_{i = 1}^n \left(x_i - x_{i - 1}\right) \inf_{x \in [x_{i - 1}, x_i]} f(x)
    \end{align*}
    And use the results in proposition~\ref{propsSupremumProps}, noting also that:
    \begin{equation*}
        \sup_{x \in [x_{i - 1}, x_i]} |f(x)| \leq \sup_{x \in [a, b]} |f(x)|
    \end{equation*}
    for use in the proof of \ref{propUpperSumSquare}.
\end{proof}
\subsection{Integral properties}
\begin{theorem}
    Let $f, g : [a, b] \mapsto \R$ be integrable (and therefore bounded).
    \begin{enumerate}
        \item If $f(x) \leq g(x)$ on $[a, b]$, then:
	    \begin{equation*}
		\int_a^b f(x) dx \leq \int_a^b g(x) dx
	    \end{equation*}
	\item If $\lambda$ is a real number then $\lambda f$ is integrable, and:
	    \begin{equation*}
		\int_a^b \lambda f(x) dx = \lambda \int_a^b f(x) dx
	    \end{equation*}
	\item $f + g$ is integrable, and:
	    \begin{equation*}
		\int_a^b (f(x) + g(x)) dx = \int_a^b f(x) dx + \int_a^b g(x) dx
	    \end{equation*}
	\item $|f|$ is integrable, and:
	    \begin{equation*}
		\left|\int_a^b f(x) dx\right| \leq \int_a^b |f(x)| dx
	    \end{equation*}
	\item The product $f(x) g(x)$ is integrable.
    \end{enumerate}
\end{theorem}
\begin{proof}
    \begin{enumerate}
    	\item For any $\dissec$ we have:
            \begin{align*}
                L(f, \dissec) &\leq L(g, \dissec) \text{ by proposition~\ref{propupperSumPreservesOrdering}} \\
                &\leq I_*(g) \\
                \therefore I_*(f) &\leq I_*(g)
            \end{align*}
            But since $f$ and $g$ are integrable, the lower integrals are equal to the integral, and we have the required result.
        \item First consider $\lambda > 0$. Since $f$ is integrable, for any $\epsilon > 0$ there exists a dissection $\dissec$ with:
            \begin{align*}
                U(f, \dissec) &\leq \int_a^b f(x) + \epsilon \\
                L(f, \dissec) &\geq \int_a^b f(x) - \epsilon
            \end{align*}
            And therefore:
            \begin{align*}
                \lambda \int_a^b f(x) dx - \lambda \epsilon &\leq L(\lambda f, \dissec) \text{by \ref{propUpperSumConstant}.} \\
                &\leq I_*(f) \\
                &\leq I^*(f) \\
                &\leq U(\lambda f, \dissec) \\
                &\leq \lambda \int_a^b f(x) dx + \lambda \epsilon \text{ by \ref{propUpperSumConstant}}
            \end{align*}
            But now, since $\epsilon$ is arbitrary:
            \begin{equation*}
                I_*(\lambda f) = I^*(\lambda f) = \lambda \int_a^b f(x) dx
            \end{equation*}
            Now we consider $\lambda = -1$ and the same dissection $\dissec$:
            \begin{align*}
                \int_a^b f(x) dx - \epsilon &\leq L(f, \dissec) \\
                &= -U(-f, \dissec) \text{ by \ref{propUpperSumNegative}}
            \end{align*}
            And therefore:
            \begin{equation*}
                U(-f, \dissec) \leq -\int_a^b f(x) dx + \epsilon
            \end{equation*}
            And:
            \begin{align*}
                \int_a^b f(x) dx + \epsilon &\geq U(f, \dissec) \\
                &= -L(-f, \dissec) \\
                \therefore -\int_a^b f(x) dx - \epsilon &\leq L(-f, \dissec)
            \end{align*}
            Then combining both results:
            \begin{equation*}
                -\int_a^b f(x) dx - \epsilon \leq I_*(-f) \leq I^*(-f) \leq -\int_a^b f(x) dx + \epsilon
            \end{equation*}
            And again we have that $\epsilon$ is arbitrary and so:
            \begin{equation*}
                I_*(-f) = I^*(-f) = -\int_a^b f(x) dx
            \end{equation*}
            Combining both these results allows for negative $\lambda$ also.
        \item Since $f$ and $g$ are integrable, for any $\epsilon > 0$ there exists dissections $\dissec_1, \dissec_2$ such that:
            \begin{align*}
                \int_a^b f(x) dx - \epsilon &\leq L(f, \dissec_1) \\
                &\leq U(f, \dissec_1) \\
                &\leq \int_a^b f(x) dx + \epsilon
            \end{align*}
            And the same for $g$:
            \begin{align*}
                \int_a^b g(x) dx - \epsilon &\leq L(g, \dissec_2) \\
                &\leq U(g, \dissec_2) \\
                &\leq \int_a^b g(x) dx + \epsilon
            \end{align*}
            Now define $\dissec = \dissec_1 \cup \dissec_2$:
            \begin{align*}
                \int_a^b f(x) dx + \int_a^b g(x) dx - 2\epsilon &\leq L(f, \dissec_1) + L(g, \dissec_2) \\
                &\leq L(f, \dissec) + L(g, \dissec) \\
                &\leq L(f + g, \dissec) \text{ by \ref{propUpperSumSum}.} \\
                &\leq I_*(f + g)
            \end{align*}
            And similarly:
            \begin{equation*}
                \int_a^b f(x) dx + \int_a^b g(x) + 2\epsilon dx \geq I^*(f + g)
            \end{equation*}
            So we have shown:
            \begin{align*}
                \int_a^b f(x) dx &+ \int_a^b g(x) dx - 2\epsilon \leq I_*(f + g)\\
                &\leq I^*(f + g) \leq \int_a^b f(x) dx + \int_a^b g(x) dx + 2\epsilon
            \end{align*}
            And since $\epsilon$ is arbitrary, $f + g$ is integrable with integral:
            \begin{equation*}
                \int_a^b f(x) dx + \int_a^b g(x) dx = \int_a^b (f(x) + g(x)) dx
            \end{equation*}
        \item Since $f$ is integrable, for any $\epsilon > 0$ there exists a dissection $\dissec$ such that:
            \begin{align*}
                \epsilon &> U(f, \dissec) - L(f, \dissec) \\
                &\leq U(|f|, \dissec) - L(|f|, \dissec) \text{ by proposition~\ref{propUpperSumAbs}}
            \end{align*}
            And therefore $|f|$ is integrable by theorem~\ref{thmRiemannCriterion}.

            Now note that:
            \begin{align*}
                -|f(x) &\leq f(x) \leq |f(x)|~\forall x \in [a, b] \\
                \implies -\int_a^b |f(x)| &\leq \int_a^b f(x) dx \leq \int_a^b |f(x)| dx
            \end{align*}
            by parts 1 and 2 above, as required.
        \item Observing that:
            \begin{equation*}
                fg = \frac{1}{2} \left[(f + g)^2 - (f - g)^2\right]
            \end{equation*}
            it suffices to show that $f^2$ is integrable (and therefore so is $g^2$).

            Suppose $|f| \leq K$. Then given $\epsilon > 0$ we can find a dissection such that:
            \begin{equation*}
                U(f, \dissec) - L(f, \dissec) < \frac{\epsilon}{2K}
            \end{equation*}
            So by proposition~\ref{propUpperSumSquare},
            \begin{align*}
                U(f^2, \dissec) - L(f^2, \dissec) &\leq 2 \sup_{x \in [a, b]} |f(x)| \left(U(f, \dissec) - L(f, \dissec)\right) \\
                &<\epsilon
            \end{align*}
            And therefore $f^2$ is integrable by theorem~\ref{thmRiemannCriterion}.
    \end{enumerate}
\end{proof}
\end{document}
