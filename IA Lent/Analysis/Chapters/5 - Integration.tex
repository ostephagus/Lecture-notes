\documentclass[../Main.tex]{subfiles}

\begin{document}
\section{Defining Integration}
We are often introduced to integration as an inverse operation to differentiation. This is fine, if we know (or can guess) an antiderivative. However, this does not serve well as a definition when we can't. The idea is that the integral should be the area under a graph. We will consider the definite integral from $a$ to $b$ to be the signed area under the graph of $f(x)$. The problem with this, however, is to define such an area in terms of known concepts.\par
We first assume that $f$ is a bounded, real-valued function over the closed interval $[a, b]$. That is,
$f : [a, b] \mapsto \R$ and there exists some bound $K$ such that $|f(x)| \leq K$ for all $x$ in $[a, b]$.
\begin{definition}{Dissection}
    A \underline{Dissection}, $\dissec$, of $[a, b]$ is a finite subset of $[a, b]$ containing the endpoints $a, b$. We write $\dissec = \{x_0, x_1, \cdots, x_n\}$ with $a = x_0 < x_1 < \cdots < x_n < b$.
\end{definition}
Associated to such a dissection are the upper and lower sums given by:
\begin{equation}
    U(f, \dissec) = \sum_{j = 1}^n (x_j - x_{j - 1}) \sup_{x \in [x_{j-1}, x_j]} f(x)
    \label{eqnUpperSum}
\end{equation}
\begin{equation}
    L(f, \dissec) = \sum_{j = 1}^n (x_j - x_{j - 1}) \inf_{x \in [x_{j-1}, x_j]} f(x)
    \label{eqnLowerSum}
\end{equation}
Note that the suprema and infinima are defined here, since $f$ is bounded.\par
Clearly we have that $L(f, \dissec) \leq U(f, \dissec)$.
\begin{lemma}
    Suppose that $\dissec$ and $\dissec'$ are dissections. Suppose that $\dissec \subseteq \dissec'$. Then:
    \begin{equation*}
        L(f, \dissec) \leq L(f, \dissec') \leq U(f, \dissec') \leq U(f, \dissec)
    \end{equation*}
    \label{lemRefinement}
\end{lemma}
\begin{remark}
    In this case, where $\dissec \leq \dissec'$, we say that $\dissec'$ is a \underline{refinement} of $\dissec$.
\end{remark}
\begin{proof}
    Let $\dissec = \{x_0, \cdots, x_n\}$. First consider $\dissec' = \dissec \cup \{y\}$, where $y \notin\dissec$.\par
    Then $y \in (x_{r - 1}, x_r)$ for some $r \in \{1, \cdots, n\}$.\par
    We must have that:
    \begin{equation*}
        \sup_{x \in [x_{r - 1}, y]} f(x), \sum_{x \in [y, x_r]} f(x) \leq \sup_{x \in [x_{r - 1}, x_r]} f(x)
    \end{equation*}
    Therefore:
    \begin{align*}
        &(y - x_{r - 1}) \sup_{x \in [x_{r - 1}, y]}f(x) + (x_r - y) \sup_{x \in [y, x_r]} f(x) \\
        &\leq (y - x_{r - 1}) \sup_{x \in [x_{r - 1}, x_r]}f(x) + (x_r - y) \sup_{x \in [x_{r - 1}, x_r]} f(x) \\
        &\leq (x_r - x_{r - 1}) \sup_{x \in [x_{r - 1}, x_r]}f(x) \\
    \end{align*}
    That is, $U(f, \dissec') \leq U(f, \dissec)$.\par
    Similarly, we know that:
    \begin{equation*}
        \inf_{x \in [x_{r - 1}, y]} f(x), \inf_{x \in [y, x_r]} f(x) \geq \inf_{x \in [x_{r - 1}, x_r]} f(x)
    \end{equation*}
    and we can use the above idea to show that:
    \begin{equation*}
        L(f, \dissec) \leq L(f, \dissec')
    \end{equation*}
    Further, if there exists more than one point in $\dissec' \backslash \dissec$, we can use the result iteratively and add each point one-by-one.
\end{proof}
\begin{lemma}
    Suppose that $\dissec_1$ and $\dissec_2$ are two dissections. Then:
    \begin{equation*}
        L(f, \dissec_1) \leq U(f, \dissec_2)
    \end{equation*}
    \label{lemUpperSumAlwaysGreater}
\end{lemma}
\begin{proof}
    Consider $\dissec = \dissec_1 \cup \dissec_2$. Then this is a refinement of both $\dissec_1$ and $\dissec_2$. Then, using lemma~\ref{lemRefinement}:
    \begin{equation*}
        L(f, \dissec_1) \leq L(f, \dissec) \leq U(f, \dissec) \leq U(f, \dissec_2)
    \end{equation*}
\end{proof}
We have also the trivial dissection, $\dissec_0 = \{a, b\}$.
\begin{align*}
    L(f, \dissec_0) &= (b - a) \inf_{x \in [a, b]} f(x) \geq -(b - a)K \\
    U(f, \dissec_0) &= (b - a) \sup_{x \in [a, b]} f(x) \leq (b - a)K
\end{align*}
So we have that:
\begin{align*}
    \subsetselect{U(f, \dissec)}{\dissec \text{ is a dissection}} \text{ is bounded above.}\\
    \subsetselect{L(f, \dissec)}{\dissec \text{ is a dissection}} \text{ is bounded below.}
\end{align*}
Note also that, by considering the trivial dissection, neither set is empty. We can therefore take a supremum and infinimum:
\begin{definition}{Upper integral}
    The \underline{upper integral} of $f$ is:
    \begin{equation*}
        I^*(f) = \inf_{\dissec\text{ in set of dissections}} U(f, \dissec)
    \end{equation*}
\end{definition}
\begin{definition}{Lower integral}
    The \underline{lower integral} of $f$ is:
    \begin{equation*}
        I_*(f) = \sup_{\dissec\text{ in set of dissections}} L(f, \dissec)
    \end{equation*}
\end{definition}
Note that these are well-defined for any bounded function. We can also bound these in the following way by lemma~\ref{lemUpperSumAlwaysGreater}:
\begin{equation*}
    (b - a)\inf_{x \in [a, b]}f(x) \leq I_*(f) \leq I^*(f) \leq (b - a) \sup_{x \in [a, b]}f(x)
\end{equation*}
\begin{definition}{Integrable function}
    A bounded function $f : [a, b] \mapsto \R$ is \underline{Riemann integrable}, or just\\\underline{integrable}, if the lower integral of $f$ is equal to the upper integral of $f$. In this case, we define the integral to be this common limit:
    \begin{equation*}
        \int_a^b f(x) dx = I_*(f) = I^*(f)
    \end{equation*}
\end{definition}
\section{Examples of Integrable Functions}
\begin{example}
    For ease, define $\dissec_k = \{0, \frac{1}{k}, \frac{2}{k}, \cdots, 1\}$, the uniform dissection.
    $f : [0, 1] \mapsto \R, f(x) = x$.
    Then consider $U(f, \dissec_k)$:
    \begin{align*}
        U(f, \dissec_k) &= \sum_{j = 1}^k \left(\frac{j}{j} - \frac{j - 1}{k}\right) \sup_{x \in [\frac{j - 1}{k}, \frac{j}{k}]} f(x) \\
        &= \sum_{j = 1}^k \frac{1}{k} \frac{j}{k} \\
        &= \frac{1}{k^2} \frac{1}{2} k(k + 1) \\
        &= \frac{1}{2} + \frac{1}{2k}
    \end{align*}
    Similarly, the lower sum is:
    \begin{equation*}
        L(f, \dissec_k) = \frac{1}{2} - \frac{1}{2k}
    \end{equation*}
    We deduce that the upper integral is bounded above by the infinimum over $k$ of the upper sums. This is $\frac{1}{2}$. Further, the lower integral is bounded below by the supremum over $k$ of the lower sum. This is $\frac{1}{2}$. Now, by the inequalities already established, the upper and lower integrals must both be $\frac{1}{2}$.
\end{example}
\begin{example}
    Consider now the function:
    \begin{align*}
        f : [0, 1] &\mapsto \R \\
        x &\mapsto
        \begin{cases}
            1 & x \in \Q \\
            0 & x \notin \Q 
        \end{cases}
    \end{align*}
    $f$ is not integrable. For any $\dissec$, every upper sum gives $1$ (there is always a rational number between any two real numbers), and every lower sum gives $0$ (there is always an irrational number between two real numbers). Therefore, $I^* = 1$ and $I_* = 0$.
\end{example}
\end{document}