\documentclass[../Main.tex]{subfiles}

\begin{document}
We want to consider functions defined by power series of the form:
\begin{equation}
    f(z) = \sum_{n=0}^\infty a_n (z - z_0)^n
\end{equation}
Where $a_n$, $z$ and $z_0$ are complex numbers. By translation, we can assume $z_0 = 0$.
\section{Convergence of power series}
\subsection{Radius of Convergence}
This series may not converge for all $z \in \C$. The first task, therefore, is to understand the set of points for which a function of this form converges.
\begin{lemma}
    If a power series $\sum_{n=0}^\infty a_n z^n$ converges, and $|w| < |z|$, then the power series:
    \begin{equation*}
        \sum_{n=0}^\infty a_n w^n
    \end{equation*}
    converges absolutely.
    \label{lemConvergenceRegionIsDisc}
\end{lemma}
\begin{proof}
    Since $\sum_{n=0}^\infty a_n z^n$ converges, $a_n z^n \to 0$.\par
    Therefore $\exists K > 0$ such that $|a_n z^n| \leq K$ for all $n$.\par
    Now consider, for $z \neq 0$ (note that the case $z = 0$ is trivial),
    \begin{align*}
        |a_n w^n| &= |a_n z^n| \left|\frac{w}{z}\right| \\
        &\leq K\rho^n
    \end{align*}
    where $\rho = \frac{|w|}{|z|}$, and is less than $1$ by assumption. Therefore, by the comparison to the sum $\sum_{n=0}^\infty K\rho^n$, the required power series converges.
\end{proof}
\begin{remark}
    We will use this result to show that every power series has a well-defined radius of convergence.
\end{remark}
\begin{theorem}
    Let $\sum_{n=0}^\infty a_n z^n$ be a power series. Then there exists some $R \in [0, \infty]$, the \underline{radius of convergence}, such that the series converges absolutely for $|z| < R$, and diverges for $|z| > R$.
    \label{thmRadiusOfConvergence}
\end{theorem}
\begin{proof}
    Let $A = \subsetselect{r \geq 0}{\exists z \in \C, |z| = r, \sum_{n=0}^\infty a_n z^n \text{ converges}}$\par
    Clearly $0 \in A$, so $A$ is non-empty. Let $R = \sup{A}$. If $A$ is not bounded above, define $R = \infty$.\par
    From the definition of $A$, $\sum_{n=0}^\infty a_n z^n$ diverges for $|z| > R$.\par
    Consider now $w \in \C$ with $|w| < R$. Then there exists $r \in A$ with $|w| < r$. This means there exists $z \in \C$ with $|z| = r$ and $\sum_{n=0}^\infty a_n z^n$ convergent. Then by Lemma~\ref{lemConvergenceRegionIsDisc}, $\sum_{n=0}^\infty a_n w^n$ converges absolutely.
\end{proof}
\begin{remarks}
    \item If $R = 0$, then the sum converges only for $z = 0$.
    \item If $R = \infty$, then the sum converges for any $z \in \C$.
    \item If $R \in \R^+ \backslash \{0\}$, the theorem gives no information about $|z| = R$. In this case, we can have any possible subset of this circle convergent.
\end{remarks}
We can get a useful way to find $R$.
\begin{lemma}
    If $\left|\frac{a_{n+1}}{a_n}\right| \to l$ as $n \to \infty$, then $R = \frac{1}{l}$/ If $l = \infty$, then $R = 0$ and if $l = 0$ then $R = \infty$.
    \label{lemRatioReciprocalRadius}
\end{lemma}
\begin{proof}
    Use the ratio test. Consider:
    \begin{align*}
	\lim_{n \to \infty} \left|\frac{a_{n+1} z^{n+1}}{a_n z^n}\right| = \lim_{n \to \infty} \left|\frac{a_{n+1}}{a_n}\right| |z| = l|z|
    \end{align*}
    Therefore by the ratio test, if $l|z| < 1$ the power series converges absolutely, and if $l|z| > 1$ then the power series diverges.\par
    The cases where $l = \infty$ and $l = 0$ are similar.
\end{proof}
\begin{lemma}
    If $|a_n|^\frac{1}{n} \to l$ then $R = \frac{1}{l}$.
    \label{lemRootReciprocalRadius}
\end{lemma}
The proof of this lemma is identical to that of lemma~\ref{lemRatioReciprocalRadius}, using the root test instead of the ratio test.
\begin{examples}{}
    \item $\sum_{n=0}^\infty \frac{z^n}{n!}$ has ratio of terms going to $0$, so the radius of convergence is $\infty$ and the series converges on the whole of $\C$.
    \item $\sum_{n=0}^\infty n! z^n$ has ratio of terms going to $\infty$, so the radius of convergence is $0$, and the series diverges everywhere except $z = 0$.
    \item $\sum_{n=0}^\infty z^n$ has radius of convergence $1$. Note that in this case the series diverges everywhere on the boundary circle.
    \item $\sum_{n=1}^\infty \frac{z^n}{n^2}$ has convergence for $|z| \leq 1$ by comparison to the series in $\frac{1}{n^2}$. Therefore, the series converges on the boundary circle.
    \item $\sum_{n=1}^\infty \frac{z^n}{n}$ has radius of convergence 1. For $z = 1$ the series diverges (this is the harmonic series). For $|z| = 1$, $z \neq 1$, consider:
        \begin{align*}
		(1-z)\sum_{n=1}^N \frac{z^n}{n} &= \sum_{n=1}^N \left(\frac{z^n}{n} - \frac{z^{n+1}}{n}\right) \\
		&= \sum_{n=1}^N \left(\frac{z^{n+1}}{n+1} - \frac{z^{n+1}}{n}\right) + z - \frac{z^{N+1}}{N+1} \\
		&= z - z \sum_{n=1}^N \frac{1}{n(n+1)} z^n - \frac{z^{N+1}}{N+1}
	\end{align*}
	And therefore this series converges absolutely by comparison to the sum in $\frac{1}{n(n+1)}$. Therefore, $\sum_{n=1}^\infty \frac{z^n}{n}$ converges for $|z| \leq 1$
\end{examples}
So we have examples of divergence, absolute convergence, and conditional convergence except at a single point. This illustrates the fact that we can find any set of convergence on the circle $|z| = R$.
\subsection{Differentiating a Power Series(*)}
We can show that inside the disc, power series are well-behaved.
\begin{theorem}
    Consider the function:
    \begin{equation*}
        f(z) = \sum_{n = 0}^\infty a_n z^n
    \end{equation*}
    and suppose it has radius of convergence $R$. Then $f$ is differentiable on $\subsetselect{z}{|z| < R}$ with derivative:
    \begin{equation*}
        f'(z) = \sum_{n = 0}^\infty n a_n z^{n-1}
    \end{equation*}
    \label{thmPowerSeriesDerivative}
\end{theorem}
\begin{remark}
    This theorem can be iterated, since the result of a differentiation is a power series, so power series are infinitely differentiable inside their radius of convergence.
\end{remark}
The rest of this subsection is \textbf{non-examinable}.\par
We need some lemmas in order to prove theorem~\ref{thmPowerSeriesDerivative}.
\begin{lemma}
    If $f(z)$ has radius of convergence $R$, then so do:
    \begin{equation*}
        \sum_{n = 0}^\infty n a_n z^{n-1} \text{ and } \sum_{n = 0}^\infty n(n-1) a_n z^{n-2}
    \end{equation*}
    \label{lemPowerSeriesDerivativesConverge}
\end{lemma}
\begin{proof}
    Suppose that $0 < |z| < R$ (note that the result trivially holds when $z = 0$).\par
    Then there exists $r$ such that $|z| < r < R$. We know that $\sum_n a_n r^n$ converges absolutely, so $a_n r^n \to 0$ and there exists some $K$ that bounds the sequence:
    \begin{equation*}
        |a_n r^n| < L~\forall n \geq 0
    \end{equation*}
    Therefore we can bound:
    \begin{align*}
        |n a_n z^{n-1}| &\leq \frac{|a_n r^n|}{|z|} n \left|\frac{z}{r}\right|^n \\
        &\leq \frac{K}{|z|} n \rho^n \text{ for } \rho = \frac{|z|}{r} < 1
    \end{align*}
    But note that this converges by the ratio test: the ratio of terms tends to $\rho < 1$ and therefore the series converges absolutely. Therefore, the required series converges by the comparison test.\par
    Also, suppose that $\sum_n n a_n w^{n-1}$ converges for some $|w| > R$. Then $\sum_n n a_n z^{n-1}$ converges absolutely for some $z$ with $|w| > |z| > R$.
    \begin{equation*}
        |n a_n z^{n-1}| \geq \frac{1}{|z|} |a_n z^n|
    \end{equation*}
    Which implies that $\sum_n a_n z^n$ converges absolutely.\contradiction\par
    So $\sum_n n a_n z^{n-1}$ has radius of convergence $R$. Iterating the same process with this new series gives the second required result.
\end{proof}
\begin{lemma}
    We can bound: 
    \begin{equation*}
        \choose{n}{r} \leq n(n-1)\choose{n-2}{r-2}$ for any $n, r \in \N$, $2 \leq r \leq n
    \end{equation*}
    And further,
    \begin{equation*}
        |(z + h)^n - z^n - nhz^{n-1}| \leq n(n-1)[|z| + |h|]^{n-2} |h|^2
    \end{equation*}
    \label{lemChooseBounds}
\end{lemma}
\begin{proof}
    The first is simple:
    \begin{align*}
        \choose{n}{r} \div \choose{n-2}{r-2} &= \frac{n!}{r!(n-r)!} \times \frac{(n-r)!(r-2)!}{(n-2)!} \\
        &= \frac{n(n-1)}{r(r-1)} \\
        &\leq n(n-1)
    \end{align*}
    as required. Then consider the second expression:
    \begin{align*}
        |(z + h)^n - z^n - nhz^{n-1}| &= \left|\sum_{r-2}^n \choose{n}{r} z^{n-r} h^r\right| \\
        &\leq \sum_{r=2}^n \choose{n}{r} |z|^{n-r} |h|^r \\ 
        &\leq \sum_{r=2}^n n(n-1) \choose{n-2}{r-2} |z|^{n-r} |h|^{r-2} |h|^2 \\ 
        &\leq n(n-1) |h|^2 \left(|h| + |z|\right)^{n-2}
    \end{align*}
    as required.
\end{proof}
Now we can prove the theorem:
\begin{proof}[of theorem~\ref{thmPowerSeriesDerivative}]
    By lemma~\ref{lemPowerSeriesDerivativesConverge}, we can define:
    \begin{equation*}
        g(z) = \sum_{n=0}^\infty na_n z^{n-1}
    \end{equation*}
    for $|z| < R$.\par
    Consider the difference quotient:
    \begin{equation*}
        I = \frac{f(z + h) - f(h) - hg(z)}{h}
    \end{equation*}
    Now, fix $z$ such that $|z| < R$ and assume that $|z| + |h| < r < R$ for some $r$.
    \begin{equation*}
        I = \frac{1}{h} \sum_{n = 0}^\infty \left[(z + h)^n - z^n - hnz^{n-1}\right]a_n
    \end{equation*}
    Then consider its magnitude:
    \begin{align*}
        |I| &= \frac{1}{|h|} \left|\lim_{N \to \infty} \sum_{n = 0}^N a_n\left[(z _ h)^n - z^n - hnz^{n-1}\right]\right| \\
        &= \lim_{N \to \infty} \left|\frac{1}{|h|} \sum_{n = 0}^N a_n\left[(z _ h)^n - z^n - hnz^{n-1}\right]\right| \text{ by continuity of } |z|
    \end{align*}
    Therefore, define:
    \begin{align*}
        I_n &= \frac{1}{n} \sum_{n = 0}^N a_n \left[(z + h)^n - z^n - hnz^{n-1}\right] \\
       |I_n| &\leq \frac{1}{|h|} \sum_{n = 0}^N |a_n| |(z + h)^n - z^n - hnz^{n-1}| \\
       &\leq \frac{1}{h} \sum_{n = 0}^N |a_n| n(n-1) \left(|z| + |h|\right)^{n-2} |h|^2 \text{ by lemma~\ref{lemChooseBounds}}\\
       &\leq |h| \sum_{n = 0}^N n(n-1) |a_n| r^{n-2} \\
       &\leq |h| \sum_{n = 0}^\infty n(n-1) |a_n| r^{n-2} = |h| A_r
    \end{align*}
    and the final sum converges by the assumption $r < R$ and lemma~\ref{lemPowerSeriesDerivativesConverge}.
    So we have that $|I| \leq |h| A_r$, and $|h| A_r \to 0$ as $h \to 0$, as required.
\end{proof}
\section{Familiar Power Series}
In this section, we will define and prove properties of familiar functions such as $\exp, \sin, \cos$.
\subsection{The Exponential Function}
We saw previously that the sum:
\begin{equation*}
    \sum_{n = 0}^\infty \frac{z^n}{n!}
\end{equation*}
We define initially a function $e$:
\begin{align*}
    e : \C &\mapsto \C \\
    z &\mapsto e(z) = \sum_{n = 0}^\infty \frac{z^n}{n!}
\end{align*}
We immediately have that $e$ is differentiable, and also that its derivative is itself.
\begin{proposition}
    \begin{equation*}
        e(a + b) = e(a) e(b)
    \end{equation*}
    \label{propExpAdditivity}
\end{proposition}
\begin{proof}
    To show this, we need the following fact:\par
    \begin{subproof}{If $F : \C \mapsto \C$ satisfies $F'(z) = 0~\forall z \in \C$, then $F$ is constant.}
        Consider $g : \R \mapsto \C$ such that $g(t) = F(tz)$. By the chain rule, $g'(t) = zF'(tz) = 0$.\par
        Then, if we write $g(t) = u(t) + iv(t)$ for real functions $u$ and $v$, we must have that $g'(t) = u'(t) + iv'(t)$, both of which must be 0. Now we must have that $u$ and $v$ are constant, and therefore $g$ is constant, and so is $F$ by corollary~\ref{corDerivativeSign}.
    \end{subproof}
    Now for any $a, b$ consider $F(z) = e(a + b - z) e(z)$. This is differentiable on all $\C$ with derivative:
    \begin{align*}
        F'(z) &= -e'(a + b - z) e(z) + e(a + b - z) - e'(z) \\
        &= -e(a + b - z) e(z) + e(a + b - z) - e(z) = 0\\
    \end{align*}
    Now we have that $F$ is constant, so $F(0) = F(b)$:
    \begin{align*}
        e(a + b) e(0) &= e(a) e(b) \\
        e(a + b) \sum_{n = 0}^\infty \frac{0^n}{n!} &= e(a) e(b) \\
        e(a + b) &= e(a) e(b)
    \end{align*}
\end{proof}
We can also consider the function $e$ with real inputs. There are no complex coefficients in the definition so $e : \R \mapsto \R$.
%TODO: Add individual labels.
\begin{propositions}{
        Consider this function $e : \R \mapsto \R$:
        \label{propsExpRealProps}
    }   
    \item $e$ is everywhere differentiable with derivative itself \label{propExpDifferentiableReal}
    \item $e(x + y) = e(x) e(y)$ \label{propExpAdditivityReal}
    \item $e$ is strictly increasing \label{propExpIncreasing}
    \item $e(x) \to \infty$ as $x \to \infty$, and $e(x) \to 0$ as $x \to -\infty$ \label{propExpLimits}
    \item $e : R \mapsto (0, \infty)$ is a bijection. \label{propExpBijection}
\end{propositions}
\begin{proof}
    The first two claims are already proven.
    \begin{enumerate}
        \setcounter{enumi}{2} %Start at 3
        \item If $x > 0$ then clearly $e(x) = \sum_{n = 0}^\infty \frac{x^n}{n!} > 0$.
            If $x = 0$, $e(0) = 1$ and if $e < 0$, note that $e(x - x) = e(x) e(-x) = 1$, so $e(x)$ must be positive since $e(-x)$ is positive by the first case.
        \item $e'(x) = e(x) > 0$ so $e$ is strictly increasing for all $x$.
        \item For $x \geq 0$, $e(x) = \sum_{n = 0}^\infty \frac{x^n}{n!} > 1 + x$, so if $x \to \infty$, $e(x) \to \infty$. For $x \leq 0$, we use the identity that $e(x) = \frac{1}{e(-x)}$ so if $e(-x) \to \infty$, then $e(x) \to 0$.
        \item Consider the map $e : \R \mapsto (0, \infty)$.
            Injectivity is immediate since $e$ is increasing (see the proof of theorem~\ref{thmInverseFunctionII}). For surjectivity, suppose that $y \in (0, \infty)$. Then due to the limits of $e(x)$ as $x$ tends to positive or negative infinity, there must exist $a$ and $b$ such that:
            \begin{equation*}
                e(a) < y < e(b)
            \end{equation*}
            We can then use theorem~\ref{thmInverseFunctionII} on this restricted domain to show that $e$ is surjective.
    \end{enumerate}
\end{proof}
\end{document}