\documentclass[../Main.tex]{subfiles}

\begin{document}
\section{Examples of Matrix Groups}
Let $M_n(\R) = \{n \times n \text{ real matrices}\}$.\par
We can check the axioms: with a group action of matrix multiplication, it is associative, has an identity matrix, but we need to be careful with inverses.
\begin{lemma}
    Any matrix $A \in M_n(\R)$ has an inverse $A^{-1}$ if and only if $det{A} \neq 0$.
\end{lemma}
For the proof, see IA Vectors and Matrices. We can then define the group of $n \times n$ matrices.
\begin{definition}{General linear group}
    Let $GL_n(\R) = \{A \in M_n(\R) | \det{A} \neq 0\}$. Then this is a group by the above statements.
\end{definition}
We also have some other knowledge about determinants from IA Vectors and Matrices.
\begin{lemma}
    For $A, B \in M_n(\R)$, $\det{A} \det{B} = \det{AB}$
\end{lemma}
Where the proof is again seen in IA Vectors and Matrices. We then have a homomorphism:
\begin{align}
    \det{} : GL_n(\R) &\mapsto \R \label{eqnDetHomism} \\
    A &\mapsto \det{A}\nonumber
\end{align}
Then the kernel of \ref{eqnDetHomism} is another interesting group:
\begin{definition}{Special linear group}
    Let $SL_n(\R) = \ker{(\det{})} = \{A \in M_n(\R) | det{A} = 1\}$
\end{definition}
\underline{Remark:} By theorem \ref{thmIsomorphism}, $SL_n(\R) \normalin GL_n(\R)$, and $GL_n(\R)/SL_n(\R) \cong im(\det{}) \cong \R$\par
This also works for complex matrices. We may define $GL_n{\C}$ and $SL_n{\C}$ in the same way.
\section{Change of Basis}
There is a natural action of $GL_n{\R}$ on $M_n{\R}$:
\begin{align*}
    GL_n{\R} &\actson M_n(\R) \\
    P \cdot A &\mapsto PAP^{-1}
\end{align*}
\begin{proposition}
    Let $\vec{v}$ be an n-dimensional vector space and $\alpha : \vec{v} \mapsto \vec{v}$ be a linear map.\par
    If $A \in M_n(\R)$ represents $\alpha$ in some basis, then the orbit:
    \begin{equation*}
        GL_n(\R)A = \{PAP^{-1} | P \in GL_n(\R)\} 
    \end{equation*}
    Consists of all matrices that represent $\alpha$ in some basis.
\end{proposition}
\begin{proof}
    A basis $|\{\vec{v_1}, \vec{v_2}, \cdots, \vec{v_n}\}$ for $\vec{v}$ defines an isomorphism of vector spaces:
    \begin{align*}
        \phi : \R^n &\mapsto \vec{v} \\
        (\lambda_1, \cdots, \lambda_n) &\mapsto \sum_{i=1}^{n}\lambda_i \vec{v}_i
    \end{align*}
    The claim that $A$ represents a linear map $\alpha$ in this basis means:
    \begin{equation*}
        A = \phi^{-1} \circ \alpha \circ \phi
    \end{equation*}
    Likewise, another basis $\{\vec{u_1}, \cdots \vec{u_n}\}$ for $\vec{v}$ corresponds to another isomorphism:
    \begin{equation*}
        \psi : \R^n \mapsto V
    \end{equation*}
    and B represents alpha in this basis:
    \begin{equation*}
        B = \psi^{-1} \circ \alpha \circ \psi
    \end{equation*}
    So now,
    \begin{align*}
        B &= (\psi^{-1} \circ \phi) \circ A \circ (\phi^{-1} \circ \psi) \\
        &= PAP^{-1}
    \end{align*}
    Where $P$ represents $\phi^{-1} \circ \psi : \R^n \mapsto \R^n$, so $p \in GL_n(\R)$\par
    Thus the set of matrices representing $\alpha$ is contain in the orbit of any one of them under this conjugation action.\par
    Conversely, if $B = PAP^{-1}$ or some $P \in GL_n(\R)$, then setting:
    \begin{equation*}
        \psi = \phi \circ P^{-1} : \R^n \mapsto V
    \end{equation*}
    We see that $\psi$ therefore defines a basis $\{\vec{u_i} = \psi(\vec{e_i})\}$ where $\{\vec{e_i}\}$ is the standard basis in $\R^n$.\par
    Now, since $P^{-1} = \phi^{-1} \circ \psi \implies P = \psi^{-1} \circ \phi$
    \begin{align*}
        B &= PAP^{-1} = \psi^{-1} \circ \phi \circ A \circ \phi^{-1} \circ \psi \\
        &= \psi^{-1} \circ \alpha \psi
    \end{align*}
    So $B$ represents $\alpha$ in this basis.
\end{proof}
\section{\Mobius~Transformations and Matrices}
Recall that multiplication in $\mobgrp$ looked similar to multiplying $2 \times 2$ matrices. This is explained by the following proposition.
\begin{proposition}
    Define the group of complex numbers under multiplication:
    \begin{equation*}
        \Ccross = \left\{
        \begin{pmatrix}
            \lambda & 0 \\
            0 & \lambda
        \end{pmatrix} \subseteq GL_2(\C) | \lambda \neq 0 \right\}
    \end{equation*}
    Then $\Ccross \normalin GL_2(\C)$ and $\mobgrp \cong GL_n(\C) / \Ccross$
\end{proposition}
\begin{proof}
    Define:
    \begin{align*}
        \Phi : GL_n(\C) &\mapsto \mobgrp \\
        \begin{pmatrix}a & b \\ c & d\end{pmatrix} &\mapsto \left(z \mapsto \frac{az + b}{cz + d}\right)
    \end{align*}
    By our computation of multiplication in $\mobgrp$, we see that $\Phi$ is a homomorphism, and is clearly surjective. A matrix $\begin{pmatrix} a & b \\ c & d\end{pmatrix}$ is is the kernel of $\Phi$ if and only if it fixes $0, 1$ and $\infty$. That is, $b = c = 0$, $a = d \neq 0$.
    \begin{equation*}
        \begin{pmatrix}a & b \\ c & d\end{pmatrix} = \begin{pmatrix}a & 0 \\ 0 & a\end{pmatrix} \subseteq \Ccross
    \end{equation*}
    Hence, the kernel of $\Phi$ is $\Ccross$, so the result follows from the isomorphism theorem.
\end{proof}
\section{Orthogonal Groups}
\subsection{The Orthogonal Group and Euclidean Distance}
We want to do geometry, so write $||\vec{v}||$ for the usual notion of length in $\R^n$:
\begin{equation}
    ||\vec{v}|| = \sqrt{\sum_{i=1}^{n}u_i^2}
    \label{eqnVecNorm}
\end{equation}
\begin{definition}{Orthogonal group}
    The $n$-dimensional \underline{orthogonal group} is the subgroup of $GL_n{\R}$ of matrices that preserve lengths in $R^n$.
    \begin{equation*}
        O(n) = \{A \in GL_n(\R) |~||A\vec{v}|| = ||\vec{v}||~\forall \vec{v} \in R^n\}
    \end{equation*}
\end{definition}
However, it is difficult to use the definition of length as given in equation~\ref{eqnVecNorm}, so we prefer to use the dot product:
\begin{equation}
    \vec{u} \cdot \vec{v} = \sum_{i=1}^{n}u_i \vec{v}_i
    \label{eqnDotProduct}
\end{equation}
\begin{lemma}[Polarisation identity]
    \label{lemPolarisation}
    For $\vec{u}, \vec{v} \in R^n$,
    \begin{equation*}
        2(\vec{u} \cdot \vec{v}) = ||\vec{u}||^2 + ||\vec{v}||^2 - ||\vec{u} - \vec{v}||^2
    \end{equation*}
\end{lemma}
\begin{proof}
    Rearranging:
    \begin{equation*}
        ||\vec{u}-\vec{v}||^2 = (\vec{u}-\vec{v}) \cdot (\vec{u}-\vec{v}) = ||\vec{u}||^2 - 2 (\vec{u} \cdot \vec{v}) + ||\vec{v}||^2
    \end{equation*}
\end{proof}
Hence, we can characterise $O(n)$ using the dot product:
\begin{lemma}[$O(n)$ and dot product]
    \begin{equation*}
        O(n) = \{A \in GL_n(\R) | (Ax) \cdot (Ay) = \vec{x} \cdot y~\forall \vec{x}, y \in \R^n\}
    \end{equation*}
    \label{lemDotProdOrthoGrp}
\end{lemma}
\begin{proof}
    If $Ax \cdot Ay = \vec{x} \cdot y \forall \vec{x}, y \in \R^n$, then for any $\vec{v} \in \R^n$,
    \begin{equation*}
        ||A\vec{v}||^2 = (A\vec{v}) \cdot (A\vec{v}) = \vec{v} \cdot \vec{v} = ||\vec{v}||^2
    \end{equation*}
    So $||A\vec{v}|| = ||\vec{v}||$, and $A \in O(n)$.\par
    Conversely, if $A \in O(n)$ then, for all $\vec{x}, y \in R^n$:
    \begin{align*}
        2(Ax) \cdot (Ay) &= ||Ax||^2 + ||Ay||^2 - ||Ax - Ay||^2 \text{ by Lemma~\ref{lemPolarisation}} \\
        &= ||Ax||^2 + ||Ay||^2 - ||A(x-y)||^2 \\
        &= ||x||^2 + ||y||^2 - ||x-y||^2 \text{ by assumption} \\
        &= 2(x \cdot y) \text{ by Lemma~\ref{lemPolarisation}}
    \end{align*}
    Thus $(Ax) \cdot (Ay) = \vec{x} \cdot y$, as required.
\end{proof}
\begin{lemma}[Matrices in $O(n)$]
    Let $A \in M_n(\R)$. Then the following are equivalent:
    \begin{enumerate}
        \item $A \in O(n)$ \label{stmtAInOrthGrp}
        \item The columns of $A$ form an orthonormal basis \label{stmtColumnsOrth}
        \item $A^T A = I$ \label{stmtTransposeInverse}
    \end{enumerate}
\end{lemma}
\begin{proof}
    Let $A = (a_{ij})$. Assume $A \in O(n)$ and let $\{e_i\}$ be the standard basis in $R^n$.\par
    Then the i-th column of $A$ is $Ae_i$ since:
    \begin{equation}
        (Ae_i) \cdot (Ae_j) = e_i \cdot e_j =
        \begin{cases}
            1 & i = j \\
            0 & i \neq j
        \end{cases}
        \label{eqnDotColumns}
    \end{equation}
    And thus the columns form an orthonormal basis. Statement~\ref{stmtAInOrthGrp} implies statment~\ref{stmtColumnsOrth}\par
    Now assume that the columns are orthogonal. As explained above, statement~\ref{stmtColumnsOrth} implies equation~\ref{eqnDotColumns}. Since $\dot{u}{v} = \vec{u^T}\vec{v}$, this means that:
    \begin{equation*}
        \vec{e_i^t}A^t A \vec{e_j} = 
        \begin{cases}
            1 & i = j \\
            0 & i \neq j
        \end{cases}
    \end{equation*}
    But this is the $i, j$ coordinate of $A^t A$, and so $A^T A = I$. Statement~\ref{stmtColumnsOrth} implies statement~\ref{stmtTransposeInverse}.\par
    Now assume $A^T A = I$. Suppose $\vec{u}, \vec{v} \in \R^n$. Then:
    \begin{align*}
        A\vec{u} \cdot A\vec{v} &= A \vec{u^T} A \vec{v} \\
        &= \vec{u^T} A^T A \vec{v} \\
        &= \vec{u^T} \vec{v} \text{ by assumption} \\
        &= \dotprod{u}{v}
    \end{align*}
    So $A$ preserves the dot product, and so by Lemma~\ref{lemDotProdOrthoGrp}, $A \in O(n)$. Statement~\ref{stmtTransposeInverse} implies statement~\ref{stmtAInOrthGrp}.\par
    These are all the implications needed to show that the three statements are equivalent.
\end{proof}
\subsection{The Special Orthogonal Group}
Recall that $\det{A^T} = \det{A}$. Therefore:
\begin{align*}
    1 &= \det{A^T A} = \det{A}\det{A^T} \\
    &= (\det{A})^2 \\
    &\implies \det{A} = \pm 1
\end{align*}
for any matrix in the orthogonal group.
\begin{definition}{Special orthogonal group}
    The \underline{special orthogonal group} is the group of orthogonal matrices with determinant 1. It can be defined in two equivalent ways:
    \begin{eqnarray*}
        SO(n) = O(n) \cap SL_n(\R) \\
        SO(n) = \{A \in O(n) | \det{A} = 1\}
    \end{eqnarray*}
\end{definition}
Note also that $SO(n)$ is the kernel of the determinant homomorphism:
\begin{align*}
    \det{} : O(n) &\mapsto C_2 \\
    A &\mapsto \det{A}
\end{align*}
which tells us that $SO(n) \normalin O(n)$. Also, from theorem~\ref{thmIsomorphism}, we know that $|O(n) : SO(n)| = 2$.
\section{Generating the Orthogonal Group}
\subsection{Reflection}
\begin{definition}{Reflection}
    Any vector $\vec{v} \in \R^n, \vec{v} \neq \vec{0}$ defines an orthogonal plane:
    \begin{equation*}
        P_\vec{v} = \subsetselect{\vec{x} \in \R^n}{\vec{x} \cdot \vec{x} = \vec{0}}
    \end{equation*}
    Then the \underline{reflection} in $P_\vec{v}$ is defined to be:
    \begin{equation*}
        S_{\vec{v}}(\vec{x}) = \vec{x} - \frac{2(\vec{x} \cdot \vec{v})}{||\vec{v}||}\vec{v}
    \end{equation*}
\end{definition}
\underline{Remarks:}
\begin{enumerate}
    \item We may write also $S_P$ for a reflection in a plane
    \item The formula is much simpler for a unit vector $\vec{\hat{v}}$:
    \begin{equation*}
        S_{\vec{\hat{v}}}(\vec{x}) = \vec{x} - 2(\vec{x} \cdot \vec{v})\vec{v}
    \end{equation*}
\end{enumerate}
\begin{lemma}
    $S_{\vec{v}}^2 = I$ and $S_\vec{v} \in O(n)$.
\end{lemma}
\begin{proof}
    We may assume that $\vec{v}$ is a unit vector. Note that $S_\vec{v}$ is linear in $\vec{x}$, so it is in $M_n(\R)$. Now note that:
    \begin{align*}
        S_{\vec{v}}(\vec{x}) \cdot \vec{v} &= (\vec{x} \cdot \vec{v}) - 2(\vec{x} \cdot \vec{v})(\vec{v} \cdot \vec{v}) \\
        &= -(\vec{x} \cdot \vec{v})
    \end{align*}
    So:
    \begin{align*}
        S_{\vec{v}}^2(\vec{x}) &= S_{\vec{v}}(\vec{x}) - 2(S_{\vec{v}}(\vec{x}) \cdot \vec{v})\vec{v} \\
        &= S_{\vec{v}}(\vec{x}) + 2(\vec{x} \cdot \vec{v})\vec{v} \\
        &= \vec{x} - 2(\vec{x} \cdot \vec{v})\vec{v} + 2(\vec{x} \cdot \vec{v})\vec{v} \\
        &= \vec{x}
    \end{align*}
    So $S_{\vec{v}}^2 = I$ and $S_\vec{v} = S_{\vec{v}}^{-1}$ and $S_\vec{v} \in GL_n(\R)$.\par
    Finally, for $\vec{x} \in \R^n$,
    \begin{align*}
        ||S_{\vec{v}}(\vec{x})||^2 &= (\vec{x} - 2(\vec{x} \cdot \vec{v})\vec{v}) \cdot (\vec{x} - 2(\vec{x} \cdot \vec{v})\vec{v}) \\
        &= ||\vec{x}||^2 - 4(\vec{x} \cdot \vec{v})^2 + 4(\vec{x} \cdot \vec{v})^2||\vec{v}||^2 \\
        &= ||\vec{x}||^2
    \end{align*}
    So $S_\vec{v} \in O(n)$ as required.
\end{proof}
\underline{Remark:}~Let $||v|| = 1$ and pick an orthonormal basis $\{\vec{v_1}, \vec{v}_2, \cdots \vec{v_{n-1}}\}$ for $P_\vec{v}$. In the basis $\{\vec{v_1}, \vec{v_2}, \cdots \vec{v_{n-1}}, \vec{v}\}$, $S_\vec{v}$ has matrix:
\begin{equation*}
    \begin{pmatrix}
        1 & 0 & 0 & \cdots & 0 \\
        0 & 1 & 0 & \cdots & 0 \\
        0 & 0 & 1 & & \vdots \\
        \vdots & \vdots & & \ddots & 0 \\
        0 & 0 & \cdots & 0 & 1
    \end{pmatrix}
\end{equation*}
\begin{theorem}
    Every $A \in O(n)$ is a product of at most $n$ reflections.
\end{theorem}
\begin{proof}
    Proceed by induction on $n$.\par
    Base case: Let $n = 1$. $O(1) \cong C_n$ so we have $1$ and $-1$ which can be written as $1$ reflected about $0$.\par
    Inductive step: Let $\{e_1, \cdots e_n\}$ be the standard basis in $\R^n$. Let $\vec{v} = e_n - Ae_n$. We then see that $S_{\vec{v}}(Ae_n) = e_n$, so $S_\vec{v} A$ preserves $e_n$. Therefore, $S_\vec{v} A$ preserves the plane orthogonal to $e_n$, which is the whole of the rest of the space.\par
    Therefore, $S_\vec{v} A$ restricts to an orthogonal transformation of $\R^{n-1}$, so by indection there exists:
    \begin{equation*}
        \vec{v}_1, cdots, \vec{v}_{n-1} \in R^{n-1} \text{ such that } S_\vec{v} A = Sv_1, \cdots Sv_{n-1}, \text{ which is } \R^{n-1}
    \end{equation*}
    Extending the $S_{v_i}$ to $R^n$ by declaring:
    \begin{equation*}
        S_{v_i}(e_n) = e_n
    \end{equation*}
    And the two sides of the equation agree on $R^n$. Therefore, $A = S_\vec{v} S_{v_1} \cdots S_{v_{n-1}}$ as required.
\end{proof}
\begin{lemma}[Elements of $O(2)$]
    Let $A \in O(2)$. Then:
    \begin{enumerate}
        \item If $A \notin SO(2)$, then $A$ is a reflection.
        \item If $A \in SO(2)$, then $A$ is a rotation about $O$.
    \end{enumerate}
\end{lemma}
\begin{proof}
    Recall that $\det{(S_{\vec{v}})} = -1$. Therefore:
    \begin{equation*}
        det{(S_{v_1} \cdots S_{v_k})} = (-1)^k
    \end{equation*}
    And by the theorem, $A = S_{v_1} \cdots S_{v_k}$ when $k \leq 2$.
    \begin{enumerate}
        \item If $A \notin SO(2)$ then $k = 1$ and $A = S_\vec{v}$ for some $\vec{v}$, as claimed.
        \item If $A \in SO(2)$, then $k$ is even so either $A = I$, or $A = S_\vec{u} S_\vec{v}$ for some linearly independent $\vec{u}, \vec{v}$. We claim that this only fixes $O$, so is a rotation.
    \end{enumerate}
    Proving the last claim, assume $\vec{x} \neq \vec{0}$ and that it is fixed by $S_\vec{u} S_\vec{v}$.
    \begin{equation*}
        S_\vec{u} S_\vec{v} \vec{x} = \vec{x} \Leftrightarrow S_\vec{u} \vec{x} = S_\vec{v} x
    \end{equation*}
    But, $\vec{x} - S_\vec{u} \vec{x}$ is parallel to $\vec{u}$, and $\vec{x} - S_\vec{v} \vec{x}$ is parallel to $\vec{v}$, so if $S_\vec{u} \vec{x} = S_\vec{v} \vec{x}$, then $\vec{u}$ and $\vec{v}$ must be parallel. \contradiction
\end{proof}
We can also get an explicit formula for elements of $SO(2)$:\par
\underline{Remark:} let $A \in SO(2)$. Then the columns of $A$ form an orthonormal basis, so:
\begin{equation*}
    A =
    \begin{pmatrix}
        a & b \\
        -b & a
    \end{pmatrix}
    \text{ with } a^2 + b^2 = 1
\end{equation*}
Then we can let:
\begin{align*}
    a &= \cos{\theta} \\
    b &= \sin{\theta}
\end{align*}
We can also analyse elements of $SO(3)$.
\begin{lemma}[Elements of $SO(3)$]
    If $A \in SO(3)$, then $A$ is a rotation, i.e. $A$ fixes a unique axis (or $A = I$)
\end{lemma}
\begin{proof}
    As above, $A$ is a product of at most $3$ reflections. However, it must be a product of an even number of reflections since its determinant must be $+1$, so we have that it is either the identity or a product of 2 reflections.\par
    Consider the case where $A = S_\vec{u} S_\vec{v}$, for linearly independent $\vec{u}, \vec{v}$. The planes $P_\vec{u}$ and $P_\vec{v}$ intersect in a line, so $A$ fixes this line $l$.\par
    Also, $S_\vec{u} S_\vec{v} \vec{x} \Leftrightarrow S_\vec{u} \vec{x} = S_\vec{v}$. So either $S_\vec{u} = S_\vec{v}$ or $\vec{x} \in P_\vec{u} \cap P_\vec{v}$. So there can be no vector fixed by $A$ that is not on the line $l$.
\end{proof}
\end{document}