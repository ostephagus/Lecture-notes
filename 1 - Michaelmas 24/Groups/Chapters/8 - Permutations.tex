\documentclass[../Main.tex]{subfiles}

\begin{document}
\section{Conjugacy in \texorpdfstring{$S_n$}{Sn}}
% Input
Theorem \ref{thmConjugacySn} makes it easy to understand conjugacy classes in $S_n$.
\begin{example}
Consider the group $S_3$.\par
We consider now $ccl_{S_3}((1~~2))$. Using Theorem \ref{thmConjugacySn}, the conjugacy class is all transpositions. These are exactly the pairs of elements where order of elements in the pair does not matter, so its size is therefore $\begin{pmatrix}3 \\ 2\end{pmatrix} = 3$.\par
The conjugacy class of the $3$-cycle, $ccl_{S_3}((1~~2~~3))$, is all the $3$-cycles. We can count these by assuming the first number is $1$, then there are $2$ choices for the second number, and the third number is determined. So in $S_3$, $ccl_{S_3}((1~~2~~3))$ has size $2$.\par
\end{example}
\begin{example}
    Consider now $S_4$.
    What is the size of the conjugacy class of $(1~~2)(3~~4)$? The options to choose the first transposition is $\begin{pmatrix}4 \\ 2\end{pmatrix}$, and then the second transposition is determined. However, for example, $(1~~2)(3~~4) = (3~~4)(1~~2)$, so we have double-counted and we need to halve the answer. We get $3$.
\end{example}
Recall that the centraliser of a group is the stabiliser of an element under the conjugation action. That is, the set of elements that commute with an element. Written $C_{S_n}(\gamma)$. Also, applying \ref{thmOrbitStab}:
\begin{equation}
|C_{S_n}(\gamma)| = \frac{|S_n|}{|ccl_{S_n}(\gamma)}
\label{eqnCentraliserSize}
\end{equation}
\begin{example}
We return to the above example, and consider the centraliser of $(1~~2)(3~~4)$. Its size, by Equation~\ref{eqnCentraliserSize}, is $\frac{4!}{3} = 8$. That is, $8$ elements of $S_n$ commute with $(1~~2)(3~~4)$.\par
The elements are then $\{e, (1~~2), (3~~4), (1~~2)(3~~4),$\newline$(1~~3)(2~~4), (1~~4)(2~~3), (1~~4~~2~~3), (1~~4~~2~~3)\}$.
\end{example}
\begin{example}
    Now we know enough to list all the conjugacy classes of $S_4$.\par
    \begin{tabular}{c|c|c}
        Element & $|ccl_{S_4}(\gamma)|$ & $|C_{S_4}(\gamma)|$ \\
        e & 1 & 24 \\
        \hline
        $(1~~2)$ & 6 & 4 \\
        $(1~~2)(3~~4)$ & 3 & 8 \\
        $(1~~2~~3)$ & 8 & 3 \\
        $(1~~2~~3~~4)$ & 6 & 4 
    \end{tabular}\par
    We know we have classified them all, since summing the sizes of the conjugacy classes gives $24 = |S_4|$.\par
    This gives us a fairly clear understanding of the elements of $S_4$.
\end{example}
\section{Conjugacy Classes in \texorpdfstring{$A_n$}{An}}
Counting conjugacy classes in $A_n$ can also be done, with a little thought.
\begin{proposition}
    If $H$, $K \leq G$, then $|H : H \cap K| \leq |G : K|$
    \label{propSubgroupIndices}
\end{proposition}
\begin{proof}
Left as an exercise.
\end{proof}%TODO: complete this exercise
\begin{lemma}[Conjugacy classes in $A_n$]
    Let $\gamma \in A_n \leq S_n$.
    \begin{enumerate}
        \item If some odd element of $S_n$ commutes with $\gamma$, then the conjugacy class of $A_n$ of $\gamma$ is equal to the conjugacy class in $S_n$. $ccl_{A_n}(\gamma) = ccl_{S_n}(\gamma)$
        \item If every element of $S_n$ that commutes with $\gamma$ is even, then $ccl_{S_n}(\gamma)$ splits into two:
        \begin{equation*}
            ccl_{S_n}(\gamma) = ccl_{A_n}(\gamma) \cup ccl_{A_n}(\tau\gamma\tau) 
        \end{equation*}
        where $\tau$ is any transposition.
    \end{enumerate}
    \label{lemConjugacyAn}
\end{lemma}
Note that in condition 1 we have an element inside $C_{S_n}(\gamma)$ but not inside $C_{A_n}(\gamma)$ (since there exists an odd element) so $C_{A_n}(\gamma) \nsubseteq C_{S_n}(\gamma)$. Condition 2 implies that $C_{S_n}(\gamma) = C_{A_n}(\gamma)$, since all the element in $C_{S_n}(\gamma)$ are even and are then present in the alternating group also.\par
\begin{proof}
    From \ref{thmOrbitStab}, 
    \begin{align*}
        |S_n| = |ccl_{S_n}(\gamma)||C_{S_n}(\gamma)| \\
        |A_n| = |ccl_{A_n}(\gamma)||C_{A_n}(\gamma)| \\
    \end{align*}
    Since $|S_n| = 2|A_n|$,
    \begin{equation}
        |ccl_{S_n}(\gamma)| = 2|ccl_{A_n}(\gamma)|\frac{|C_{A_n}(\gamma)|}{|C_{S_n}(\gamma)|}
        \label{eqnSizeCcl1}
    \end{equation}
    Then by Proposition~\ref{propSubgroupIndices}, $|C_{S_n}(\gamma) : C_{A_n}(\gamma)| \leq 2$. This means, therefore, that $C_{S_n}(\gamma)$ is either equal to $C_{A_n}(\gamma)$, or is partitioned into two by it. That is, $|C_{S_n}(\gamma) : C_{A_n}(\gamma)| = 1$ or $2$. Using Lagrange's Theorem on Equation~\ref{eqnSizeCcl1}, we get:
    \begin{equation}
        |ccl_{S_n}(\gamma)| = \frac{2|ccl_{A_n}(\gamma)|}{|C_{S_n}(\gamma) : C_{A_n}(\gamma)|}
        \label{eqnSizeCcl2}
    \end{equation}
    And the denominator is either 1 or 2.We can therefore consider the cases:\par
    \begin{case}{Denominator is $2$}
        Here there is an odd element of $C_{S_n}(\gamma)$, $C_{A_n}(\gamma) \neq C_{S_n}(\gamma)$.\par
        Therefore, $|ccl_{S_n}(\gamma)| = |ccl_{A_n}(\gamma)|$, and since $C_{S_n}(\gamma)$ must be contained within $C_{A_n}(\gamma)$, the two must be equal as required.\par
    \end{case}
    \begin{case}{Denominator is $1$}
        In this case, the centralisers are equal, so \ref{eqnSizeCcl2} becomes:
        \begin{equation*}
            |ccl_{S_n}(\gamma)| = 2|ccl_{A_n}(\gamma)|
        \end{equation*}
        So the conjugacy class of $\gamma$ in $A_n$ is only half the size.\par
        Now consider a transposition $\tau \in S_n$. This is an odd permutation, so $\tau \notin S_n$.\par
        Note that $\tau\gamma\tau^{-1} \in ccl_{S_n}(\gamma)$ by definition.\par
        We then see what happens if $\tau\gamma\tau^{-1} \in ccl_{A_n}(\gamma)$.\par
        Then $\tau\gamma\tau^{-1} = \alpha\gamma\alpha^{-1}$ for some $\alpha \in A_n$, which is even. But rearranging this gives $(\tau\alpha)\gamma(\tau\alpha)^{-1} = \gamma$, so the odd element $\tau\alpha \in C_{S_n}(\gamma)$. This contradicts the assumption of case 2.\par
        Hence, $\tau\gamma\tau^{-1} \notin ccl_{A_n}(\gamma)$, so $ccl_{S_n}(\gamma) = ccl_{A_n}(\gamma) \cup ccl_{A_n}(\tau\gamma\tau^{-1})$ as required.
    \end{case}
\end{proof}
This lemma makes it possible to determine the conjugacy classes in $A_n$.
\begin{example}
    Consider the group $A_4$.\par
    All elements are either $e$, $(2, 2)$-cycles or $3$-cycles.\par
    Since the odd element $(1~~2)$ commutes with $(1~~2)(3~~4)$, the conjugacy class of $(2, 2)$-cycles remains intact.\par
    For $3$-cycles, the elements that commute with it are the identity, itself, and itself squared. These are all even elements, so we are in case 2 of Lemma~\ref{lemConjugacyAn}, and our conjugacy class splits.
    We can now classify all the conjugacy classes in $A_4$:\par
    \begin{tabular}{c|c}
        Element & $|ccl_{A_n}(\gamma)$ \\
        $e$ & $1$ \\
        $(1~~2)(3~~4)$ & $3$ \\
        $(1~~2~~3)$ & $4$ \\
        $(3~~2~~1)$ & $4$ 
    \end{tabular}
\end{example}
\begin{example}
    Consider the group $A_5$ and $S_5$.\par
    \begin{tabular}{c|c|c}
        Element $\gamma$ & $|ccl_{S_5}(\gamma)$ & $|C_{S_5}(\gamma)|$ \\
        \hline
        $e$ & $1$ & $120$ \\
        $(1~~2)$ & $\begin{pmatrix}5 \\ 2\end{pmatrix} = 10$ & $12$ \\
        $(1~~2~~3)$ & $2\times\begin{pmatrix}5 \\ 4\end{pmatrix} = 20$ & $6$ \\
        $(1~~2)(3~~4)$ & $\begin{pmatrix}5 \\ 2\end{pmatrix}\times\begin{pmatrix}4 \\ 2\end{pmatrix} \div 2 = 15$ & $8$ \\
        $(1~~2~~3)(4~~5)$ & $2\times\begin{pmatrix}5 \\ 4\end{pmatrix} = 20$ & $6$ \\
        $(1~~2~~3~~4)$ & $\begin{pmatrix}5 \\ 4\end{pmatrix}\times3! = 30$ & $4$ \\
        $(1~~2~~3~~4~~5)$ & $4! = 24$ & $5$
    \end{tabular}\par
    Now we can understand conjugacy in $A_5$. We first need to figure out whether the centralisers contain an odd element.\par
    \begin{itemize}
        \item $(4~~5)$ commutes with $(1~~2~~3)$, so the conjugacy class of $(1~~2~~3)$ does not split in $A_5$.
        \item $(1~~2)$ commutes with  $(1~~2)(3~~4)$, so its conjugacy class does not split.
        \item The size of the centraliser of $(1~~2~~3~~4~~5)$ is $5$, so the centraliser is equal to the subgroup generated by it, that is, only 5-cycles commute with the 5-cycle. So we have no odd elements in the centraliser, and the conjugacy class splits into two in $A_n$.
    \end{itemize}
    We now have enough information to categorise the conjugacy classes of $A_n$.\par
    \begin{tabular}{c|c}
        Element $\gamma$ & $|ccl_{A_n}(\gamma)$ \\
        \hline
        $e$ & $1$ \\
        $(1~~2~~3)$ & $20$ \\
        $(1~~2)(3~~4)$ & $15$ \\
        $(1~~2~~3~~4~~5)$ & $12$ \\
        $(2~~1~~3~~4~~5)$ & $12$
    \end{tabular}
\end{example}
Recall: a \underline{simple} group is one where there exist no non-trivial normal subgroups\par % WHAT?
\begin{theorem}
    $A_5$ is simple. \label{thmA5Simple}
\end{theorem}
\begin{proof}
    Suppose we have a proper normal subgroup $N$ of $A_5$.\par
    Then $N$ is a union of conjugacy classes in $A_5$.\par
    It must contain the identity, and we consider the possible orders of $N$.
    \begin{itemize}
        \item Just the identity: $|N| = 1$
        \item The 3-cycles: $|N| = 1+20=21$
        \item $(2-2)$-cycles also: $|N| = 1+20+15 = 36$
        \item Also half the 5-cycles: $|N| = 1+20+15+12=48$
        \item All elements: $|N| = 60$
        \item $(2-2)$-cycles: $|N| = 1+15 = 16$
        \item Also some 5-cycles: $|N| = 1+15+12 = 28$
        \item The rest of the 5-cycles: $|N| = 1+15+12+12 = 40$
        \item Only half the 5-cycles: $|N| = 1+12 = 13$
        \item All the 5-cycles: $|N| = 1+12+12 = 25$
        \item 3-cycles and 5-cycles: $|N| = 1+20+12 = 33$
        \item The rest of the 5-cycles: $|N| = 1+20+12+12 = 45$
    \end{itemize}
    But by \ref{thmLagrange}, none of these (except $A_5$ and $\{e\}$) can be subgroups.
\end{proof}
\end{document}