\documentclass[../Main.tex]{subfiles}

\begin{document}
\section{Properties}
A complex number, $z$, is in the set of complex numbers $\mathbb{C}$. It can be written in component form as $z=a+ib$ where $a, b \in \mathbb{R}$, $i = \sqrt{-1}$.\par
We also define $Re(z) = a$, $Im(z) = b$. Note $\mathbb{R} \subset \mathbb{C}$.
\subsection{Arithmetic}
Here define $z_1 = a_1 + b_1 i$ and $z_2 = a_2 + b_2 i$.\par
Addition and subtraction is elementwise: $z_1 \pm z_2 = (a_1 \pm a_2) + (b_1 \pm b_2)i$.
Multiplication gives: $z_1 z_2 = (a_1 a_2 - b_2 b_2) + (a_1 b_2 + a_2 b_1) i$.
We then see that $\mathbb{C}$ is closed under addition and multiplication.
By a similar process to rationalising the denominator, we can arise at:
\begin{equation}
    z^{-1} = \frac{a - b i}{a^2 + b^2}
\end{equation}
\subsection{Complex Conjugation}
We also define the complex conjugate, denoted $\bar{z}$ or $z^*$ as the negation of the complex part $a - b i$. The magnitude of $z$ is $\sqrt{a^2 + b^2}$, and so $z\bar{z} = |z|^2$. Also, we can define $z^{-1}$ in this way:
\begin{equation*}
    z^{-1} = \frac{\bar{z}}{|z|^2}
\end{equation*}
\section{Argand Diagram}
% TODO: insert image here
The complex number $z$ can be represented by the position vector in the argand diagram. Then the complex conjugate is the reflection of this vector in the real axis. Addition is akin to tip-to-tail vector addition, and scalar multiplication is equivalent.\par
From this representation we get the triangle inequality:
\begin{equation}
    |z_1 + z_2| \leq |z_1| + |z_2|
    \label{eqnTriIneq}
\end{equation}
and the similar inequality:
\begin{equation}
    |z_1 - z_2| \geq \left||z_1| - |z_2|\right|
    \label{eqnTriIneqNegative}
\end{equation}
Which can be derived from Equation~\ref{eqnTriIneq}:
\begin{align*}
    \text{Let } z_1 = z'_1 - z'_2; z_2 = z'_2 \\
    \ref{eqnTriIneq} \implies |z'_1| &\leq |z'_1 - z'_2| + |z'_2| \\
    |z'_1 - z'_2| &\geq |z'_1| - |z'_2|
\end{align*}
Swapping $z'_1 \leftrightarrow z'_2$ should not change the equation so modulus needed on RHS, this then gives \ref{eqnTriIneqNegative}.
\section{Polar Representation}
A complex number $z = x + i y$ can also be represented by specifying the modulus and argument (angle ACW from the real axis). $z = x + i y = r(\cos(\theta) + i\sin(\theta))$, where $r = |z|$, $\theta = \arctan(\frac{y}{x})$. In this representation each modulus-argument form represents a distinct complex number, but each complex number does not have a distinct modulus-argument form due to the periodicity of trigonometric functions. Thus $\theta$ is generally restricted to \underline{principal values}, $-\pi < \theta \leq \pi$ or (less commmonly) $0 \leq \theta < 2\pi$.\par
In this system, we derive multiplication: $z_1 z_2 = r_1 r_2 (\cos(\theta_1 + \theta_2) + i\sin(\theta_1 + \theta_2))$.
The exponential function $\exp(z)$ can be easily extended to the complex plane using its series expansion:
\begin{equation*}
    \exp(z) = \sum_{n=0}^{\infty}{\frac{z^2}{n!}}
\end{equation*}
And we have an alternative modulus-argument format $z = re^{i\theta}$.
\begin{proposition}
    $\exp(z_1)\exp(z_2) = \exp(z_1 + z_2) \forall z_1, z_2 \in \mathbb{C}$
\end{proposition}
\begin{proof}
    \begin{align*}
        \exp(z_1)\exp(z_2) &= \sum_{m, n = 0}^{\infty}\frac{z_1^m z_2^n}{m!n!} \\
        &=\sum_{r=0}^{\infty}\sum_{m=0}^{r}\frac{z_1^{r-m}z_2^m}{m!(r-m)!} \\
        &=\sum_{r=0}^{\infty}\frac{1}{r!}\sum_{m=0}^{r}\frac{r!}{m!(r-m)!}z_1^{r-m}z_2^m \\
        &=\sum_{r=0}^{\infty}\frac{1}{r!}\left(z_1 + z_2\right)^r \\
        &=\exp(z_1 + z_2)
    \end{align*}
\end{proof}
\end{document}