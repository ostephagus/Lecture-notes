\documentclass[../Main.tex]{subfiles}

\begin{document}
\section{Parallels with Linear Algebra}
Quantum Mechanics uses a \underline{functional space}, which has many similarities with a vector space.

\begin{tabularx}{0.9\textwidth}{|>{\centering\arraybackslash}X|>{\centering\arraybackslash}X|}
    \hline
    Linear Algebra concept & Quantum Mechanics concept \\
    \hline
    Vector (n-dimensional complex vector) & State \\
    $\vec{v}$ & $\psi$ \\
    Basis $\{e_J\}$ which allows $\vec{v}$ to be written as a linear combination & $\{\vec{X}\}$ continuous basis, $\psi(\vec{x}, t)$. \\
    Vector space $\C^n$ & $L^2(\R^3)$, complex-valued square-integrable functions \\
    Inner product $\langle \vec{v}~|~\vec{u}\rangle$ & Inner product $(\psi, \phi)$ defined by equation~\ref{eqnInnerProduct} \\
    Linear maps $\C^n \mapsto \C^n$ represented by a matrix & Operator $\op$ \\
    \hline
\end{tabularx}

The inner product in $L^2(\R^3)$ is:
\begin{equation}
    \int_{\R^3} \psi^*(\vec{x}, t) \phi(\vec{x}, t)d^3 x
    \label{eqnInnerProduct}
\end{equation}
\section{Wave Function and Probabilistic Interpretation}
\subsection{The Wave Function}
In classical mechanics, we have that $\vec{x}$ and $\dvec{x}$ determine the dynamics of a particle in a deterministic way.

In quantum mechanics, $\psi(\vec{x}, t)$ determines the dynamics of a particle in a probabilistic way.

\begin{definition}{State}
    The \underline{state} of a particle is a mathematical description represented by $\psi$.
\end{definition}
\begin{definition}{Wave function}
    The \underline{wave function} of a particle is the complex coefficient of $\psi$ in the continuous basis $\vec{x}$ at a given time $t$. $\psi(\vec{x}, t)$ is the $\vec{x}$ representation.
    \begin{equation*}
        \psi(\vec{x}, t) : \R^3 \mapsto \C~~\forall t \in \R
    \end{equation*}
\end{definition}
\begin{remark}
    We could define other representations, such as the $\vec{p}$-representation (for momentum), but we will consider only the $\vec{x}$ representation in this course.
\end{remark}
The physical interpretation of $\psi(\vec{x}, t)$ is given by:
\begin{equation}
    \rho(\vec{x}, t) \propto |\psi(\vec{x}, t)|^2
    \label{eqnProbAmplitude}
\end{equation}
where $\rho(\vec{x}, t)$ is the probability density for a particle described by a state $\psi$ to sit at a point $\vec{x}$ at a given time $t$.

\subsection{Mathematical Properties}
\begin{align}
    &\int_\R^3 |\psi(\vec{x}, t)|^2 d^3 x = N \in \R,~~N \text{ finite, non-zero} \label{eqnWFTotIntegral} \\
    &\ybar(\vec{x}, t) = \frac{1}{\sqrt{N}} \psi(\vec{x}, t) \label{eqnPhiBar} \\
    &\int_\R^3 |\ybar(\vec{x}, t)|^2 d^3 x = 1 \label{eqnWFInt1} \\
    &\rho(\vec{x}, t) = |\ybar(\vec{x}, t)|^2 \label{eqnWFProbDens}
\end{align}
\begin{definition}{Equivalent state}
    Two states $\psi$ and $\tilde{\psi}$ are \underline{equivalent state} if the amplitudes of their corresponding wavefunctions are the same,
    \begin{equation*}
        |\tilde{\psi}(\vec{x}, t)|^2 = |\psi(\vec{x}, t)|^2
    \end{equation*}
    this is when $\tilde{\psi} = e^{i\alpha} \psi$.
\end{definition}
\begin{remark}
    As an aside, a state $\psi$ corresponds only to a series of rays in the space of functions, not an individual wavefunction. We can always define a new equivalent wavefunction up to a complex unit constant.
\end{remark}
\section{The Hilbert Space}
We now set up a vector space of functions that allows us to define important properties in Quantum Mechanics.
\subsection{Hilbert Space as a Vector Space}
\begin{definition}{Hilbert space}
    The set of all square-integrable functions in $\R^3$ is called the \underline{Hilbert space} ($\hilb$ or $L^2(\R^3)$)
\end{definition}
We have the property that if $\psi_1, \psi_2 \in \hilb$

\begin{theorem}[Superposition principle]
    If $\psi_1(\vec{x}, t)$ and $\psi_2(\vec{x}, t)$ are square-integrable, then $\psi = a_1 \psi_2 + a_2 \psi_2$ is square integrable.
    \label{thmSuperposition}
\end{theorem}
\begin{proof}
    \begin{align*}
        \int_{\R^3} |\psi_1(\vec{x}, t)|^2 d^3x &= N_1 < \infty \\
        \int_{\R^3} |\psi_2(\vec{x}, t)|^2 d^3x &= N_2 < \infty 
    \end{align*}
    then by the triangle identity for complex numbers ($|z_1| + |z_2| \leq |z_1 + z_2|$), and taking $z_1 = a_1 \psi_1(\vec{x}, t)$ and $z_2 = \psi_2(\vec{x}, t)$,
    \begin{align*}
        \int_{\R^3}& |\psi(\vec{x}, t)|^2 d^3x = \int_{\R^3} |a_1 \psi_1(\vec{x}, t) + a_2 \psi_2 (\vec{x}, t)|^2 d^3x \\
        &\leq \int_{\R^3} \left(|a_1 \psi_1(\vec{x}, t)| + |a_2 \psi_2 (\vec{x}, t)|\right)^2 d^3x \\
        &= \int_{\R^3} \left(|a_1 \psi_1(\vec{x}, t)|^2 + |a_2 \psi_2 (\vec{x}, t)|^2+ 2 |a_1 \psi_1||a_2 \psi_2|\right) d^3x \\
        &\leq \int_{\R^3} \left(2|a_1 \psi_1(\vec{x}, t)|^2 + 2|a_2 \psi_2 (\vec{x}, t)|^2\right) d^3x \\
        &= 2|a_1|^2 N_1 + 2|a_2|^2 N_2 < \infty
    \end{align*}
\end{proof}
\begin{remark}
    As we will see later, it is necessary to require $\psi$ to be twice differentiable, and that its second derivative is square-integrable. We also require that $\psi$ goes to $0$ exponentially fast as $|x| \to \infty$.
\end{remark}
\subsection{Inner Product}
\begin{definition}{Inner product}
    The \underline{inner product} in $\hilb$ is defined, for two functions $\psi$ and $\phi \in \hilb$:
    \begin{equation}
        \inn{\psi}{\phi} = \int_{\R^3} \phi^*(\vec{x}, t) \psi(\vec{x}, t) d^3x
        \label{eqnHilbInnProd}
    \end{equation}
\end{definition}
\begin{theorem}
    If $\psi, \phi \in \hilb$, then their inner product is guaranteed to exist.
    \label{thmInnProdExists}
\end{theorem}
\begin{proof}
    \begin{align*}
        \int_{\R^3} |\psi(\vec{x}, t)|^2 d^3x &= N_1 < \infty \\
        \int_{\R^3} |\phi(\vec{x}, t)|^2 d^3x &= N_2 < \infty 
    \end{align*}
    Therefore, consider the inner product:
    \begin{align*}
        \inn{\phi}{\psi}& \left|\int_{\R^3} \psi^*(\vec{x}, t)\phi(\vec{x}, t) d^3 x\right| \\
        &\leq \sqrt{\int_{\R^3} |\psi(\vec{x}, t)|^2 d^3x \int_{\R^3} |\phi(\vec{x}, t)|^2 d^3x} \\
        &= \sqrt{N_1 N_2} < \infty
    \end{align*}
    by the Cauchy-Schwarz inequality.
\end{proof}
\begin{propositions}{
        Let $\phi, \psi \in \hilb$. Let $a_1, a_2$ be real coefficients.
        \label{propsInnProdProps}
    }
    \item $\inn{\psi}{\phi} = \inn{\phi}{\psi}^*$
    \item The inner product is antilinear in the first argument, linear in the second argument.
    \item $\inn{\psi}{\psi} \geq 0$
\end{propositions}
\begin{proof}
    \begin{enumerate}
        \item Immediate from the definition
        \item Consider $\psi, \psi_1, \psi_2, \psi, \psi_1, \psi_2 \in \hilb$, $a_1, a_2 \in \C$.
            \begin{align*}
                \inn{a_1 \psi_1 + a_2 \psi_2}{\phi} &= a_1^* \inn{\psi_1}{\phi} + a_2^* \inn{\psi_2}{\phi} \\
                \inn{\psi}{a_1 \phi_1 + a_2 \phi_2} &= a_1 \inn{\psi}{\phi_1} + a_2 \inn{\psi}{\phi_2}
            \end{align*}
        \item Note that when a function is input to both sides of the inner product, the resulting integrand is square so the inner product must be $\geq 0$. Equality holds when the integrand is equal to $0$ on all $\R^3$, which is when $\psi \equiv 0$. However, this is not a valid probability density function because it cannot be normalised.
    \end{enumerate}
\end{proof}
\begin{definition}{Norm}
    The \underline{norm} of a wavefunction $\psi$ is the real function:
    \begin{equation}
        ||\psi|| = \sqrt{\inn{\psi}{\psi}}
        \label{eqnWFNorm}
    \end{equation}
\end{definition}
\begin{definition}{Normalised wavefunction}
    A wavefunction $\psi$ is \underline{normalised} if its norm is $1$. We denote this $\ybar$.
\end{definition}
\begin{definition}{Orthogonal wavefunctions}
    Wavefunctions $\phi, \psi \in \hilb$ are  \underline{orthogonal} if their inner product is zero.
\end{definition}
\begin{definition}{Orthonormal wavefunctions}
    A set of wavefunctions $\{\psi_n\}$ is  \underline{orthonormal} if their inner products satisfy:
    \begin{equation*}
        \inn{\psi_n}{\psi_m} = \delta_{mn}
    \end{equation*}
\end{definition}
\begin{definition}{Completeness}
    A set of wavefunctions is \underline{complete} if any $\psi \in \hilb$ can be written as a linear combination of elements of the set:
    \begin{equation}
        \phi = \sum_{n = 0}^{\infty} c_n \psi_n
        \label{eqnHilbLinCombo}
    \end{equation}
    where $c_n \in \C$
\end{definition}
\begin{lemma}
    If $\{\psi_n\}$ forms a complete and orthonormal basis of $\hilb$, then  in equation~\ref{eqnHilbLinCombo} $c_n = \inn{\psi_n}{\phi}$.
    \label{lemLinComboCoeffts}
\end{lemma}
\begin{proof}
    \begin{align*}
        \inn{\psi_n}{\phi} 
        &= \inn{\psi_n}{\sum_{m = 0}^\infty c_m \psi_m} \\
        &= \sum_{m = 0}^{\infty}c_m \inn{\psi_n}{\psi_m} \\
        &= \sum_{m = 0}^{\infty} c_m \delta_mn \\
        &= c_n
    \end{align*}
\end{proof}
\section{Time-Dependent Schr\"odinger Equation}
Recall that the first postulate of Quantum Mechanics is Born's Rule that the probability density is given by the wavefunction. The second postulate is the time-dependent \schr equation (TDSE):
\begin{equation}
    i\hbar \frac{\partial \psi(\vec{x}, t)}{\partial t} = -\frac{\hbar}{2m}\nabla^2 \psi(\vec{x}, t) + U(\vec{x})\psi(\vec{x}, t)
    \label{eqnTDSE}
\end{equation}
where here:
\begin{equation*}
    \nabla^2 = \frac{d^2}{dx^2} + \frac{d^2}{dy^2} + \frac{d^2}{dy^2}
\end{equation*}
and $U(\vec{x}) \in \R$ is the \underline{potential}.

\begin{remarks}
    \item Since this gives the first derivative of $\psi$ with respect to time, if the wave function is known at a given time $t_0$ then we can find it at all time.
    \item We have asymmetry in $t \leftrightarrow \vec{x}$, so this is a non-relativistic equation.
\end{remarks}
\begin{example}
    Consider an electron behaving like a wave, and so the solution to the wave equation is $\exp\left[i(kx - \omega t)\right]$. Then, if we substitute this into equation~\ref{eqnTDSE} with $U = 0$, we find $\omega \propto k^2$, where $k$ is the wave number.

    De Broglie's hypothesis gives that:
    \begin{align*}
        k &= \frac{p}{\hbar} \\
        \omega &= \frac{E}{k}
    \end{align*}
    and so, since $E = \frac{p^2}{2m}$, we can find $\omega$ in terms of $k$:
    \begin{align*}
        \omega &= \frac{p^2}{2m \hbar} = \frac{\hbar}{2m} k^2 \\
        \therefore \omega \propto k^2
    \end{align*}
    as expected.

    Note that this is different from an electromagnetic wave, where $\omega \propto k$. This is because we are comparing massless particles (photons) with massive particles (electrons).
\end{example}
\begin{proposition}
    Let $\psi$ obey TDSE. Then the normalisation of $\psi$ is constant.
    \label{propConstNormaln}
\end{proposition}
\begin{proof}
    Normalisation is: $\int_{\R^3} |\psi(\vec{x}, t)|^2 d^3x = N$.
    \begin{align*}
        \frac{d^{}N}{dt^{}} &= \frac{d}{dt} \int_{\R^3} |\psi(\vec{x}, t)|^2 d^3x \\
        &= \int_{\R^3} \frac{\partial}{\partial t} \left(\psi(\vec{x}, t) \psi^*(\vec{x}, t)\right) d^3x
    \end{align*}
    and by TDSE:
    \begin{align*}
        \frac{\partial \psi}{\partial t} &= \frac{i \hbar}{2m} \nabla^2 \psi - \frac{iU}{\hbar}\psi \\
        \frac{\partial \psi^*}{\partial t} &= \frac{i \hbar}{2m} \nabla^2 \psi* - \frac{iU}{\hbar}\psi* \\
        \therefore \frac{\partial \psi^* \psi}{\partial t} &= \nabla \cdot \left(\frac{i\hbar}{2m} \left(\psi^* \nabla \psi - \psi \nabla \psi^*\right)\right) \\
        &=0 \text{ by divergence theorem}
    \end{align*}
    because we have the boundary conditions that $\psi \to 0$ when $\vec{x}$ gets far from $0$.
\end{proof}
\begin{remark}
    The equation:
    \begin{equation*}
        \frac{\partial}{\partial t} (\psi^* \psi) = \nabla \cdot \left(\frac{i\hbar}{2m}(\psi^* \nabla \psi - \psi \nabla \psi^*)\right) = 0
    \end{equation*}
    that appeared in the proof of proposition~\ref{propConstNormaln} shows us that the probability density is conserved. We can define the quantity:
    \begin{equation*}
        \vec{J} = \frac{i\hbar}{2m}(\psi^* \nabla \psi - \psi \nabla \psi^*)
    \end{equation*}
\end{remark}
\section{Measurements in Quantum Mechanics}
We want to know how to extract information from a state $\psi$.
\begin{definition}{Observability}
    A property of a particle described by $\psi$ is \underline{observable} if it can be measured. In QM, observable properties correspond to an operator acting on $\psi$, and a measurement corresponds to the expected value of this operation.
\end{definition}
\subsection{Heuristic Derivation of Position and Momentum}
\label{sscHeuristicMesurements}
Consider only one dimension for this subsection.

The two most important observables are the position, $x$, and the momentum, $p$.

From the probabilistic interpretation, if we want to measure the position of a particle, we calculate:
\begin{align*}
    \E{x} &= \int_{-\infty}^{\infty} x |\psi(x, t)|^2 dx \\
    &= \int_{-\infty}^\infty \psi^*(x, t) x \psi(x, t) dx
\end{align*}
where here $\E{x}$ is the expected value for $x$. We see that this has the same form as in IA Probability.

Now the expected value of the momentum is defined:
\begin{align*}
    \E{p} &= \frac{d\E{x}}{dt} \\
    &= m \frac{d}{dt} \int_{-\infty}^{\infty} \psi^*(x, t) x \psi(x, t) dx \\
    &= m \int_{-\infty}^{\infty} x \frac{\partial}{\partial t}(\psi^* \psi) dx \\
    &= \frac{i\hbar m}{2m} \int_{-\infty}^{\infty} x \frac{\partial}{\partial x}\left(\psi^* \frac{\partial \psi}{\partial x}- \psi \frac{\partial \psi^*}{\partial x}\right) dx \\
    &\text{Integrate by parts, boundary term goes to }0: \\
    &= \frac{-i\hbar}{2} \int_{-\infty}^{\infty} \left(\psi^* \frac{\partial \psi}{\partial x} - \psi \frac{\partial \psi^*}{\partial x}\right) dx \\
    &= -i\hbar \int_{-\infty}^{\infty} \psi^* \frac{\partial \psi}{\partial x} dx \\
\end{align*}
Therefore we have derived the operators for position, multiplication by $x$, and for momentum, application of $-i\hbar \frac{\partial}{\partial x}$. Now any quantity that is a function of position and momentum, $Q(x, p)$, can be translated into such an integral:
\begin{equation*}
    \E{Q(x, p)} = \int_{-\infty}^{\infty} \psi^* Q(x, -i\hbar \frac{\partial}{\partial x})\psi dx 
\end{equation*}
\subsection{Hermitian Operators}
In $\C^n$, we have linear maps that are represented by matrices. We define a Hermitian matrix by $A^\dagger = A$.

In QM, we consider the Hilbert space $\hilb$ and define operators similarly.
\begin{definition}{Operator}
    An \underline{operator} $\op$ is a linear map $\op : \hilb \mapsto \hilb$ such that:
    \begin{equation*}
        \op(a_1 \psi_1 + a_2 \psi_2) = a_1 \op(\psi_1) + a_2 \op(\psi_2)
    \end{equation*}
    for all $a_1, a_2 \in C,~\psi_1, \psi_2 \in \hilb$.
\end{definition}
\begin{examples}{}
    \item Finite difference operators:
        \begin{equation*}
            \sum_{n = 0}^N q_n(x) \frac{\partial^{n}}{\partial x^{n}}
        \end{equation*}
        where $q_n(x)$ is a polynomial. The operators we have already seen for position and momentum are special cases.
    \item Translation: $\psi(x) \mapsto \psi(x - a)$
    \item Parity: $\psi(x) \mapsto \psi(-x)$.
\end{examples}
\begin{definition}{Hermitian conjugate}
    The \underline{Hermitian conjugate} $\opdg$ of an operator $\op$ is the operator such that:
    \begin{equation*}
        (\opdg \psi_1, \psi_2) = (\psi_1, \op \psi_2)
    \end{equation*}
    for all $\psi_1, \psi_2 \in \hilb$.
\end{definition}
We can easily verify that the usual Hermitian properties apply.
\begin{definition}{Hermitian operator}
    An operator $\op$ is a \underline{Hermitian operator} if $\op = \opdg$.
\end{definition}
\begin{remark}
    All physics quantities in Quantum Mechanics are represented by Hermitian operators.
\end{remark}
\begin{examples}{
        For the proofs of these, see the first example sheet.
    }
    \item The operator $\hat{x} : \psi(x, t) \mapsto x\psi(x, t)$ is Hermitian. We can check this easily:
        \begin{align*}
            (\hat(x)\psi_1, \psi_2) &= \int_{-\infty}^\infty (x\psi_1)^* \psi_2 dx \\
            &= \int_{-\infty}^\infty \psi_1^* x \psi_2 dx \\
            &= (\psi_1, \hat{x} \psi_2)
        \end{align*}
    \item We can similarly verify that $\hat{p} : \psi(x, t) \mapsto i\hbar \frac{\partial \psi}{\partial x}$ is Hermitian.
    \item The kinetic energy operator $\hat{T} : \psi(x, t) \mapsto \frac{\hat{p}^2}{2m}\psi$ is Hermitian.
    \item The potential energy operator $\hat{U} : \psi(x, t) \mapsto U(x) \psi(x,t)$ is Hermitian.
    \item The total energy operator $\hat{H} : \psi(x, t) \mapsto (\hat{T} + \hat{U}) \psi(x, t)$.
\end{examples}
\begin{theorem}
    The eigenvalues of Hermitian operators are real.
    \label{thmReaEvals}
\end{theorem}
\begin{proof}
    Let $\op$ be Hermitian. Let there exist an eigenvalue $a$ with eigenfunction $\psi$, $\op \psi = a\psi$. For ease, assume $\psi$ is normal ($||\psi|| = 1$).

    \begin{align*}
        (\psi, \op\psi) &= (\psi, a\psi) = a(\psi, \psi) \\
        &= a \\
        (\psi, \op\psi) &= (\op\psi, \psi) = (a\psi, \psi) \\
        &= a^* (\psi, \psi) = a^* \\
        \implies& a = a^*, a \in \R
    \end{align*}
\end{proof}
\begin{theorem}
    Let $\op$ be a Hermitian operator, $\psi_1, \psi_2$ normalised eigenfunctions of $\op$ with eigenvalues $a_1, a_2$. Then if $a_1 \neq a_2$, $\psi_1$ and $\psi_2$ are orthogonal.
    \label{thmOrthoEFuncs}
\end{theorem}
\begin{proof}
    We have:
    \begin{align*}
        \op \psi_1&= a_1 \psi_1 \\
        \op \psi_2&= a_2 \psi_2
    \end{align*}
    then consider:
    \begin{align*}
        a_1 \inn{\psi_1}{\psi_2} &= a_1^* \inn{\psi_1}{\psi_2} \text{ since }a\text{ real} \\
        &= \inn{a_1 \psi_1}{\psi_2} \\
        &= \inn{\op \psi_1}{\psi_2} \\
        &= \inn{\psi_1}{\op\psi_2} \text{ since }\op\text{ is Hermitian} \\
        &= \inn{\psi_1}{a_2\psi_2} \\
        &= a_2 \inn{\psi_1}{\psi_2}
    \end{align*}
    That is,
    \begin{equation*}
        (a_1 - a_2)\inn{\psi_1}{\psi_2}
    \end{equation*}
    which, since $a_1 \neq a_2$, gives that the eigenfunctions are orthogonal.
\end{proof}
\begin{theorem}
    The discrete (or continuous) set of eigenfunctions of any Hermitian operator forms a complete orthonormal basis for the Hilbert space $\hilb$.
    \label{thmHermitianBasis}
\end{theorem}
This theorem will be used without proof.
\begin{remark}
    We say discrete or continuous because some states can be written in a discrete basis,
    \begin{equation*}
        \psi(x, t) = \sum_{i=1}^{N}a_1 \psi_i(x, t),~N \in [1, \infty]
    \end{equation*}
    However, sometimes we need a \textit{continuous basis}:
    \begin{equation*}
        \psi(x, t) = \int_{\Delta \lambda} a(\lambda) \psi_\lambda(x, t) d\lambda
    \end{equation*}
    where here $\lambda \in \R, \Delta \lambda \subseteq \R$.
\end{remark}
\subsection{Expectation Values}
The Copenhagen Interpretation of Quantum Mechanics requires the following postulates:
\begin{enumerate}
    \item[I.] The possible outcomes of a measurement of an observable $O$ are the eigenvalues of the corresponding operator $\op$.
    \item[II.] If $\op$ has a discrete set of normalised eigenfunctions $\{\psi_i\}$ with distinct eigenvalues $\{\lambda_i\}$ then the measurement of the observable $\op$ on a particle described by the state $\psi$ has probability:
        \begin{equation*}
            \mathbb{P}(O = \lambda_i) = |a_i|^2 = |\inn{\psi_i}{\psi}|^2
        \end{equation*}
        where we assume that $\psi = \sum_i a_i \lambda_i$.
    \item[III.] If $\{\psi_i\}$ is a set of orthonormal eigenfunctions of $\op$ and $\{\psi_i\}_{i \in I_\lambda}$ is a complete set of orthonormal eigenfunctions with eigenvalue $\lambda$,
        \begin{equation*}
            P(O = \lambda) = \sum_{i \in I} |a_i|^2 = \sum_{i \in I} |\inn{\psi_i}{\psi}|^2
        \end{equation*}
    \item[IV.] (Projection Postulate): If $O$ is measured on a particle described by a state $\psi$ at time $t$ and the outcome of the measurement is $\lambda_i$, then the wavefunction of $\psi$ instantaneously becomes $\psi(x, t) = \psi_i$ after the measurement. This is the famous \textit{wavefunction collapse}.
\end{enumerate}
\begin{definition}{Projection operator}
    Given a wavefunction $\psi(x, t) = \sum_i a_i \psi_i(x,t)$, \underline{projection operator} is the operator:
    \begin{equation*}
        \hat{P}_i : \psi \mapsto a_i \psi_i = \inn{\psi_i}{\psi} \psi_i
    \end{equation*}
\end{definition}
\begin{definition}{Expectation value}
    Given a state $\psi$ and observable $O$ with operator $\op$, the \underline{expectation value} of $O$ on $\psi$ is:
    \begin{align*}
        \E{O}_\psi &= \sum_i \mathbb{P}(O = \lambda_i) \\
        &= \inn{\psi}{\op\psi}
    \end{align*}
    as we saw earlier in subsection~\ref{sscHeuristicMesurements}.
\end{definition}
Recall Born's Rule: If $\phi$ is the state that gives the desired outcome of a measurement of a particle described by a state $\psi$, then the probability that a measurement on $\psi$ gives this outcome is given by:
\begin{equation*}
    |\inn{\phi}{\psi}|^2 = \left|\int_{-\infty}^{\infty} \phi^*(x, t) \psi(x, t) dx \right|^2
\end{equation*}
That is, we no longer require an eigenfunction.
\section{Time-Independent Schr\"odinger Equation}
We can re-write the TDSE in one dimension:
\begin{align*}
    i\hbar \frac{\partial \psi}{\partial t} &= -\frac{\hbar^2}{2m} \frac{\partial^{2}\psi}{\partial x^{2}} + U(x) \psi(x, t) \\
    &= \hat{H} \psi(x, t)
\end{align*}
where $\hat{H} = \hat{T} + \hat{U}$. We now consider $\psi(x, t) = T(t)\chi(x)$. That is, we separate out the time and space dependences.
\begin{align*}
    i\hbar \frac{\partial T(t)\chi(x)}{\partial t} &= \hat{H}(T(t)\chi(x)) \\
    i\hbar \frac{\partial T(t)}{\partial t}\chi(x) &= T(t)\hat{H}(\chi(x)) \\
    \frac{1}{T(t)} i\hbar \frac{\partial T}{\partial t} &= \frac{\hat{H} \chi(x)}{\chi(x)}
\end{align*}
but here we have that the LHS is a function of $t$, where the RHS is a function of $x$. Therefore, we must have that both are equal to some constant $E$:
\begin{align}
    \frac{1}{T(t)} i\hbar \frac{\partial T}{\partial t} &= E \implies T(t) = e^{-\frac{iEt}{\hbar}} \label{eqnTimeTDSE} \\
    \hat{H} \chi(x) &= E\chi(x) \label{eqnSpaceTDSE}
\end{align}
and here equation~\ref{eqnSpaceTDSE} is the Time-Independent \schr Equation (TISE). We can write it explicitly:
\begin{equation}
    -\frac{\hbar^2}{2m} \frac{\partial^{2}\chi}{\partial x^{2}} + U(x) \chi(x) = E\chi(x)
    \label{eqnTISE}
\end{equation}
However, it is often more useful to consider the Hamiltonian-operator form (\ref{eqnSpaceTDSE}).
\section{Stationary States}
We now have the TISE, and by finding a solution of the TISE we can convert this into a solution of the TDSE:
\begin{definition}{Stationary state}
    For a solution $\chi(x)$ of the TISE, there is a corresponding \underline{stationary state} which solves the TDSE given by:
    \begin{equation}
        \psi(x, t) = e^{-\frac{iEt}{\hbar}} \chi(x)
        \label{eqnStnState}
    \end{equation}
\end{definition}
These are called stationary states because the probability density does not change:
\begin{align*}
    \rho(x, t) &= |\psi(x, t)|^2 \\
    &= \left|e^{-\frac{iEt}{\hbar}} \chi(x)\right| \\
    &= \left|\chi(x)\right|
\end{align*}
\begin{theorem}
    Every solution of the TDSE can be written as a linear combination of stationary states.
    \label{thmStationaryBasis}
\end{theorem}
\begin{remark}
    This is the application of theorem~\ref{thmHermitianBasis} to the operator $\hat{H}$.
\end{remark}
\begin{itemize}
    \item For a system with a discrete set of eigenvalues $E_1, E_2, \cdots$ of $\hat{H}$,
        \begin{equation*}
            \sum_{n \in \N} a_n \chi_n(x) e^{-\frac{iEnt}{\hbar}}
        \end{equation*}
    \item For a system with a continuous set of eigenvalues $E(\alpha)$ for $\alpha \in \R$,
        \begin{equation*}
            \psi(x, t) = \int_{\Delta \alpha} A(\alpha) \chi_\alpha(x) e^{-\frac{iE\alpha t}{\hbar}} d\alpha
        \end{equation*}
        where $A : \R\mapsto\C$.
\end{itemize}
Then using the Copenhagen Interpretation postulates, $|a_n|^2$ or $|A(\alpha)|^2 d\alpha$ is the probability of measuring the energy of a particle to be $E_n$ or $E(\alpha)$.
\begin{example}
    Consider a state with 2 distinct energy eigenvalues $E_1, E_2$.
    \begin{equation*}
        \psi(x, t) = a_1 \chi_1(x) e^{-\frac{iE_1t}{\hbar}} + a_2 \chi_2(x) e^{-\frac{iE_2t}{\hbar}}
    \end{equation*}
    Then at $t = 0$, $\psi(x, t) = a_1 \chi_1(x) + a_2 \chi_2(x)$. If $a_1$ or $a_2$ is zero, then $\psi$ itself is a stationary state.

    The probability density for $\psi$ is:
    \begin{align*}
        \rho(x, t) &= |\psi(x, t)| = \inn{\psi}{\psi} \\
        &= \inn{a_1 \chi_1(x) e^{-\frac{iE_1t}{\hbar}} + a_2 \chi_2(x) e^{-\frac{iE_2t}{\hbar}}}{a_1 \chi_1(x) e^{-\frac{iE_1t}{\hbar}} + a_2 \chi_2(x) e^{-\frac{iE_2t}{\hbar}}} \\
        &= |a_1|^2 \inn{\chi_1}{\chi_1} + |a_2|^2 \inn{\chi_2}{\chi_2} + a_1^* a_2 \chi_1^* \chi_2 e^{-\frac{i(E_1 - E_2) t}{\hbar}} \\
        &+ a_2^* a_1 \chi_2^* \chi_1 e^{-\frac{i(E_2 - E_1) t}{\hbar}} \\
    \end{align*}
    Now let $a_1 = |a_1|e^{i\alpha_1}$, same for $a_2$.
    \begin{align*}
        \rho(x, t) &= |a_1|^2 + |a_2|^2 + |a_1||a_2|\left(e^{-i(\alpha_1 - \alpha_2)} e^{-\frac{i(E_1 - E_2)t}{\hbar}}\right. \\
        &\left.+ e^{i(\alpha_1 - \alpha_2)}e^{\frac{i(E_1 - E_2)t}{\hbar}}\right) \\
        &= 1 + 2|a_2||a_2|\chi_1(x) \chi_2(x) \cos\left(\frac{(E_1 - E_2)t}{\hbar} + \alpha_1 - \alpha_2\right)
    \end{align*}
\end{example}
\end{document}