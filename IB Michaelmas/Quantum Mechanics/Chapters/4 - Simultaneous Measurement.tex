\documentclass[../Main.tex]{subfiles}

\begin{document}
Recall that, for an operator $\op$, the possible outcomes of a measurement are given by its eigenvalues. Given a wavefunction of the form $\psi = \sum_{i=1}^{\infty} a_i \psi_i$, the probability that a measurement results in a given eigenvalue $\lambda_i$ is:
\begin{equation*}
    \P(O = \lambda_i) = |a_i|^2 = |\inn{\psi_i}{\psi}|^2
\end{equation*}
Once a measurement has been performed, we have the projection postulate which describes wavefunction collapse, and the wavefunction becomes $\psi_i$.
\section{Commutators}
\begin{definition}{Commutator}
    The \underline{commutator} of two Hermitian operators $\hat{A}, \hat{B}$ is the operator defined as:
    \begin{equation*}
        [\hat{A}, \hat{B}] = \hat{A}\hat{B} - \hat{B}\hat{A}
    \end{equation*}
\end{definition}
\begin{propositions}{
        Let $\hat{A}, \hat{B}$ be hermitian operators.
        \label{propsCommutator}
    }
    \item $[\hat{A}, \hat{B}] = -[\hat{B}, \hat{A}]$
    \item $\cmmt{A}{A} = 0$
    \item $[\hat{A}, \hat{B}\hat{C}] = \cmmt{A}{B}\hat{C} + \hat{B} \cmmt{A}{C}$
    \item $[\hat{A}\hat{B}, \hat{C}] = \hat{A}\cmmt{B}{C} + \cmmt{A}{C}\hat{B}$
\end{propositions}
\begin{example}
    We consider the commutator $\cmmt{x}{p}$ in one dimension. Consider $\psi \in \hilb$:
    \begin{align*}
        \hat{x} \hat{p} \psi &= x\left(-i\hbar \frac{\partial }{\partial x}\right)\psi(x) \\
        &= -i\hbar x \frac{\partial \psi(x)}{\partial x} \\
        \hat{p} \hat{x} \psi &= \left(-i\hbar \frac{\partial }{\partial x}\right) \left(x \psi(x)\right) \\
        &= -i\hbar \psi(x) - i\hbar x \frac{\partial \psi(x)}{\partial x} \\
        \cmmt{x}{p} &= -i\hbar x \frac{\partial \psi(x) }{\partial x} + i \hbar \psi(x) + i \hbar x \frac{\partial \psi(x)}{\partial x} \\
        &= i \hbar \psi(x)
    \end{align*}
    Therefore we conclude $\cmmt{x}{p} = i\hbar \hat{I}$ where $\hat{I}$ is the identity operator. This is called the \underline{canonical commutator relation}.
\end{example}
\begin{definition}{Simultaneous diagonalisability}
    Two Hermitian operators $\hat{A}$, $\hat{B}$ are \underline{simultaneously diagonalisable} in $\hilb$ if there exists a complete basis of joint eigenfunctions $\psi_i$ such that:
    \begin{equation}
        \begin{aligned}
            \hat{A}\psi_i &= a_i \psi_i \\
            \hat{B}\psi_i &= b_i \psi_i
        \end{aligned}
        \label{eqnSimDiag}
    \end{equation}
\end{definition}
\begin{theorem}
    Two Hermitian operators $\hat{A}, \hat{B}$ are simultaneously diagonalisable if and only if $\cmmt{A}{B} = 0$.
    \label{thmSimDiagIffCmmtZero}
\end{theorem}
\begin{proof}
    \begin{proofdirection}{$\Rightarrow$}{Suppose that $\hat{A}, \hat{B}$ are simultanously diagonalisable.}
        Therefore, we have eigenfunctions $\psi_i$ that form a complete basis of $\hilb$ as in equation~\ref{eqnSimDiag}. Therefore:
        \begin{align*}
            \cmmt{A}{B}\psi_i &= \hat{A} \hat{B} \psi_i - \hat{B}\hat{A} \psi_i \\
            &= a_i b_i \psi_i - b_i a_i \psi_i = 0
        \end{align*}
        Then we can apply this to any linear combination:
        \begin{align*}
            \cmmt{A}{B} \sum_{i} a_i \psi_i &= \sum_i a_i \cmmt{A}{B} \psi_i \\
            &= 0
        \end{align*}
    \end{proofdirection}
    \begin{proofdirection}{$\Leftarrow$}{Suppose that the commutator is zero}
        Let $\psi_i$ be the eigenfunctions of $\hat{A}$ with eigenvalues $a_i$.
        \begin{align*}
            0 &= \cmmt{A}{B}\psi_i = \hat{A} \hat{B} \psi_i - \hat{B} \hat{A} \psi_i \\
            &= \hat{A} \hat{B} \psi_i = a_i \hat{B} \psi_i
        \end{align*}
        That is,
        \begin{equation*}
            \hat{A}(\hat{B}\psi_i) = a_i \hat{B} \psi_i
        \end{equation*}
        Then $\hat{B}$ maps the eigenspace $E_i$ of $\hat{A}$ with eigenvalue $a_i$ into itself. Therefore, the restricted map $B|_{E_i}$ is a Hermitian operator of $E_i$. Similarly, because the $E_i$ are complete and thus span $\hilb$, we can find a basis of joint eigenfunctions of $\hat{A}$ and $\hat{B}$.
    \end{proofdirection}
\end{proof}
\section{Uncertainty of Simultaneous Measurement}
\begin{definition}{Uncertainty}
    The \underline{uncertainty} in a measurement of an observable $O$ on a state $\psi$ is defined as:
    \begin{equation*}
        \Delta_\psi O = \sqrt{(\Delta_\psi O)}
    \end{equation*}
    where:
    \begin{align*}
        (\Delta_\psi O)^2 &= \E{(\op - \E{op}_\psi \hat{I})}_\psi \\
        &= \E{\op^2}_\psi - \left(\E{\op}_\psi\right)^2
    \end{align*}
\end{definition}
Note that the two definitions are equivalent:
\begin{align*}
    \E{(\hat{A} - \E{\hat{A}}_\psi \hat{I})^2}&= \inn{\psi}{(\hat{A} - \E{\hat{A}}_\psi \hat{I})^2\psi} \\
    &= \inn{\psi}{\left(\hat{A}^2 - 2\E{\hat{A}}_\psi \hat{A} + \left(\E{\hat{A}}_\psi\right)^2\hat{I}\right)\psi} \\
    &= \inn{\psi}{\hat{A}^2 \psi - 2\E{\hat{A}}_\psi \hat{A}\psi + \left(\E{\hat{A}}_\psi\right)^2\psi} \\
    &= \inn{\psi}{\hat{A}^2 \psi} - 2\E{\hat{A}}_\psi \inn{\psi}{\hat{A}\psi} + \left(\E{\hat{A}}_\psi\right)^2\inn{\psi}{\psi} \\
    &= \E{\hat{A}^2}_\psi - 2\left(\E{\hat{A}}_\psi\right)^2 + \left(\E{\hat{A}}_\psi\right)^2 \\
    &= \E{\hat{A}^2}_\psi - \left(\E{\hat{A}}_\psi\right)^2
\end{align*}
\begin{lemma}
    Let $\psi$ be a state in $\hilb$ and $A$ an observable. Then $(\Delta_{\psi} A)^2 \geq 0$ with equality if and only if $\psi$ is an eigenfunction of $\hat{A}$.
    \label{lemUncProps}
\end{lemma}
\begin{proof}
    \begin{align*}
        (\Delta_\psi A)^2 &= \inn{\psi}{(\hat{A} - \E{\hat{A}}_\psi \hat{I})^2 \psi} \\
        &= \inn{(\hat{A} - \E{\hat{A}}_\psi \hat{I})\psi}{(\hat{A} - \E{\hat{A}}_\psi \hat{I})\psi} \text{ because $A$ is Hermitian} \\
        &= ||(\hat{A} - \E{\hat{A}}_\psi \hat{I})\psi||^2 \\
        &= ||\phi||^2 \geq 0
    \end{align*}
    Then this new state $\phi$ as defined above has zero norm if and only if it is in fact zero.
    For the second part, consider both directions:
    \begin{proofdirection}{$\Rightarrow$}{Suppose that $\phi = 0$.}
        Then $\psi$ obeys:
        \begin{equation*}
            \hat{A} \psi = \E{\hat{A}}_\psi \psi
        \end{equation*}
        so $\psi$ must be an eigenfunction.
    \end{proofdirection}
    \begin{proofdirection}{$\Leftarrow$}{Suppose that $\psi$ is an eigenfunction of $\hat{A}$, with eigenvalue $a$.}
        Then $\E{\hat{A}}_\psi = \inn{\psi}{\hat{A}\psi} = a\inn{\psi}{\psi} = a$.

        Similarly, $\E{\hat{A}^2}_\psi = \inn{\psi}{\hat{A}^2\psi} = a^2\inn{\psi}{\psi} = a^2$.

        Then we have that $(\Delta_\psi A)^2 = \E{\hat{A}^2} - \left(\E{\hat{A}}\right)^2 = 0$.
    \end{proofdirection}
\end{proof}
\begin{lemma}[Cauchy-Schwarz inequality]
    For states $\psi, \phi \in \hilb$,
    \begin{equation*}
        |\inn{\psi}{\phi}|^2 \leq ||\psi||^2 ||\phi||^2
    \end{equation*}
    with equality only if $\phi = a\psi$ for $a \in \C, |a| = 1$. That is, they are equivalent states.
    \label{lemStateCSIneq}
\end{lemma}
\begin{theorem}[Generalised Uncertainty Principle]
    If $A$, $B$ are observables and $\psi \in \hilb$,
    \begin{equation}
        \left(\Delta_\psi A\right)\left(\Delta_\psi B\right) \geq \frac12 \left|\inn{\psi}{\cmmt{A}{B}\psi}\right|
        \label{eqnUncertaintyPrinciple}
    \end{equation}
    \label{thmUncertaintyPrinciple}
\end{theorem}
\begin{proof}
    First note that:
    \begin{align*}
        (\Delta_\psi A)^2 &= \inn{(\hat{A} - \E{\hat{A}}_\psi \hat{I})\psi}{(\hat{A} - \E{\hat{A}}_\psi \hat{I})\psi} \\
        (\Delta_\psi B)^2 &= \inn{(\hat{B} - \E{\hat{B}}_\psi \hat{I})\psi}{(\hat{B} - \E{\hat{B}}_\psi \hat{I})\psi}
    \end{align*}

    Define also:
    \begin{align*}
        \hat{A}' &= \hat{A} - \E{\hat{A}}_\psi \hat{I} \\
        \hat{B}' &= \hat{B} - \E{\hat{B}}_\psi \hat{I} \\
        (\Delta_\psi A)^2 &= \inn{\hat{A}' \psi}{\hat{A}' \psi} \\
        (\Delta_\psi B)^2 &= \inn{\hat{B}' \psi}{\hat{B}' \psi}
    \end{align*}
    By lemma~\ref{lemStateCSIneq},
    \begin{equation}
        \begin{split}
            (\Delta_\psi A)^2 (\Delta_\psi B)^2 &\geq |\inn{\hat{A}'\psi}{\hat{B}'\psi}|^2 \\
            &= |\inn{\psi}{\hat{A}'\hat{B}'\psi}
            \label{eqnCSUncertainty}
        \end{split}
    \end{equation}
    Define:
    \begin{gather}
        [\hat{A}', \hat{B}'] = \hat{A}' \hat{B}' - \hat{B}' \hat{A}' \label{eqnCommutator} \\
        \{\hat{A}', \hat{B}'\} = \hat{A}' \hat{B}' + \hat{B}' \hat{A}' \label{eqnAntiCommutator}
    \end{gather}
    where here equation~\ref{eqnAntiCommutator} is the anti-commutator. Given that $\hat{A}'$ and $\hat{B}'$ are Hermitian:
    \begin{equation}
        \begin{split}
            &[\hat{A}', \hat{B}']^\dagger = -[\hat{A}', \hat{B}'] \\
            &\{\hat{A}', \hat{B}'\}^\dagger = \{\hat{A}', \hat{B}'\}
        \end{split}
        \label{eqnCommutatorHermitian}
    \end{equation}
    Now write:
    \begin{equation}
        \hat{A}' \hat{B}' = \frac12 \left([\hat{A}', \hat{B}'] + \{\hat{A}', \hat{B}'\}\right)
        \label{eqnCommutatorSum}
    \end{equation}
    Note that, by equation~\ref{eqnCommutatorHermitian},
    \begin{align*}
        &\inn{\psi}{\{\hat{A}', \hat{B}'\}\psi} \text{ is real,} \\
        &\inn{\psi}{[\hat{A}', \hat{B}']\psi} \text{ is pure imaginary.} \\
    \end{align*}
    Then substitute equation~\ref{eqnCommutatorSum} into equation~\ref{eqnCSUncertainty}:
    \begin{align*}
        (\Delta_\psi A)^2& (\Delta_\psi B)^2\geq \left|\inn{\psi}{\frac12 \left([\hat{A}', \hat{B}'] + \{\hat{A}', \hat{B}'\}\right)}\right|^2 \\
        &= \frac14 \left|\inn{\psi}{[\hat{A}', \hat{B}']} + \inn{\psi}{\{\hat{A}', \hat{B}'\}}\right|^2 \\
        &= \frac14 \left|\inn{\psi}{[\hat{A}', \hat{B}']}\right|^2 + \left|\inn{\psi}{\{\hat{A}', \hat{B}'\}}\right|^2 \text{ by real-imaginary} \\
        &\geq \frac14 \left|\inn{\psi}{[\hat{A}', \hat{B}']}\right|^2 \\
        &= \frac14 \left|\inn{\psi}{\cmmt{A}{B}}\right|^2 \\
    \end{align*}
    Then take square roots for the answer.
\end{proof}
\begin{corollary}
    For two operators $\hat{A}, \hat{B}$ that commute ($\cmmt{A}{B} = 0$), if $\psi \in \hilb$ then $A$ and $B$ can be measured simultaneously on $\psi$ with theoretically zero uncertainty.
    \label{corCommuteImpliesSimultaneous}
\end{corollary}
\begin{proof}
    Observe that this fact drops out of the statement of theorem~\ref{thmUncertaintyPrinciple}
\end{proof}
\begin{corollary}[Heisenberg Uncertainty Principle]
    For operators $\hat{x}$ and $\hat{p}$, the minimum uncertainty for a simultaneous measurement is:
    \begin{equation*}
        (\Delta_\psi x) (\Delta_\psi p) \geq \frac\hbar2
    \end{equation*}
    \label{corHeisenbergUncertainty}
\end{corollary}
\begin{proof}
    We found earlier that $\cmmt{x}{p} = i \hbar \hat{I}$, so applying this to theorem~\ref{thmUncertaintyPrinciple} gives the result.
\end{proof}
We have shown that indeed this lower bound is achieved by the Gaussian Wavepacket. In fact, we have the following statements:
\begin{proposition}
    For $\psi \in \hilb$, $\psi$ is a state of minimum uncertainty with respect to $\hat{x}$ and $\hat{p}$ if and only if $\hat{x} \psi = i a \hat{p} \psi$ for $a \in \R$.

    This condition is equivalent to:
    \begin{equation*}
        \psi(x) = C e^{-b x^2}
    \end{equation*}
    for $c \in \C, b \in \R^+$.
    \label{propMinUncConditions}
\end{proposition}
\section{Ehrenfest Theorem}
\begin{theorem}[Ehrenfest theorem]
    The expectation of an operator $\hat{A}$ for a state $\psi \in \hilb$ evolves according to:
    \begin{equation}
        \frac{d}{dt} \E{\hat{A}}_\psi = \frac{i}{\hbar} \E{\cmmt{H}{A}}_\psi + \E{\frac{\partial \hat{A}}{\partial t}}_\psi
        \label{eqnEhrenfest}
    \end{equation}
    \label{thmEhrenfest}
\end{theorem}
\begin{proof}
    \begin{align*}
        \frac{d}{dt} \E{\hat{A}}_\psi &= \frac{d}{dt}\int_{-\infty}^{\infty} \psi^*(x, t) \hat{A} \psi(x, t) dx \\
        &= \int_{-\infty}^{\infty} \frac{\partial}{\partial t} (\psi^* \hat{A} \psi) dx \\
        &= \int_{-\infty}^{\infty} \left(\frac{\partial \psi^*}{\partial t} \hat{A} \psi + \psi^* \frac{\partial \hat{A}}{\partial t}\psi + \psi^* \hat{A} \frac{\partial \psi}{\partial t}\right) dx \\
        \intertext{Apply TDSE:}
        &=\int_{-\infty}^\infty \left(-\frac{i}{\hbar}\hat{H}\psi^*\right)^*\hat{A}\psi + \int_{-\infty}^{\infty} -\frac{i}{\hbar}\psi^* \hat{A}\hat{H}\psi dx + \E{\frac{\partial \hat{A}}{\partial t}}_\psi\\
        &=\int_{-\infty}^\infty \frac{i}{\hbar}\psi^*\hat{H}\hat{A}\psi + \int_{-\infty}^{\infty} -\frac{i}{\hbar}\psi^* \hat{A}\hat{H}\psi dx + \E{\frac{\partial \hat{A}}{\partial t}}_\psi\\
        &= \frac{i}{\hbar} \int_{-\infty}^{\infty} \psi^* (\hat{H} \hat{A} - \hat{A} \hat{H})\psi dx + \E{\frac{\partial \hat{A}}{\partial t}}_\psi \\
        &= \frac{i}{\hbar} \E{\cmmt{H}{A}}_\psi + \E{\frac{\partial \hat{A}}{\partial t}}_\psi
    \end{align*}
\end{proof}
\begin{examples}{
        Considering different operators $\hat{A}$ in theorem~\ref{thmEhrenfest} gives us results that we can understand from Classical Mechanics:
    }
    \item Taking $\hat{A} = \hat{H}$ gives $\frac{dE}{dt} = 0$, conservation of energy.
    \item Taking $\hat{A} = \hat{P}$ gives:
        \begin{equation*}
            \frac{d}{dt} \E{\hat{p}}_\psi = -\E{\frac{\partial U}{\partial t}}_\psi
        \end{equation*}
        where $U$ is the potential.
    \item Taking $\hat{A} = \hat{x}$ gives:
        \begin{equation*}
            \frac{d\E{\hat{x}}_\psi}{dt} = \frac{\E{\hat{p}}_\psi}{m}
        \end{equation*}
\end{examples}
\nonexaminablesection{Harmonic Oscillator, Revisited}
Write the Hamiltonian operator in terms of $\hat{p}$ and $\hat{x}$:
\begin{equation*}
    \hat{H} = \frac{\hat{p}^2}{2m} + \frac12 m \omega^2 \hat{x}^2
\end{equation*}
Then we want to know the eigenvalues and eigenfunctions of the Hamiltonian. We can re-write as:
\begin{align*}
    \hat{H} &= \frac{1}{2m}\left(\hat{p} + i m \omega \hat{x}\right)(\hat{p} - i m \omega \hat{x}) + \frac{i \omega}{2} \cmmt{p}{x} \\
    &= \frac{1}{2m}\left(\hat{p} + i m \omega \hat{x}\right)(\hat{p} - i m \omega \hat{x}) + \frac{\hbar \omega}{2} \hat{I}
\end{align*}
Define the \underline{ladder operators}:
\begin{align*}
    \hat{a} &= \frac{1}{\sqrt{2m}} (\hat{p} - i \omega \hat{x}) \\
    \hat{a}^\dagger &= \frac{1}{\sqrt{2m}} (\hat{p} + i \omega \hat{x})
\end{align*}    
Therefore we can rewrite the Hamiltonian once again:
\begin{equation*}
    \hat{H} = \hat{a}^\dagger \hat{a} + \frac{\hbar \omega}{2} \hat{I}
\end{equation*}
Then we commute the following commutators:
\begin{align*}
    [\hat{a}, \hat{a}^\dagger] &= \frac{1}{2m} [\hat{p} - i m \omega \hat{x}, \hat{p} + i m \omega \hat{x}] \\
    &\qquad= -\frac{i m \omega}{2m} \cmmt{x}{p} + \frac{i m \omega}{2m} \cmmt{p}{x} \\
    &\qquad= \hbar \omega \hat{I} \\
    \cmmt{H}{a} &= [\hat{a}^\dagger \hat{a}, \hat{a}] = -\hbar \omega \hat{a} \\
    [\hat{H}, \hat{a}^\dagger] &= [\hat{a}^\dagger \hat{a}, \hat{a}^\dagger] = \hbar \omega \hat{a}^\dagger
\end{align*}
Now suppose that $\chi$ is an eigenfunction of $\hat{H}$ with eigenvalue $E$. Then $\hat{H} \chi = E\chi$. Now take $\hat{a} \chi$ and consider its energy:
\begin{align*}
    \hat{H}(\hat{a}\chi) &= \hat{H} \hat{a} \chi = \cmmt{H}{a} \chi + \hat{a} \hat{H} \chi \\
    &= -\hbar \omega \hat{a} \chi + E\hat{a} \chi \\
    &= (E - \hbar \omega) \hat{a} \chi
\end{align*}
So we find that $\hat{a} \chi$ is an energy eigenfunction with eigenvalue $E - \hbar \omega$. We have found that $\hat{a}$ ``steps down'' an energy eigenstate. Analogously, $\hat{a}^\dagger \chi$ is an eigenfunction with energy $(E + \hbar \omega)$. This is why we called $\hat{a}$ and $\hat{a}^\dagger$ the \textit{ladder operators}.

Then we can easily show (by induction), that:
\begin{align*}
    (\hat{a}^n \chi)&\text{ is an energy eigenstate with energy eigenvalue }E - n \hbar \omega \\
    ((\hat{a}^\dagger)^n \chi)&\text{ is an energy eigenstate with energy eigenvalue }E + n \hbar \omega
\end{align*}
Using the fact that, for the Quantum Harmonic Oscillator, the energy must be non-negative, we can find the ground state by solving:
\begin{equation*}
    \hat{a} \chi_0 = 0
\end{equation*}
Solving this:
\begin{align*}
    0&= \frac{1}{\sqrt{2m}}(\hat{p} - i m \omega \hat{x}) \chi_0 \\
    &= -i \hbar \frac{d\chi_0}{dx} - i m \omega x \chi_0 \\
    \implies \chi_0(x) &= Ce^{-m \omega x^2 / 2\hbar} \\
    \hat{H} \chi_0(x) &= \hat{a}^\dagger \hat{a} \chi_0 + \frac{\hbar \omega}{2} \hat{I} \chi_0 \\
    &= 0 + \frac{\hbar \omega}{2} \chi_0
\end{align*}
So we have found the ground state with energy $\frac{\hbar \omega}{2}$ and eigenfunction $\chi_0(x) = Ce^{-m \omega x^2 / 2\hbar}$. To find the higher energy states, we can simply apply $\hat{a}$ to $\chi_0$ to get $\chi_1$, etc. We find, as we expect, that these will have energy eigenvalues:
\begin{equation*}
    E_n = \left(n + \frac12\right)\hbar \omega
\end{equation*}
\end{document}