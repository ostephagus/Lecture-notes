\documentclass[../Main.tex]{subfiles}

\begin{document}
Recall that, for an operator $\op$, the possible outcomes of a measurement are given by its eigenvalues. Given a wavefunction of the form $\psi = \sum_{i=1}^{\infty} a_i \psi_i$, the probability that a measurement results in a given eigenvalue $\lambda_i$ is:
\begin{equation*}
    \P(O = \lambda_i) = |a_i|^2 = |\inn{\psi_i}{\psi}|^2
\end{equation*}
Once a measurement has been performed, we have the projection postulate which describes wavefunction collapse, and the wavefunction becomes $\psi_i$.
\section{Commutators}
\begin{definition}{Commutator}
    The \underline{commutator} of two Hermitian operators $\hat{A}, \hat{B}$ is the operator defined as:
    \begin{equation*}
        [\hat{A}, \hat{B}] = \hat{A}\hat{B} - \hat{B}\hat{A}
    \end{equation*}
\end{definition}
\begin{propositions}{
        Let $\hat{A}, \hat{B}$ be hermitian operators.
        \label{propsCommutator}
    }
    \item $[\hat{A}, \hat{B}] = -[\hat{B}, \hat{A}]$
    \item $\cmmt{A}{A} = 0$
    \item $[\hat{A}, \hat{B}\hat{C}] = \cmmt{A}{B}\hat{C} + \hat{B} \cmmt{A}{C}$
    \item $[\hat{A}\hat{B}, \hat{C}] = \hat{A}\cmmt{B}{C} + \cmmt{A}{C}\hat{B}$
\end{propositions}
\begin{example}
    We consider the commutator $\cmmt{x}{p}$ in one dimension. Consider $\psi \in \hilb$:
    \begin{align*}
        \hat{x} \hat{p} \psi &= x\left(-i\hbar \frac{\partial }{\partial x}\right)\psi(x) \\
        &= -i\hbar x \frac{\partial \psi(x)}{\partial x} \\
        \hat{p} \hat{x} \psi &= \left(-i\hbar \frac{\partial }{\partial x}\right) \left(x \psi(x)\right) \\
        &= -i\hbar \psi(x) - i\hbar x \frac{\partial \psi(x)}{\partial x} \\
        \cmmt{x}{p} &= -i\hbar x \frac{\partial \psi(x) }{\partial x} + i \hbar \psi(x) + i \hbar x \frac{\partial \psi(x)}{\partial x} \\
        &= i \hbar \psi(x)
    \end{align*}
    Therefore we conclude $\cmmt{x}{p} = i\hbar \hat{I}$ where $\hat{I}$ is the identity operator. This is called the \underline{canonical commutator relation}.
\end{example}
\begin{definition}{Simultaneous diagonalisability}
    Two Hermitian operators $\hat{A}$, $\hat{B}$ are \underline{simultaneously diagonalisable} in $\hilb$ if there exists a complete basis of joint eigenfunctions $\psi_i$ such that:
    \begin{equation}
        \begin{aligned}
            \hat{A}\psi_i &= a_i \psi_i \\
            \hat{B}\psi_i &= b_i \psi_i
        \end{aligned}
        \label{eqnSimDiag}
    \end{equation}
\end{definition}
\begin{theorem}
    Two Hermitian operators $\hat{A}, \hat{B}$ are simultaneously diagonalisable if and only if $\cmmt{A}{B} = 0$.
    \label{thmSimDiagIffCmmtZero}
\end{theorem}
\begin{proof}
    \begin{proofdirection}{$\Rightarrow$}{Suppose that $\hat{A}, \hat{B}$ are simultanously diagonalisable.}
        Therefore, we have eigenfunctions $\psi_i$ that form a complete basis of $\hilb$ as in equation~\ref{eqnSimDiag}. Therefore:
        \begin{align*}
            \cmmt{A}{B}\psi_i &= \hat{A} \hat{B} \psi_i - \hat{B}\hat{A} \psi_i \\
            &= a_i b_i \psi_i - b_i a_i \psi_i = 0
        \end{align*}
        Then we can apply this to any linear combination:
        \begin{align*}
            \cmmt{A}{B} \sum_{i} a_i \psi_i &= \sum_i a_i \cmmt{A}{B} \psi_i \\
            &= 0
        \end{align*}
    \end{proofdirection}
    \begin{proofdirection}{$\Leftarrow$}{Suppose that the commutator is zero}
        Let $\psi_i$ be the eigenfunctions of $\hat{A}$ with eigenvalues $a_i$.
        \begin{align*}
            0 &= \cmmt{A}{B}\psi_i = \hat{A} \hat{B} \psi_i - \hat{B} \hat{A} \psi_i \\
            &= \hat{A} \hat{B} \psi_i = a_i \hat{B} \psi_i
        \end{align*}
        That is,
        \begin{equation*}
            \hat{A}(\hat{B}\psi_i) = a_i \hat{B} \psi_i
        \end{equation*}
        Then $\hat{B}$ maps the eigenspace $E_i$ of $\hat{A}$ with eigenvalue $a_i$ into itself. Therefore, the restricted map $B|_{E_i}$ is a Hermitian operator of $E_i$. Similarly, because the $E_i$ are complete and thus span $\hilb$, we can find a basis of joint eigenfunctions of $\hat{A}$ and $\hat{B}$.
    \end{proofdirection}
\end{proof}
\end{document}