\documentclass[../Main.tex]{subfiles}

\begin{document}
In three dimensions, the Time Independent Schr\"odinger Equation is:
\begin{equation}
    -\frac{\hbar^2}{2m} \nabla^2 \chi(\vec{x}) + U(\vec{x}) \chi(\vec{x}) = E \chi(\vec{x})
    \label{eqnTISE3D}
\end{equation}
\section{Reminder of Coordinate Systems}
We will use Cartesian coordinates $(x, y, z)$ and Spherical coordinates $(r, \theta, \phi)$. We denote the angle from the positive $z$-axis by $\theta$.
In these coordinate systems, the Laplacian is:
\begin{align*}
    \text{Cartesian: } \nabla^2&= \frac{\partial^{2}}{\partial x^{2}} + \frac{\partial^{2}}{\partial y^{2}} + \frac{\partial^{2}}{\partial z^{2}} \\
    \text{Spherical: } \nabla^2 &= \frac{1}{r^2} \frac{\partial}{\partial r}\left(r^2\frac{\partial}{\partial r}\right) + \frac{1}{r^2 \sin{\theta}} \frac{\partial }{\partial \theta}\left(\sin{\theta} \frac{\partial}{\partial \theta}\right)\\
    &\qquad+ \frac{1}{r^2 \sin^2{\theta}}\frac{\partial^{2}}{\partial \phi^{2}}
\end{align*}
For the spherical polars expression, we can simplify the term in $r$:
\begin{equation*}
    \nabla f = \frac{1}{r} \frac{\partial^2}{\partial r^2} (rf) + \frac{1}{r^2 \sin^2{\theta}}\left[\sin{\theta} \frac{\partial}{\partial \theta}\left(\sin{\theta} \frac{\partial f}{\partial \theta}\right) + \frac{\partial^{2}f}{\partial \phi^{2}}\right] 
\end{equation*}
Recall the different volume integrals:
\begin{align*}
    \text{Cartesian: }&\int_{\R^3} dV = \int_{x = -\infty}^{\infty} \int_{y = -\infty}^{\infty} \int_{z = -\infty}^{\infty} f(x, y, z) dz~dy~dx \\
    \text{Spherical: }&\int_{\R^3} dV = \int_{r = 0}^{\infty} \int_{\phi = 0}^{2\pi} \int_{\theta = 0}^{\pi} f(r, \phi, \theta) r^2 \sin{\theta}~d\theta~d\phi~dr \\
\end{align*}
We will consider only spherical potentials $U(\vec{x}) = U(r)$.

\section{Angular Momentum}
\subsection{Definition and Commutativity}
In classical mechanics, we used the constant angular momentum to simplify 3-dimensional problems with a central potential into 2-dimensional ones. We will also define the angular momentum in Quantum Mechanics:
\begin{definition}{Angular momentum}
    The \underline{angular momentum} in Quantum mechanics is an operator:
    \begin{equation}
        \opvec{L} = \opvec{x} \times \opvec{p} = -i\hbar \opvec{x} \times \nabla
        \label{eqnQAngMomentum}
    \end{equation}
\end{definition}
We can show (an indeed do in Ex3) that $\opvec{L}$ is Hermitian, and so are its components $\hat{L}_i$.
\begin{proposition}
    For distinct $i, j \in \{1, 2, 3\}$, the commutator of different elements of the angular momentum operator is:
    \begin{equation}
        [\hat{L}_i, \hat{L}_j] = i \hbar \epsilon_{ijk} \hat{L}_k
        \label{eqnAngMomCmmt}
    \end{equation}
    \label{propAngMomCmmt}
\end{proposition}
\begin{remark}
    The commutator is non-zero, which tells us that we cannot measure the angular momentum to arbitrary precision, because if we measure one element to arbitrary precision then we have a lower bound on the precision of the other two elements' measurements.
\end{remark}
\begin{proof}
    The proof is simply by multiplying the commutator out. First we need the commutator of $\hat{x}_i$ and $\hat{p}_j$:
    \begin{align*}
        [\hat{x}_i, \hat{p}_j] &= \hat{x}_i \hat{p}_j - \hat{p}_j \hat{x}_i \\
        &= -i\hbar \left(x_i \frac{\partial}{\partial x_j} - \frac{\partial}{\partial x_j} (x_i \cdot)\right) \\
        &= -i\hbar \left(x_i \frac{\partial}{\partial x_j} - x_i\frac{\partial}{\partial x_j} - \left(\frac{\partial x_i}{\partial x_j} \times \cdot \right)\right) \\
        &= -i\hbar \left(- \delta_{ij} \hat{I}\right) \\
        &= i\hbar \delta_{ij} \hat{I}
    \end{align*}
    Now we can use this to find $\hat{L}_i \hat{L}_j$:
    \begin{align*}
        \hat{L}_i \hat{L}_j&= \epsilon_{ikl} \hat{x}_k \hat{p}_l \epsilon_{jmn} \hat{x}_m \hat{p}_n \\
        &= \epsilon_{ikl} \epsilon_{jmn} \hat{x}_k \hat{p}_l \hat{x}_m \hat{p}_n \\
        &= \epsilon_{ikl} \epsilon_{jmn} \hat{x}_k (\hat{x}_m \hat{p}_l - [\hat{p}_l, \hat{x}_m]) \hat{p}_n \\
        &= \epsilon_{ikl} \epsilon_{jmn} \hat{x}_k (\hat{x}_m \hat{p}_l - i\hbar \delta_{lm} \hat{I}) \hat{p}_n \\
        &= \epsilon_{ikl} \epsilon_{jmn} \hat{x}_k \hat{x}_m \hat{p}_l \hat{p}_n - i \hbar\epsilon_{ikl} \epsilon_{jln} \hat{x}_k \hat{p}_n \\
        &= \epsilon_{ikl} \epsilon_{jmn} \hat{x}_k \hat{x}_m \hat{p}_l \hat{p}_n - i \hbar (\delta_{in} \delta_{kj} - \delta_{ij} \delta_{kn}) \hat{x}_k \hat{p}_n \\ 
        &= \epsilon_{ikl} \epsilon_{jmn} \hat{x}_k \hat{x}_m \hat{p}_l \hat{p}_n - i \hbar \hat{x}_j \hat{p}_i
    \end{align*}
    Then similarly, $\hat{L}_j \hat{L}_i = \epsilon_{jkl} \epsilon_{imn} \hat{x}_k \hat{x}_m \hat{p}_l \hat{p}_n - i \hbar \hat{x}_i \hat{p}_j$. Re-label the indices to match with the previous expression:
    \begin{align*}
        \hat{L}_j \hat{L}_i &= \epsilon_{jmn} \epsilon_{ikl} \hat{x}_m \hat{x}_k \hat{p}_n \hat{p}_l - i \hbar \hat{x}_i \hat{p}_j \\
        \hat{L}_j \hat{L}_i &= \epsilon_{jmn} \epsilon_{ikl} \hat{x}_k \hat{x}_m \hat{p}_l \hat{p}_n - i \hbar \hat{x}_i \hat{p}_j  \text{ by commutativity}
    \end{align*}
    which has the same first part as $\hat{L}_i \hat{L}_j$.

    Now find the commutator we want:
    \begin{align}
        [\hat{L}_i, \hat{L}_j] &= -i\hbar(\hat{x}_j \hat{p}_i - \hat{x}_i \hat{p}_j) \nonumber\\
        &= i\hbar(\hat{x}_i \hat{p}_j - \hat{x}_j \hat{p}_i) \label{eqnAngMomCmmt2}
    \end{align}
    and, by expanding out the $\epsilon_{ijk}$ in equation~\ref{eqnAngMomCmmt} in terms of index notation, we find that this is indeed the required result.
\end{proof}
\begin{remark}
    Equation~\ref{eqnAngMomCmmt2} gives an interesting alternate form to the commutator.
\end{remark}
\begin{definition}{Total angular momentum}
    The \underline{total angular momentum} operator is:
    \begin{equation}
        \hat{L}^2 = \hat{L}_1^2 + \hat{L}_2^2 + \hat{L}_3^2 = |\opvec{L}|^2
        \label{eqnTotAngMomentum}
    \end{equation}
\end{definition}
We can prove that the commutator is:
\begin{equation*}
    [\hat{L}^2, \hat{L}_i] = 0
\end{equation*}
This tells us that we can find the magnitude of $\opvec{L}$ and one of its components with arbitrary precision. We generally find the $z$-component of the angular momentum and its magnitude. This is equivalent to simultaneously diagonalising the two operators.
\subsection{Eigenfunctions of Angular Momentum}
\label{secSphericalHarmonics}
To find the joint set of eigenfunctions of $\hat{L}^2$ and $\hat{L}_3$, we write $\opvec{L}$ in spherical coordinates. This results in:
\begin{align*}
    \hat{L}_3 &= -i \hbar \frac{\partial}{\partial \phi} \\
    \hat{L}^2 &= -\frac{\hbar^2}{\sin^2{\theta}} \left[\sin{\theta} \frac{\partial}{\partial \theta}\left(\sin{\theta} \frac{\partial}{\partial \theta}\right) + \frac{\partial^{2}}{\partial \phi^{2}}\right]
\end{align*}
Then we find $Y(\theta, \phi)$, the joint eigenfunctions of these operators.
The equations for $Y$ are:
\begin{gather}
    -i\hbar \frac{\partial }{\partial \phi} Y(\theta, \phi) = \hbar m Y(\theta, \phi) \label{eqnAngMomEFunc1} \\
    -\frac{\hbar^2}{\sin^2{\theta}} \left[\sin{\theta} \frac{\partial}{\partial \theta}\left(\sin{\theta} \frac{\partial}{\partial \theta}\right) + \frac{\partial^{2}}{\partial \phi^{2}}\right] Y(\theta, \phi) = \lambda Y(\theta, \phi) \label{eqnAngMomEFunc2}
\end{gather}
Now we can look for separable solutions:
\begin{equation*}
    U(\theta, \phi) = \Theta(\theta) \Phi(\phi)
\end{equation*}
Substituting this into equation~\ref{eqnAngMomEFunc1},
\begin{align*}
    -i \hbar \Phi'(\phi) \Theta(\theta) &= \hbar m \Phi(\phi) \Theta(\theta) \\
    \Phi'(\phi) &= im \Phi(\phi) \\
    \therefore \Phi(\phi) &= e^{i m \phi}
\end{align*}
Now we note that the wavefunction $Y$ must be single-valued on $\R^3$, so we need invariance under $\phi \mapsto \phi + 2\pi$. Therefore, $m$ must be an integer (quantisation).

Now we can solve for $\Theta$. Using the separable solution in equation~\ref{eqnAngMomEFunc2}:
\begin{align}
    -\frac{\hbar^2}{\sin^2{\theta}}&\left[\sin{\theta} \frac{\partial}{\partial \theta}\left(\sin{\theta} \Theta'(\theta)\right)\Phi(\phi) + \Theta(\theta)\Phi''(\phi)\right] = \lambda \Theta(\theta) \Phi(\phi)\nonumber \\
    &= -\frac{\hbar^2}{\sin{\theta}} \frac{\partial }{\partial \theta} \left(\sin(\theta) \Theta'(\theta)\right) \Phi(\phi) +\frac{m^2 \hbar^2}{\sin^2{\theta}} \Phi(\phi) \Theta(\theta)
\end{align}
by substituting in $\Phi(\phi)$. Then divide through by $-\hbar^2 \Phi(\phi)$:
\begin{equation}
    \frac{1}{\sin{\theta}} \frac{\partial }{\partial \theta} \left(\sin(\theta) \Theta'(\theta)\right) -\frac{m^2}{\sin^2{\theta}} \Theta(\theta) = -\frac{\lambda}{\hbar^2} \Theta(\theta)
    \label{eqnAssocLegendre}
\end{equation}
This is similar to the Legendre equation from IB Methods. In fact, equation~\ref{eqnAssocLegendre} is the \textit{associated Legendre Equation}. Its solutions are the \textit{associated Legendre functions}:
\begin{align*}
    \Theta(\theta) &= P_{l, m} (\cos(\theta)) \\
    &= (\sin{\theta})^{|m|} \frac{d^{|m|}}{d(\cos(\theta))^{|m|}} P_l(\cos(\theta))
\end{align*}
where the $P_l(\cos(\theta))$ are the normal Legendre Polynomials.

Note that, if $|m| > l$, we have that $y(\theta) = 0$ because these polynomials have degree $l$ so taking too many derivatives means that the output is zero. This gives us a bound on $m$ in terms of $l$:
\begin{equation*}
    -l \leq m \leq l
\end{equation*}
Then the eigenvalues of $\hat{L}^2$ are given by:
\begin{equation*}
    \lambda = \hbar^2 l(l+1)
\end{equation*}
Putting everything together:
\begin{align*}
    Y_{l,m}(\theta, \phi) &= P_{l, m}(\cos(\theta)) e^{i m \phi} \\
    \hat{L}^2 Y_{l, m}(\theta, \phi) &= \hbar^2 l(l+1)Y_{l, m}(\theta, \phi) \\
    \hat{L}_3 Y_{l, m}(\theta, \phi) &= \hbar mY_{l, m}(\theta, \phi)
\end{align*}
These are called \underline{spherical harmonics}.

The parameters $l, m$ are quantum numbers that characterise the total angular momentum ($l$), and the $z$ component of angular momentum ($m$, the \underline{azimuthal number}).

The first few states are:
\begin{align*}
    Y_{0, 0}(\theta, \phi) &= \frac{1}{\sqrt{4\pi}} \\
    Y_{1, 0}(\theta, \phi) &= \sqrt{\frac{3}{4\pi}} \cos(\theta) \\
    Y_{1, \pm 1}(\theta, \phi) &= \mp \sqrt{\frac{3}{8\pi}} \sin(\theta) e^{\pm \phi} \\
\end{align*}
Then like all eigenfunctions, we have an orthogonality relation:
\begin{equation*}
    \inn{Y_{l, m}}{Y_{l', m'}} = \delta_{ll'} \delta_{mm'}
\end{equation*}
\section{Spherically Symmetric Potential}
In spherical coordinates, the Laplacian is:
\begin{equation*}
    \nabla^2 = \frac{1}{r^2} \frac{\partial}{\partial r}\left(r^2\frac{\partial}{\partial r}\right) + \frac{1}{r^2 \sin{\theta}} \frac{\partial }{\partial \theta}\left(\sin{\theta} \frac{\partial}{\partial \theta}\right)
\end{equation*}
Then looking at the definition of $\hat{L}^2$, we observe:
\begin{equation*}
    -\hbar^2 \nabla^2 = \frac{\hbar^2}{r^2} \frac{\partial}{\partial r}\left(r^2\frac{\partial}{\partial r}\right) + \frac{\hat{L}^2}{r^2}
\end{equation*}
Then the Hamiltonian operator is:
\begin{equation}
    \hat{H} = \frac{\hbar^2}{2mr^2} \frac{\partial}{\partial r}\left(r^2\frac{\partial}{\partial r}\right) + \frac{\hat{L}^2}{2mr^2} + U(\vec{x})
    \label{eqnHamiltonianAngMom}
\end{equation}
\begin{proposition}
    For a spherically symmetric potential $U(\vec{x}) = U(r)$,
    \begin{equation*}
        [\hat{H}, \hat{L}^2] = 0, \qquad [\hat{H}, \hat{L}_i] = 0
    \end{equation*}
    \label{propHamiltonianCommutes}
\end{proposition}
\begin{proof}
    We need to look at the following commutators:
    \begin{align*}
        [\hat{L}_i, \hat{x}_j]&= [\epsilon_{imn} \hat{x}_m \hat{p}_n, \hat{x}_j] \\
        &= \epsilon_{imn} \left(\hat{x}_m [\hat{p}_n, \hat{x}_j] + [\hat{x}_m, \hat{x}_j] \hat{p}_n\right) \\
        &= \epsilon_{imn} \left(\hat{x}_m [\hat{p}_n, \hat{x}_j] + 0\right)
    \end{align*}
    Note also that we found $[\hat{p}_n, \hat{x}_j] = -i\hbar \delta_{nj}$:
    \begin{align*}
        [\hat{L}_i, \hat{x}_j]&= -i \hbar\epsilon_{imj} \hat{x}_m \\
        &= i \hbar \epsilon_{ijk} \hat{x}_k
    \end{align*}
    Similarly we find:
    \begin{align*}
        [\hat{L}_i, \hat{x}_j^2] &= [\hat{L}_i, \hat{x}_j]\hat{x}_j + \hat{x}_j [\hat{L}_i, \hat{x}_j] \\
        &= i \hbar \epsilon_{ijk} \left(\hat{x}_k \hat{x}_j + \hat{x}_j \hat{x}_k\right) \\
        &= 0
    \end{align*}
    because this is a product of an antisymmetric tensor with a symmetric tensor.
    Therefore, since $U(r)$ is a function containing only $\hat{x}_i^2$,
    \begin{equation*}
        [\hat{L}_i, U(r)] = 0
    \end{equation*}
\end{proof}
\begin{remark}
    We can analogously find:
    \begin{align*}
        [\hat{L}_i, \hat{P}_j] &= i \epsilon_{ijk} \hat{p}_k \\
        [\hat{L}_i, \hat{P}_j^2] &= 0 \\
        [\hat{L}_i, \hat{p}^2] &= [\hat{L}_i, \hat{P}_1^2 + \hat{p}_2^2 + \hat{p}_3^2] = 0 \\
    \end{align*}
    and note also that $\hat{p}^2 = -\hbar^2 \nabla^2$. This tells us that:
    \begin{align*}
        [\hat{H}, \hat{L}^2] &= 0 \\
        [\hat{H}, \hat{L}_i] &= 0
    \end{align*}
\end{remark}
In particular we find that $\{\hat{H}, \hat{L}^2, \hat{L}_i\}$ is a set of \underline{mutually commuting}\\\underline{operators}. This means we can find joint eigenstates of these three operators, that form a basis of $\hilb$.

Further, this means that the eigenvalues $E, |\vec{L}|^2, L_z$ can be simultaneously measured, with arbitrary precision.

It is very hard to prove, but is true, that this set of operators is \underline{maximal}. That is, we cannot construct another nontrivial operator that commutes with elements in this set.

We will start by looking for the solutions of the TISE such that $\chi(r)$ satisfies $\hat{L}^2 \chi(r) = 0$.

Substituting this in:
\begin{align}
    \nabla^2 \chi(r) &= \frac{1}{r} \frac{d^{2}}{d x^{2}}(r\chi(r)) \nonumber \\
    \implies -\frac{\hbar^2}{2m} &\left(\frac{d^{2}\chi(r)}{dr^{2}}+ \frac{2}{r} \frac{d\chi(r)}{dr}\right) + U(r) \chi(r) = E\chi(r) \label{eqnTISESpherical}
\end{align}
We must also consider the normalisation conditions for $\chi$. These are slightly more complex in 3 dimensions. We need:
\begin{align*}
    1 &= \int_{\R^3} |\chi(r)|^2 dV \\
    &= 4\pi \int_{0}^\infty |\chi(r)|^2 r^2 dr
\end{align*}
Then for this to hold, require that the eigenfunction $\chi(r)$ goes to $0$ as $r \to \infty$ sufficiently fast. This we have used often for 1-dimensional solutions.

However, we also require that $\chi$ is well-behaved at $0$. We can be fairly relaxed with this condition, because we are multiplying by $r$ and then squaring, so we can permit, at worst, solutions containing terms like $\frac{1}{r}$.

To solve equation~\ref{eqnTISESpherical}, we define $\sigma(r) = r \chi(r)$. Then:
\begin{equation}
    -\frac{\hbar^2}{2m} \frac{d^{2}\sigma(r)}{dr^{2}} + U(r) \sigma(r) = E\sigma(r)
    \label{eqnTISESpherical2}
\end{equation}
This looks very familiar! However, we only define it on $r \geq 0$, and the normalisation condition is instead:
\begin{equation*}
    \int_{0}^{\infty} |\sigma(r)|^2 dr = \frac{1}{4\pi}
\end{equation*}
To understand our boundary conditions, we have to exclude an important case.
\begin{proposition}
    If $\sigma(r)$, as above defined, tends to a non-zero constant $a$ as $r \to 0$, the Hamiltonian operator $\hat{H}$ is not Hermitian.
    \label{propSigmaConstNoHerm}
\end{proposition}
\begin{proof}
    For $\hat{H}$ to be Hermitian we want $\inn{\phi}{\hat{H}\chi} = \inn{\hat{H} \phi}{\chi}$ for all $\phi, \chi \in \hilb$.

    We have bound states, so we can assume that the above $\phi, \chi$ are real (by possibly rotating to an equivalent state).
    \begin{align*}
        \inn{\phi}{\hat{H} \chi} &= \int_{0}^{\infty} \phi(r) \hat{H} \chi(r) r^2 dr \\
        &= -\frac{\hbar^2}{2m} \int_{0}^{\infty} \phi(r) \frac{d}{dr}\left(r^2 \chi \frac{d\chi}{dr}\right) dr \\
        &= -\frac{\hbar^2}{2m} \left[r^2 \phi \frac{d\chi}{dr} - r^2 \chi \frac{d\phi}{dr}\right]_0^\infty \\
        &\qquad - \frac{\hbar^2}{2m} \int_{0}^{\infty} \frac{d}{dr} \left(r^2 \frac{d\phi}{dr}\right)\chi(r) dr \\
        &= -\frac{\hbar^2}{2m} \left[r^2 \phi \frac{d\chi}{dr} - r^2 \chi \frac{d\phi}{dr}\right]_0^\infty + \inn{\hat{H} \phi}{\chi}
    \end{align*}
    %TODO: Check, what happens to potential?
    So we need the boundary term to vanish, and this is only if $r\chi(r) \to 0$ as $r\to 0$, so when $\sigma(0) = 0$.
\end{proof}
Now our boundary conditions are:
\begin{equation*}
    \sigma(0) = 0,\quad \sigma'(0) \text{ is finite}
\end{equation*}

The easiest way to solve this is to extend the potential $U(r)$ to negative values of $r$ by an even extension, $U(-r) = -U(r)$. We then look for $\sigma$ defined on the whole real line. Only odd solutions will suffice, because we have the condition $\sigma(0) = 0$
\begin{example}[3D spherically symmetric finite potential well]
    Consider a potential $U(r)$ defined on $r \geq0$:
    \begin{equation*}
        U(r) =
        \begin{cases}
            0 & r \leq a \\
            U_0 & r > a
        \end{cases}
    \end{equation*}
    Then we use an even extension to extend $U$ to $\R$:
    \begin{equation*}
        U'(r) =
        \begin{cases}
            0 & -a \leq r \leq a \\
            U_0 & \text{otherwise}
        \end{cases}
    \end{equation*}
    Then we solve this exactly as we have done before, but we must also observe the boundary condition $\sigma(0) = 0$.

    We look for bound states, which have energy $E \in [0, U_0]$.

    We define $k$ and $\bar{k}$ as we did in the 1D case:
    \begin{equation*}
        k = \sqrt{\frac{2mE}{\hbar^2}},\qquad \bar{k} = \sqrt{\frac{2m(U_0 - E)}{\hbar^2}}
    \end{equation*}
    Then we find the solution for $\sigma(r)$ is as in the 1D case:
    \begin{equation*}
        \sigma(r) =
        \begin{cases}
            A\sin(kr) + B\cos(kr) & |r| \leq a \\
            Ce^{-\bar{k}r} & r > a \\
            De^{\bar{k}r} & r < -a
        \end{cases}
    \end{equation*}
    Then applying the boundary condition (different from the 1D case), we have $B = 0$. Therefore, we find $\sigma$ is odd and $D = -C$. The standard boundary conditions, applying at $r = \pm a$,
    \begin{align*}
        \sigma(a) \text{ cts}& \implies A \sin(ka) = Ce^{-\bar{k}a} \\
        \sigma'(a) \text{ cts}& \implies kA \cos(ka) = -\bar{k}Ce^{-\bar{k}a}
    \end{align*}
    Then taking the ratio, $-k \cot(ka) = \bar{k}$. From the definition of $k$ and $\bar{k}$, we have that $k^2 + \bar{k}^2 = \frac{2mU_0}{\hbar^2}$. This was considered in Ex2Q3, and we find that the solutions are the intersections of these two curves. See figure~\ref{figFPWellOddSoln}.

    \begin{figure}
        \centering
        \begin{tikzpicture}[scale=1]
            \begin{axis}[
                axis lines=middle,
                xmin=-0.5,xmax=4,ymin=-0.5,ymax=4,
                xlabel={$ak$},
                ylabel={$a\bar{k}$},
                extra x ticks={1.5708},
                extra x tick labels={$\frac\pi2$}
            ]
            \addplot[samples=50, domain=0:3] {-cot(deg(x))} node[right]{$y=-cot(ka)$};
            \addplot[samples at={-0.05, -0.04, ..., 2.1}, domain=-0.1:3] {sqrt(4.1 - x * x)} node[pos=0.2, anchor=south west] {$k^2 + \hat{k}^2 = r^2$};
            \end{axis}
        \end{tikzpicture}
        \caption{Graph of odd solutions of finite potential well}
        \label{figFPWellOddSoln}
    \end{figure} 

    We require a minimum value for $U_0$, $U_0 \geq \frac{\pi^2 \hbar^2}{8ma^2}$. Below this, we do not have a bound state in the 3-dimensional case. Now we have found $\sigma$, we can find the wavefunction $\chi$:
    \begin{equation*}
        \chi(r) = 
        \begin{cases}
            \frac{A\sin(kr)}{r} & r < a \\
            \frac{Ce^{-\bar{k}r}}{r} & r \geq a
        \end{cases}
    \end{equation*}
\end{example}
\section{The Hydrogen Atom}
This section deals with the Quantum description of the hydrogen atom. We found that the various classical descriptions of the atom were insufficient, and so as a culmination of the various ideas of this course, we will provide a better mathematical understanding using the operators $\hat{L}^2, \hat{L}_z, \hat{H}$.

We will model the hydrogen atom as a stationary nucleus of one proton at the origin. The electrostatic force is given by:
\begin{align}
    F_{\text{Coulomb}}(r) &= -\frac{e^2}{4\pi \epsilon_0} \frac{1}{r^2} = -\frac{\partial U_{\text{Coulomb}}}{\partial r} \label{eqnCoulombForce} \\
    U_{\text{Coulomb}}(r) &= -\frac{e^2}{4\pi \epsilon_0} \frac{1}{r} \label{eqnCoulombPot}
\end{align}
We look for bound states with energy less than $0$ (note that the potential is negative, and tends to $0$ as $r \to \infty$). The TISE is:
\begin{equation*}
    -\frac{\hbar^2}{2m_e} \nabla^2 \chi(r, \theta, \phi) - \frac{e^2}{4\pi \epsilon_0} \frac{1}{r} \chi(r, \theta, \phi) = E\chi(r, \theta, \phi)
\end{equation*}
writing this in terms of  the angular momentum:
\begin{equation*}
    E\chi = -\frac{\hbar^2}{2m_e} \frac{1}{r} \frac{\partial^{2}}{\partial r^{2}} \left(r \chi\right) + \frac{\hat{L}^2}{2m_e r^2} \chi - \frac{e^2}{4\pi \epsilon_0 r} \chi
\end{equation*}
Then because $\chi$ must be also an eigenfunction for $\hat{L}^2$ and $\hat{L}_z$, we can separate variables and look for solutions:
\begin{equation*}
    \chi(r, \theta, \phi) = R(r) Y_{l, m}(\theta, \phi)
\end{equation*}
where $Y_{l, m}$ are the spherical harmonics as found in section~\ref{secSphericalHarmonics}. After separating variables, we find the equation for $R$:
\begin{equation}
    -\frac{\hbar^2}{2m_e} \left(\frac{d^{2}R}{dr^{2}} + \frac{2}{r}\frac{dR}{dr}\right) + \frac{\hbar^2 l(l+1)}{2m_er^2}R(r) - \frac{e^2}{4\pi \epsilon_0} \frac{R(r)}{r} = ER(r)
    \label{eqnHydrogenRadial}
\end{equation}
Here we can notice that we have an \textit{effective potential}, as in the classical orbits case, that depends on the angular momentum.
\end{document}