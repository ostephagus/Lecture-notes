\documentclass[../Main.tex]{subfiles}

\begin{document}
In three dimensions, the Time Independent Schr\"odinger Equation is:
\begin{equation}
    -\frac{\hbar^2}{2m} \nabla^2 \chi(\vec{x}) + U(\vec{x}) \chi(\vec{x}) = E \chi(\vec{x})
    \label{eqnTISE3D}
\end{equation}
\section{Reminder of Coordinate Systems}
We will use Cartesian coordinates $(x, y, z)$ and Spherical coordinates $(r, \theta, \phi)$. We denote the angle from the positive $z$-axis by $\theta$.
In these coordinate systems, the Laplacian is:
\begin{align*}
    \text{Cartesian: } \nabla^2&= \frac{\partial^{2}}{\partial x^{2}} + \frac{\partial^{2}}{\partial y^{2}} + \frac{\partial^{2}}{\partial z^{2}} \\
    \text{Spherical: } \nabla^2 &= \frac{1}{r^2} \frac{\partial}{\partial r}\left(r^2\frac{\partial}{\partial r}\right) + \frac{1}{r^2 \sin{\theta}} \frac{\partial }{\partial \theta}\left(\sin{\theta} \frac{\partial}{\partial \theta}\right)\\
    &\qquad+ \frac{1}{r^2 \sin^2{\theta}}\frac{\partial^{2}}{\partial \phi^{2}}
\end{align*}
For the spherical polars expression, we can simplify the term in $r$:
\begin{equation*}
    \nabla f = \frac{1}{r} \frac{\partial^2}{\partial r^2} (rf) + \frac{1}{r^2 \sin^2{\theta}}\left[\sin{\theta} \frac{\partial}{\partial \theta}\left(\sin{\theta} \frac{\partial f}{\partial \theta}\right) + \frac{\partial^{2}f}{\partial \phi^{2}}\right] 
\end{equation*}
Recall the different volume integrals:
\begin{align*}
    \text{Cartesian: }&\int_{\R^3} dV = \int_{x = -\infty}^{\infty} \int_{y = -\infty}^{\infty} \int_{z = -\infty}^{\infty} f(x, y, z) dz~dy~dx \\
    \text{Spherical: }&\int_{\R^3} dV = \int_{r = 0}^{\infty} \int_{\phi = 0}^{2\pi} \int_{\theta = 0}^{\pi} f(r, \phi, \theta) r^2 \sin{\theta}~d\theta~d\phi~dr \\
\end{align*}
We will consider only spherical potentials $U(\vec{x}) = U(r)$.

\section{Three-Dimensional Operators}
\subsection{Angular Momentum}
In classical mechanics, we used the constant angular momentum to simplify 3-dimensional problems with a central potential into 2-dimensional ones. We will also define the angular momentum in Quantum Mechanics:
\begin{definition}{Angular momentum}
    The \underline{angular momentum} in Quantum mechanics is an operator:
    \begin{equation}
        \opvec{L} = \opvec{x} \times \opvec{p} = -i\hbar \opvec{x} \times \nabla
        \label{eqnQAngMomentum}
    \end{equation}
\end{definition}
We can show (an indeed do in Ex3) that $\opvec{L}$ is Hermitian, and so are its components $\hat{L}_i$.
\begin{proposition}
    For distinct $i, j \in \{1, 2, 3\}$, the commutator of different elements of the angular momentum operator is:
    \begin{equation}
        [\hat{L}_i, \hat{L}_j] = i \hbar \epsilon_{ijk} \hat{L}_k
        \label{eqnAngMomCmmt}
    \end{equation}
    \label{propAngMomCmmt}
\end{proposition}
\begin{remark}
    The commutator is non-zero, which tells us that we cannot measure the angular momentum to arbitrary precision, because if we measure one element to arbitrary precision then we have a lower bound on the precision of the other two elements' measurements.
\end{remark}
\begin{proof}
    The proof is simply by multiplying the commutator out. First we need the commutator of $\hat{x}_i$ and $\hat{p}_j$:
    \begin{align*}
        [\hat{x}_i, \hat{p}_j] &= \hat{x}_i \hat{p}_j - \hat{p}_j \hat{x}_i \\
        &= -i\hbar \left(x_i \frac{\partial}{\partial x_j} - \frac{\partial}{\partial x_j} (x_i \cdot)\right) \\
        &= -i\hbar \left(x_i \frac{\partial}{\partial x_j} - x_i\frac{\partial}{\partial x_j} - \left(\frac{\partial x_i}{\partial x_j} \times \cdot \right)\right) \\
        &= -i\hbar \left(- \delta_{ij} \hat{I}\right) \\
        &= i\hbar \delta_{ij} \hat{I}
    \end{align*}
    Now we can use this to find $\hat{L}_i \hat{L}_j$:
    \begin{align*}
        \hat{L}_i \hat{L}_j&= \epsilon_{ikl} \hat{x}_k \hat{p}_l \epsilon_{jmn} \hat{x}_m \hat{p}_n \\
        &= \epsilon_{ikl} \epsilon_{jmn} \hat{x}_k \hat{p}_l \hat{x}_m \hat{p}_n \\
        &= \epsilon_{ikl} \epsilon_{jmn} \hat{x}_k (\hat{x}_m \hat{p}_l - [\hat{p}_l, \hat{x}_m]) \hat{p}_n \\
        &= \epsilon_{ikl} \epsilon_{jmn} \hat{x}_k (\hat{x}_m \hat{p}_l - i\hbar \delta_{lm} \hat{I}) \hat{p}_n \\
        &= \epsilon_{ikl} \epsilon_{jmn} \hat{x}_k \hat{x}_m \hat{p}_l \hat{p}_n - i \hbar\epsilon_{ikl} \epsilon_{jln} \hat{x}_k \hat{p}_n \\
        &= \epsilon_{ikl} \epsilon_{jmn} \hat{x}_k \hat{x}_m \hat{p}_l \hat{p}_n - i \hbar (\delta_{in} \delta_{kj} - \delta_{ij} \delta_{kn}) \hat{x}_k \hat{p}_n \\ 
        &= \epsilon_{ikl} \epsilon_{jmn} \hat{x}_k \hat{x}_m \hat{p}_l \hat{p}_n - i \hbar \hat{x}_j \hat{p}_i
    \end{align*}
    Then similarly, $\hat{L}_j \hat{L}_i = \epsilon_{jkl} \epsilon_{imn} \hat{x}_k \hat{x}_m \hat{p}_l \hat{p}_n - i \hbar \hat{x}_i \hat{p}_j$. Re-label the indices to match with the previous expression:
    \begin{align*}
        \hat{L}_j \hat{L}_i &= \epsilon_{jmn} \epsilon_{ikl} \hat{x}_m \hat{x}_k \hat{p}_n \hat{p}_l - i \hbar \hat{x}_i \hat{p}_j \\
        \hat{L}_j \hat{L}_i &= \epsilon_{jmn} \epsilon_{ikl} \hat{x}_k \hat{x}_m \hat{p}_l \hat{p}_n - i \hbar \hat{x}_i \hat{p}_j  \text{ by commutativity}
    \end{align*}
    which has the same first part as $\hat{L}_i \hat{L}_j$.

    Now find the commutator we want:
    \begin{align}
        [\hat{L}_i, \hat{L}_j] &= -i\hbar(\hat{x}_j \hat{p}_i - \hat{x}_i \hat{p}_j) \nonumber\\
        &= i\hbar(\hat{x}_i \hat{p}_j - \hat{x}_j \hat{p}_i) \label{eqnAngMomCmmt2}
    \end{align}
    and, by expanding out the $\epsilon_{ijk}$ in equation~\ref{eqnAngMomCmmt} in terms of index notation, we find that this is indeed the required result.
\end{proof}
\begin{remark}
    Equation~\ref{eqnAngMomCmmt2} gives an interesting alternate form to the commutator.
\end{remark}
\begin{definition}{Total angular momentum}
    The \underline{total angular momentum} operator is:
    \begin{equation}
        \hat{L}^2 = \hat{L}_1^2 + \hat{L}_2^2 + \hat{L}_3^2 = |\opvec{L}|^2
        \label{eqnTotAngMomentum}
    \end{equation}
\end{definition}
We can prove that the commutator is:
\begin{equation*}
    [\hat{L}^2, \hat{L}_i] = 0
\end{equation*}
This tells us that we can find the magnitude of $\opvec{L}$ and one of its components with arbitrary precision. We generally find the $z$-component of the angular momentum and its magnitude. This is equivalent to simultaneously diagonalising the two operators. To find the joint set of eigenfunctions of $\hat{L}^2$ and $\hat{L}_3$, we write $\opvec{L}$ in spherical coordinates. This results in:
\begin{align*}
    \hat{L}_3 &= -i \hbar \frac{\partial}{\partial \phi} \\
    \hat{L}^2 &= -\frac{\hbar^2}{\sin^2{\theta}} \left[\sin{\theta} \frac{\partial}{\partial \theta}\left(\sin{\theta} \frac{\partial}{\partial \theta}\right) + \frac{\partial^{2}}{\partial \phi^{2}}\right]
\end{align*}
Then we find $Y(\theta, \phi)$, the joint eigenfunctions of these operators.
\end{document}