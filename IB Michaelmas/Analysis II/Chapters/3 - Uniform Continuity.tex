\documentclass[../Main.tex]{subfiles}

\begin{document}
\section{Finding Uniformly Continuous Functions}
\begin{definition}{Uniform continuity}
    Let $E \subseteq \R$, and let $f : E \mapsto \R$. Then $f$ is \underline{uniformly continuous} on $E$ if, for all $\epsilon > 0$, there exists $\delta(\epsilon) > 0$ such that for all $x, y$,
    \begin{equation*}
        |x - y| < \delta \implies |f(x) - f(y)| < \epsilon
    \end{equation*}
\end{definition}
\begin{remark}
    Similar to uniform convergence, we now require that $\delta$ does not depend on the specific point chosen.
\end{remark}
We have uniform continuity implies continuity at every point in $E$ (pointwise continuity). However, the converse does not hold:
\begin{example}[Pointwise continuity does not imply uniform continuity]
    We consider:
    \begin{align*}
        f : (0, 1) &\mapsto \R\\
        x &\mapsto \frac{1}{x}
    \end{align*}
    then as $x \to 0$, we see that any $\delta(\epsilon)$ will not work for smaller $x$ because we can go arbitrarily close to $0$ to get $|x-y| < \delta$ but $|f(x) - f(y)| > \epsilon$.
\end{example}
\begin{example}[Bounded continuous does not imply uniform continuity]
    We can use the classic example:
    \begin{align*}
        f : (0, 1) &\mapsto \R\\
        x &\mapsto (0, 1)
    \end{align*}
    Then we can use the sequence definition of continuity. Taking $x_n = (2n\pi)^{-1}$ and $y_n = ((2n + \frac12)\pi)^{-1}$ gives that $|x_n - y_n| \to 0$ but $|f(x_n) - f(y_n) = 1$.
\end{example}
\begin{theorem}
    Consider a closed, bounded interval $[a, b]$, and let $f : [a, b] \mapsto \R$ be a continuous function. Then $f$ is uniformly continuous.
    \label{thmUCTClosedBdd}
\end{theorem}
\begin{proof}
    Assume, on the contrary, that $f$ is not uniformly continuous on $[a, b]$. Then there exists an $\epsilon > 0$ for which there does not exist a $\delta > 0$ such that $|x - y| < \delta \implies |f(x) - f(y)| < \epsilon$. 

    Therefore take $\delta = \frac{1}{n}$. Take $x_n, y_n \in [a, b]$ where $|x_n - y_n| < \frac{1}{n}$ and $|f(x_n) - f(y_n)| > \epsilon$. Use the Bolzano-Weierstrass theorem to get a subsequence $(x_{n_k})_k$ that converges to a limit $x \in [a, b]$.
    \begin{align*}
        |x-y_{n_k}| &\leq |x-x_{n_k}| + |x_{n_k} - y_{n_k}| \\
        &\leq |x - x_{n_k}| + \frac{1}{n_k} \\
        &\to 0 \text{ as } k \to \infty
    \end{align*}
    Therefore by continuity of $f$, $f(x_{n_k}) \to f(x)$ as $k \to \infty$. However, now we can get $|f(x_{n_k}) - f(y_{n_k})| < \epsilon$.\contradiction
\end{proof}
\nonexaminablesection{Classifying Integrable Functions}
\begin{theorem}
    Let $f : [a, b] \mapsto \R$ be any function. Suppose that there is a collection $\collec$ for intervals $I \subset \R$ such that, given the set:
    \begin{equation*}
        F = [a, b] \backslash \bigcup_{I \in \collec} I
    \end{equation*}
    then $f$ is continuous at every point in $F$. That is, the set of discontinuities is a subset of the union. Then for each $\epsilon > 0$, there exists a $\delta(\epsilon)$ such that if $x \in F, y \in [a, b]$,
    \begin{equation*}
        |x - y| < \delta \implies |f(x) - f(y)| < \epsilon
    \end{equation*}
    \label{thmUCTSubsetClosedBdd}
\end{theorem}
\begin{remarks}
    \item This requires weaker assumptions on $f$ than theorem~\ref{thmUCTClosedBdd}, but reaches roughly the same conclusion.
    \item Note that the hypothesis here is stronger than simply saying that the function $f_F$ restricted to $f$ is continuous on $F$. For example, the irrational indicator function $f(x) = 1$ if $x$ irrational, $0$ otherwise, is continuous on the rational numbers but nowhere continuous on the real numbers. This would not be a valid function.
\end{remarks}
\begin{proof}
    The proof is very similar to that of theorem~\ref{thmUCTClosedBdd}, but the main difference is that we must show that the limit point found, $x$, is in $F$ and not any of the intervals in $\collec$. The argument is that, since each $I \in \collec$ is an open interval, we cannot get there by taking a limit of points in $F$.
\end{proof}
\section{Integrability}
\subsection{A Reminder of Riemann Integration}
Recall that we defined Riemann Integration in terms of upper and lower sums on a partition of an interval.

Let $f : [a, b] \mapsto \R$ be a bounded function, $m \leq f(x) \leq M$ with finite $m, M$.

Let $P = \{a_0 = a, a_1, \cdots, a_{n-1}, a_n = b\}$ be a \underline{partition} of $[a, b]$, where the $a_i$ are strictly increasing. Write $I_j = [a_j, a_{j+1}],~0 \leq j \leq n-1$. Then define the upper and lower sums $U$ and $L$:
\begin{align*}
    U(f, P) &= \sum_{j=0}^{n-1} (a_{j+1} - a_j) \sup_{x\in I_j} f(x) \\
    L(f, P) &= \sum_{j=0}^{n-1} (a_{j+1} - a_j) \inf_{x\in I_j} f(x)
\end{align*}
Then in IA Analysis I we showed that $m(b-a) \leq L(f, P) \leq U(f, P) \leq M(b-a)$. When we refine the partition the upper sum only stays the same or decreases, and vice versa for the lower sum. We then define the upper and lower integals:
\begin{align*}
    I^*(f) &= \inf_{P} U(P, f) \\
    I_*(f) &= \sup_{P} L(P, f)
\end{align*}
and $f$ is \underline{Riemann Integrable} on $[a, b]$ if, on this interval, $I_* = I^*$. In this case, we call the common value the \underline{integral} of $f$ and denote it:
\begin{equation*}
    \int_{a}^{b} f(x) dx = \int_a^b f
\end{equation*}
Recall also the Riemann Criterion for integrability:
\begin{theorem}[Riemann Criterion for Integrablility]
    A bounded function $f : [a, b] \mapsto \R$ is integrable if and only if, given $\epsilon > 0$, there exists a partition $P$ such that $U(P, f) - L(P, f) < \epsilon$.
    \label{thmRiemannCriterion}
\end{theorem}
The proof was given in IA Analysis I.
\subsection{Continuity and Integrability}
We can now use our notion of uniform continuity to prove results about integration.
\begin{theorem}
    Let $f : [a, b] \mapsto [A, B] \subseteq \R$ be integrable. Let $g : [A, B] \mapsto \R$ be continuous. Then the composition $g \circ f : [a, b] \mapsto \R$ is an integrable function.
    \label{thmCtsComposeIBL}
\end{theorem}
\begin{proof}
    First, by uniform continuity of $g$ we invoke theorem~\ref{thmUCTClosedBdd}. For all $\epsilon > 0$ there exists a $\delta > 0$ such that for all $x, y \in [A, B]$, $|x - y| < \delta \implies |g(x) - g(y)| < \epsilon$.
    We also have integrability of $f$, so there exists a partition $P$ of $[a, b]$ such that $U(P, f) - L(P, f) < \epsilon'$. Then:
    \begin{align*}
        U(P, g\circ f) - L(P, g \circ f) = \sum_{j=0}^{n - 1} (a_{j+1} - a_j) \left(\sup_{I_j} g\circ f - \inf_{I_j} g \circ f\right)
    \end{align*}
    We now want these terms to be small. Therefore, we can consider terms for which the difference of supremum and infimum is small, and terms for which there is a large difference, separately. First let the terms with difference less than $\delta$ have indices $j \in J$:
    \begin{equation*}
        J = \subsetselect{j}{\sup_{I_j} f - \inf_{I_j} f < \delta}
    \end{equation*}
    Indeed, if $j \in J$ then the smallest and largest values of $f$ are less than $\delta$, which means the input to $g$ is less than $\delta$ apart and so by uniform continuity of $g$ we must have:
    \begin{equation*}
        \sup_{I_j} g\circ f - \inf_{I_j} g\circ f < \epsilon~\forall j \in J
    \end{equation*}
    and so summing:
    \begin{equation}
        \sum_{j \in J} (a_{i+1} - a_i) \sup_{I_j} g\circ f - \inf_{I_j} g\circ f < \sum_{j \in J} \epsilon
        \label{eqnContinuityBound}
    \end{equation}
    Now consider the $j \notin J$, that is, $\sup_{I_j} f - \inf_{I_j} f \geq \delta$, then:
    \begin{align*}
        \delta (a_{j+1} - a_j) &\leq (a_{j+1} - a_j) \left(\sup_{I_j} f - \inf_{I_j} f\right) \\
        \delta \sum_{j \notin J}(a_{j+1} - a_j) &\leq \sum_{j \notin J} (a_{j+1} - a_j) \left(\sup_{I_j} f - \inf_{I_j} f\right) \\
        &\leq \epsilon' \text{ by integrability of }f \\
        \sum_{j \notin J}(a_{j+1} - a_j) &\leq \frac{\epsilon'}{\delta}
    \end{align*}
    Therefore, we can now set $\epsilon' = \delta \epsilon$, so that this sum above is less than $\epsilon$. Using this:
    \begin{equation*}
        \sum_{j \notin J}(a_{j+1} - a_j) (\sup_{I_j} g \circ f - \inf_{I_j} g \circ f) <\epsilon(2\sup_{[A, B]} |g|)
    \end{equation*}
    Now combining this with equation~\ref{eqnContinuityBound},
    \begin{equation*}
        U(P, g \circ f) - L(P, g \circ f) < \left((b-a) + 2 \sup_{[A, B]} |g|\right) \epsilon
    \end{equation*}
    This is a fixed multiple of $\epsilon$.
\end{proof}
\begin{corollary}
    If $g : [a, b] \mapsto \R$ is continuous the $g$ is integrable
    \label{corCtsIntegrable}
\end{corollary}
\begin{proof}
    We can compose $g$ with the identity function, which is integrable.
\end{proof}
\begin{remark}
    This proof is much easier than that which was given in IA Analysis I.
\end{remark}
\begin{theorem}
    Let $f_n : [a, b] \mapsto \R$ be bounded, integrable functions. Let $(f_n) \to f$ uniformly on $[a,b]$, so $f : [a, b] \mapsto \R$. Then $f$ is bounded and integrable. Further,
    \begin{equation*}
        \int_{a}^{b} f_n(x) dx \to \int_{a}^{b} f(x) dx 
    \end{equation*}
    \label{thmUCIntegrable}
\end{theorem}
\begin{remark}
    We have already seen in theorem~\ref{thmUCIntegrals} that the integrals converge, so it is only necessary to prove boundedness and integrability.
\end{remark}
\begin{proof}
    We have boundedness fairly easily:
    \begin{equation*}
        \sup_{[a, b]} |f| \leq \sup_{[a, b]} |f - f_n| + \sup_{[a, b]} |f_n| < \epsilon + \sup_{[a,b]} |f_n|
    \end{equation*}
    This is bounded by boundedness of $f_n$.

    Now consider any partition $P = \{a = a_0, a_1, \cdots, a_{m-1}, a_m = b\}$. Let $\epsilon > 0$:
    \begin{align*}
        &U(P, f) - L(P, f) = \sum_{j=1}^{m-1} (a_{j+1} - a_j) \left(\sup_{I_j} f - \inf_{I_j} f\right) \\
        &= \sum_{j=1}^{m-1} (a_{j+1} - a_j) \left(\sup_{I_j} (f-f_n+f_n) - \inf_{I_j} (f-f_n+f_n)\right) \\
        &\leq \sum_{j=1}^{m-1} (a_{j+1} - a_j) \left(\sup_{I_j} (f-f_n) + \sup_{I_j} f_n - \inf_{I_j} (f-f_n) - \inf_{I_j}f_n\right) \\
        &\leq \sum_{j=1}^{m-1} (a_{j+1} - a_j) \left(\sup_{I_j} (f-f_n) - \inf_{I_j} (f-f_n) \right) \\
        &+ \sum_{j=1}^{m-1} (a_{j+1} - a_j) \left(\sup_{I_j} f_n - \inf_{I_j}f_n\right) \\
        &\leq U(P, f_n) - L(P, f_n) + 2(b-a)\sup_{[a, b]} |f - f_n|
    \end{align*}
    Now given some $\epsilon$, choose $N$ such that $2(b-a) \sup_{[a, b]} |f - f_N| < \frac12 \epsilon$ by uniform convergence, and also choose a partition $P_N$ such that $U(P, f_N) - L(P, f_N) < \frac12 \epsilon$ by integrability of $f_N$. Therefore now:
    \begin{equation*}
        U(f, P_N) - L(f, P_N) < \epsilon
    \end{equation*}
\end{proof}
\end{document}