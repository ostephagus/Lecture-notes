\documentclass[../Main.tex]{subfiles}

\begin{document}
\section{Communication Classes}
\begin{definition}{Leads to}
    A state $i$ \underline{leads to} another state $j$ if:
    \begin{equation*}
        \P_i(X_n = j \text{ for some } n\geq 0) > 0
    \end{equation*}
    and we write $i \to j$.
\end{definition}
\begin{definition}{Communication}
    A state $i$ \underline{communicates} with another state $j$ if $i$ and $j$ lead to each other. We write $i \comms j$.
\end{definition}
\begin{theorem}
    The following are equivalent:
    \begin{enumerate}
        \item $i \to j$
        \item There exists a path $x_0 = i, x_1 = x_1, \cdots, x_n = j$ with each step having non-zero transition probability:
            \begin{equation*}
                P_{i, x_1} \Pmat{1}{2} \cdots \Pmat{n-2}{n-1} P_{x_{n-1}, j} > 0
            \end{equation*}
        \item $p_{ij}(n) > 0$ for some $n$.
    \end{enumerate}
    \label{thmCommEquivalence}
\end{theorem}
\begin{proof}
    \begin{subproof}{Statement 1 is equivalent to statement 3}
        \begin{equation*}
            \left\{X_n = j \text{ for some } n \geq 0\right\} = \bigcup_{n \geq 0} \left\{X_n = j\right\}
        \end{equation*}
        so if $\P_i(X_n = j \text{ for some } n \geq 0) > 0$, then there exists some $n \geq 0$ such that $\P_i(X_n = j) > 0$. Clearly if $\P_i(X_n = j) > 0$ then $i \to j$.
    \end{subproof}
    \begin{subproof}{Statement 2 is equivalent to statement 3}
        \begin{align*}
            P_{ij}(n) &= \P_i(X_n = j) \\
            &= \sum_{x_1, \cdots, x_{n-1}} \P_i(X_1 = x_1, \cdots, X_{n-1} = x_{n-1}, X_n = j) \\
            &= \sum_{x_1, \cdots, x_{n-1}} P_{i, x_1} \Pmat{1}{2} \cdots P_{x_{n-1}, j}
        \end{align*}
        so if $p_{ij}(n) > 0$, there must be at least one element of the sum that is non-zero. Also, if an element of the sum is non-zero then $p_{ij}(n) > 0$.
    \end{subproof}
\end{proof}
\begin{corollary}
    Communication is an equivalence relation on the state space $I$.
    \label{corCommEquivalence}
\end{corollary}
\begin{proof}
    We clearly have that $x \comms x$, and $x \comms y \equiv y \comms x$. By statement $2$ in theorem~\ref{thmCommEquivalence}, we can put the paths together and we have that $x \comms y$ and $y \comms z \implies x \comms z$.
\end{proof}
\begin{definition}{Communication classes}
    An equivalence class defined by communication is a \underline{communication class}.
\end{definition}
\begin{definition}{Closed class}
    A communication class $C \subseteq I$ is \underline{closed} if $x \to y$ for some $x \in C, y \in I$ implies that $y \in C$.
\end{definition}
\begin{definition}{Absorbing state}
    A state $x$ is an \underline{absorbing state} if $\{x\}$ is a closed communication class.
\end{definition}
\begin{definition}{Irreducibility}
    A transition matrix $P$ is \underline{irreducible} if $I$ is a communicating class.
\end{definition}
\section{Hitting Time}
\begin{definition}{First hitting time}
    Let $A \subseteq I$. The \underline{first hitting time} for $A$ is:
    \begin{equation*}
        T_A = \inf \subsetselect{n \geq 0}{X_n \in A}
    \end{equation*}
    Note that this is not necessarily finite, since this is infinite if the Markov chain never gets to $A$, because we define $\inf \emptyset = \infty$.
\end{definition}
\begin{definition}{Hitting probability}
    The \underline{hitting probability} is the function:
    \begin{equation*}
        h^A : I \mapsto [0, 1]
    \end{equation*}
    defined by $h_i^A = \P_i(T_A < \infty), i \in I$.
\end{definition}
\begin{definition}{Mean hitting time}
    The \underline{mean hitting time} is a function $k^A : I \mapsto [0, \infty]$ defined by:
    \begin{equation*}
        k_i^A = \E_i[T_A]
    \end{equation*}
    If $T_A$ is infinite then the mean hitting time must be infinite. If, however, $\P(T_a = \infty) = 0$ we get:
    \begin{equation*}
        k_i^A = \sum_{n = 0}^\infty \P_i (T_A = n)
    \end{equation*}
\end{definition}
\end{document}