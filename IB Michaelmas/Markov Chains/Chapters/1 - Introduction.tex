\documentclass[../Main.tex]{subfiles}

\begin{document}
\section{Definitions}
Throughout, all our random variables and random processes will be assumed to be defined on an appropriate underlying probability space.

\begin{definition}{Discrete-time Markov Chain}
    A \underline{discrete-time Markov Chain} is a sequence $\chain{X} = (X_n)_{n \geq 0}$ of random variables taking values in the same discrete, countably infinite \underline{state space} I, such that:
    \begin{equation*}
        \P(X_{n+1} = x_{n + 1} | X_n = x_n, \cdots, X_0 = 0) = \P(X_{n+1} = x_{n + 1} | X_n = x_n)
    \end{equation*}
\end{definition}
\begin{definition}{Time-homogeneous Markov Chain}
    A Markov Chain is \underline{time-homogeneous} if $\P(X_{n+1} = y | X_n = x)$ always depends on $n$.
\end{definition}
From now on, we will assume any Markov Chain is time-homogeneous.

In this case, we introduce notation. Let $P(x, y) = P_{xy} = \P(X_{n + 1} = y | X_n = x)$. The matrix $P$ that is formed is a \underline{stochastic matrix}, where $P_{xy} \geq 0$ and $\sum_{y \in I} P_{xy} = 1$.

\begin{example}
    A simple binary transition Markov Chain considers 2 states, with probability $\alpha$ to transition from $1$ to $2$, and probability $\beta$ to go back. The transition matrix $P$ is:
    \begin{equation*}
        P = 
        \begin{pmatrix}
            1 - \alpha & \alpha \\
            \beta & 1 - \beta
        \end{pmatrix}
    \end{equation*}
\end{example}
\begin{definition}{Initial distribution}
    A vector $\vec{\lambda}$ is an \underline{initial distribution} for a Markov chain $\chain{X}$ if $\vec{\lambda}$ is a probability mass function (so components sum to 1) and $P(X_0 = x) = \lambda_x, x \in I$.
\end{definition}
\section{Identifying a Markov Chain}
\begin{theorem}
    A set of random variables $\chain{X}$ is a Markov Chain with initial distribution $\vec{\lambda}$ and transition matrix $P$ if and only if:
    \begin{equation*}
        \P(X_0 = x_0, X_1 = x_1, \cdots, X_n = x_n) = \lambda_{x_0} P_{x_0 x_1} \cdots P(x_{n-1} x_n)
    \end{equation*}
    for all $n \geq 1$ and $x_i \in I$.
    \label{thmMarkovPathEquivalence}
\end{theorem}
\begin{proof}
    \begin{proofdirection}{$\Rightarrow$}{Suppose that $\chain{X}$ is Markov$(\lambda, P)$}
        \begin{align*}
            &\P(X_0 = x_0, \cdots, X_n = x_n) = \\
            &= \P(X_0 = x_0, \cdots, X_{n-1} = x_{n-1}) \P(X_n = x_n | X_{n-1} = x_{n-1}) \\
            &= \P(X_0 = x_0, \cdots, X_{n-1} = x_{n-1}) P_{x_{n-1}, x_n} \\
            &= \cdots = P(X_0 = x_0) P_{x_0, x_1} \cdots P_{x_{n-1}, x_n} \\
        \end{align*}
        and replacing the first term with $\lambda_{x_0}$ gives the required result.
    \end{proofdirection}
    \begin{proofdirection}{$\Leftarrow$}{Let the probability of a path be $\lambda_{x_0} P_{{x_0}, {x_1}}, \cdots$}
        \begin{align*}
            &\P(X_n = x_n | X_{n - 1} = x_{n-1}, \cdots, X_0 = x_0) = \\
            &= \frac{\P(X_0 = x_0, \cdots, X_n = x_n)}{\P(X_0 = x_0, \cdots, X_{n-1} = x_{n-1})} \\
            &= \frac{\lambda_{x_0} P_{x_0, x_1} \cdots P_{x_{n-1}, x_n}}{\lambda_{x_0} P_{x_0, x_1} \cdots P_{x_{n-2}, x_{n-1}}} \\
            &= P_{x_{n-1}, x_n}
        \end{align*}        
        and this defines a Markov Chain.
    \end{proofdirection}
\end{proof}
\begin{definition}{Unit mass function}
    For $i \in I$, the \underline{unit mass function} $\delta_i$ is the probability mass function with $1$ at position $i$.
\end{definition}
\begin{definition}{Independence of processes}
    Two processes $(X_n)_{n \geq 0}$ and $(Y_n)_{n \geq 0}$ are \underline{independent} if and only if, for any two finite sets of indices $\{t_1, t_2, \cdots, t_k\}$ and $\{s_1, \cdots, s_m\}$:
    \begin{align*}
        &\P(X_{t_1} = x_{t_1}, \cdots, X_{t_k} = x_{t_k}, Y_{s_1} = y_{s_1}, \cdots, Y_{s_m} = y_{s_m}) = \\
        &\P(X_{t_1} = x_{t_1}, \cdots, X_{t_k} = x_{t_k}) \times \P(Y_{s_1} = y_{s_1}, \cdots, Y_{s_m} = y_{s_m})
    \end{align*}
\end{definition}
Note that for a Markov chain, $X_{n+1}$ is \textit{conditionally} independent of $X_{n-1}$ given $X_n$, but $X_{n+1}$ is not fully independent of $X_{n-1}$.
\begin{theorem}[Markov property]
    If $\chain{X}$ is a Markov chain, $\chain{X} = \text{Markov}(\vec{\lambda}, P)$, then for any $m \geq 1$ and $i \in I$, conditional on $\{X_m = i\}$,
    \begin{enumerate}
        \item The process $(X_{m + n})_{n \geq 0}$ is a Markov chain with initial distribution $\delta_i$ and transition matrix $P$.
        \item The process is independent of $(X_0, \cdots, X_m)$.
    \end{enumerate}
    \label{thmMarkovProperty}
\end{theorem}
\begin{proof}
    Clearly, $\ind{X_m = j | X_m = i} = \delta_{ij}$, so:
    \begin{align*}
        &\P(X_{n + m} = x_{m + n} | X_m = x_m, \cdots, X_{m + n - 1} = x_{n + m - 1}) = \\
        &= \P(X_{m + n} = x_{m + n} | X_{m + n - 1} = x_{m + n - 1}) \\
        &= \Pmat{m + n - 1}{m + n}
    \end{align*}
    so $\chain{X}$ is a Markov chain.

    \begin{align*}
        &\frac{\P(X_0 = x_0, \cdots, X_m = x_m, X_{m + 1} = x_{m + 1}, \cdots, X_{m + n} = x_{m + n})}{\P(X_m = i)} = \\
        &= \frac{\lambda_{x_0} \Pmat{0}{1} \cdots \Pmat{m-1}{m} \P(X_{m + 1} = x_{m + 1} \cdots X_{m + n} = x_{m + n} | X_m = x_m)}{\P(X_m = i)} \\
        &= P(X_0 = x_0, \cdots, X_m = x_m | X_m = i) \\
        &\times \ind{X_{m + 1} = x_{m + 1}, \cdots, X_{m + n} = x_{m + n} | X_m = x_m}
    \end{align*}
\end{proof}
\section{More on the Transition Matrix}
Suppose $\chain{X}$ is a Markov chain with initial distribution $\vec{\lambda}$ and transition matrix $P$.

We want to consider what $\P(X_n = n)$ is for large $n$.

\begin{align*}
    P(X_n = n) &= \sum_{x_0, \cdots, x_{n - 1}} \P(X_0 = x_0, \cdots, X_n = x_n) \\
    &= \sum_{x_0, \cdots, x_{n-1}} \lambda_{x_0} \Pmat{0}{1} \cdots \Pmat{n-1}{n} \\
    &= \left(\vec{\lambda} P^n\right)_{x_n}
\end{align*}
So to understand the long-term distribution of the chain $\chain{X}$ it suffices to understand the behaviour of $P^n$ for stochastic matrices $P$.

\begin{theorem}
    Suppose $\chain{X}$ is a Markov chain with initial distribution $\vec{\lambda}$ and transition matrix $P$. Then:
    \begin{enumerate}
        \item $\P(X_n = x) = \left(\vec{\lambda} P^n\right)_n$ for all $x \in J$.
        \item $\P(X_{m + n} = y | X_m = n) = \left(P^n\right)_{x, y}$ for all $x, y \in J$.
    \end{enumerate}
    \label{thmMarkovMatPower}
\end{theorem}
\begin{proof}
    The first statement has already been shown.

    The second statement can be found by using theorem~\ref{thmMarkovProperty}, and then it becomes the first statement.
\end{proof}
\begin{example}
    Consider the simplest Markov chain with 2 states.
    The transition matrix is:
    \begin{equation*}
        P = 
        \begin{pmatrix}
            1 - \alpha & \alpha \\
            \beta & 1 - \beta
        \end{pmatrix}
    \end{equation*}
    We want the probability that $X_n = 0$ given that $X_0 = 0$.

    Since $P^{n+1} = P^n P$, 
    \begin{align*}
        P_{0, 0}^{n+1} &= P_{0, 0}^n (1 - \alpha) + P_{0, 1}^n (\beta) \\
        &= P_{0, 0}^n (1 - \alpha) + (1 - P_{0, 0}^n) (\beta) \\
        &= (1 - \alpha - \beta) P_{0, 0}^n + \beta
    \end{align*}
    and so solving this:
    \begin{equation*}
        P_{0, 0}^n =
        \begin{cases}
            \frac{\beta}{\alpha + \beta} + \frac{\alpha}{\alpha + \beta} (1 - \alpha - \beta)^n & \alpha + \beta \neq 0 \\
            1 & \alpha + \beta = 0
        \end{cases}
    \end{equation*}
\end{example}
Introduce notation:
$\P_x(A) = \P(A | X_0 = x)$ and $\E_x(A) = \E[A | X_0 = x]$ for any event $A$. For transition matrix $P$, let $(P^n)_{ij}$ = $p_{ij}(n)$.

We want to compute $p_{ij}(n)$ for a Markov Chain. Suppose $P$ is a $k \times k$ matrix.
\begin{itemize}
    \item If $P$ has $k$ distinct real eigenvalues, then it is diagonalisable, $P = UDU^{-1}$ for diagonal $D$. Then we can easily compute $P^n = UD^n U^{-1}$ where $D^n$ is very simple to compute. If the eigenvalues of $P$ are $\lambda_i$, then we find that $p_{11}(n)$ is a polynomial of $k$ terms. Therefore, we can compute the coefficients using values of $P^n$ for small $k$, and get the resulting distribution for $n$ large.
    \item If $P$ has some complex eigenvalues, these will come in complex conjugate pairs. In this case, given that $P^n$ is real for each $n$, we introduce $r^n\cos{n\theta}$ and $r^n\sin{n\theta}$ for each $\lambda_i = re^{i\theta}$.
    \item If all eigenvalues are distinct except the last two, $\lambda_{k - 1} = \lambda_k$, we use Jordan Normal Form to get that the same expansion holds for $p_{11}(n)$ except that we must include a term of the form $(a + bn) \lambda_{k-1}^n$.
\end{itemize}
\begin{example}
    Consider a Markov Chain with 3 states:
    \begin{equation*}
        P=
        \begin{pmatrix}
            0 & 1 & 0 \\
            0 & \frac12 & \frac12 \\
            \frac12 & 0 & \frac12
        \end{pmatrix}
    \end{equation*}
    Our intuition tells us that the final distribution should be:
    \begin{equation*}
        \begin{pmatrix}\frac15 \\ \frac25 \\ \frac25\end{pmatrix}
    \end{equation*}
    because half the time we stay at 2 or 3, but always leave 1. Let's check.

    The eigenvalues of $P$ are $1, \frac{i}{2}, -\frac{i}{2}$, so:
    \begin{equation*}
        p_{11}(n) = a + b\left(\frac12\right)^n \cos\left(\frac{n\pi}{2}\right) + c\left(\frac{1}{2}\right)^n \sin\left(\frac{n\pi}{2}\right)
    \end{equation*}
    But we know that $p_{11}(0) = 1$, $p_{11}(1) = 0$ and $p_{11}(2) = 0$. Therefore:
    \begin{equation*}
        p_{11}(n) = \frac15 + \left(\frac12\right)^n \left[\frac45 \cos\left(\frac{n\pi}{2}\right) - \frac25 \sin\left(\frac{n\pi}{2}\right)\right]
    \end{equation*}
    which goes to $\frac15$ exponentially fast.
\end{example}
\end{document}