\documentclass[../Main.tex]{subfiles}

\begin{document}
\begin{definition}{Span}
    Let $V$ be a vector space over $\F$. The \underline{span} of a subset $S \subseteq V$ is the set:
    \begin{equation*}
        \spn{S} = \left\{\left.\sum_{s \in S} \lambda_s s \right| \lambda_s \in \F, \text{ finitely many } \lambda _s \text{ are non-zero} \right\}
    \end{equation*}
\end{definition}
\begin{remark}
    If $S$ is infinite, we have not defined infinite summation so cannot add infinitely many elements of $S$. We call a sum of not-infinitely-many scaled elements of $S$ a \underline{linear combination}.
\end{remark}
\begin{definition}{Spanning set}
    A subset $S$ of a vector space $V$ is a \underline{spanning set} for $V$ if $\spn{S} = V$. We say $S$ \underline{spans} $V$.
\end{definition}
$V$ is \underline{finite-dimensional} if it has a finite spanning set.
\end{document}