\documentclass[../Main.tex]{subfiles}

\begin{document}
\section{The Dirac Delta Function}
The Dirac Delta function was introduced in IA Differential Equations with the following properties:
\begin{align*}
    \delta(x) &= 0 \text{ if }x \neq 0 \\
    \forall \epsilon > 0, &\int_{-\epsilon}^{\epsilon} \delta(x) dx = 1
\end{align*}
No such function exists.

However, we can attempt to understand this by considering the Dirac Delta Function as the limit of a sequence of functions:
\begin{equation*}
    \delta_n(x) =
    \begin{cases}
        \frac{n}{\alpha}\exp\left[-\frac{1}{1-n^2x^2}\right] & |x| < \frac1n \\
        0 & |x| \geq \frac1n
    \end{cases}
\end{equation*}
where:
\begin{equation*}
    \int_{-1}^{1} \exp\left[-\frac{1}{1-y^2}\right] dy 
\end{equation*}
We note that:
\begin{enumerate}
    \item The function $\delta_n$ is smooth for each $n$;
    \item $\delta_n(x)$ is zero on $|x| \geq \frac1n$;
    \item For all $\epsilon > 0,~\exists N > 0$ such that:
        \begin{equation*}
            n \geq N \implies \int_{-\epsilon}^{\epsilon} \delta_x dx = 1
        \end{equation*}
\end{enumerate}
For the third remark, we simply take $N > \frac1\epsilon$. This results in:
\begin{align*}
    \int_{-\epsilon}^{\epsilon} \delta_n(x) dx &= \int_{-\frac1n}^{\frac1n} \delta_n(x) dx \\
    &= \frac1\alpha \int_{-1}^{1} \exp\left[-\frac{1}{1-y^2}\right] dy \text{ by taking }y = nx \\
    &= 1
\end{align*}
Now consider figure~\ref{figDiracDeltaGraph}. We see that $\lim_{n \to \infty} \delta_n(x) = 0$ if $x \neq 0$ and, for $\epsilon > 0$, we have the required integral property.

Therefore, this \underline{almost} gives us the limit:
\begin{equation*}
    \delta(x) = \lim_{n \to \infty} \delta_n(x)
\end{equation*}
However, the pointwise limit does not exist everywhere:
\begin{equation*}
    \lim_{n \to \infty} \delta_n(0) = \lim_{n \to \infty} \frac{n}{e\alpha}
\end{equation*}
which is infinite.

For this course, we will say that the limit does exist, but in a ``weaker sense'' (see III Distribution Theory \& Applications).

The Dirac Delta function rarely appears in isolation. For example, the key ``picking out'' property:
\begin{equation*}
    \int_{-\infty}^{\infty} f(x)\delta(x - a) dx = f(a)
\end{equation*}
It also has important applications in modelling an impulse:
\begin{equation*}
    \ddot{y} + \omega^2 y = \delta(t)
\end{equation*}
To deal with this, we solve $\ddot{y}_n + \omega^2 y_n = \delta_n(t)$. Then we define $y(t)$ as the pointwise limit $\lim_{n \to \infty} t_n(t)$.

Using this limiting procedure, we can also consider $\delta'(x)$:
\begin{align*}
    \int_{-\infty}^{\infty} \delta'(x)f(x) dx &= \lim_{n \to \infty} \int_{-\infty}^{\infty} \delta_n'(x) f(x) dx \\
    &= -\lim_{n \to \infty} \int_{-\infty}^{\infty} \delta_n(x) f'(x) dx \\
    &= -\int_{-\infty}^{\infty} \delta(x) f'(x) dx = -f'(0)    
\end{align*}
It gets irritating writing all these limits, so we usually deal with the Dirac Delta function itself (where we understand that, behind the scenes, a limiting process is occurring).
\subsection{Periodic delta functions}
Let $\delta^L(x)$ denote the $L$-periodic delta function outside the interval $\left[\left.-\frac{L}{2}, \frac{L}{2}\right)\right.$
We can even take the Fourier series:
\begin{equation*}
    \frac{1}{L} \int_{-\frac{L}{2}}^{\frac{L}{2}} \delta(x) e^{-\frac{2\pi}{L}inx} dx
\end{equation*}
This gives:
\begin{equation*}
    \delta^L(x) \sim \frac{1}{L}\sum_{n \in \Z} e^{\frac{2\pi}{L}inx}
\end{equation*}
Plausibility test:
\begin{align*}
    f(0) &= \int_{-\frac{L}{2}}^{\frac{L}{2}} \delta^L(x) f(x) dx \\
    &= \sum_{n \in Z} \left[\frac{1}{L} \int_{-\frac{L}{2}}^{\frac{L}{2}} e^{\frac{2\pi}{L}inx}f(x) dx \right] \\
    &= \sum_{n \in \Z} \hat{f}_{-n} = \sum_{n \in \Z} \hat{f}_{n} \\
    &= f(0)
\end{align*}
This, although we have dealt with non-convergent sums to find it, is our expected result. Note that, since $\delta_n(x)$ is smooth its Fourier coefficients must decay rapidly, so this is slightly less bad than it looks.
\subsection{Eigenfunction Expansion of \texorpdfstring{$\delta(x)$}{the Delta Function}}
Let $L$ be a Sturm-Liouville operator on the vector space $V \subseteq C^2[a, b]$, where we require that $y \in V$ satisfy real, homogeneous boundary conditions at each non-singular endpoint.

Let $\{Y_k\}_{k=1}^\infty$ be a set of normalised eigenfunctions. Fix $\xi \in [a, b]$, and consider functions $\delta_n(x - \xi)$. For $n$ sufficiently large, $\delta_n(x - \xi) \in V$. By completeness,
\begin{align*}
    \delta(x - \xi) &= \sum_{k=1}^{\infty} \inn{\delta_n(t-\xi)}{Y_k(t)}_W Y_k(x) \\
    &= \sum_{k=1}^{\infty} \left[\int_{a}^{b} \delta_n(t - \xi) Y_k(t) w(t) dt \right] Y_k(x)
\end{align*}
Formally, letting $n \to \infty$,
\begin{equation*}
    \delta(x - \xi) = \sum_{k=1}^{\infty} w(\xi) Y_k(\xi) Y_k(x)
\end{equation*}
We can consider another plausibility test:
\begin{align*}
    f(x) &= \int_{a}^{b} \delta(x - \xi)f(\xi) d\xi \\
    &= \sum_{k=1}^{\infty} \int_{a}^{b} f(\xi) Y_k(\xi) w(\xi) d\xi Y_k(x) \\
    &= \sum_{k=1}^{\infty} \hat{f}_k Y_k(x) \sim f(x)
\end{align*}
\end{document}