\documentclass[../Main.tex]{subfiles}

\begin{document}
\section{Definitions and Simple Properties}
We can understand the Fourier Transform as the limit $L \to \infty$ of the Fourier Series that we discussed in Chapter 1. Indeed, in the book ``Methods of Modern Mathematical Physics: Functional Analysis Volume I'', Michael Reed and Barry Simon, 1980 (Academic Press), all the following properties are derived this way.

\begin{definition}{Fourier Transform}
    For $f : \R \mapsto \C$, define its \underline{Fourier transform} (FT) by:
    \begin{equation}
        \hat{f}(\lambda) = \int_{-\infty}^{\infty} e^{-i\lambda x} f(x) dx
        \label{eqnFT}
    \end{equation}
    for $\lambda \in \R$.
\end{definition}
We will use $\FT$ to denote the linear map that sends $f$ to $\hat{f}$, $\FT(f) = \hat{f}$.
By the Riemann-Lebesgue lemma,
\begin{equation*}
    \int_{-\infty}^{\infty} e^{-i\lambda x} g(x) dx \to 0 \text{ as } |\lambda| \to \infty
\end{equation*}
we have that $\hat{f}(\lambda) \to 0$ as $|\lambda| \to \infty$.
\begin{propositions}{
        Let $f : \R \mapsto \C$ be any function. Denote the input to $f$ by $x$, and the input to $\FT(f)$ by $\lambda$.
        \label{propsFTProps}
    }
    \item Differentiation is given by:
        \begin{equation*}
            \FT\left[\left(\frac{d^{k}}{dx^{k}}\right)f(x)\right] = (i\lambda)^k \hat{f}(\lambda)
        \end{equation*}
        \label{propFTDiff}
    \item Multiplication is given by:
        \begin{equation*}
            \FT\left[x^k f(x)\right] = \left(i \frac{d}{d\lambda}\right)^k \hat{f}(\lambda)
        \end{equation*}
        \label{propFTMult}
    \item Translation is given by:
        \begin{equation*}
            \FT[f(x-k)] = e^{-i\lambda k} \hat{f}(\lambda)
        \end{equation*}
        \label{propFTTrans}
    \item Phase shift is given by:
        \begin{equation*}
            \FT\left[e^{-ikx}f(x)\right] = \hat{f}(x - k)
        \end{equation*}
        \label{propFTPhase}
\end{propositions}
\begin{proof}
    \begin{enumerate}
        \item \begin{align*}
            \int_{-\infty}^{\infty} f^{(k)}(x) e^{-i \lambda x} dx &= \begin{split}
                \int_{-\infty}^{\infty} f(x) &\left(-\frac{d}{dx}\right)^k e^{-i \lambda x} dx\\
                &\text{ by integration by parts}
            \end{split}\\
            &= (i\lambda)^k \hat{f}(\lambda)
        \end{align*}
        \item \begin{align*}
            &\left(i \frac{d}{d\lambda}\right)^k \int_{-\infty}^{\infty} e^{-i\lambda x}f(x) dx \\
            &= \int_{-\infty}^{\infty} e^{-i \lambda x} x^k f(x) dx \\
            &= \FT[x^k f(x)]            
        \end{align*}
        \item \begin{align*}
            \FT[f(x - a)] &= \int_{-\infty}^{\infty} e^{-i \lambda x}f(x - k) dx \\
            \intertext{Change variables $x \mapsto x + k$:}
            &= \int_{-\infty}^{\infty} e^{-i \lambda (x + k)} f(x) dx \\
            &= e^{-i \lambda k} \int_{-\infty}^{\infty} e^{-i \lambda x} f(x) dx \\
            &= e^{-i \lambda k} \hat{f}(\lambda)            
        \end{align*}
        \item The phase shift example is very similar to the previous, but the logic works in reverse.
    \end{enumerate}
\end{proof}
Proposition~\ref{propFTDiff} is extremely important in solving differential equations. For example, consider:
\begin{equation*}
    P\left(\frac{d}{dx}\right)y = F(x), P\text{ is a polynomial}
\end{equation*}
Then taking Fourier transforms of both sides:
\begin{align*}
    P(i\lambda) \hat{y}(\lambda) &= \hat{F}(\lambda) \\
    \hat{y}(\lambda) &= \frac{\hat{F}(\lambda)}{P(i\lambda)}
\end{align*}
This is fairly easy to do once the Fourier Transform has been computed, since we just evaluate a polynomial. However, to be able to use this technique we need to be able to reconstruct $y$ from $\hat{y}$.
\begin{theorem}[Fourier Inversion Theorem]
    Given a function $\hat{f}$, the function $f$ that satisfies $\FT(f) = \hat{f}$ is given by:
    \begin{equation}
        \frac{1}{2\pi} \int_{-\infty}^{\infty} e^{i \lambda x} \hat{f}(\lambda) d\lambda
        \label{eqnInverseFT}
    \end{equation}
    \label{thmInverseFT}
\end{theorem}
\begin{proof}
    \begin{equation*}
        \lim_{n \to \infty} \frac{1}{2\pi} \int_{-n}^{n} e^{i \lambda x} \hat{f}(\lambda) d\lambda = \lim_{n \to\infty}\frac{1}{2\pi} \int_{-n}^{n} \left[\int_{-\infty}^{\infty} e^{-i \lambda y} f(y) dy \right] d\lambda
    \end{equation*}
    Exchange the order of integration:
    \begin{align*}
        &= \lim_{n \to \infty} \int_{-\infty}^{\infty} f(y) \left[\frac{1}{2\pi} \int_{-n}^{n} e^{i \lambda(x - y)} d\lambda \right] dy \\
        &= \lim_{n \to \infty} \int_{-\infty}^{\infty} f(y) \frac{\sin\left[n(x-y)\right]}{\pi(x - y)} dy \\
        \intertext{Change variables $y \mapsto x + y$:}
        &= \lim_{n \to \infty} \int_{-\infty}^{\infty} f(x + y) \frac{\sin(ny)}{\pi y} dy
    \end{align*}
    Now recall a result on the first example sheet if IA Differential Equations:
    \begin{equation*}
        \forall n > 0, \int_{-\infty}^{\infty} \frac{\sin(ny)}{\pi y} dy = 1
    \end{equation*}
    Therefore,
    \begin{align*}
        \frac{1}{2\pi} &\lim_{n \to \infty} \int_{-n}^{n} \hat{f}(\lambda) d\lambda  - f(x) \\
        &= \lim_{n \to \infty} \int_{-n}^{n} \sin(ny) \left[\frac{f(x + y) - f(x)}{\pi y}\right] dy \\
        %TODO: check lecture notes.
        &\text{this will be finished later.}
    \end{align*}
\end{proof}
Note that from this we get (removing the limits):
\begin{equation*}
    f(x) = \int_{-\infty}^{\infty} f(y) \left[\frac{1}{2\pi} \int_{-\infty}^{\infty} e^{i \lambda (x - y)} d\lambda \right] dy
\end{equation*}
And we can conclude:
\begin{equation*}
    \delta(x - y) = \frac{1}{2\pi} \int_{-\infty}^{\infty} e^{i\lambda(x - y)} d\lambda
\end{equation*}
\begin{proposition}[Parseval's theorem for Fourier transforms]
    For $f, g : \R \mapsto \C$,
    \begin{align*}
        \int_{-\infty}^{\infty} f(x)\overline{g(x)} dx &= \frac{1}{2\pi} \int_{-\infty}^{\infty} \hat{f}(\lambda) \overline{\hat{g}(\lambda)} d\lambda \\
        \intertext{hence}
        \int_{-\infty}^{\infty} |f(x)|^2 dx &= \frac{1}{2\pi} \int_{-\infty}^{\infty} |f(\lambda)|^2 d\lambda 
    \end{align*}
    \label{thmParsevalFT}
\end{proposition}
\begin{proof}
    \begin{align*}
        \frac{1}{2\pi}& \int_{-\infty}^{\infty} \left[\int_{-\infty}^{\infty} e^{-i \lambda y}f(y) dy \right] \left[\int_{-\infty}^{\infty} e^{-i \lambda x}g(x) dx \right] d\lambda \\
        &= \int_{\R^2} f(y) \overline{g(x)} \left[\frac{1}{2\pi} \int_{-\infty}^{\infty} e^{i \lambda(x - y)} d\lambda \right] dx~dy \\
        &= \int_{\R^2} f(y) \overline{g(x)} \delta(x - y) dx~dy \\
        &= \int_{-\infty}^{\infty} f(x) \overline{g(x)} dx 
    \end{align*}
\end{proof}
\begin{definition}{Convolution}
    For $f, g : \R \mapsto \C$, define the \underline{convolution} $f \star g : \R \mapsto \C$ by:
    \begin{equation*}
        (f \star g)(x) = \int_{-\infty}^{\infty} f(x - y)g(y) dy
    \end{equation*}
\end{definition}
\begin{remark}
    Note that convolutions are commutative ($f \star g = g \star f$)
\end{remark}
\begin{proposition}
    \begin{equation*}
        \FT\left[(f \star g)(x)\right] = \hat{f}(\lambda) \hat{g}(\lambda)
    \end{equation*}
    \label{propFTConvol}
\end{proposition}
\begin{proof}
    \begin{align*}
        \text{LHS}&= \int_{-\infty}^{\infty} e^{-i \lambda x} \left[\int_{-\infty}^{\infty} f(x - y) g(y) dy \right] dx  \\
        &= \int_{\R^2} e^{-i \lambda x} f(x - y) g(y) dx~dy
    \end{align*}
    then consider a change of variables:
    \begin{equation*}
        \begin{pmatrix} X \\ Y\end{pmatrix} = \begin{pmatrix} x - y \\ y\end{pmatrix} = \begin{pmatrix} 1 & -1 \\ 0 & 1\end{pmatrix} \begin{pmatrix}x \\ y\end{pmatrix}
    \end{equation*}
    Therefore the Jacobian is $\det{J} = 1$,
    \begin{align*}
        \text{LHS} &= \int_{\R^2} e^{-i \lambda(X + Y)} f(X) g(Y) dX~dY \\
        &= \left[\int_{-\infty}^{\infty} e^{-i \lambda X}f(X) dX \right]\left[\int_{-\infty}^{\infty} e^{-i \lambda Y} g(Y) dY \right] \\
        &= \hat{f}(\lambda) \hat{g}(\lambda).
    \end{align*}
\end{proof}
\section{Important Examples}
We want to be able to use FTs to solve partial differential equations. To do this, we need to understand how it acts on certain functions.
\subsection{Exponentials}
Let $\sigma \in \C$ with $\Re(\sigma) > 0$. Let:
\begin{equation*}
    f(x) = H(x) e^{-\sigma x} =
    \begin{cases}
        e^{-\sigma x} & x \geq 0 \\
        0 & x < 0
    \end{cases}
\end{equation*}
Then we find the Fourier transform:
\begin{align}
    \hat{f}(\lambda) &= \int_{-\infty}^{\infty} f(x)e^{-i\lambda x} dx \nonumber \\
    &= \int_{0}^{\infty} e^{-(\sigma + i \lambda) x} dx \nonumber \\
    &= \frac{1}{\sigma + i \lambda} \label{eqnFTExp}
\end{align}
Note that indeed $|\hat{f}(\lambda)| \to 0$ as $|\lambda| \to \infty$, but only with $\frac1\lambda$. This fairly slow convergence is because the function has a discontinuity, and we recall that a ``less nice'' input function gives a slower decaying Fourier transform.

From theorem~\ref{thmInverseFT},
\begin{align*}
    H(x) e^{-\sigma x} &= \frac{1}{2\pi} \int_{-\infty}^{\infty} \frac{e^{i \lambda x}}{\sigma + i \lambda} d\lambda \\
    &= \frac{1}{2\pi i} \int_{-\infty}^{\infty} \frac{e^{i \lambda x}}{\lambda - i \sigma} d\lambda
\end{align*}
This is easy to evaluate with ideas from IB Complex Methods. However, we will use differentiation under the integral sign to evaluate it:
\begin{align*}
    H(x) x^k e^{-\sigma x}&= \left(-\frac{\partial}{\partial \sigma}\right)\frac{1}{2\pi} \int_{-\infty}^{\infty} \frac{e^{i \lambda x}}{i \lambda + \sigma} d\lambda \\
    &= \frac{k!}{2\pi} \int_{-\infty}^{\infty} \frac{e^{i \lambda x}}{(\sigma + i \lambda)^{k + 1}} d\lambda 
\end{align*}
Then we can see that as $k$ gets larger, the LHS gets more regular and the RHS decays faster as $\lambda$ gets large.
\subsection{Gaussian}
Consider $f(x) = \frac{1}{\sqrt{2\pi}} e^{-x^2 / 2}$, the standard Gaussian.
Then we try to evaluate:
\begin{equation*}
    \hat{f}(\lambda) = \frac{1}{2\pi} \int_{-\infty}^\infty e^{-i \lambda x - x^2 / 2} dx
\end{equation*}
This is difficult without IB Complex Methods. However, we can notice the following:
\begin{align*}
    \left(\frac{d}{dx}\right)f &= -x f \\
    \implies (i \lambda) \hat{f} &= -\left(i \frac{d}{d\lambda}\right)\hat{f} \\
    \implies \left(\frac{d}{dx}\right) \hat{f} &= -\lambda \hat{f}.
\end{align*}
And so:
\begin{align*}
    \hat{f}(\lambda) &=\hat{f}(0) e^{-\lambda^2 / 2} \\
    \intertext{Because $f$ is a Gaussian,}
    \hat{f}(0) &= \int_{-\infty}^{\infty} f(x) dx = 1 \\
    \intertext{So:}
    \hat{f}(\lambda) &= e^{-\frac{\lambda^2}{2}}
\end{align*}
And so Gaussians are eigenfunctions of the operator $\FT$. $\FT[f] = \sqrt{2\pi} f$. We have also evaluated our original integral:
\begin{equation*}
    e^{-\frac{\lambda^2}{2}} = \frac{1}{\sqrt{2\pi}} \int_{-\infty}^\infty e^{-i \lambda x - x^2 / 2} dx
\end{equation*}
\subsection{Dirac Delta Function}
From the definition of $\delta(x)$,
\begin{equation*}
    \hat{\delta}(x) = \int_{-\infty}^{\infty} e^{-i \lambda x}\delta(x) dx = 1
\end{equation*}
From the inversion formula,
\begin{equation*}
    \delta(x) = \frac{1}{2\pi} \int_{-\infty}^{\infty} e^{i \lambda x} d\lambda
\end{equation*}
\section{Initial Value Problems, Revisited}
Recall the initial value problem from the previous chapter:
\begin{equation}
    \begin{cases}
        Ly = f(t) & t > 0 \\
        y(0) = \dot{y}(0) = 0 &
    \end{cases}
    \tag{\ref{eqnInhomogIVP}}
\end{equation}
Now assume that $L$ has constant coefficients,
\begin{equation*}
    L = \alpha \frac{d^{2}}{dt^{2}} + \beta \frac{d}{dt} + \gamma
\end{equation*}
Then take $\alpha = 1$ without loss of generality (by dividing through). Extend the definition of $t \mapsto y(t)$ by taking $y = 0$ on $t < 0$. Same for $f$.

Because we are dealing with time ($t$), it is customary to write $\omega$ rather than $\lambda$ for the parameter to the FT. Take the FT of the problem:
\begin{equation*}
    \left[(i \omega)^2 + \beta(i \omega) + \gamma\right] \hat{y}(\omega) = \hat{f}(\omega)
\end{equation*}
Therefore:
\begin{equation*}
    \hat{y}(\omega) = \frac{\hat{f}(\omega)}{P(\omega)},\quad P(\omega) = (i \omega)^2 + \beta(i \omega) + \gamma
\end{equation*}
Then write $P$ in terms of its roots, $P(\omega) = (i \omega + \sigma_1)(i \omega + \sigma_2)$. Assume that these have positive real parts, and initially assume that the roots are distinct. Apply partial fraction decomposition:
\begin{align*}
    \frac{1}{P(\omega)} &= \frac{1}{\sigma_2 - \sigma_1} \left[\frac{1}{i \omega + \sigma_1} - \frac{1}{i \omega + \sigma_2}\right] \\
    \intertext{And by the section on exponentials,}
    &= \frac{1}{\sigma_2 - \sigma_1} \left[\FT[H(t)e^{-\sigma_1 t}] - \FT[H(t) e^{-\sigma_2t}]\right]
\end{align*}
If we define the \underline{response function}:
\begin{equation*}
    R(t) = H(t) \left[\frac{e^{-\sigma_1 t} - e^{-\sigma_2 t}}{\sigma_2 - \sigma_1}\right]
\end{equation*}
Then $\hat{R}(\omega) = \frac{1}{P(\omega)}$ and so:
\begin{equation*}
    \hat{y}(\omega) = \hat{R}(\omega) \hat{f}(\omega)
\end{equation*}
In the degenerate case $\sigma_1 = \sigma_2$, detuning (taking the limit as $\sigma_2 \to \sigma_1$) gives that instead $R(t) = H(t) te^{-\sigma t}$. In either case, by proposition~\ref{propFTConvol},
\begin{align*}
    y(t) &= R \star f(t) \\
    &= \int_{-\infty}^{\infty} R(t - \tau) f(\tau) d\tau \\
    &= \int_{0}^{t} R(t - \tau)f(\tau) d\tau 
\end{align*}
Now $R$ looks like a Green's function for the problem, $R(t - \tau) = G(t;\tau)$.

Note that we did not explicitly apply the initial conditions. We did, however, \textit{implicitly} use them:
\begin{align*}
    &\int_{0}^{\infty} \left(\ddot{y} + \beta \dot{y} + \gamma y\right) e^{-i \omega t} dt \\
    &= \int_{0}^{\infty} \ddot{y} e^{-i \omega t} dt + \int_0^\infty \beta \dot{y} e^{-i \omega t}dt + \int_0^\infty \gamma y e^{-i \omega t} dt \\
    &= \left[\dot{y}(t) e^{-i \omega t}\right]_0^\infty + (i\omega) \int_{0}^{\infty} \dot{y}(t) e^{-i \omega t} dt + \left[\beta y(t) e^{-i \omega t}\right]_0^\infty \\
        &\qquad + (i\omega) \int_{0}^{\infty} \beta y(t) e^{-i \omega t} dt + \int_{0}^{\infty} \gamma y e^{-i \omega t} dt \\
    \intertext{Use initial conditions:}
    &= 0 + (i\omega) \int_{0}^{\infty} \dot{y}(t) e^{-i \omega t} dt + 0 + (i\omega) \int_{0}^{\infty} \beta y(t) e^{-i \omega t} dt\\
        &\qquad+ \int_{0}^{\infty} \gamma y e^{-i \omega t} dt \\
    &= (i \omega)^2 \int_{0}^{\infty} y(t) e^{-i \omega t} dt + (i\omega) \int_{0}^{\infty} y(t) e^{-i \omega t} dt + \int_{0}^{\infty} \gamma y e^{-i \omega t} dt \\
    &= \left[(i \omega)^2 + \beta(i \omega) + \gamma\right] \hat{f}(\omega)
\end{align*}
This problem arose because we dealt with the domain $[0, \infty)$, rather than the real line where the boundary terms are zero without further assumption on $y$.
\begin{example}[Damped SHM]
    Consider the problem:
    \begin{equation*}
        \begin{cases}
            \ddot{y} + 2\gamma \dot{y} + y = f(t) & t > 0 \\
            y(0) = \dot{y} = 0 & t \leq 0
        \end{cases}
    \end{equation*}
    for $\gamma > 0$.

    By FT,
    \begin{equation*}
        \hat{y}(\omega) = \frac{\hat{f}(\omega)}{P(\omega)},\quad P(\omega) = \left(i \omega + \gamma + \sqrt{\gamma^2 - 1}\right) \times \left(i \omega + \gamma - \sqrt{\gamma^2 - 1}\right)
    \end{equation*}
    Note that the real part of both roots is positive, so we can use our result.

    In the case $\gamma > 1$, set $\sigma_{\pm} = \gamma \pm \sqrt{\gamma^2 -1}$
    \begin{align*}
        R(t) &= H(t) \left[\frac{e^{-\sigma_- t} - e^{-\sigma_+ t}}{\sigma_+ - \sigma_-}\right] \\
        &= H(t) e^{-\gamma t} \frac{\sinh(t\sqrt{\gamma^2 - 1})}{\sqrt{\gamma^2 - 1}}
    \end{align*}
    and this is overdamping.

    In the case $\gamma < 1$,
    \begin{equation*}
        R(t) = H(t) e^{-\gamma t} \frac{\sin(t \sqrt{1 - \gamma^2})}{\sqrt{1 - \gamma^2}}
    \end{equation*}
    this is underdamping.

    For $\gamma = 1$, we can take a limit:
    \begin{equation*}
        R(t) = H(t) te^{-t}
    \end{equation*}
    and this is critical damping.
\end{example}
\section{Solving Partial Differential Equations}
For PDEs on bounded domains, we often found solutions of the form:
\begin{equation*}
    f(x) = \sum_{n=1}^{\infty} \hat{f}_n Y_n(x)
\end{equation*}
In an unbounded domain, we need to instead take a continuous set of eigenfunctions, using an integral rather than a sum. This is closely linked to the Fourier Transform.
\section{Important Applications}
\subsection{Poisson Summation Formula}
\begin{proposition}
    For a suitable function $f$,
    \begin{equation*}
        \sum_{n \in\Z} f(x + n) = \sum_{n \in \Z} \hat{f}(2 \pi n) e^{2 \pi n x}
    \end{equation*}
    \label{propPoissonSum}
\end{proposition}
\begin{proof}
    First note that the LHS and RHS are periodic in $x$ with period 1. Let the LHS be $F$, and RHS be $G$. Therefore, they have a Fourier series:
    \begin{align*}
        \hat{F}_n &= \int_{0}^{1} e^{-2 \pi i n x} F(x) dx \\
        &= \sum_{m \in \Z} e^{-2 \pi i n x} f(x + m) dx \\
        &= \sum_{m \in \Z} \int_{m}^{m+1} e^{-2 \pi inx} f(x) dx \\
        &= \int_{-\infty}^{\infty} e^{-2 \pi i n x} f(x) dx \\
        &= \hat{f}(2 \pi n)
    \end{align*}
    Therefore $\hat{F}_n = \hat{G}_n$, so $F(x) = G(x)$.
\end{proof}
\begin{example}
    Consider $f(x) = H(x) x e^{-\sigma x}$ with $\Re(\sigma) > 0$.

    Computing the Fourier Transform:
    \begin{align*}
        \text{Let } g(x) &= H(x) e^{-\sigma x} \\
        \text{So } f(x) &= x f(x) \\
        \hat{f}(x) &= i \frac{d}{d\lambda} \hat{g}(\lambda) \text{ by proposition~\ref{propFTMult}} \\
        &= i \frac{d}{d\lambda} \left(\frac{1}{\sigma + i \lambda}\right) \\
        &= i \frac{-i}{(\sigma + i \lambda)^2} \\
        &= (\sigma + i \lambda)^{-2}
    \end{align*}
    Now apply proposition~\ref{propPoissonSum} with $x = 0$:
    \begin{align*}
        \sum_{n \in \Z} \hat{f}(2 \pi n) e^{2 \pi n} &= \sum_{n \in \Z} f(n) \\
        \sum_{n \in \Z} \frac{e^{2 \pi n}}{(2 \pi i n + \sigma)^2} &= \sum_{n \in \Z} n e^{-\sigma n} \\
        &= \frac{e^{\sigma}}{(1 - e^\sigma)^2} \\
        &= \frac{4}{\sinh^2\left(\frac{\sigma}{2}\right)}
    \end{align*}
\end{example}
\begin{example}
    Setting $f(x) = \frac{1}{\sqrt{2\pi}} e^{-\frac{x^2}{2}}$, we recall that $\hat{f}(\lambda) = e^{-\frac{\lambda^2}{2}}$. Applying this:
    \begin{equation*}
        \frac{1}{\sqrt{2\pi}} \sum_{n \in \Z} e^{-(x + n)^2 / 2} = \sum_{n \in \Z} e^{-(2 \pi n)^2 / 2} e^{2 \pi i n x}
    \end{equation*}
\end{example}
\subsection{Heisenberg Uncertainty Principle}
\begin{proposition}
    If $\psi$ is a normalised eigenfunction, $\int_{-\infty}^{\infty} |\psi(x, t)|^2 dx = 1$,
    \begin{equation*}
        \left(\int_{-\infty}^{\infty} x^2 |\psi(x, t)|^2 dx \right)\left(\int_{-\infty}^{\infty} \lambda^2 |\hat{\psi}(\lambda, t)|^2 d\lambda \right) \geq \frac{\pi}{2}
    \end{equation*}
    \label{propHeisUncertainty}
\end{proposition}
\begin{proof}
    Start with $1 = \int_{-\infty}^{\infty} 1 \times \psi(x, t) \overline{\psi}(x, t) dx$.
    We can integrate by parts:
    \begin{align*}
        1 &= -\int_{-\infty}^{\infty} x \frac{\partial}{\partial x}\left(\psi \overline{\psi}\right) dx  \\
        &= -\int_{-\infty}^{\infty} x \psi \overline{\psi}' + x \psi' \overline{\psi} dx \\
        \intertext{Then by triangle inequality:}
        &\leq 2 \int_{-\infty}^{\infty} x|\psi||\psi'| dx \\
        &\leq 2 \left(\int_{-\infty}^{\infty} x^2 |\psi|^2 dx \right)^\frac12 \left(\int_{-\infty}^{\infty} |\psi'|^2 dx \right)^\frac12 
    \end{align*}
    Recall the product rule for Fourier transforms, $\FT[f'(x)] = (i \lambda) \FT[f(x)]$.
    \begin{align*}
        \int_{-\infty}^{\infty} |f(x)|^2 dx &= \frac{1}{2\pi} \int_{-\infty}^{\infty} |\hat{f}(\lambda)|^2 d\lambda \\
        \implies \int_{-\infty}^{\infty} |f'(x)|^2 dx &= \frac{1}{2\pi} \int_{-\infty}^{\infty} \lambda^2 |\hat{f}(\lambda)|^2 d\lambda
    \end{align*}
    And hence:
    \begin{equation*}
        1 \leq \left(\int_{-\infty}^{\infty} x^2 |\psi|^2 dx \right)^\frac12 \left(\frac{1}{2\pi}\int_{-\infty}^{\infty} \lambda^2|\hat{\psi}|^2 d\lambda \right)^\frac12 
    \end{equation*}
    and the result follows.
\end{proof}
Note now that replacing $\psi(x, t)$ with $e^{- i \lambda_0x} \psi(x + x_0, t)$,
\begin{align*}
    \left(\int_{-\infty}^{\infty} x^2 |\psi(x + x_0, t)|^2 dx \right)^\frac12 \left(\frac{1}{2\pi}\int_{-\infty}^{\infty} \lambda^2|\hat{\psi}(\lambda + \lambda_0, t)|^2 d\lambda \right)^\frac12 &\geq \frac\pi2 \\
    \left(\int_{-\infty}^{\infty} (x - x_0)^2|\psi(x, t)|^2 dx \right)^\frac12 \left(\frac{1}{2\pi}\int_{-\infty}^{\infty} (\lambda - \lambda_0)^2|\hat{\psi}(\lambda, t)|^2 d\lambda \right)^\frac12 &\geq \frac\pi2 \\
\end{align*}
And then choosing:
\begin{align*}
    x_0 &= \int_{-\infty}^{\infty} x |\psi(x, t)|^2 dx \\
    \lambda_0 &= \int_{-\infty}^{\infty} \lambda |\hat{\psi}(\lambda, t)|^2 d\lambda
\end{align*}
gives the Heisenberg uncertainty principle.
\subsection{Extending Fourier Series to Fourier Transforms}
Consider a smooth function $f : \R \mapsto \C$ and assume that $f(x)$ is identically zero outside $(-L, L)$. Then on the interval $(-L, L)$, we can construct the Fourier series:
\begin{align*}
    f(x) &= \sum_{n \in \Z} \hat{f}_n e^{- \pi nx / L} \\
    \hat{f}_n &= \frac{1}{2L} \int_{-L}^{L} e^{-i\pi nx/L}f(x) dx \\
    &= \frac{1}{2L} \int_{-\infty}^{\infty} e^{-i\pi nx/L}f(x) dx
\end{align*}
because $f$ is zero outside the considered range $(-L, L)$.

Now write $\lambda_n = \frac{n \pi }{L}$, $\delta \lambda_n = \lambda_{n+1} - \lambda_n = \frac{\pi}{L}$. Then:
\begin{equation*}
    \hat{f}_n = \frac{1}{2\pi}\delta \lambda_n \hat{f}(\lambda_n).
\end{equation*}
Now for $x \in (-L, L)$,
\begin{align*}
    f(x) &= \frac{1}{2\pi} \sum_{n \in \Z} e^{i \lambda_n x} \hat{f}(\lambda_n) \delta \lambda_n\\
    \intertext{Now take $l \to \infty$, and so $\delta \lambda_n \to 0$:}
    &= \frac{1}{2\pi} \int_{-\infty}^{\infty} e^{i \lambda x} \hat{f}(\lambda) d\lambda 
\end{align*}
Also, from Parseval's theorem:
\begin{align*}
    \int_{-L}^{L} |f(x)|^2 dx &= 2L \sum_{n \in \Z} |\hat{f}_n|^2 \\
    &= \frac{1}{2\pi} \sum_{n \in \Z} |\hat{f}(\lambda_n)|^2 \delta \lambda_n \\
    \intertext{then taking $L \to \infty$:}
    \int_{-\infty}^{\infty} |f(x)|^2 dx &= \frac{1}{2\pi} \int_{-\infty}^{\infty} |\hat{f}(\lambda)|^2 d\lambda
\end{align*}
So we can derive Parseval's theorem for Fourier Transforms from Parseval's theorem for Fourier Series.
\subsection{Central Limit Theorem}
Let $X_1, \cdots, X_n$ be independent and identically distributed random variables with expectation $\mu$ and variance $\sigma^2$. WLOG, after possibly shifting each random variable, let $\mu = 0$.

Define $S_n = \frac{1}{\sqrt{n}} \sum_{i = 1}^n X_i$.

Denote the probability density function of $S_n$ by $f(x)$.
\begin{align*}
    \hat{f}_n(\lambda) &=\E\left(e^{-i \lambda S_n}\right) \\
    &=\E\left(e^{-i \lambda (X_1 + \cdots + X_n) / \sqrt{n}}\right) \\
    &= \prod_{i=1}^{n} \E\left(e^{- i \lambda X_i / \sqrt{n}}\right) \\
    &= \left(\E\left(e^{-i \lambda X_1 / \sqrt{n}}\right)\right)^n \\
    &= \varphi_{X_1}\left(\frac{\lambda}{\sqrt{n}}\right)^n
\end{align*}
where $\varphi$ is the moment generating function for $X_1$.
Now notice:
\begin{align*}
    \varphi_{X_1}(0) &= 1 \\
    \varphi_{X_1}'(0) &= \mu = 0 \\
    \varphi_{X_1}''(0) &= -\sigma^2
\end{align*}
For ease, set $\sigma = 1$. Now the Fourier transform is:
\begin{align*}
    \hat{f}_n(\lambda) &= \left(1 - \frac12 \frac{\lambda^2}{n} + o\left(\frac{1}{n}\right)\right)^n \\
    &= e^{-\lambda^2 / 2} + o(1) \text{ as $n \to \infty$}
\end{align*}
Now recall that the Gaussian is an eigenfunction of the Fourier Transform, so by the Inverse Fourier Transform:
\begin{align*}
    \lim_{n \to \infty} f_n(x) &= \lim_{n \to \infty} \frac{1}{2\pi} \int_{-\infty}^{\infty} e^{i \lambda x} \hat{f}_n(\lambda) d\lambda  \\
    &= \frac{1}{\sqrt{2\pi}} e^{-x^2 / 2}
\end{align*}
That is, $S_n$ tends to a Gaussian random variable in distribution.
\section{The Discrete Fourier Transform}
\subsection{Definition and Connection with Continuous FT}
The discrete Fourier transform does what the Fourier Transform does but to sequences. That is, it allows us to define a new sequence with possibly different properties.

\begin{definition}{Discrete Fourier transform}
    Given a sequence $(X_n)_{n = 0}^{N-1} \in \C^{N}$, define the \underline{discrete Fourier transform} (DFT) of this sequence by the new sequence $(\hat{X}_k)_{k = 0}^{N-1}$ defined by:
    \begin{equation*}
        \hat{X}_k = \sum_{n=0}^{N-1} X_n e^{-(\frac{2\pi n}{N})k}
    \end{equation*}
\end{definition}
Now let $f : [0, 2\pi) \mapsto \C$ be given and suppose that $X_n = f(x_n)$ where $(x_n)$ is the sequence:
\begin{equation*}
    x_n = \frac{2\pi n}{N} \in [0, 2\pi)
\end{equation*}
Further, setting $\delta x_n = x_{n+1} - x_n = \frac{2\pi }{N}$,
\begin{align*}
    \hat{X}_k&= \frac{N}{2\pi} \sum_{n=0}^{N-1} f(x_n) e^{-ikx_n} \delta x_n
\end{align*}
If we then extend $f(x) = 0$ when $x \notin [0, 2\pi)$, we find:
\begin{equation*}
    \hat{X}_k \approx \frac{N}{2\pi} \int_{-\infty}^{\infty}  e^{-ikx} f(x) dx = \frac{N}{2\pi} \hat{f}(k)
\end{equation*}
Then for $N$ large,
\begin{equation*}
    \hat{f}(k) \approx \frac{2\pi}{N} \hat{X}_k
\end{equation*}
So indeed we have that the DFT is the discrete analogue of the normal, continuous Fourier transform.
\subsection{The Inverse DFT}
\begin{proposition}
    Given $(\hat{X}_k)_{k = 0}^{N-1}$, the series $(X_n)_{n = 0}^{N-1}$ given by:
    \begin{equation*}
        X_n = \frac{1}{N} \sum_{k=0}^{N-1} \hat{X}_k e^{i \frac{2\pi k}{N}n}
    \end{equation*}
    is such that $(\hat{X}_k)$ is the DFT of $(X_n)$.
    \label{propInverseDFT}
\end{proposition}
\begin{proof}
    By direct computation:
    \begin{align*}
        &\sum_{k=0}^{N-1} \hat{X}_k e^{i \frac{2\pi n}{N}k} \\
        &= \sum_{k=0}^{N-1} \sum_{m=0}^{N-1} X_m e^{2 \pi i (n - m) k / N} \\
        \intertext{Write $\omega_N = e^{2 \pi i / N}$:}
        &= \sum_{m=0}^{N-1} X_m \sum_{k=0}^{N-1} \omega_N^{n-m}k
    \end{align*}
    Then for each term, if $n = m$ then the term is 1. When summed, this gives $N$. If they are different, the sum is given by:
    \begin{align*}
        \sum_{k=0}^{N-1} \omega_N^{n-m}k &= \frac{1 - \omega_N^{n-m}N}{1 - \omega_N^{n-m}} \\
        &= \frac{1 - 1}{1 - \omega_N^{n-m}} = 0
    \end{align*}
    Therefore,
    \begin{align*}
        \sum_{k=0}^{N-1} \hat{X}_k e^{2\pi i k n / N}&= \sum_{m = 0}^{N-1} X_m N \delta_{nm} \\
        &= N X_n
    \end{align*}
\end{proof}
\subsection{Connection to Linear Algebra}
We can also represent these finite sequences as vectors in $\C^N$. Therefore:
\begin{equation*}
    \hvec{X} = \FT \vec{X}
\end{equation*}
and so $\FT$, in this case, is an $N \times N$ matrix, with elements given by:
\begin{equation*}
    \FT_{kn} = e^{-i \frac{2\pi n}{N} k}
\end{equation*}
From the previous, we know that:
\begin{equation*}
    \FT \FT^\dagger = \FT^\dagger \FT = N I_N
\end{equation*}
and so $\FT$ is \textit{almost} unitary. In fact, the matrix $\frac{1}{\sqrt{n}}\FT$ is unitary.
\begin{example}
    Take $X_n$ to be the binomial coefficients, $X_n = \binom{N-1}{n}$. Taking the DFT:
    \begin{equation*}
        \hat{X}_k = \sum_{n=0}^{N-1} \binom{N-1}{n} e^{-i (2\pi i n / N)k} = (1 + e^{-2 \pi i k / N})^{N-1}
    \end{equation*}
\end{example}
\subsection{Properties of the DFT}
\begin{proposition}[Parseval's theorem for DFT]
    \begin{equation*}
        \sum_{n=0}^{N-1} X_n \overline{Y_n} = \frac{1}{N} \sum_{k=0}^{N-1} \hat{X}_k \overline{\hat{Y}_k}
    \end{equation*}
    \label{propParsevalDFT}
\end{proposition}
\begin{proof}
    Starting from the RHS, and using the matrix form:
    \begin{align*}
        &= \frac{1}{N} (\FT \vec{X})\cdot(\FT \vec{Y}) \\
        &= \left(\frac{1}{\sqrt{n}} \FT \vec{X}\right) \cdot \left(\frac{1}{\sqrt{n}} \FT\vec{Y}\right) \\
        &= \vec{X} \cdot \vec{Y} \text{ because $\frac1{\sqrt{n}}\FT$ is unitary}
    \end{align*}
\end{proof}
\begin{corollary}
    \begin{equation*}
        \sum_{n=0}^{N-1} |X_n|^2 = \frac1n \sum_{k=0}^{N-1} |\hat{X}_k|^2
    \end{equation*}
    \label{corParsevalDFT}
\end{corollary}
\begin{example}
    Again take $X_n$ to be the binomial coefficients, $X_n = \binom{N-1}{n}$. Use proposition~\ref{propParsevalDFT}:
    \begin{align*}
        \sum_{n=0}^{N-1} \binom{N-1}{n}^2 &= \frac1N \sum_{k=0}^{N-1} (1 + e^{-2 \pi i k / N})^{N-1} (1 + e^{2 \pi i k / N})^{N-1} \\ 
        &= \frac1n \sum_{k=0}^{N-1} \left(e^{i \pi k / N} + e^{- i \pi k / N}\right)^{2N-2} \\
        &= \frac1n \sum_{k=0}^{N-1} \left(2 \cos\left(\frac{n \pi}{k}\right)\right)^{2N-2}
    \end{align*}
    Then by IA Numbers and Sets, we can simplify the LHS:
    \begin{align*}
        &= \sum_{n=0}^{N-1} \binom{N-1}{n} \binom{N-1}{n} \\
        &= \sum_{n=0}^{N-1} \binom{N-1}{n} \binom{N-1}{N-1-n} \\
        &= \binom{2N-2}{N-1}
    \end{align*}
    Therefore, setting $M = N-1$:
    \begin{equation*}
        \frac{1}{2^{2M}} \binom{2M}{M} = \frac{1}{M+1} \sum_{k=0}^{M} \cos^{2M}\left(\frac{\pi k}{M+1}\right)
    \end{equation*}
    And we note that this is the probability of getting exactly $M$ heads after tossing a fair coin $2M$ times.
\end{example}
As with the continuous Fourier Transform, we can define the convolution of sequences:
\begin{definition}{Convolution (of sequences)}
    Given sequences $(X_n)_{n = 0}^{N-1}$ and $(Y_n)_{n = 0}^{N-1}$, define the \underline{convolution} as:
    \begin{equation*}
        (X \star Y)_n = \sum_{m=0}^{N-1} X_{(n - m) \operatorname{mod} N}Y_m
    \end{equation*}
\end{definition}
\begin{proposition}
    \begin{equation*}
        \left[\FT(X \star Y)\right]_k = \hat{X}_k \hat{Y}_k
    \end{equation*}
    \label{propConvProdDFT}
\end{proposition}
\begin{proof}
    Compute directly:
    \begin{align*}
        \text{LHS } &=\sum_{n=0}^{N-1} \left(\sum_{m=0}^{N-1} X_{(n - m) \operatorname{mod} N}Y_m\right)e^{-i \left(\frac{2\pi n}{N}\right)k} \\
        &= \sum_{m=0}^{N-1} Y_m e^{-i (2 \pi m / N)k} \sum_{n=0}^{N-1} X_{(n - m) \operatorname{mod} N} e^{-i (2\pi (n-m) / N)k} \\
        &= \sum_{m=0}^{N-1} Y_m e^{-i (2 \pi m / N)k} \hat{X}_k \\
        &= \hat{X}_k \hat{Y}_k
    \end{align*}
\end{proof}
\nonexaminablesubsection{Computational Complexity and FFT}
The Discrete Fourier transform, given by:
\begin{equation*}
    \hat{X}_k = \sum_{n=0}^{N-1} X_n e^{-(2 \pi n / N) k}
\end{equation*}
has (if computed directly), a time complexity of $O(n^2)$, consistent with matrix multiplication. However, in the 1960s it was discovered that there exists an algorithm (the Fast Fourier Transform, FFT) which can perform the computation in time complexity $O(N \log_2(N))$. This was an enormous speedup. Incidentally, Gauss actually discovered this all the way back in 1820.

In the 1990s, with the advent of (theoretical) quantum computing, a Quantum algorithm was developed that can do this in time complexity $O((\log_2(N))^2)$. This is fast enough to break RSA encryption, but practial limitations mean that such an algorithm will not run for many years on quantum computers.
\end{document}