\documentclass[../Main.tex]{subfiles}

\begin{document}
\section{Superposition}
We will be interested in solving Boundary-Value problems (BVPs) and Initial Boundary-Value problems (IBVPs):
\begin{equation}
    \begin{cases}
        P\phi(\vec{x}) = 0 & \vec{x} \in \Omega \\
        \text{Appropriate boundary conditions} & \vec{x} \in \partial \Omega
    \end{cases}
    \label{eqnGeneralBVP}
\end{equation}
\begin{equation}
    \begin{cases}
        Q\phi(\vec{x}, t) = 0 & (\vec{x}, t) \in \Omega \times (0, T) \\
        \text{Appropriate initial conditions} & (\vec{x}, t) \in \Omega \times \{t = 0\} \\
        \text{Appropriate boundary conditions} & (\vec{x}, t) \in \partial \Omega \times (0, T)
    \end{cases}
    \label{eqnGeneralIBVP}
\end{equation}
Where here $P, Q$ are \underline{linear partial differential operators} and $\Omega$ is a bounded, open subset of $\R^n$.

Note that we can split equation~\ref{eqnGeneralIBVP} into:
\begin{align*}
    &\begin{cases}
        Q\phi_1(\vec{x}, t) = 0 & (\vec{x}, t) \in \Omega \times (0, T) \\
        \text{Initial conditions} = 0 & (\vec{x}, t) \in \Omega \times \{t = 0\} \\
        \text{Appropriate boundary conditions} & (\vec{x}, t) \in \partial \Omega \times (0, T)
    \end{cases} \\
    &\begin{cases}
        Q\phi_2(\vec{x}, t) = 0 & (\vec{x}, t) \in \Omega \times (0, T) \\
        \text{Appropriate initial conditions} & (\vec{x}, t) \in \Omega \times \{t = 0\} \\
        \text{boundary conditions} = 0 & (\vec{x}, t) \in \partial \Omega \times (0, T)
    \end{cases}
\end{align*}
and then $\phi = \phi_1 + \phi_2$ is a valid solution.

In this course, $\Omega$ will always be (possibly after a change of variables) a line/rectangle/cuboid. So without loss of generality we can always deal with boundary conditions that are zero everywhere except from on one endpoint/edge/face. For example:
\begin{equation}
    \begin{cases}
        P\psi(\vec{x}) = 0 & x \in (0, 1) \times (0, 1) \\
        \psi = f_i & \text{ on side } i \in \{1, \cdots, 4\}
    \end{cases}
    \label{eqneqnBVPSquare}
\end{equation}
and we can split this into 4 functions $\psi_i$:
\begin{equation*}
    \begin{cases}
        P\psi_i(\vec{x}) = 0 & \vec{x} \in (0, 1)^2 \\
        \psi_i = 0 & \text{on side } \neq i \\
        \psi_i = f_i & \text{on side } = i
    \end{cases}
\end{equation*}
Then $\psi = \sum_{i=1}^4 \psi_i$ solves the problem.
\section{Laplace's Equation}
Recall Laplace's Equation for $\phi : \R^n \mapsto \R$:
\begin{equation}
    \Delta \phi = 0
    \label{eqnLaplace}
\end{equation}
where $\Delta = \nabla^2 = \nabla \cdot \nabla$. We say $\phi$ is \underline{harmonic} if it satisfies this equation. Harmonic functions are always infinitely differentiable (smooth).
\begin{example}[Incompressible irrotational fluid]
    Consider $\vec{u} : \R^3 \mapsto \R^3$ is a velocity field of an incompressible fluid ($\nabla u = 0$), and if it is also irrotational ($\nabla \times u = \vec{0}$), we can solve using Laplace's equation since there must exist a potential $\phi : \R^3 \mapsto \R$ such that $\vec{u} = \nabla \phi$. Then:
    \begin{equation*}
        0 = \nabla \cdot \vec{u} = \nabla \cdot (\nabla \phi) = \Delta \phi
    \end{equation*}
\end{example}
We now consider BVPs of the form:
\begin{equation*}
    \begin{cases}
        \Delta \phi(\vec{x}) = 0 & \vec{x} \in \Omega \\
        B\phi = f(\vec{x}) & \vec{x} \in \partial \Omega
    \end{cases}
\end{equation*}
where $B\phi = \phi$ (Dirichlet BCs), $B\phi = \frac{\partial \phi}{\partial \vec{n}}$ (Von Neumann BCs), or more generally $B\phi = \phi + \frac{\partial \phi}{\partial \vec{n}}$ (Robin BCs).
\subsection{Separation of Variables on a Square}
Consider:
\begin{equation*}
    \begin{cases}
        \phi_{xx} + \phi_{yy} = 0 & (x, y) \in (0, 1)^2 \\
        \phi(x, y) = 0 & \{\{0, 1\}, y\} \cup \{x, 0\} \\
        \phi(x, 1) = f(x) & y = 1
    \end{cases}
\end{equation*}
Now try a \underline{separable} solution of the form:
\begin{equation*}
    \phi = X(x) Y(y)
\end{equation*}
Then:
\begin{equation*}
    X''(x) Y(y) + X(x)Y''(y) = 0
\end{equation*}
Dividing through by $XY$:
\begin{align*}
    \frac{X''(x)}{X(x)} + \frac{Y''(y)}{Y(y)} &= 0 \\
    -\frac{X''(x)}{X(x)} = \lambda = \frac{Y''(y)}{Y(y)}
\end{align*}
because we have 2 terms with different dependences, so they may only be constant. We can now simultaneously solve the $x$ equation and the $y$ equation:
\begin{equation*}
    \begin{cases}
        X'' + \lambda X = 0 & x \in (0,1) \\
        X(x) = 0 & x \in \{0, 1\}
    \end{cases}
\end{equation*}
which is simply a Sturm-Liouville problem with $w = 1, p = 1, q = 0$. Therefore, there exist eigenfunction-eigenvalue pairs $(X_n, \lambda_n)$ with increasing $\lambda$ and orthogonality condition $\inn{X_n}{X_m} = 0$ for $n \neq m$.

After checking, we find that $\lambda > 0$ for non-trivial solutions. Then we can solve:
\begin{equation*}
    X(x) = A\sin(\sqrt{\lambda}x) + B\cos(\sqrt{\lambda}x)
\end{equation*}
then applying the boundary conditions gives $B = 0$, $\lambda = \lambda_n = (n\pi)^2$.
\begin{equation*}
    X_n(x) = \sin(n\pi x), \inn{X_n}{X_m} = \frac12 \delta_{mn}
\end{equation*}
Therefore the $y$ problem becomes:
\begin{equation*}
    Y'' - (n\pi)^2 Y = 0
\end{equation*}
and we solve this to get $Y(y) = B \sinh(n\pi y)$, or $Y_n(y) = B_n \sinh(n\pi y)$. Combining these together:
\begin{equation*}
    \phi_n(x, y) = A_n \sin(n\pi x) \sinh(n\pi x)
\end{equation*}
each satisfies $\Delta y_n = 0$ on $(x, y) \in (0,1)^2$ and $\phi_n = 0$ on $x = 0$, $x = 1$ and $y = 0$. Therefore, the most general solution $\phi$ is:
\begin{equation*}
    \phi(x, y) = \sum_{n = 1}^\infty A_n \sin(n\pi x) \sinh(n\pi x)
\end{equation*}
but we need to apply the final boundary condition, $\phi(x, 1) = f(x)$ Taking an inner product:
\begin{align*}
    \inn{f}{X_n} &= \sum_{m=0}^{\infty} A_m \inn{X_m}{X_n} \sinh(n\pi) \\
    &= \frac{A_n}{2}\sinh(n\pi) \\
    \therefore A_n &= \frac{2}{\sinh(n\pi)} \int_{0}^{1} f(x) \sin(n\pi x) dx 
\end{align*}
\end{document}