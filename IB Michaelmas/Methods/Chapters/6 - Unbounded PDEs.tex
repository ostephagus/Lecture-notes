\documentclass[../Main.tex]{subfiles}

\begin{document}
\section{Well-Posedness}
\begin{definition}{Well-posedness}
    A problem is \underline{well-posed} if:
    \begin{enumerate}
        \item A solution exists
        \item The solution is unique
        \item The solution depends continuously on the given data (such as boundary conditions and initial conditions).
    \end{enumerate}
\end{definition}
Conditions 1 and 2 are obvious, but 3 is more subtle: to mention continuity implicitly means that we have to assign some topology which permits this definition.

For example, if we have an IBVP whose initial data and boundary data lie in a space $X$, and a time $t > 0$ the solution to this problem $u(x, t)$ belongs to another space $Y$, we get some abstract map:
\begin{equation*}
    S_t : X \mapsto Y
\end{equation*}
and we need to know whether this map is continuous. This depends on the metrics of $X$ and $Y$. Further, we want to have \textit{sensible} metrics - the discrete metric (assigning distance 1 to nonequal points, and 0 to equal points) will not do.
\begin{example}
    Consider the initial-value problem:
    \begin{equation*}
        \frac{dx}{dt} = -x,\qquad x(0) = x_0
    \end{equation*}
    Then clearly the first and second requirements of well-posedness are satisfied. The solution is:
    \begin{equation*}
        X_0(t) = x_0 e^{-t}
    \end{equation*}
    Now consider a different initial condition, $x(0) = x_1$. This has solution $X_1(t) = x_1 e^{-t}$. The difference in these solutions at time $t$ is:
    \begin{equation*}
        |X_1(t) - X_0(t)| = e^{-t}|x_1 - x_0| \leq |x_1 - x_0|
    \end{equation*}
    So, in the sense of the Euclidean norm, the problem is well-posed.
\end{example}
\begin{example}
    We can try to find an ill-posed problem by looking at the heat equation. This has the behaviour that two different initial conditions give similar solutions after sufficient time, because the heat simply spreads out. However, we can reverse the direction of time and this means that two similar initial solutions will evolve very differently.

    Consider the IBVP on $\Omega = (0, \pi)$:
    \begin{equation*}
        \begin{cases}
            \phi_t + \phi_{xx} = 0 & \Omega \times (0, \infty) \\
            \phi = 0 & \partial \Omega \times (0, \infty) \\
            \phi = f & \Omega \times \{t = 0\}
        \end{cases}
    \end{equation*}
    Then if $f(x) = 0$, we get the solution $\phi(x, t) = 0$.

    If we take instead $f(x) = f_n(x) = \frac{1}{n} \sin(nx)$, we get the solution:
    \begin{equation*}
        \phi_n(x, t) = \frac{1}{n} e^{n^2 t} \sin(nx)
    \end{equation*}
    We see here that two initial conditions that are close give two solutions that are far apart. Consider the supremum norm on $(0, \pi)$:
    \begin{align*}
        ||f - f_n||_\infty &= \sup_{(0, \pi)} |f(x) - f_n(x)| = \frac1n \to 0 \\
        ||\phi(x, t) - \phi_n(x, t)|_\infty &= \frac{1}{n} e^{n^2 t} \to \infty
    \end{align*}
    Then this is ill-posed. We have another notion (local well-posedness) when we consider initial conditions that are close together, and time not too long. However, this problem is not even locally well-posed.
\end{example}
\section{The Method of Characteristics}
We want to solve a PDE of the following form:
\begin{equation*}
    a(x, y) \frac{\partial u}{\partial x} + b(x, y) \frac{\partial u}{\partial y} = c(x, y, u)
\end{equation*}
This is a quasi-linear problem, because the nonlinearity occurs only on the RHS.

We consider this equation, along with initial conditions $u(x, y, 0) = \phi(x, y)$ on a curve $C \subseteq \R^2$.

The key idea for this is attributed to Bernhard Riemann. This idea is to look at curves $(x, y) = (x(t), y(t))$ defined by:
\begin{gather*}
    \frac{dx}{dt} = a(x, y) \\
    \frac{dy}{dt} = b(x, y) \\
    (x(0),y(0)) \in C
\end{gather*}
We call these the \underline{characteristic curves}. Using these, we get a whole family of solutions determined by starting point $(x(0), y(0))$. Consider the evolution of $u(x, y)$ along a given characteristic. Set:
\begin{equation*}
    z(t) = u(x(t), y(t))
\end{equation*}
By the chain rule:
\begin{align*}
    \frac{dz}{dt} &= \frac{\partial u}{\partial x} \frac{dx}{dt} + \frac{\partial u}{\partial y} \frac{dy}{dt} \\
    &= a(x, t) \frac{\partial u}{\partial x} + b(x, t) \frac{\partial u}{\partial y} \\
    &= c(x, y, z) \text{ where $x, y, z$ are all functions of $t$.}
\end{align*}
Now we have an ordinary differential equation for $z$, with initial condition:
\begin{equation*}
    z(0) = u(x(0), y(0)) = \phi(x(0), y(0)) \text{ which is given.}
\end{equation*}
Instead of $(x(0), y(0))$, we will consider the curve $C$ to be parameterised by $s$, so then any point $(x, y) \in \R^2$ is given by $(s, t)$, where $s$ defines the characteristic and $t$ defines the point on the characteristic. We want to invert the relationship between $(x, y)$ and $(s, t)$.

\begin{example}
    Consider the problem:
    \begin{equation*}
        \frac{\partial u}{\partial x} + \frac{\partial u}{\partial y} = u,\qquad u(x, 0) = f(x)
    \end{equation*}
    Therefore our curve $C$ is $\{(s, 0), s \in \R\}$.

    The derivatives $\frac{dx}{dt}$ and $\frac{dy}{dt}$ are both $1$, so the characteristic curves are:
    \begin{equation*}
        \left\{
        \begin{split}
            x &= t + x_0  \\
            y &= t + y_0
        \end{split}
        \right.
    \end{equation*}
    We want $(x_0, y_0) \in C$ so take $(x_0, y_0) = (s, 0)$. Then $x = t + s, y = t$.
    \begin{align*}
        \frac{dz}{dt} &=c(x, y, z) \\
        &= z \\
        \therefore z(t) &= z_0 e^t \\
        z_0 &= u(x_0, y_0) \\
        &= u(s, 0) = f(s) \\
        \therefore z(t, s) &= f(s) e^t
    \end{align*}
    We now want to invert the relationship between $(x, y)$ and $(s, t)$ to get $s$ and $t$ in terms of $x$ and $y$:
    \begin{equation*}
        t = y,\qquad s = x - y
    \end{equation*}
    Then our final solution is:
    \begin{equation*}
        u(x, y) = z(t(x, y), s(x, y)) = f(x - y) e^y
    \end{equation*}
\end{example}
\begin{example}
    Consider the problem:
    \begin{equation*}
        (1 + x^2) \frac{\partial u}{\partial x} + \frac{\partial u}{\partial y} = u + 1, \qquad u(0, y) = f(y)
    \end{equation*}
    Then our curve $C$ is $\{(0, s), s \in \R\}$.

    The time derivatives for $x$ and $y$ are $\dot{x} = (1 + x^2)$, $\dot{y} = 1$, so:
    \begin{equation*}
        x(t) = \tan(t + \arctan(x_0)), \qquad y(y) = t + y_0
    \end{equation*}
    Then we need $(x_0, y_0) \in C$, $(x_0, y_0) = (0, s)$.
    \begin{equation*}
        \left\{
        \begin{split}
            x &= \tan(t) \\
            y &= t + s
        \end{split}
        \right.
    \end{equation*}
    Then we find $z$:
    \begin{align*}
        \frac{dz}{dt}&= z + 1 \\
        z(t) &= -1 + [z_0 + 1]e^t \\
        z_0 &= u(x_0, y_0) = f(s) \\
        \intertext{Hence}
        z(t, s) &= -1 + \left[f(s) + 1\right]e^t
    \end{align*}
    Then inverting the $(t, s) \mapsto (x, t)$ map:
    \begin{equation*}
        \left\{
        \begin{split}
            t &= \arctan(x) \\
            s &= y - \arctan(x)
        \end{split}
        \right.
    \end{equation*}
    We get our solution:
    \begin{equation*}
        u(x, y) = -1 + \left[f(y - \arctan(x)) + 1\right] e^{\arctan{x}}
    \end{equation*}
\end{example}
However, we have a problem: what if the characteristic curves cross? In this case we would have two different characteristics for a single point, which means that the solution would take multiple values at that point. In fact this tells us that the problem has non-unique solutions, so is not well-posed.
\section{Classifying Second-Order PDEs}
We will classify PDEs of the form $Lu = g$ where the operator $L$ is given by:
\begin{equation}
    a(x, y) \frac{\partial^{2}}{\partial x^{2}} + 2b(x, y) \frac{\partial^{2}}{\partial x\partial y} + c(x, y) \frac{\partial^{2}}{\partial y^{2}} + d(x, y) \frac{\partial }{\partial x} + e(x, y) \frac{\partial }{\partial y} + f(x, y)
    \label{eqnGeneralSOPDE}
\end{equation}
The terms we most want to control are the higher-order terms, the second derivatives. We will focus on these, and introduce a change of variables $(x, y) \mapsto (\xi, \eta)$ to simplify the problem.

Introduce vectors:
\begin{equation*}
    \vec{x} = \begin{pmatrix}x \\ y\end{pmatrix} = \begin{pmatrix}x_1 \\ x_2\end{pmatrix},\quad \vec{\xi} = \begin{pmatrix}\xi \\ \eta\end{pmatrix} = \begin{pmatrix}\xi_1 \\ \xi_2\end{pmatrix}
\end{equation*}
Then with this new notation the operator $L$ is:
\begin{equation*}
    L = \sum_{i, j = 1}^2 a_{ij} \frac{\partial^{2}u}{\partial x_i\partial x_j} + \text{lower order terms}
\end{equation*}
where $(a_{ij})$ is a matrix given by:
\begin{equation*}
    (a_{ij})(x, y) = \begin{pmatrix}
        a(x, y) & b(x, y) \\
        b(x, y) & c(x, y)
    \end{pmatrix}
\end{equation*}
The following proposition tells us what happens under this coordinate transform:
\begin{proposition}
    Consider any transform $\vec{x} \mapsto \vec{\xi}$.
    Let $U(\vec{\xi}) = u(\vec{x})$.
    Then the operator $L$ for $u$ transforms to the operator $\tilde{L}$ for $U$ defined by:
    \begin{equation*}
        \tilde{L}U = \sum_{p, q = 1}^{2} A_{pq} \frac{\partial^{2}U}{\partial \xi_p\partial \xi_q}
    \end{equation*}
    where:
    \begin{equation*}
        A_{pq} = \sum_{i, j = 1}^{2} a_{ij} \frac{\partial \xi_p}{\partial x_i} \frac{\partial \xi_q}{\partial x_j}
    \end{equation*}
    \label{propTransformPDE}
\end{proposition}
\begin{proof}
    Use summation convention. Then by the chain rule:
    \begin{equation*}
        \frac{\partial u}{\partial x_i} = \frac{\partial u}{\partial \xi_p} \frac{\partial \xi_p}{\partial x_i}
    \end{equation*}
    and so,
    \begin{align*}
        a_{ij} \frac{\partial^{2}u}{\partial x_i\partial x_j}&= a_{ij} \frac{\partial^{2}U}{\partial \xi_p\partial \xi_q} \frac{\partial \xi_p}{\partial x_i} \frac{\partial \xi_q}{\partial x_j} + a_{ij} \frac{\partial U}{\partial \xi_p} \frac{\partial^{2}\xi_p}{\partial x_i\partial x_j} \\
        &= A_{pq} \frac{\partial^{2}U}{\partial \xi_p\partial \xi_q} + \text{lower order terms}
    \end{align*}
\end{proof}
Then we use proposition~\ref{propTransformPDE} to get the elements of the matrix $(A_{pq})$, back in terms of $\xi$ and $\eta$:
\begin{equation*}
    (A_{pq}) =
    \begin{pmatrix}
        a \xi_x^2 + 2b \xi_x \xi_y + c\xi_y^2 & a\xi_x\eta_x + b(\xi_x \eta_y + \xi_y \eta_x) + c \xi_y \eta_y \\
        a\xi_x\eta_x + b(\xi_x \eta_y + \xi_y \eta_x) + c \xi_y \eta_y & a\eta_x^2 + 2b\eta_x\eta_y + c^2 \eta_y^2
    \end{pmatrix}
\end{equation*}
We can simplify the PDE greatly if we can make $A_{11}$ and $A_{22}$ zero. This happens when the variables $M = \xi_x / \xi_y$ or $N = \eta_x / \eta_y$ satisfy the quadratic equation:
\begin{equation*}
    az^2 + 2bz + c = 0
\end{equation*}
Then by the quadratic formula, the solutions are:
\begin{equation*}
    z = \frac{-b \pm \sqrt{b^2 - ac}}{a}
\end{equation*}
When $M$ or $N$ are chosen to satisfy this equation we call $\xi(x, y) = \text{const}$ and $\eta(x, y) = \text{const}$ the \underline{characteristic curves} of the PDE. When such curves exist, along them we have:
\begin{equation*}
    \xi_x + \frac{dy}{dx} \xi_y = 0,\qquad \eta_x + \frac{dy}{dx} \eta_y = 0
\end{equation*}
We can alternatively write this as $(x, y) = (x, y(x))$ where $y(x)$ is defined implicitly $\xi(x, y(x)) = \text{const}$ or $\eta(x, y(x)) = \text{const}$.

We then have a family of characteristic curves defined by:
\begin{equation}
    \frac{dy}{dx} = -\frac{-b \pm \sqrt{b^2 - ac}}{a}
    \label{eqnCCurvesImplicit}
\end{equation}
Then we can, at last, provide classifications for the PDE in equation~\ref{eqnGeneralSOPDE}:
\begin{definition}{Ellipitic, parabolic, hyperbolic operators}
    \begin{itemize}
        \item If $b^2 < ac$ then $L$ is an \underline{elliptic operator}, with no real characteristic curves.
        \item If $b^2 = ac$ then $L$ is a \underline{parabolic operator}, with 1 family of real characteristic curves.
        \item If $b^2 > ac$ then $L$ is a \underline{hyperbolic operator}, with 2 families of real characteristic curves.
    \end{itemize}
\end{definition}
The case of a hyperbolic operator is when we introduce the coordinates $(\xi, \eta)$ defined by equation~\ref{eqnCCurvesImplicit}, giving us the new partial differential operator of the form:
\begin{equation*}
    \tilde{L} = \frac{\partial^{2}}{\partial \xi\partial \eta} + \text{lower order terms}
\end{equation*}
and the new problem $\tilde{L}U = G(\xi, \eta)$ where $G(\xi, \eta) = g(x, y)$.
\begin{example}[Hyperbolic operator: wave equation]
    Consider the wave equation with speed $c = 1$:
    \begin{equation*}
        \frac{\partial^{2}u}{\partial x^{2}} - \frac{\partial^{2}u}{\partial y^{2}} = 0
    \end{equation*}
    Then our coefficient functions are, $a = 1, b = 0, c = -1$ and $b^2 - ac = 1$. This is greater than $0$, so the problem is hyperbolic. Because this holds for all $x$, we say the problem is \underline{globally hyperbolic}. The characteristic curves are defined by equation~\ref{eqnCCurvesImplicit}, which becomes, in this case,
    \begin{equation*}
        \frac{dy}{dx} = \mp 1
    \end{equation*}
    Integrating these to find the characteristic curves:
    \begin{equation*}
        y \pm x = \text{const}
    \end{equation*}
    Then we introduce the coordinates $\xi(x, y) = x - y$, $\eta(x, y) = x + y$. In these coordinates we indeed have $A_{11} = A_{22} = 0$, and the off-diagonal entries are:
    \begin{equation*}
        A_{12} = A_{21} = 2
    \end{equation*}
    Then for the new function $U(\xi, \eta) = u(x, y)$, the problem becomes:
    \begin{equation*}
        4 \frac{\partial^{2}U}{\partial x\partial y} = 0
    \end{equation*}
    which has solutions $U(\xi, \eta) = A(\xi) + B(\eta)$, where $A$ and $B$ are arbitrary functions. We can then find the solution for $u(x, y)$:
    \begin{equation*}
        u(x, y) = A(x - y) + B(x + y)
    \end{equation*}
    Now suppose we want to solve the initial value problem for $u(x, t)$ on a domain $\Omega = \R$:
    \begin{equation*}
        \begin{cases}
            Lu = 0 & \Omega \times (0, \infty) \\
            u = f & \Omega \times \{t = 0\} \\
            u_t = g &\Omega \times \{t = 0\} 
        \end{cases}
    \end{equation*}
    Now we know that the solution must have the form:
    \begin{equation*}
        u(x, t) = A(x - t) + B(x + t)
    \end{equation*}
    Matching the initial conditions, we find:
    \begin{equation*}
        u(x, t) = \frac12 \left[f(x + t) - f(x - t)\right] + \frac12 \int_{x - t}^{x + t} g(s) ds 
    \end{equation*}
\end{example}
\begin{example}[Locally hyperbolic operator]
    Consider the partial differential equation:
    \begin{equation*}
        xy \frac{\partial^{2}u}{\partial x^{2}} - \frac{\partial^{2}u}{\partial y^{2}} = 0
    \end{equation*}
    Then our coefficient functions are $a = xy, b = 0, c = 1$. The discriminant $b^2 - ac$ is $xy$, so we have that the PDE is hyperbolic only on the region $xy > 0$. In these two quarters of the plane we have the characteristic curves defined by the differential equation:
    \begin{equation*}
        \frac{dy}{dx} = \mp \frac{1}{\sqrt{xy}}
    \end{equation*}
    Integrating this up:
    \begin{equation*}
        \frac13 y^{3/2} \pm x^{1/2} = \text{const}
    \end{equation*}
    therefore, introduce the new coordinates $(\xi, \eta)$:
    \begin{equation*}
        \xi(x, y) = \frac13 y^{3/2} + x^{1/2},\quad\eta(x, y) = \frac13 y^{3/2} - x^{1/2}
    \end{equation*}
    Then our new PDE is:
    \begin{equation*}
        -\frac12 y \frac{\partial^{2}U}{\partial \xi\partial \eta} + \text{lower order terms} = 0
    \end{equation*}
\end{example}
\end{document}