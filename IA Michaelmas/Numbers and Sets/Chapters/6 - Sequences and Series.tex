\documentclass[../Main.tex]{subfiles}

\begin{document}
\section{Sequences and Limits}
\begin{definition}{Sequence}
    A \underline{sequence} is an enumerated collection of objects in which repetitions are allowed and order matters.
\end{definition}
We write the terms of a sequence $a$ as $a_1, a_2, a_3, \cdots$ or $(a_n)_{n=1}^\infty$.\par
Key question: what does it mean for a sequence to tend to a limit $l$?\par
We cannot simply say that a sequence gets closer to $l$. Consider the sequence $0, \frac{1}{2}, \frac{2}{3}, \frac{3}{4}, \cdots$. This does not tend to 100, but the terms do get closer to it.\par
We also cannot say that terms get arbitrarily close to $l$. Consider the sequence $0, 10, \frac{1}{2}, 10, \frac{2}{3}, 10, \frac{3}{4}, \cdots$. This gets arbitrarily close to $1$, but cannot tend to it because of the 10s every other term. We need the sequence to get close to $l$, and stay close to $l$. We can rigorously define this:
\begin{definition}{Limit of a sequence}
    A sequence $a_n$ \underline{tends to a limit} $l \in \R$ as $n$ tends to infinity if,
    \begin{equation*}
        \forall \epsilon > 0~\exists N \in \N \text{ such that } \forall n \geq N, l - \epsilon < a_n < l + \epsilon.
    \end{equation*}
\end{definition}
We can also write this as:
\begin{equation}
    \forall \epsilon > 0, \exists N \in \N \text{ such that } \forall n \geq N, |a_n - l| < \epsilon
    \label{eqnSequenceLimit}
\end{equation}
Note that this introduces modulus signs, for which the triangle inequality can be used:
\begin{equation}
    \forall a, b, c \in \R, |a - c| \leq |a - b| + |b - c|
    \label{eqnTriangleInequality}
\end{equation}
When a sequence tends to a limit $l$, we can also write:
\begin{itemize}
    \item $a_n \rightarrow l$ as $n \rightarrow \infty$
    \item $\lim_{n \rightarrow \infty} = l$
    \item $(a_n)_{n=1}^\infty$ \underline{converges} to $l$.
\end{itemize}
\begin{definition}{Convergence}
    A sequence $a_n$ is \underline{convergent} if it has a limit $l$.
\end{definition}
The opposite of convergence is \underline{divergence}.
\section{Convergence of Sequences}
We begin by demonstrating how to show convergence of different sequences:
\begin{examples}{}
    \item $a_n = 1-\frac{1}{n}$.\par
        Given $\epsilon > 0$, choose $N > \frac{1}{\epsilon}$ by proposition~\ref{propArchimedes}. Note that the limit to be shown is 1.\par
        Then $|a_n - 1| = \frac{1}{n}$, so using $n \leq N$ we have $\frac{1}{n} < \epsilon$, as required.
    \item $a_n = \begin{cases}\frac{1}{n} & n \text{ even} \\ 0 & \text{ otherwise}\end{cases}$.\par
        As above, choose $N > \frac{1}{\epsilon}$, so $\frac{1}{n} < \epsilon$ as required.
    \item $a_n = (-1)^n$\par
        Here we want to show divergence, so proceed by contradiction.\par
        Assume $a_n \rightarrow l$. Then use equation~\ref{eqnSequenceLimit} with $\epsilon = 1$. Therefore, $|1-l| < 1$ and $|-1-l| < 1$. Using equation~\ref{eqnTriangleInequality}, $|1 - l + 1 + l| \leq |1 - l| + |1 + l| < 2$, but $1 - l + 1 + l = 2$. This contradiction proves divergence.
\end{examples}
\begin{lemma}
    The limit of a convergent sequence is unique.
    \label{lemUniqueLimit}
\end{lemma}
\begin{proof}
    Assume, on the contrary, that a sequence $a_n$ tends to the distinct real numbers $l$ and $k$.\par
    Then consider $\epsilon = \frac{1}{2} |l - k|$.\par
    From equation~\ref{eqnSequenceLimit},
    \begin{align*}
        &\exists N \in \N \text{ such that } |a_n - l| < \epsilon \forall n \geq N \\
        &\exists M \in \N \text{ such that } |a_n - k| < \epsilon \forall n \geq M
    \end{align*}
    So for any $n \geq \max{\{M, n\}}$, consider $|l - k|$. By the definition of $\epsilon$, this is equal to $2\epsilon$.\par
    Also, $|a_n - k - (a_n - l)| \leq |a_n - k| + |a_n - l| < 2\epsilon$ \contradiction
\end{proof}
\begin{definition}{Bounded sequence}
    A sequence $a_n$ is \underline{bounded} if there exists a real number $B$ such that, for all $n$, $a_n \leq B$.
\end{definition}
Note that any sequence that converges is also bounded. To see this, choose some tolerance $\delta$. Then by the definition of convergence, all the sequence terms after the $N$th are within $\delta$ of the limit $l$. Then we consider the maximum of the finite set:
\begin{equation*}
    \max{\{a_1, a_2, a_3, \cdots, a_{N-1}, l + \delta\}}
\end{equation*}
To see that this is the bound.
\begin{definition}{Monotonic}
    A sequence $a_n$ is \underline{monotonic} if it is either increasing or decreasing. Increasing means that for all $n$, $a_{n+1} \geq a_n$, and decreasing is the reverse.
\end{definition}
\begin{theorem}
    Every bounded monotonic sequence of real numbers converges.
\end{theorem}
\begin{proof}
    Suppose a sequence $a_n$ is increasing and bounded above.\par
    Consider the set $\subsetselect{a_n}{n \in \N}$. It is non-empty and bounded above, so we have a supremum $l$.\par
    For any $\epsilon > 0$, consider $l - \epsilon$ which cannot be an upper bound for the set. Therefore $\exists N \in \N$ such that $a_N > l - \epsilon$.\par
    But also $a_N \leq a_n$ for all $n \geq N$ by assumption that $a_n$ is increasing, so equation~\ref{eqnSequenceLimit} is satisfied.\par
    The same method of proof can be used to show that a decreasing bounded sequence must converge.
\end{proof}
\begin{remarks}
    \item For an increasing sequence to converge, we only need it to be bounded above.
    \item THe bounded condition is necessary - monotonic sequences can (and often do) diverge.
    \item This theorem is actually equivalent to the least upper bound axiom (each implies the other)
    \item It can be shown that every sequence has an infinite monotonic subsequence (see Analysis I)
\end{remarks}
\begin{proposition}
    If $a_n \leq d \forall n$ and $a_n \rightarrow c$ as $n \rightarrow \infty$, then $c \leq d$.
    \label{propLimitLessThanBound}
\end{proposition}
\begin{proof}
    Via real axiom~\ref{realAxiomOrdering}, it suffices to show that $c \not > d$.\par
    Suppose, on the contrary, that we have $c > d$. Let $\epsilon = c - d > 0$.\par
    Then $\exists N \in \N$ such that $\forall n > N, |a_n - c| < \epsilon$.\par
    Note that $a_n \leq d < c$, so $a_n < c$, and $|a_n - c| = c - a_n$. But we therefore have $c - a_n < c - d$, or $a_n > d$\contradiction\par
    Therefore $c \not > d$ so $c \leq d$.
\end{proof}
\begin{remark}
    If $a_n < d \forall n$ (strict inequality in assumption), we still only get $c \leq d$ (non-strict inequality in result).
\end{remark}
\begin{proposition}
    If $a_n \rightarrow c$ and $b_n \rightarrow d$ as $n \rightarrow \infty$, then $a_n + b_n \rightarrow c + d$.
    \label{propSumSequences}
\end{proposition}
\begin{proof}
    Given $\epsilon > 0$, we have:
    \begin{align*}
        &\exists N \in \N \text{ such that } \forall n \geq N, |a_n - c| < \epsilon \\
        &\exists M \in \N \text{ such that } \forall n \geq M, |b_n - d| < \epsilon
    \end{align*}
    Now let $N' = \max{\{N, M\}}$. Then, $\forall n \geq N'$,
    \begin{align*}
        |a_n + b_n - (c + d)| &\leq |a_n - c| + |b_n - d| \text{ by equation~\ref{eqnTriangleInequality}} \\
        &< \epsilon + \epsilon = 2\epsilon
    \end{align*}
    And a new variable $\delta$ can be defined to be $2\epsilon$ for equation~\ref{eqnSequenceLimit} to hold.
\end{proof}
\end{document}