\documentclass[../Main.tex]{subfiles}

\begin{document}
\section{Sequences and Limits}
\begin{definition}{Sequence}
    A \underline{sequence} is an enumerated collection of objects in which repetitions are allowed and order matters.
\end{definition}
We write the terms of a sequence $a$ as $a_1, a_2, a_3, \cdots$ or $(a_n)_{n=1}^\infty$.\par
Key question: what does it mean for a sequence to tend to a limit $l$?\par
We cannot simply say that a sequence gets closer to $l$. Consider the sequence $0, \frac{1}{2}, \frac{2}{3}, \frac{3}{4}, \cdots$. This does not tend to 100, but the terms do get closer to it.\par
We also cannot say that terms get arbitrarily close to $l$. Consider the sequence $0, 10, \frac{1}{2}, 10, \frac{2}{3}, 10, \frac{3}{4}, \cdots$. This gets arbitrarily close to $1$, but cannot tend to it because of the 10s every other term. We need the sequence to get close to $l$, and stay close to $l$. We can rigorously define this:
\begin{definition}{Limit of a sequence}
    A sequence $a_n$ \underline{tends to a limit} $l \in \R$ as $n$ tends to infinity if,
    \begin{equation*}
        \forall \epsilon > 0~\exists N \in \N \text{ such that } \forall n \geq N, l - \epsilon < a_n < l + \epsilon.
    \end{equation*}
\end{definition}
We can also write this as:
\begin{equation}
    \forall \epsilon > 0, \exists N \in \N \text{ such that } \forall n \geq N, |a_n - l| < \epsilon
    \label{eqnSequenceLimit}
\end{equation}
Note that this introduces modulus signs, for which the triangle inequality can be used:
\begin{equation}
    \forall a, b, c \in \R, |a - c| \leq |a - b| + |b - c|
    \label{eqnTriangleInequality}
\end{equation}
When a sequence tends to a limit $l$, we can also write:
\begin{itemize}
    \item $a_n \rightarrow l$ as $n \rightarrow \infty$
    \item $\lim_{n \rightarrow \infty} = l$
    \item $(a_n)_{n=1}^\infty$ \underline{converges} to $l$.
\end{itemize}
\begin{definition}{Convergence}
    A sequence $a_n$ is \underline{convergent} if it has a limit $l$.
\end{definition}
The opposite of convergence is \underline{divergence}.
\section{Convergence of Sequences}
We begin by demonstrating how to show convergence of different sequences:
\begin{examples}{}
    \item $a_n = 1-\frac{1}{n}$.\par
        Given $\epsilon > 0$, choose $N > \frac{1}{\epsilon}$ by proposition~\ref{propArchimedes}. Note that the limit to be shown is 1.\par
        Then $|a_n - 1| = \frac{1}{n}$, so using $n \leq N$ we have $\frac{1}{n} < \epsilon$, as required.
    \item $a_n = \begin{cases}\frac{1}{n} & n \text{ even} \\ 0 & \text{ otherwise}\end{cases}$.\par
        As above, choose $N > \frac{1}{\epsilon}$, so $\frac{1}{n} < \epsilon$ as required.
    \item $a_n = (-1)^n$\par
        Here we want to show divergence, so proceed by contradiction.\par
        Assume $a_n \rightarrow l$. Then use equation~\ref{eqnSequenceLimit} with $\epsilon = 1$. Therefore, $|1-l| < 1$ and $|-1-l| < 1$. Using equation~\ref{eqnTriangleInequality}, $|1 - l + 1 + l| \leq |1 - l| + |1 + l| < 2$, but $1 - l + 1 + l = 2$. This contradiction proves divergence.
\end{examples}
\begin{lemma}
    The limit of a convergent sequence is unique.
    \label{lemUniqueLimit}
\end{lemma}
\begin{proof}
    Assume, on the contrary, that a sequence $a_n$ tends to the distinct real numbers $l$ and $k$.\par
    Then consider $\epsilon = \frac{1}{2} |l - k|$.\par
    From equation~\ref{eqnSequenceLimit},
    \begin{align*}
        &\exists N \in \N \text{ such that } |a_n - l| < \epsilon \forall n \geq N \\
        &\exists M \in \N \text{ such that } |a_n - k| < \epsilon \forall n \geq M
    \end{align*}
    So for any $n \geq \max{\{M, n\}}$, consider $|l - k|$. By the definition of $\epsilon$, this is equal to $2\epsilon$.\par
    Also, $|a_n - k - (a_n - l)| \leq |a_n - k| + |a_n - l| < 2\epsilon$ \contradiction
\end{proof}
\begin{definition}{Bounded sequence}
    A sequence $a_n$ is \underline{bounded} if there exists a real number $B$ such that, for all $n$, $a_n \leq B$.
\end{definition}
Note that any sequence that converges is also bounded. To see this, choose some tolerance $\delta$. Then by the definition of convergence, all the sequence terms after the $N$th are within $\delta$ of the limit $l$. Then we consider the maximum of the finite set:
\begin{equation*}
    \max{\{a_1, a_2, a_3, \cdots, a_{N-1}, l + \delta\}}
\end{equation*}
To see that this is the bound.
\begin{definition}{Monotonic}
    A sequence $a_n$ is \underline{monotonic} if it is either increasing or decreasing. Increasing means that for all $n$, $a_{n+1} \geq a_n$, and decreasing is the reverse.
\end{definition}
\begin{theorem}
    Every bounded monotonic sequence of real numbers converges.
\end{theorem}
\begin{proof}
    Suppose a sequence $a_n$ is increasing and bounded above.\par
    Consider the set $\subsetselect{a_n}{n \in \N}$. It is non-empty and bounded above, so we have a supremum $l$.\par
    For any $\epsilon > 0$, consider $l - \epsilon$ which cannot be an upper bound for the set. Therefore $\exists N \in \N$ such that $a_N > l - \epsilon$.\par
    But also $a_N \leq a_n$ for all $n \geq N$ by assumption that $a_n$ is increasing, so equation~\ref{eqnSequenceLimit} is satisfied.\par
    The same method of proof can be used to show that a decreasing bounded sequence must converge.
\end{proof}
\begin{remarks}
    \item For an increasing sequence to converge, we only need it to be bounded above.
    \item THe bounded condition is necessary - monotonic sequences can (and often do) diverge.
    \item This theorem is actually equivalent to the least upper bound axiom (each implies the other)
    \item It can be shown that every sequence has an infinite monotonic subsequence (see Analysis I)
\end{remarks}
\begin{proposition}
    If $a_n \leq d \forall n$ and $a_n \rightarrow c$ as $n \rightarrow \infty$, then $c \leq d$.
    \label{propLimitLessThanBound}
\end{proposition}
\begin{proof}
    Via real axiom~\ref{realAxiomOrdering}, it suffices to show that $c \not > d$.\par
    Suppose, on the contrary, that we have $c > d$. Let $\epsilon = c - d > 0$.\par
    Then $\exists N \in \N$ such that $\forall n > N, |a_n - c| < \epsilon$.\par
    Note that $a_n \leq d < c$, so $a_n < c$, and $|a_n - c| = c - a_n$. But we therefore have $c - a_n < c - d$, or $a_n > d$\contradiction\par
    Therefore $c \not > d$ so $c \leq d$.
\end{proof}
\begin{remark}
    If $a_n < d \forall n$ (strict inequality in assumption), we still only get $c \leq d$ (non-strict inequality in result).
\end{remark}
\begin{proposition}
    If $a_n \rightarrow c$ and $b_n \rightarrow d$ as $n \rightarrow \infty$, then $a_n + b_n \rightarrow c + d$.
    \label{propSumSequences}
\end{proposition}
\begin{proof}
    Given $\epsilon > 0$, we have:
    \begin{align*}
        &\exists N \in \N \text{ such that } \forall n \geq N, |a_n - c| < \epsilon \\
        &\exists M \in \N \text{ such that } \forall n \geq M, |b_n - d| < \epsilon
    \end{align*}
    Now let $N' = \max{\{N, M\}}$. Then, $\forall n \geq N'$,
    \begin{align*}
        |a_n + b_n - (c + d)| &\leq |a_n - c| + |b_n - d| \text{ by equation~\ref{eqnTriangleInequality}} \\
        &< \epsilon + \epsilon = 2\epsilon
    \end{align*}
    And a new variable $\delta$ can be defined to be $2\epsilon$ for equation~\ref{eqnSequenceLimit} to hold.
\end{proof}
\section{Series}
We have defined the sum of two numbers, so by induction finite sums are also defined. However, we have not defined infinite sums. Nevertheless, we can find examples for which infinite summation converges.\par
\begin{definition}{Partial sum}
    Let $a_n$ be a sequence. Then the $k$th \underline{partial sum} of this sequence is:
    \begin{equation*}
        s_k = \sum_{n=1}^{k} a_n
    \end{equation*}
\end{definition}
\begin{definition}{Infinite sum}
    If the sequence of partial sums of a sequence converges, then we write the limit $\sum_{n=1}^{\infty} a_n$.
\end{definition}
\begin{examples}{}
    \item Consider $a_n = r^n$. When $|r| < 1$, this is a convergent series (``Geometric series''). The limit is $\frac{a_1}{1-r}$
    \item Now let $a_n = \frac{1}{n}$. \textit{The harmonic series}
        \begin{equation*}
            s_{2^k} = 1 + \frac{1}{2} + \frac{1}{3} + \frac{1}{4} + \frac{1}{5} + \frac{1}{6} + \frac{1}{7} + \frac{1}{8} + \cdots + \frac{1}{2^k}
        \end{equation*}
        And by deleting every term whose denominator is not a power of 2,
        \begin{equation*}
            s_{2^k} = 1 + \frac{1}{2} + \frac{1}{4} + \frac{1}{8} + \cdots + \frac{1}{2^k}
        \end{equation*}
        Note that in general, $\frac{1}{2^m + 1} + \frac{1}{2^m + 2} + \cdots + \frac{1}{2^{m+1}} \geq \frac{2^m}{2^{m+1}} = \frac{1}{2}$.\par
        And so $s_{2^k}$ is bounded below by $1 + \frac{k}{2}$, so as $k \rightarrow \infty$, the series diverges.
    \item Let $a_n = \frac{1}{n^2}$. We have the partial sum:\par
        \begin{equation*}
            s_{2^k - 1} = 1 + \frac{1}{2^2} + \frac{1}{3^2} + \frac{1}{4^2} + \frac{1}{5^2} + \frac{1}{6^2} + \frac{1}{7^2} + \cdots + \frac{1}{(2^k - 1)^2}
        \end{equation*}
        And bounding each term by the start of each power-of-two block:
        \begin{equation*}
            s_{2^k - 1} \geq 1 + \frac{1}{2^2} + \frac{1}{2^2} + \frac{1}{4^2} + \frac{1}{4^2} + \frac{1}{4^2} + \frac{1}{4^2} + \cdots + \frac{1}{(2^k - 1)^2}
        \end{equation*}
        And so in general each block looks like: $\frac{1}{(2^m)^2} + \frac{1}{(2^m + 1)^2} + \cdots + \frac{1}{(2^{m + 1} - 1)^2} \leq \frac{2^m}{(2^m)^2} = \frac{1}{2^m}$.
        So $s_k$ is bounded by a geometric series, and converges. In fact, it converges to $\frac{\pi^2}{6}$.
\end{examples}
\subsection{Decimal Expansions}
\begin{definition}{Decimal expansion}
    Let $d_n$ be a sequence of integers 0 to 9. Then the series $\sum_{n=1}^{k} \frac{d_n}{10^n}$ is a \underline{decimal expansion}.
\end{definition}
We can be sure that it converges because it is bounded by a geometric series: $\sum_{n=1}^{k} \frac{d_n}{10^n} \leq \sum_{n=1}^{k} \frac{9}{10^n}$.
If $x = \sum_{n=1}^{\infty} \frac{d_n}{10^n}$, then this is the decimal expansion of $x$.
\begin{proposition}
    Any $x \in \R, 0 \leq x \leq 1$ has a decimal expansion.
    \label{propDecimalExpansion}
\end{proposition}
\begin{proof}
    Select $d_1$ as the largest integer such that $\frac{d_1}{10} \leq x$. Then $d_1$ is between 0 and 9, and also $0 \leq x - \frac{d_1}{10} < \frac{1}{10}$.\par
    Then select $d_2$ as the largest integer such that $\frac{d_2}{100} \leq x - \frac{d_1}{10}$.\par
    Now we get $0 \leq d_2 \leq 9$, and $0 \leq x - \frac{d_1}{10} - \frac{d_2}{100} < \frac{1}{100}$.\par
    Continuing this for $d_k$ as $k \rightarrow \infty$, note that $\frac{1}{10^k} \rightarrow \infty$, so $x - \sum_{n=0}^{k} \frac{d_n}{10^n} \rightarrow 0$, so in the limit we have equality.
\end{proof}
\begin{proposition}
    Decimal expansions are not unique.
    \label{propDecimalExpansionUniqueness}
\end{proposition}
\begin{proof}
    Let $a = \sum_{j=1}^{\infty}\frac{a_j}{10^j}$, $b = \sum_{j=1}^{\infty}\frac{b_j}{10^j}$. Let $a = b$, and suppose $a_j = b_j \forall j < k$, and that for some $k$, $a_j \neq b_j$. After relabelling, let $a_k < b_k$.\par
    \begin{equation*}
        \sum_{j=k+1}^{\infty} \frac{a_j}{10^j} \leq \sum_{j=k+1}^{\infty} \frac{9}{10^j} = \frac{1}{10^k} \text{ by geometric series.}
    \end{equation*}
    \begin{equation*}
        b-a = \frac{b_k}{10^k} - \frac{a_k}{10^k} + \sum_{j=k+1}^{\infty} \frac{b_j}{10^j} - \sum_{j=k+1}^{\infty} \frac{a_j}{10^j}
    \end{equation*}
    And $b = a$, so this expression must be 0.\par
    The largest that $\left|\sum_{j=k+1}^{\infty} \frac{b_j}{10^j} - \sum_{j=k+1}^{\infty} \frac{a_j}{10^j}\right|$ can be is $\frac{1}{10^k}$, which means $b_k - a_k \leq 1$.\par
    $b_k - a_k = 1$ when $\sum_{j=k+1}^{\infty} \frac{b_j}{10^j} - \sum_{j=k+1}^{\infty} \frac{a_j}{10^j} = \frac{-1}{10^k}$, which is when $b_j = 0$ and $a_j = 9~\forall j > k$.
\end{proof}
\begin{remark}
    This means that the only non-unique decimal expansions are those of the form $0.a_2 a_2 \cdots a_k 9999\cdots$ and $0.a_1 a_2 \cdots b_k 0000 \cdots$.
\end{remark}
\begin{definition}{Periodic}
    A decimal expansion is \underline{periodic} if, after a finite number of terms, it repeats in blocks of the same length. That is,
    \begin{equation}
        \exists l, k \text{ such that } \forall n > l, d_{n+k} = d_n
        \label{eqnPeriodicExpansion}
    \end{equation}
\end{definition}
\begin{proposition}
    The decimal expansion of a number $x$ is periodic if and only if $x \in \Q$.
    \label{propDecimalExpansionPeriodic}
\end{proposition}
\begin{proof}
    \begin{proofdirection}{$\Rightarrow$}{Assume $x$ has a periodic decimal expansion, as in equation~\ref{eqnPeriodicExpansion}.}
        Let $x = \sum_{n=1}^{\infty} \frac{d_n}{10^n}$
        \begin{align*}
            10^l x - \sum_{n=1}^{l} 10^{n-l} d_{n+l} &= \sum_{n=l+1}^{\infty}\frac{d_n}{10^{n-l}} \\
            &= \sum_{n=1}^{\infty} \frac{d_{l + (n \text{ mod } k)}}{10^n} \\
            &= \left(\sum_{n=1}^{k}10^{n-l} d_{n+l}\right) \sum_{n=1}^{\infty} \frac{1}{10^{kn}} \\
            &= \left(\sum_{n=1}^{k}10^{n-l} d_{n+l}\right) \frac{1}{10^k} \times \frac{1}{1-\frac{1}{10^k}}
        \end{align*}
        Which is an integer multiplied by two rational numbers, and is therefore rational.\par
        Therefore $10^l x - \sum_{n=1}^{l} 10^{n-l} d_n \in \Q$, and $\sum_{n=1}^{l} 10^{n-l} d_n \in \Z$, so $x \in \Q$.
    \end{proofdirection}
    \begin{proofdirection}{$\Leftarrow$}{Assume $x$ is rational.}
        Let:
        \begin{equation*}
            x = \frac{p}{2^a 5^b q}
        \end{equation*}
        Where $a, b, p, q$ are integers and $a, b, p \geq 0$, and $\hcf{(q, 10)} = 1$.
        \begin{align*}
            \text{ Then } 10^{\max{\{a, b\}}} x &= \frac{t}{q} \\
            &= n + \frac{c}{q}
        \end{align*}
        for some integers $n, c$ and with $0 \leq c < q$. Now we have $q$ coprime with $10$, so we can use theorem~\ref{thmEulerFermat},
        \begin{equation*}
            10^{\varphi(q)} - 1 = kq \text{ for some } k \in \N
        \end{equation*}
        Now:
        \begin{align*}
            \frac{c}{q} &= \frac{kc}{kq} \\
            &= \frac{kc}{10^{\varphi(q)} - 1}
        \end{align*}
        And note that this is in the form of a geometric series:
        \begin{equation*}
            \frac{c}{q} = \sum_{j=1}^{\infty} \frac{1}{10^{\varphi(q) j}}
        \end{equation*}
        But we need to ensure that $kc < 10^{\varphi(q) + 1}$. This is satisfied because $0 \leq kc < kq$, and $kq = 10^{\varphi(q)} - 1$.\par
        And therefore $\frac{c}{q}$ has a periodic decimal expansion.\par
        $x = \frac{1}{10^{\max{\{a, b\}}}} \left(n + \frac{c}{q}\right)$, and so its decimal expansion is also periodic.
    \end{proofdirection}
\end{proof}
\subsection{Euler's Number}
Euler's number, $e$, is defined by the infinite series:
\begin{equation}
    e = 1 + \frac{1}{1!} + \frac{1}{2!} + \frac{1}{3!} + \frac{1}{4!} + \cdots
    \label{eqnESeries}
\end{equation}
We can see by truncating terms that $e > 3$, and also the geometric sequence with ratio $\frac{1}{2}$, which converges to $3$, bounds it from above.
Thus $e$ is well-defined, and we have:
\begin{equation}
    2 < e < 3
    \label{eqnEBounds}
\end{equation}
If we define $0! = 1$, equation~\ref{eqnESeries} becomes:
\begin{equation}
    e = \sum_{n=0}^{\infty} \frac{1}{n!}
    \label{eqnESum}
\end{equation}
\begin{proposition}
    The number $e$ is irrational.
    \label{propEIrrational}
\end{proposition}
\begin{proof}
    Suppose, on the contrary, that there exist integers $p$ and $q$ such that $e = \frac{p}{q}$. Note also that $p \neq 1$, by equation~\ref{eqnEBounds}.\par
    Then $q!e = (q-1)!p$, which is a natural number.\par
    By equation~\ref{eqnESeries},
    \begin{equation*}
        q!e = q! + \frac{q!}{1!} + \frac{q!}{2!} + \cdots + \frac{q!}{(q-1)!} + 1 + \frac{1}{q+1} + \frac{1}{(q + 1)(q + 2)} + \cdots
    \end{equation*}
    Define the terms before, and including, 1 to be $N \in \N$, and the terms after (that are not integers) to be $x$. That is, $q!e = N + x$.\par
    We can bound $x$ by a geometric series:
    \begin{equation*}
        x \leq \sum_{n=1}^{\infty} \frac{1}{(q+1)^n} = \frac{1}{q}
    \end{equation*}
    But since $q \geq 1, \frac{1}{q} \leq 1$, and so $x < \frac{1}{q}$ cannot be an integer.\par
    However, $q!e$ is an integer and so is $N$, but $q!e = N + x$ \contradiction\par
    Therefore $q!e$ is not an integer, and $e$ is irrational.
\end{proof}
\section{Polynomials}
\begin{definition}{Polynomial}
    A \underline{polynomial} is a function $P(x)$, which has the form:
    \begin{equation}
        P(x) = a_d x^d + a_{d-1} x^{d-1} + \cdots + a_1 x + a_0
        \label{eqnPolynomial}
    \end{equation}
    Where $d$ is the \underline{degree} of the polynomial, and $a_d \neq 0$.
\end{definition}
\begin{definition}{Root}
    A \underline{root} of the polynomial $P(x)$ is a number $x_0$ that satisfies $P(x_0) = 0$.
\end{definition}
\begin{definition}{Algebraic number}
    A real number $x$ is \underline{algebraic} if it is a root of a polynomial that has integer coefficients.
\end{definition}
The opposite of an algebraic number is a \underline{transcendental number}. For a long time, it was not known whether transcendental numbers existed. Also note that, at the time these terms were defined, algebra related only to the study of polynomials.
\begin{lemma}
    For any polynomial $P(x)$, there exists a real number $K$ such that $|P(x) - P(y)| \leq K|x - y|$ for $0 \leq x, y \leq 1$.
    \label{lemPolynomialBounds}
\end{lemma}
\begin{proof}
    From equation~\ref{eqnPolynomial}, we get:
    \begin{align*}
        &|P(x) - P(y)| = |a_n(a^n - y^n) + a_{n-1}(x^{n-1} - y^{n-1}) + \cdots + a_1(x-y)| \\
        &= |x - y| |a_n(x^{n-1} + x^{n-2} y + \cdots + y^{n-1}) + a_{n-1} (\cdots) + \cdots + a_1| \\
        &\leq |x - y| \left[d(|a_n| + |a_{n-1}| + \cdots + |a_1|)\right]
    \end{align*}
    So we have a value $K$ independent of $x$ and $y$.
\end{proof}
\begin{theorem}[Fundamental Theorem of Algebra]
    A non-constant polynomial of degree $d$ has at most $d$ real roots.
    \label{thmFundamentalAlgebra}
\end{theorem}
\begin{proof}
    \induction{$d=1$}{
        A polynomial of degree 1 is of the form $ax + b$ and has exactly one real root $-\frac{b}{a}$.
    }
    {$d \leq k - 1$}{}
    {$d = k$}{
        If this polynomial has no roots then we are done.\par
        If not, we have a root $r$. Then we can factor $P(x) = (x - r)Q(x)$, where $Q(x)$ is a polynomial of degree $k - 1$, so has at most $k - 1$ roots by assumption. Therefore $P(x)$ has at most $k$ roots.
    }
\end{proof}
The first number that was proven to be transcendental was Liouville's constant.
\begin{theorem}{Existence of transcendental numbers}
    The number $L = \sum_{n=1}^{\infty} \frac{1}{10^{n!}}$, Liouville's constant, is transcendental.
\end{theorem}
\begin{proof}
    Let the $n$th partial sum of $L$ be $L_n = \sum_{k=1}^{\infty} \frac{1}{10^{k!}}$.\par
    Note that $0 < L_n \leq L < 1$, and that as $n \rightarrow \infty, L_n \rightarrow L$.\par
    Now suppose that there exists a polynomial with integer coefficients for which $L$ is a root. By lemma~\ref{lemPolynomialBounds}, we have a real number $K$ such that:
    \begin{equation*}
        |P(x) - P(y)| \leq K|x - y|
    \end{equation*}
    and note that:
    \begin{equation*}
        |L - L_n| = \sum_{k=n+1}^{\infty} \frac{1}{10^{k!}} \leq \frac{2}{10^{(n+1)!}}
    \end{equation*}
    due to the very fast convergence of the terms of $L$.\par
    Suppose also that $P(x)$ has degree $d$. Also, note that $L_n = \frac{s}{10^{n!}}$ for some integer $s$ by putting all terms over a common denominator, and $P(L_n) = \frac{t}{10^{d(n!)}}$ for some integer $t$ by the same method.\par
    Then we can choose $n$ sufficiently large so that $L_n$ is not a root of $P(x)$, because by theorem~\ref{thmFundamentalAlgebra} there are only finitely many roots.\par
    We have $|P(L_n) - P(L)| = |P(L_n)| \geq \frac{1}{10^{d(n!)}}$, and therefore:
    \begin{align*}
        \frac{1}{10^{d(n!)}} &\leq K|L-L_n| \\
        &\leq K \times \frac{2}{10^{(n+1)!}} \\
        \frac{1}{10^d} &\leq 2K \times \frac{1}{10^{n+1}}
    \end{align*}
    And as $n \rightarrow \infty$, this inequality does not hold.\contradiction\par
    $L$ cannot be a root of any polynomial with integer coefficients.
\end{proof}
\begin{remarks}
    \item This is Liouville's constant, and other numbers with the property that $0 < |x - \frac{p}{q}| < \frac{1}{q^n}$ are Liouville numbers, and these are numbers that are very well-approximated by a rational number. The proof above can work for any Liouville number
    \item The above proof does not work for other constants like $\pi$ and $e$, which are transcendental.
\end{remarks}
\end{document}