\documentclass[../Main.tex]{subfiles}

\begin{document}
\section{The Problem With the Rational Numbers}
We have so far seen $\mathbb{N} \subset \mathbb{Z} \subset \mathbb{Q}$.
\begin{proposition}
    There is no rational number $x$ that satisfies $x^2 = 2$. \label{propIrrationalRootTwo}
\end{proposition}
\begin{proof}
    Assume $x>0$ without loss of generality as $x^2 = (-x)^2$.\par
    On the contrary, assume that $x \in \mathbb{Q}$, let $a = \frac{a}{b}$, $a, b \in \mathbb{N}$.\par
    $x^2 = \frac{a^2}{b^2} = 2$\par
    $\therefore a^2 = 2b^2$
    By Theorem~\ref{thmFundamentalArith}, the prime factorisation of any number is unique. The LHS of the equation, however, has an even exponent of 2, but the RHS has an odd exponent of 2. \contradiction
\end{proof}
\begin{corollary}
    If $\exists x \in \mathbb{Q}$, with $x^2=n$ for some $n \in \mathbb{N}$, then $n$ is a square.
\end{corollary}
An alternative proof of \ref{propIrrationalRootTwo}:
\begin{proof}
    Suppose on the contrary that $x^2 = 2$ for $x = \frac{a}{b}$, $a, b \in \mathbb{N}$. Then for any $c, d \in \mathbb{N}$, $cx+d$ is of the form $\frac{e}{b}$, for some $e \in \mathbb{Z}$.\par
    Thus if $cx+d>0$, then $cx+d \geq 1/b$\par
    We now use the fact that $1<x<2 \implies 0<x-1<1$.\par
    Therefore, for sufficiently large $n$, consider $(x-1)^n$.\par
    Then we have $0<(x-1)^n<1/b$.\par
    For any $n$, $(x-1)^n$ is of the form $cx+d$ because we reduce all the higher order terms using $x^2 = 2$.\par
    Thus we have $cx+d > 0$ but $cx+d<1/b$ \contradiction
\end{proof}
\underline{Remark}: This expresses the idea that $\mathbb{Q}$ has "gaps", where other numbers can exist. \par
To illustrate this, consider the set $\{x:x^2<2\}$. This clearly has an upper bound: from the rational numbers we can choose $2$, $1.5$, $1.42$, ...\par
Then $\sqrt{2}$ is the least upper bound of this set, but in the rational numbers there is no such least upper bound.
\section{Defining the Real Numbers}
Some preliminary definitions are needed:
\begin{definition}{Bounded above}
    A set $S$ is \underline{bounded above} if there exists a number $x$ such that $x\geq y \forall y \in S$. Such an $x$ is called an upper bound.
\end{definition}
\begin{definition}{Least upper bound}
    A number $x$ is the \underline{least upper bound} if it is an upper bound and every other bound is larger than it. Also known as the supremum of a set.
\end{definition}
Note that the least upper bound is unique.
\begin{definition}{Real numbers}
    Written $\mathbb{R}$, the \underline{real numbers} are a set with elements $0$ and $1$, $0\neq 1$, equipped with operators $+$ and $\times$, and an ordering $<$, that satisfies:
    \begin{enumerate}
        \item $+$ is commutative and associative, with identity $0$, and every real number has an inverse under $+$;\label{realAxiomAddition}
        \item $\times$ is commutative and associative, with identity $1$, and every real number has an inverse under $\times$;\label{realAxiomMultn}
        \item $\times$ distributes over $+$: for $a, b \in \mathbb{R}$, $a\times(b+c)=a\times b + a \times c$;\label{realAxiomDist}
        \item $\forall a, b \in \mathbb{R}$, exactly one of $a<b$, $a=b$, $a>b$ holds; $\forall a, b, c \in \mathbb{R}$, if $a<b$ and $b<c$, then $a<c$;\label{realAxiomOrdering}
        \item $\forall a, b, c \in \mathbb{R}$, $a<b \implies a+c<b+c$;\label{realAxiomAddOrder}
        \item $\forall a, b, c \in \mathbb{R}$, $a<b \implies a\times c<b\times c$ if $c>0$;\label{realAxiomMultOrder}
        \item Given any set $S$ of real number that is non-empty and bounded above, $S$ has a least upper bound.\label{realAxiomLUB}
    \end{enumerate}
\end{definition}
\begin{remarks}
    \item From the axioms, we can check, for example, that $0<1$. We have already $0 \neq 1$, so by axiom~\ref{realAxiomOrdering}, the only other possibility is $0>1$. Assume this for contradiction. Note that this also implies $-1 > 0$ by axiom~\ref{realAxiomAddOrder}.
    \begin{align*}
        1 &< 0 \\
        1-1 &< 0-1 \text{ by (\ref{realAxiomAddOrder})}\\
        0 &< (-1) \\
        (0) \times (-1) &< (-1) \times (-1) \text{ which is valid under (\ref{realAxiomMultOrder}) as $-1 > 0$} \\
        0 &< 1.
    \end{align*} \contradiction
    \item We may have $\mathbb{Q} \subset \mathbb{R}$ by considering the generic $\frac{a}{b} \in \mathbb{Q}$ with $ab^{-1} \in \mathbb{R}$.
    \item  $\mathbb{Q}$ does not satisfy the least upper bound axiom as shown in the previous section.
    \item The wording of the upper bound axiom is important: if S were empty then any real number is an upper bound so there is no least such. If S is not bounded above then no number is an upper bound and there cannot be a supremum.
    \item It is possible to construct $\mathbb{R}$ from $\mathbb{Q}$, and thus the axioms above arise.
\end{remarks}
\subsection{Suprema}
For a set $S$ bounded above and defined by an inclusive upper bound, it follows immediately that the greatest upper bound of such a set is $\max(S)$. However, if the set does not trivially define a $\max(S)$ then the least upper bound is a very useful concept as it can be used instead of $\max(S)$. We then want a way of defining a least upper bound for such sets.
\begin{proposition}[Proposition of Archimedes]
    The set of natural numbers is not bounded above in $\mathbb{R}$.
    \label{propArchimedes}
\end{proposition}
\begin{proof}
    Suppose, on the contrary, that $\mathbb{N}$ is bounded above. $\mathbb{N} \neq \emptyset$, so $\exists c = \sup(\mathbb{N})$.\par
    We must have $c - 1 \neq \sup(\mathbb{N})$, so $\exists n \in \mathbb{N}$ such that $n > c - 1$.\par
    But then $n+1 \in \mathbb{N}$, and $n + 1 > c$ \contradiction~as c is the supremum.
\end{proof}
\begin{corollary}
    For any $t \in \mathbb{R}^+$, $\exists n \in \mathbb{N}$ with $\frac{1}{n} < t$.\par
    \label{corollNaturalReciprocal}
\end{corollary}
\begin{proof}
    Let $s = \frac{1}{t}$. By Proposition~\ref{propArchimedes}, $\exists n \in \mathbb{N}$ with $n > s$\par
    $\therefore \frac{1}{n} < \frac{1}{s} \Leftrightarrow \frac{1}{n} < t$
\end{proof}
\subsection{Infima}
\begin{definition}{Bounded below}
    A set $S$ is \underline{bounded below} if there exists $x$ such that $x \leq y \forall y \in S$. Such an $x$ is a \underline{lower bound}.
\end{definition}
If $S \neq \emptyset$ and is bounded below, then define $-S = \{-s | s \in S\}$. Now we have $-S \neq \emptyset$ and bounded above, so there exists a supremum. Hence the infimum (greatest lower bound) of $S$ is $-\sup(-S)$. This is denoted $\inf{S}$.\par
Corollary~\ref{corollNaturalReciprocal} then gives us a method of finding infima, and the above statement allows us to relate sets bounded above to those bounded below.
\section{Closing the Gaps in the Rationals}
\begin{theorem}
    There exists a rational number $x$ with $x^2 = 2$.
\end{theorem}
\begin{proof}
    Let $S = \{x \in \mathbb{R} | x^2 < 2\}$.\par
    Then $1 \in S$ so $S$ is non-empty, and all elements of $S$ are less than $2$ so $S$ is also bounded above.\par
    Thus $\sup(S)$ exists by axiom~\ref{realAxiomLUB}.\par
    Let $c = \sup(S)$.\par
    Observe that $1 < c < 2$ by above.\par
    By axiom~\ref{realAxiomOrdering}, $c^2$ is either equal to, greater than or less than 2.
    % TODO: include a sub-proof environment.
    Consider $c^2 < 2$.\par
    Let $t \in \mathbb{R} : 0 < t < 1$.
    \begin{align*}
        (c+t)^2 &= c^2 + 2ct + t^2 \\
        &< c^2 + 4t + t \text{ as} c < 2 \text{ and } t^2 < t \\
        &< c^2 + 5t
    \end{align*}
    Thus choose $t \leq \frac{2-c^2}{5}$ so we have $c^2 + 5t \leq 2$.\par
    But now we have $(c+t)^2 < 2$ and so $(c+t) \in S$.\contradiction~c cannot be an upper bound, so $c^2 \nless 2$\par
    Then consider $c^2 > 2$:
    Let $t \in \mathbb{R} : 0 < t < 1$\par
    and then $(c - t)^2 = c^2 - 2ct + t^2 > c^2 - 4t$.\par
    So if $t \leq \frac{c^2 - 2}{4}$, we get $c^2 - 4t \geq 2$ so it is an upper bound. \contradiction~as c can no longer be the least upper bound since $(c-t) < c$ is an upper bound, so $c^2 \ngtr 2$.\par
    Thus we are left with only $c^2 = 2$
\end{proof}
\underline{Remark}: The same format can be used to show $\sqrt[n]{x}~\forall n \in \mathbb{N}, x \in \mathbb{R}^+$.\par
\begin{definition}{Irrational Number}
    A real number that is not rational is \underline{irrational}.
\end{definition}
\begin{proposition}
    The rational numbers are \underline{dense} in $\mathbb{R}$, that is, for any two real numbers $a$ and $b$ with $a<b$, there exists $c \in \mathbb{Q}$ with $a<c<b$.
    \label{propDenseRationals}
\end{proposition}
\begin{proof}
    We assume $a\geq 0$ without loss of generality, since the argument can be used for $a$ translated along the real line by another real number.\par
    By Corollary~\ref{corollNaturalReciprocal}, $\exists n \in \mathbb{N}$ with $\frac{1}{n} < b - a$. Let $T = \{k \in \mathbb{N} | \frac{k}{n} \geq b\}$.\par
    By Proposition~\ref{propArchimedes}, $\exists N \in \mathbb{N}$ s.t. $N > b$.\par
    So $Nn \in T$, so $T \neq \emptyset$\par
    By Well-Ordering Principle, $T$ has a minimal element $m$. Set $c=(m-1)\frac{1}{n}$.\par
    Since $m-1 \notin T$, we have $c < b$ by axiom~\ref{realAxiomOrdering}.
    To prove $c > a$, we assume $c \leq a$ for contradiction. Then $\frac{m}{n}=c + \frac{1}{n} < a + (b-a)$.\par
    Then $\frac{m}{n} < b$ \contradiction\par
    Then $a<c<b$ as required.
\end{proof}
\begin{proposition}
    The irrational numbers are also dense in $\mathbb{R}$.
    \label{propDenseIrrationals}
\end{proposition}
\begin{proof}
    Using Proposition~\ref{propDenseRationals}, find that $a\sqrt{2} < c < b\sqrt{2}$, then divide through to get $a < \frac{c}{\sqrt{2}} < b$, where clearly $\frac{c}{\sqrt{2}}$ is an irrational number.
\end{proof}
\end{document}