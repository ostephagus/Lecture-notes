\documentclass[../Main.tex]{subfiles}

\begin{document}
We have a vague intuition for the sizes of infinite sets. We understand that $\mathbb{N}$ seems a lot smaller than $\mathbb{Z}$ or $\mathbb{Q}$, or especially $\mathbb{R}$. However, we would like to be able to formally define this notion.
\section{Countable sets}
\begin{definition}{Countability}
    A set $X$ is \underline{countable} if $X$ is finite or there exists a bijection $X \mapsto \mathbb{Z}$.
\end{definition}
\begin{example}
    $\Z$ is a countable set. We can define a bijection:
    \begin{align*}
        \N &\mapsto \Z \\
        n &\mapsto \frac{n}{2} \text{ if n even}\\
        n &\mapsto -\frac{n-1}{2} \text{ if n odd}
    \end{align*}
\end{example}
\begin{lemma}
    Any subset of $\N$ is countable.
    \label{lemCountableNaturalSubset}
\end{lemma}
\begin{proof}
    If $S \subseteq \N$ is non-empty, by the well-ordering principle there is a least element $s_1 \in S$. If $S \backslash \{s_1\} \neq \emptyset$, then by the well-ordering principle there is a least element $s_2$, and we use the same process.\par
    If this process terminates, then at some point we have the greatest element of the set $s_m$, and so it is finite and countable.\par
    If this process does not terminate, we define the function:\par
    \begin{align*}
        g : \N &\mapsto S \\
        n &\mapsto s_n
    \end{align*}
    It is well-defined (for every $n$, there is a unique $s_n$), and is injective.\par
    It is also surjective since if $k \in S$, then $k \in \N$ and there are $<k$ elements of $S$ that are less than $k$, so $k = s_m$ for some $m \leq k$. So every element $k$ in $S$ has a corresponding input. We thus have a bijection with the natural numbers.
\end{proof}
\begin{theorem}
    \label{thmCountStatements}
    The following are equivalent:\par
    \begin{enumerate}
        \item The set $X$ is countable \label{stmtCountable}
        \item there is an injection $X \mapsto \N$ \label{stmtInjection}
        \item $X$ is empty, or there exists a surjection $\N \mapsto X$ \label{stmtSurjection}
    \end{enumerate}
\end{theorem}
\begin{proof}
    If $X$ is countable, it is either finite (and will inject into $\N$), or there exists a bijection $X \mapsto \N$ so there is an injection. Statement~\ref{stmtCountable} implies statement~\ref{stmtInjection}.\par
    Now suppose that there exists an injection $f : X \mapsto \N$, and we can define a new range to be the image of $f$, label this $S$, and so $f : X \mapsto S$ is a bijection. If $S$ is finite, then so is $X$. However, if $S$ is infinite, then we have a subset of the natural numbers. By Lemma~\ref{lemCountableNaturalSubset}, there is a bijection $g : S \mapsto \N$, so the composition $gf : X \mapsto \N$ is a bijection, so $X$ is countable. Statement~\ref{stmtInjection} implies statement~\ref{stmtCountable}.\par
    Now assume $X$ is countable. That is, $X$ is finite or there is a bijection $X \mapsto \N$. Now if $X$ is finite and non-empty, we can define an injection from $X$ to the first $|X|$ natural numbers. If $X$ is not finite, then we have a bijection $X \mapsto \N$, so we certainly have an injection $N \mapsto X$.\par
    Suppose that $X \neq \emptyset$ and that there is a surjection $f : \N \mapsto X$. We define:
    \begin{align*}
        g : X &\mapsto \N \\
        a &\mapsto \text{the least element with } f(b) = a
    \end{align*}
    So by construction we have an injective map $g : X \mapsto \N$. Statement~\ref{stmtSurjection} implies statement~\ref{stmtInjection}.
\end{proof}
\begin{corollary}
    Any subset of a countable set is itself countable.
    \label{corCountableSubet}
\end{corollary}
\begin{proof}
    If we have a subset $Y \subseteq X$, and $X$ is countable, then consider the injection $X \mapsto \N$ which exists by theorem~\ref{thmCountStatements}, and restrict this to the injection $g : Y \mapsto \N$, then $Y$ is countable by theorem~\ref{thmCountStatements}.
\end{proof}
\begin{theorem}
    $N \times N$ is countable.
    \label{thmDoubleNaturalCountable}
\end{theorem}
Two proofs will be provided.
\begin{proof}[diagonal argument]
    Considering the set of pairs of natural numbers, we count them as follows:\par
    \begin{equation*}
        (1, 1), (2, 1), (1, 2), (1, 3), (2, 2), (3, 1), \cdots
    \end{equation*}
    This counts the natural numbers diagonally. We then proceed inductively.\par
    Define the sequence $\left(a_n\right)^\infty_{n=1}$ by $a_1 = (1, 1)$, and $a_n = (p-1, q+1)$ if $p \neq 1$, or $a_n = (q+1, 1)$ if $p = 1$.
    Then this list includes every point $(x, y) \in \N \times \N$.
\end{proof}
\begin{proof}[defining an injection]
    Define:
    \begin{align*}
        f : \N \times \N &\mapsto \N \\
        (x, y) &\mapsto 2^x \times 3^y
    \end{align*}
    Then we have a unique factorisation by theorem~\ref{thmFundamentalArith}, so the map $f$ is injective. By theorem~\ref{thmCountStatements}, we must have that $\N \times \N$ is countable.
\end{proof}
\begin{corollary}
    $\Z \times \Z$ is countable.
    \label{corDoubleIntCountable}
\end{corollary}
\begin{proof}
    Since $\Z$ is countable, there is an injection $f : \Z \mapsto \N$, and since $\N \times \N$ is countable, there is an injection $g : \N \times \N \mapsto \N$, so the composition:
    \begin{equation*}
        g(f, f) : \Z \times \Z \mapsto \N \times \N \mapsto \N
    \end{equation*}
    is an injection. So by theorem~\ref{thmCountStatements}, $\Z \times \Z$ is countable.
\end{proof}
\begin{theorem}
    A countable union of countable sets is itself countable
    \label{thmCountableUnion}
\end{theorem}
\begin{proof}
    Assume that the countable sets are indexed by the natural numbers. So we label the countable sets: $A_1, A_2, \cdots$. We wish to show that the union: 
    \begin{equation*}
        \bigcup_{n \in \N} A_n
    \end{equation*}
    is countable.\par
    For each integer $i$, since $A_i$ is countable, we may list its elements $a_1^{(i)}, a_2^{(i)}, \cdots$. Note that this list may or may not terminate.\par
    We then define a function:
    \begin{align*}
        f: \bigcup_{n\in\N} A_n &\mapsto \N \\
        x &\mapsto 2^i \times 3^j
    \end{align*}
    Where $x = a_j^{(i)}$. But we must be careful of duplicates: by the well-ordering principle we choose the least $i$ such that $x \in A_i$.\par
    Then we have an injection to the natural numbers, so we must have that this union is countable.
\end{proof}
\begin{corollary}
    \label{corQCountable}
    The set of rational numbers is countable.
\end{corollary}
\begin{proof}
    Let the set of rational numbers be defined:
    \begin{equation*}
        \Q = \bigcup_{n \in \N} \left\{\frac{m}{n}~|~m \in \Z\right\}
    \end{equation*}
    Then this is a countable union (it is indexed by natural numbers) of countable sets (they are indexed by the integers), so $\Q$ is countable.
\end{proof}
\begin{theorem}
    \label{thmAlgebraicCountable}
    The set of algebraic numbers, $\A$, is countable.
\end{theorem}
\begin{proof}
    We can define a bijection between the set of polynomials with integer coefficients of degree $d$ and the set $\Z^{d+1}$, by mapping each coefficient to an integer in the $(d+1)$-tuple that is an element of $\Z^{d+1}$. Thus, the set of polynomials of degree $d$ is countable by theorem~\ref{thmCountableUnion}.\par
    Then the set of polynomials with integer coefficients is the union of polynomials of degree $d$, where $d \in \N$, so the set of polynomials is a contable union of countable sets, and by theorem~\ref{thmCountableUnion} is coubtable.\par
    Each such polynomial has a finite number of roots (at most $d$), so the set of roots of polynomials with integer coefficients, $\A$, must be countable.
\end{proof}
\section{Uncountable sets}
The first uncountable set we met was $R$.
\begin{theorem}
    $\R$ is uncountable.
    \label{thmRUncountable}
\end{theorem}
\begin{proof}
    Assume, on the contrary, that $\R$ is countable. That is, we can list the elements of $\R$, indexed by $\N$:
    \begin{equation*}
        r_1, r_2, r_3, \cdots
    \end{equation*}
    Then we write their decimal expansion:
    \begin{align*}
        &r_1 = n_1 \cdot d_{11} d_{12} d_{13} \cdots \\
        &r_2 = n_2 \cdot d_{21} d_{22} d_{23} \cdots \\
        &r_3 = n_3 \cdot d_{31} d_{32} d_{33} \cdots \\
        &\vdots
    \end{align*}
    Now define a new real number $r$ by its decimal expansion $0\cdot d_1 d_2 d_3 \cdots$, where:
    \begin{equation*}
        d_n=
        \begin{cases}
            1 & d_{nn} \neq 1 \\
            2 & d_{nn} = 1
        \end{cases}
    \end{equation*}
    Note that this number has a unique decimal expansion (since the digits are neither all 9s nor all 0s), and by construction is not one of the $r_i$ listed above.\contradiction\par
    $\R$ is not countable
\end{proof}
\begin{remark}
    The whole number parts $n_i$ were not used, so the same proof actually shows that the interval $(0,1)$ is uncountable.
\end{remark}
\begin{corollary}
    \label{corTranscUncountable}
    There are uncountably many transcendental numbers.
\end{corollary}
\begin{proof}
    Assume on the contrary that the set of transcendental numbers, $\R \backslash \A$, is countable.
    Then the union $(\R \backslash \A) \cup \A$ is a union of two countable sets, and so must be countable. But this union is $\R$, which by theorem~\ref{thmRUncountable} is uncountable. \contradiction\par
    So the set of transcendental numbers cannot be countable.
\end{proof}
\begin{theorem}
    The power set $\powerset{\N}$ is uncountable.
    \label{thmPowerSetNUncountable}
\end{theorem}
Two proofs are provided, by diagonal argument and by injections.
\begin{proof}[diagonal argument]
    Suppose, on the contrary, that $\powerset{\N}$ is countable.\par
    Then $\powerset{\N} = \{S_1, S_2, \cdots\}$.
    Define $S \subseteq \N$, so $S \in \powerset{\N}$, as:
    \begin{equation*}
        S = \subsetselect{n \in \N}{n \notin S_n}
    \end{equation*}
    Then $S \neq S_i~\forall i \in \N$ since the sets always differ on the element $i$.\par
    Thus we have constructed a subset of the natural numbers not in the list.\contradiction\par
    $\powerset{\N}$ cannot be countable.
\end{proof}
\begin{proof}[injections with $\N$]
    Consider the function $f : (0, 1) \mapsto \powerset{\N}$.\par
    We consider the binary expansion (like decimal expansion but in base 2) of $x \in (0, 1)$:
    \begin{equation*}
        0 \cdot x_1 x_2 x_3 \cdots \text{ where } x_i \in \{0, 1\}
    \end{equation*}
    Note that we do not include the expansion with all $1$s.\par
    Now we define $f(x) = \subsetselect{n \in \N}{x_n = 1}$. The function $f$ therefore selects natural numbers for the subset $f(x)$ based on whether the corresponding digit in the binary expansion is 1.\par
    This is an injection of $(0, 1) \mapsto \powerset{\N}$, so $\powerset{\N}$ must be uncountable.
\end{proof}
A wider statement about power sets is encompassed by the following theorem:
\begin{theorem}
    For any set $X$, there is no bijection between $X$ and its power set $\powerset{X}$.
    \label{thmNoPowerSetBijections}
\end{theorem}
\begin{proof}
    Consider the function $f : X \mapsto \powerset{X}$.\par
    Define $S = \{x \in X : x \notin f(x)\}$.
    Thus we have a subset of $X$, and so $S \in \powerset{X}$, but that is not in the image of $f$ (since any element of $im(f)$ differs by the element $x$).\par
    Thus $f$ is not surjective, so it is not a bijection. The set $S$ was defined for any map $f$, so there can not exist a bijection of the form $f$.
\end{proof}
\begin{remarks}
    \item This is reminiscent of Russel's Paradox, but is permissible since we define $S$ as a subset of $X$.
    \item This proof can be used to show that there is no universal set. If we suppose $V$ is the universal set, then $\powerset{V} \in V$, but there is no surjection $V \mapsto \powerset{V}$.
\end{remarks}
\begin{lemma}
    Let $\subsetselect{A_i}{i \in I}$ be a family of open, pairwise-disjoint intervals in $\R$ indexed by the index set $I$. Then $I$ must be countable.
\end{lemma}
Two similar proofs are offered:
\begin{proof}[Bijections with $\Q$]
    Each interval $A_i$ must contain a rational number by proposition~\ref{propDenseRationals}. Thus, by the well-ordering principle we can choose the least such, and let this represent the $A_i$. Thus we must have a bijection between $I$ and a subgroup of $\Q$, which by corollary~\ref{corCountableSubet} is countable.\par
    Thus we have a bijection from $I$ to a countable set, so $I$ is countable.
\end{proof}
\begin{proof}[Countable union of countable sets]
    The set $\subsetselect{i \in I}{A_i \text{ has length } \geq 1}$ can be indexed by $\Z$, and so must be countable.\par
    The set $\subsetselect{i \in I}{A_i \text{ has length } \geq \frac{1}{2}}$ can be indexed by $\frac{1}{2}\Z$. Continuing this process ad infinitum, we see that $I$ is a countable union of countable sets, and is countable by theorem~\ref{thmCountableUnion}.
\end{proof}
We now have some tools to show countability:
\begin{itemize}
    \item Listing its elements and showing they can be indexed by $\N$ (or another countable set)
    \item Show an injection of the set into $\N$
    \item Use theorem~\ref{thmCountableUnion} if it can be seen as a countable union of countable sets
    \item Consider $\Q$ if the set looks like a subset of $\R$.
\end{itemize}
And some tools to show uncountability:
\begin{itemize}
    \item Use a diagonal argument and reach a contradiction
    \item Show there exists an injection from an uncountable set to the set being studied
\end{itemize}
\subsection{Functions and Sizes of Sets}
We have now the intuition that a bijection between two sets implies they are the same \textit{size}. Similarly, an injection $X \mapsto Y$ implies that the \textit{size} of $X$ is at most as big as $Y$, and a surjection $X \mapsto Y$ implies that the \textit{size} of $X$ is at least as big as $Y$.\par
For these to make sense, we need that if $X$ is at most as big as $Y$, we can say the converse ($Y$ is at least as big as $X$).
\begin{lemma}
    Consider two non-empty sets $A$ and $B$.\par
    If there exists an injection $f : A \mapsto B$, then this is equivalent to there existing a surjection $g : B \mapsto A$.
\end{lemma}
\begin{proof}
    \begin{proofdirection}{$\Rightarrow$}{Suppose $f : A \mapsto B$ is an injection}
        Now we are able to define $g$ in a similar way to how we defined inverses in the proof of propositions~\ref{propInjectiveIffLeftInverse} and \ref{propSurjectiveIffRightInverse}\par
        Now we can define, for any random point $a_0$ in $A$:
        \begin{align*}
            g : B &\mapsto A \\
            b &\mapsto \begin{cases}
                \text{the unique } a \in A \text{ such that } f(a) = b & \text{if it exists} \\
                a_0 & \text{otherwise}
            \end{cases}
        \end{align*}
        And we have a well-defined surjective map $g$.
    \end{proofdirection}
    \begin{proofdirection}{$\Leftarrow$}{Suppose $g : B \mapsto A$ is a surjection}
        Let $f : A \mapsto B$ be defined by choosing any of the points in $B$ that have $g(b) = a$ to be $f(a)$. Then $f$ is a well-defined injection.
    \end{proofdirection}
\end{proof}
We also need that if $A$ is at most as big as $B$, and $B$ is at most as big as $B$, then $A$ and $B$ have the same size.
\begin{theorem}[Schr\"oder-Bernstein Theorem]
    \label{thmSchroderBernstein}
    If $f : A \mapsto B$ and $g : B \mapsto A$ are injections, then there exists a bijection $h : A \mapsto B$.
\end{theorem}
\begin{proof}
    For an element $a \in A$, write $g^{-1}(a)$ for the point $b \in B$, if it exists, such that $g(b) = a$. Similarly, write $f^{-1}(b)$ for the $a \in A$, if it exists, such that $f(a) = b$.\par
    We call the sequence of elements 
    \begin{equation*}
        \{g^{-1}(a), f^{-1}(g^{-1}(a)), g^{-1}(f^{-1}(g^{-1}(a))), \cdots\}
    \end{equation*}
    the \underline{ancestor sequence} of $a \in A$. The sequence will terminate or repeat. We also define the ancestor sequence of $b \in B$.\par
    We can then let:
    \begin{align*}
        A_0 &= \subsetselect{a \in A}{\text{ancestor sequence terminates after an even number of terms}} \\
        A_1 &= \subsetselect{a \in A}{\text{ancestor sequence terminates after an odd number of terms}} \\
        A_\infty &= \subsetselect{a \in A}{\text{ancestor sequence does not terminate}}
    \end{align*}
    And define $B_0, B_1, B_\infty$ similarly. We now have a partition of both sets into 3 subsets.\par
    Note that $f$ gives a bijection $A_0 \mapsto B_1$, since everything in the set $B_1$ has at least one ancestor, and this ancestor must have an even number of ancestors (exactly the set $A_0$). The same argument shows that $g$ then provides a bijection from $B_0 \mapsto A_1$.\par
    We can use either $f$ or $g$ as the bijection $A_\infty \mapsto B_\infty$\par
    Then we define the function $h$:
    \begin{align*}
        h : A &\mapsto B \\
        a &\mapsto
        \begin{cases}
            f(a) & a \in A_0 \\
            g^{-1}(a) & a \in A_1 \\
            f(a) & a \in A_\infty
        \end{cases}
    \end{align*}
\end{proof}
We can now show how this theorem can be used:
\begin{example}
    Is there a bijection from the interval $[0, 1] \mapsto [0, 1] \cup [2, 3]$?\par
    Observe that we clearly have an injection:
    \begin{align*}
        f : [0, 1] &\mapsto [0, 1] \cup [2, 3] \\
        x &\mapsto x
    \end{align*}
    But we can also define an injection the other way:
    \begin{align*}
        g : [0, 1] \cup [2, 3] &\mapsto [0, 1] \\
        x &\mapsto \frac{x}{3}
    \end{align*}
    And so by theorem~\ref{thmSchroderBernstein} we have a bijection, as required.
\end{example}
\begin{remarks}
    \item It would be nice to say that \textit{For any two sets $A$ and $B$, either $A$ injects into $B$ or $B$ injects into $A$}. This is indeed true but the proof is not seen until Part II Logic and Set Theory.
    \item The continuum hypothesis says that any subset of the real numbers is either finite, countably infinite (in bijetion with $\N$), or uncountable (in bijection with $\R$). That is, there does not exist a set whose size lies between those of $\N$ and $\R$.
\end{remarks}
\end{document}