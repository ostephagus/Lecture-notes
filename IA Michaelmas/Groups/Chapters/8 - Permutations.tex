\documentclass[../Main.tex]{subfiles}

\begin{document}
\section{Permutations}
\begin{definition}{Permutation}
    A \underline{permutation} of a set $X$ is a bijection $X \mapsto X$.
\end{definition}
The group of all such permutations is, as we have already seen, $\Sym(X)$.
\subsection{Representing Permutations as Lists}
Any list of $k$ distinct elements determines a $k$ cycle. For example, consider the permutation:
\begin{equation*}
    \sigma = (a_1~~a_2~~\cdots~~a_k) \in S_n
\end{equation*}
Then this works as follows:
\begin{equation*}
    \sigma(b) =
    \begin{cases}
        a_i + 1 & b = a_i, i<k \\
        a_1 & b = a_k \\
        b & b \text{not in list}
    \end{cases}
\end{equation*}
\subsection{Composing Permutations}
We can then compose two permutations.\par
Consider $\tau = (1~~2), \sigma = (1~~3~~2)$ acting on the set $\{1, 2, 3\}$.\par
Then the composition $\tau \circ \sigma = (1~~2)(1~~2~~3)$ is simplified by the following process:
\begin{enumerate}
    \item Start at the number $1$
    \item Compute $\sigma(1) = 3$
    \item Compute $\tau(3) = 3$
    \item Compute $\sigma(3) = 2$
    \item Compute $\tau(2) = 1$
    \item We are back at $1$, so write the 2-cycle $(1~~3)$
    \item Now we know that $2 \mapsto 2$, but we can check: $\sigma(2) = 1$, $\tau(1) = 2$.
\end{enumerate}
\begin{example}
    $(1~~4~~3~~2)(2~~4~~3) = (1~~4~~2~~3)$
\end{example}
Note that a representation of a cycle like this can be cyclically shifted without changing its effect. $(1~~2~~3) = (2~~3~~1)$.\par
\begin{definition}{Disjoing cycles}
    Two cycles $(a_1~~\cdots~~a_k)$ and $(b_1~~\cdots~~b_l)$ are \underline{disjoint} if $a_i \neq b_j~\forall i, j$.
\end{definition}
\begin{theorem}{Disjoint cycles}
    Every $\sigma \in S_n$ can be written as a product of disjoint cycles. This product is unique up to cyclical shifts and reordering.
\end{theorem}
\begin{proof}
    The action of $\langle \sigma \rangle$ on $X_n$ partitions $X_n$ into orbits.\par
    Then $\langle \sigma \rangle i_1 \cup \langle \sigma \rangle i_2 \cup \cdots \cup \langle \sigma \rangle i_k = X_n$.\par
    Setting $n_j = |\langle \sigma \rangle i_j|$, we see:
    \begin{align*}
        \sigma = &(i_1~~\sigma(i_1)~~\cdots~~\sigma^{n_1 - 1}(i_1)) \cdot \\
        &(i_2~~\sigma(i_2)~~\cdots~~\sigma^{n_2 - 1}(i_2)) \cdot \\
        &~~~~~~~~\vdots \\
        &(i_k~~\sigma(i_k)~~\cdots~~\sigma^{n_k - 1}(i_k))
    \end{align*}
    And note that these are disjoint because orbits partition $X_n$.\par
    We have shown that the representation exists. Note also that the only choices made were the orbit representatives $i_j$, and the labelling $j$ of each $i_j$. Changing the orbit representatives corresponds to cyclically shifting terms in the disjoint cycle including $i_j$, and changing the labelling corresponds to changing the ordering of the disjoint cycles.\par
    Therefore the representation is unique up to the aforementioned changes.
\end{proof}
\begin{definition}{Cycle type}
    If $\sigma = (a_1^{(1)}~~a_2^{(1)}~~\cdots~~a_{k_1}^{(1)})(a_1^{(2)}~~a_2^{(2)}~~\cdots~~a_{k_2}^{(2)})\cdots(a_1^{(l)}~~a_2^{(l)}~~\cdots~~a_{k_l}^{(l)})$, a product of disjoint cycles, then $(k_1, k_2, \cdots, k_l)$ is the \underline{cycle type} of $\sigma$. We say that $\sigma$ is a $(k_1, k_2, \cdots, k_l)$-cycle. 1-cycles are not included.
\end{definition}
If $\sigma$ is a $k$-cycle, then $|\sigma| = k$. More generally, if $\sigma$ is a $(k_1, k_2, \cdots, k_n)$-cycle, then $|\sigma| = \lcm{\{k_1, k_2, \cdots, k_n\}}$.
\section{Transpositions and \texorpdfstring{$S_n$}{Sn}}
\subsection{Generating \texorpdfstring{$S_n$}{Sn}}
\begin{definition}{Transposition}
    A 2-cycle $(i~~j)$ is also known as a \underline{transposition}
\end{definition}
We can now consider some generating sets for $S_n$.
\begin{theorem}
    The set of transpositions generates $S_n$.
    \label{thmTranspositionsGenerateSn}
\end{theorem}
\begin{proof}
    \induction{$n = 1$}{
        Here $S_1 = \{e\}$, and there is nothing to prove.
    }{$n = k$}{Suppose $S_k$ is generated by transpositions}
    {$n = k + 1$}{
        Note that $S_k \leq S_{k + 1}$.\par
        Consider an arbitrary transform $\sigma \in S_{k+1}$.\par
        If $\sigma(k+1) = k+1$, then $\sigma \in S_k$ and is generated by transpositions by assumption.\par
        If not, consider the transposition $\tau = (k+1~~\sigma(k+1))$. Then $\tau \sigma(k+1) = k+1$, and so $\tau \sigma \in S_k$.\par
        Therefore let $\tau \sigma = T$, where $T$ is a product of transpositions by assumption. Therefore $\sigma = \tau T$, a product of transpositions.
    }
\end{proof}
\begin{remark}
    This gives a generating set of size $\begin{pmatrix}n & 2\end{pmatrix}$. This is better than $n!$, but still quite large.
\end{remark}
\begin{definition}{Adjacent transposition}
    A transposition of the form $(i~~i+1)$ is an \underline{adjacent transposition}.
\end{definition}
\begin{lemma}
    Any transposition can be written as a product of an odd number of transpositions.
    \label{lemAdjacentTranspositions}
\end{lemma}
\begin{proof}
    For a transposition $(i~~j)$, assume $i < j$ after relabelling. Then prove the result by induction on $j - i$.
    \induction{$j = i + 1$}{Then $(i~~j)$ is already adjacent, so it is the the product of 1 adjacent transposition}
    {$j = i + k$}{Assume that $(i~~i+k)$ is generated by an odd number of transpositions}
    {$j = i + k + 1$}{
        Now $(i~~i+k+1) = (i+k~~i+k+1)(i~~i+k)(i+k~~i+k+1)$, and so there are two extra adjacent transpositions, which is still an odd number.
    }
\end{proof}
\begin{remark}
    This lemma now gives us a generating set for $S_n$ of size $n - 1$.
\end{remark}
\subsection{Parity of a Permutation}
Before we define parity, we need a useful lemma:
\begin{lemma}
    If $\tau_1, \tau_2, \cdots, \tau_k$ are all transpositions, and their product, $\sigma$, is the identity, then $k$ is even.
    \label{lemEvenProductIdentity}
\end{lemma}
\begin{proof}
    First, note that by lemma~\ref{lemAdjacentTranspositions}, we can assume all of the transpositions are adjacent because this will not change the number of transpositions modulo 2.\par
    A pair $(i, j) \in \{1, 2, \cdots, n\} \times \{1, 2, \cdots, n\}$ is called an inversion for a transformation $\sigma$ if $i < j$ but $\sigma(i) > \sigma(j)$. That is, $\sigma$ switches the order of $i$ and $j$.
    \begin{subproof}{If $\sigma$ is defined as above, then $k$ is congruent to the number of inversions of $\sigma$, modulo 2.}
        \induction{$k = 0$}{Here $\sigma = Id$, which has 0 inversions and the congruence trivially holds.}
        {$k = m - 1$}{Let $\sigma'$ be a product of $k$ transpositions.}
        {$k = m$}{
            Let $\sigma = \tau \sigma'$. Then $\tau = (l~~l+1)$ for $1 \leq l < n$.\par
            We consider which pairs could be inversions under $\sigma$ but not $\sigma'$, and vice versa.\par
            All elements other than $l$ and $l + 1$ are not inverted, so consider $(i, j)$ such that $(\sigma'(i), \sigma'(j)) = (l, l+1)$.\par
            Then $\sigma(i) = l+1, \sigma(j) = l$. $\sigma(i) > \sigma(j)$.\par
            If $i < j$, then $(i, j)$ is not an inversion under $\sigma'$, but is under $\sigma$, which means we have 1 more inversion under $\sigma$ than $\sigma'$.\par
            If instead $i > j$, then $(i, j)$ is an inversion under $\sigma'$, but not under $\sigma$, which means we have 1 less inversion under $\sigma$ than $\sigma'$.\par
            So the number of inversions under $\sigma$ differs by 1 from $\sigma'$, and therefore the congruence holds because we have 1 more transposition.
        }
    \end{subproof}
    Now, if $\sigma = Id$, then we must have $k \equiv 0$ modulo 2, which means $k$ even.
\end{proof}
We can now define the sign homomorphism:
\begin{theorem}[Sign homomorphism]
    The map:
    \begin{align*}
        \sign : S_n &\mapsto C_2 \\
        \tau_1 \circ \cdots \circ \tau_k &\mapsto (-1)^k
    \end{align*}
    where the $t_i$ are transpositions, is a well-defined group homomorphism.
    \label{thmSignHism}
\end{theorem}
\begin{proof}
    To show that $\sign$ is well-defined, consider the products of transpositions:
    \begin{align*}
        &\tau_1 \tau_2 \cdots \tau_k \\
        &\tau_1' \tau_2' \cdots \tau_l'
    \end{align*}
    such that these products are equal. Thus, multiply one by the inverse of the other:
    \begin{align*}
        &\tau_k \tau_{k-1} \cdots \tau_1 \tau_1' \tau_2' \cdots \tau_l' \\
        &= \tau_k \tau_{k-1} \cdots \tau_1 \tau_1 \cdots \tau_{k-1} \tau_k \\
        &= e \text{ by repeated cancellation}
    \end{align*}
    So by lemma~\ref{lemEvenProductIdentity}, $k + l \equiv 0~(\text{mod}~2)$, which means $k \equiv l~(\text{mod}~2)$. And therefore $\sign$ is well-defined.\par
    Now to show it is a homomorphism:\par
    \begin{align*}
        \sign{(\tau_1 \tau_2 \cdots \tau_k \tau_1' \tau_2' \cdots \tau_l')} &= (-1)^{k + l} \\
        &= (-1)^k (-1)^l \\
        &= \sign{(\tau_1 \tau_2 \cdots \tau_k)} \sign{(\tau_1' \tau_2' \cdots \tau_l')}
    \end{align*}
\end{proof}
Now we can have a notion of parity:
\begin{definition}{Parity of a permutatation}
    Let $\sigma$ be a permutation. Define the following:\par
    If $\sign{\sigma} = 1$, then $\sigma$ is even. \par
    If $\sign{\sigma} = -1$, then $\sigma$ is odd.
\end{definition}
\begin{definition}{Alternating group}
    The alternating group, $A_n$, is defined to be the kernel of the homomorphism:
    $\sign : S_n \mapsto C_2$.    
\end{definition}
\begin{example}[Elements of $A_3$]
    We consider the even elements of $S_3$. Those are the identity and the two 3-cycles.\par
    Therefore $A_n = \{Id, (1~~2~~3), (1~~3~~2)\}$. Note that this is isomorphic to $C_3$, because it only has 3 elements. This should be unsurprising since $S_3 \cong D_6$.
\end{example}
We can easily determine parity from cycle type. For a $(k_1, k_2, \cdots, k_l)$-cycle, it is even if the number of even $k_i$ is even.
\begin{examples}[Determining parity from cycle-type]{}
    \item A $(2, 2)$-cycle is even because there is an even number of even numbers in the cycle type (two 2s)
    \item A $(2, 3, 4, 5)$-cycle is even because there is an even number of even numbers in the cycle type (a 2 and a 4)
    \item A $(2, 2, 3, 4)$-cycle is odd because there is an odd number of even numbers in the cycle type (3 of them: two 2s and a 4).
\end{examples}
\section{Conjugacy in \texorpdfstring{$S_n$}{Sn}}
We can easily relate conjugacy to cycle type in $S_n$.
\begin{theorem}[Conjugacy in $S_n$]
    Two permutations, $\sigma_1$ and $\sigma_2$ in $S_n$ are conjugate if and only if they have the same cycle type.
    \label{thmConjugacySn}
\end{theorem}
\begin{proof}
    \begin{proofdirection}{$\Leftarrow$}{Assume $\sigma_1$ and $\sigma_2$ have the same cycle type.}
        Let $\sigma_1 = (a_1^{(1)} a_2^{(2)} \cdots a_{l_1}^{(1)})(a_1^{(2)} a_2^{(2)} \cdots a_{l_2}^{(2)})\cdots(a_1^{(k)} a_2^{(2)} \cdots a_{l_k}^{(k)})$. This is a product of disjoint cycles so $\sigma_1(a_j^{(i)}) = a_{j + 1~\text{mod}~l_i}^{(i)}$.\par
        Define $\sigma_2$ similarly: $\sigma_2 = (b_1^{(1)} b_2^{(2)} \cdots b_{l_1}^{(1)})(b_1^{(2)} b_2^{(2)} \cdots b_{l_2}^{(2)})\cdots(b_1^{(k)} b_2^{(2)} \cdots b_{l_k}^{(k)})$ Note also that $\sigma_2(b_j^{(i)}) = b_{j + 1~\text{mod}~l_i}^{(i)}$
        Now let $\tau(a_j^{(i)}) = b_j^{(i)}$ be a permutation of $X_n$.\par
        Therefore:
        \begin{align*}
            \tau \sigma_1 \tau^{-1}(b_j^{(i)}) &= \tau \sigma_1(a_j^{(i)}) \\
            &= \tau(a_{j + 1~\text{mod}~l_i}^{(i)}) \\
            &= b_{j + 1~\text{mod}~l_i}^{(i)} \\
            &= \sigma_2(b_j^{(i)})
        \end{align*}
        So $\sigma_1$ and $\sigma_2$ are conjugates by $\tau$.
    \end{proofdirection}
    \begin{proofdirection}{$\Rightarrow$}{Assume $\sigma_1$ and $\sigma_2$ are conjugate.}
        Let $\sigma_2 = \tau \sigma_1 \tau^{-1}$\par
        The above argument shows that, if $\sigma_1$ is defined as above, $\sigma_2 = (b_1^{(1)} \cdots b_{l_1}^{(1)}) \cdots (b_1^{(k)} \cdots b_{l_k}^{(k)})$, which is the same cycle type as required.
    \end{proofdirection}
\end{proof}
Theorem \ref{thmConjugacySn} makes it easy to understand conjugacy classes in $S_n$.
\begin{example}
Consider the group $S_3$.\par
We consider now $\ccl_{S_3}((1~~2))$. Using Theorem \ref{thmConjugacySn}, the conjugacy class is all transpositions. These are exactly the pairs of elements where order of elements in the pair does not matter, so its size is therefore $\begin{pmatrix}3 \\ 2\end{pmatrix} = 3$.\par
The conjugacy class of the $3$-cycle, $\ccl_{S_3}((1~~2~~3))$, is all the $3$-cycles. We can count these by assuming the first number is $1$, then there are $2$ choices for the second number, and the third number is determined. So in $S_3$, $\ccl_{S_3}((1~~2~~3))$ has size $2$.\par
\end{example}
\begin{example}
    Consider now $S_4$.
    What is the size of the conjugacy class of $(1~~2)(3~~4)$? The options to choose the first transposition is $\begin{pmatrix}4 \\ 2\end{pmatrix}$, and then the second transposition is determined. However, for example, $(1~~2)(3~~4) = (3~~4)(1~~2)$, so we have double-counted and we need to halve the answer. We get $3$.
\end{example}
Recall that the centraliser of a group is the stabiliser of an element under the conjugation action. That is, the set of elements that commute with an element. Written $C_{S_n}(\gamma)$. Also, applying \ref{thmOrbitStab}:
\begin{equation}
|C_{S_n}(\gamma)| = \frac{|S_n|}{|\ccl_{S_n}(\gamma)}
\label{eqnCentraliserSize}
\end{equation}
\begin{example}
We return to the above example, and consider the centraliser of $(1~~2)(3~~4)$. Its size, by Equation~\ref{eqnCentraliserSize}, is $\frac{4!}{3} = 8$. That is, $8$ elements of $S_n$ commute with $(1~~2)(3~~4)$.\par
The elements are then $\{e, (1~~2), (3~~4), (1~~2)(3~~4),$\newline$(1~~3)(2~~4), (1~~4)(2~~3), (1~~4~~2~~3), (1~~4~~2~~3)\}$.
\end{example}
\begin{example}
    Now we know enough to list all the conjugacy classes of $S_4$.\par
    \begin{tabular}{c|c|c}
        Element & $|\ccl_{S_4}(\gamma)|$ & $|C_{S_4}(\gamma)|$ \\
        e & 1 & 24 \\
        \hline
        $(1~~2)$ & 6 & 4 \\
        $(1~~2)(3~~4)$ & 3 & 8 \\
        $(1~~2~~3)$ & 8 & 3 \\
        $(1~~2~~3~~4)$ & 6 & 4 
    \end{tabular}\par
    We know we have classified them all, since summing the sizes of the conjugacy classes gives $24 = |S_4|$.\par
    This gives us a fairly clear understanding of the elements of $S_4$.
\end{example}
\section{Conjugacy in \texorpdfstring{$A_n$}{An}}
Counting conjugacy classes in $A_n$ can also be done, with a little thought.
\begin{proposition}
    If $H$, $K \leq G$, then $|H : H \cap K| \leq |G : K|$
    \label{propSubgroupIndices}
\end{proposition}
\begin{proof}
Left as an exercise.
\end{proof}%TODO: complete this exercise
\begin{lemma}[Conjugacy classes in $A_n$]
    Let $\gamma \in A_n \leq S_n$.
    \begin{enumerate}
        \item If some odd element of $S_n$ commutes with $\gamma$, then the conjugacy class of $A_n$ of $\gamma$ is equal to the conjugacy class in $S_n$. $\ccl_{A_n}(\gamma) = \ccl_{S_n}(\gamma)$
        \item If every element of $S_n$ that commutes with $\gamma$ is even, then $\ccl_{S_n}(\gamma)$ splits into two:
        \begin{equation*}
            \ccl_{S_n}(\gamma) = \ccl_{A_n}(\gamma) \cup \ccl_{A_n}(\tau\gamma\tau) 
        \end{equation*}
        where $\tau$ is any transposition.
    \end{enumerate}
    \label{lemConjugacyAn}
\end{lemma}
Note that in condition 1 we have an element inside $C_{S_n}(\gamma)$ but not inside $C_{A_n}(\gamma)$ (since there exists an odd element) so $C_{A_n}(\gamma) \nsubseteq C_{S_n}(\gamma)$. Condition 2 implies that $C_{S_n}(\gamma) = C_{A_n}(\gamma)$, since all the element in $C_{S_n}(\gamma)$ are even and are then present in the alternating group also.\par
\begin{proof}
    From theorem~\ref{thmOrbitStab}, 
    \begin{align*}
        |S_n| = |\ccl_{S_n}(\gamma)||C_{S_n}(\gamma)| \\
        |A_n| = |\ccl_{A_n}(\gamma)||C_{A_n}(\gamma)| \\
    \end{align*}
    Since $|S_n| = 2|A_n|$,
    \begin{equation}
        |\ccl_{S_n}(\gamma)| = 2|\ccl_{A_n}(\gamma)|\frac{|C_{A_n}(\gamma)|}{|C_{S_n}(\gamma)|}
        \label{eqnSizeCcl1}
    \end{equation}
    Then by Proposition~\ref{propSubgroupIndices}, $|C_{S_n}(\gamma) : C_{A_n}(\gamma)| \leq 2$. This means, therefore, that $C_{S_n}(\gamma)$ is either equal to $C_{A_n}(\gamma)$, or is partitioned into two by it. That is, $|C_{S_n}(\gamma) : C_{A_n}(\gamma)| = 1$ or $2$. Using Lagrange's Theorem on Equation~\ref{eqnSizeCcl1}, we get:
    \begin{equation}
        |\ccl_{S_n}(\gamma)| = \frac{2|\ccl_{A_n}(\gamma)|}{|C_{S_n}(\gamma) : C_{A_n}(\gamma)|}
        \label{eqnSizeCcl2}
    \end{equation}
    And the denominator is either 1 or 2.We can therefore consider the cases:\par
    \begin{case}{Denominator is $2$}
        Here there is an odd element of $C_{S_n}(\gamma)$, $C_{A_n}(\gamma) \neq C_{S_n}(\gamma)$.\par
        Therefore, $|\ccl_{S_n}(\gamma)| = |\ccl_{A_n}(\gamma)|$, and since $C_{S_n}(\gamma)$ must be contained within $C_{A_n}(\gamma)$, the two must be equal as required.\par
    \end{case}
    \begin{case}{Denominator is $1$}
        In this case, the centralisers are equal, so \ref{eqnSizeCcl2} becomes:
        \begin{equation*}
            |\ccl_{S_n}(\gamma)| = 2|\ccl_{A_n}(\gamma)|
        \end{equation*}
        So the conjugacy class of $\gamma$ in $A_n$ is only half the size.\par
        Now consider a transposition $\tau \in S_n$. This is an odd permutation, so $\tau \notin S_n$.\par
        Note that $\tau\gamma\tau^{-1} \in \ccl_{S_n}(\gamma)$ by definition.\par
        We then see what happens if $\tau\gamma\tau^{-1} \in \ccl_{A_n}(\gamma)$.\par
        Then $\tau\gamma\tau^{-1} = \alpha\gamma\alpha^{-1}$ for some $\alpha \in A_n$, which is even. But rearranging this gives $(\tau\alpha)\gamma(\tau\alpha)^{-1} = \gamma$, so the odd element $\tau\alpha \in C_{S_n}(\gamma)$. This contradicts the assumption of case 2.\par
        Hence, $\tau\gamma\tau^{-1} \notin \ccl_{A_n}(\gamma)$, so $\ccl_{S_n}(\gamma) = \ccl_{A_n}(\gamma) \cup \ccl_{A_n}(\tau\gamma\tau^{-1})$ as required.
    \end{case}
\end{proof}
This lemma makes it possible to determine the conjugacy classes in $A_n$.
\begin{example}
    Consider the group $A_4$.\par
    All elements are either $e$, $(2, 2)$-cycles or $3$-cycles.\par
    Since the odd element $(1~~2)$ commutes with $(1~~2)(3~~4)$, the conjugacy class of $(2, 2)$-cycles remains intact.\par
    For $3$-cycles, the elements that commute with it are the identity, itself, and itself squared. These are all even elements, so we are in case 2 of Lemma~\ref{lemConjugacyAn}, and our conjugacy class splits.
    We can now classify all the conjugacy classes in $A_4$:\par
    \begin{tabular}{c|c}
        Element & $|\ccl_{A_n}(\gamma)$ \\
        $e$ & $1$ \\
        $(1~~2)(3~~4)$ & $3$ \\
        $(1~~2~~3)$ & $4$ \\
        $(3~~2~~1)$ & $4$ 
    \end{tabular}
\end{example}
\begin{example}
    Consider the group $A_5$ and $S_5$.\par
    \begin{tabular}{c|c|c}
        Element $\gamma$ & $|\ccl_{S_5}(\gamma)$ & $|C_{S_5}(\gamma)|$ \\
        \hline
        $e$ & $1$ & $120$ \\
        $(1~~2)$ & $\begin{pmatrix}5 \\ 2\end{pmatrix} = 10$ & $12$ \\
        $(1~~2~~3)$ & $2\times\begin{pmatrix}5 \\ 4\end{pmatrix} = 20$ & $6$ \\
        $(1~~2)(3~~4)$ & $\begin{pmatrix}5 \\ 2\end{pmatrix}\times\begin{pmatrix}4 \\ 2\end{pmatrix} \div 2 = 15$ & $8$ \\
        $(1~~2~~3)(4~~5)$ & $2\times\begin{pmatrix}5 \\ 4\end{pmatrix} = 20$ & $6$ \\
        $(1~~2~~3~~4)$ & $\begin{pmatrix}5 \\ 4\end{pmatrix}\times3! = 30$ & $4$ \\
        $(1~~2~~3~~4~~5)$ & $4! = 24$ & $5$
    \end{tabular}\par
    Now we can understand conjugacy in $A_5$. We first need to figure out whether the centralisers contain an odd element.\par
    \begin{itemize}
        \item $(4~~5)$ commutes with $(1~~2~~3)$, so the conjugacy class of $(1~~2~~3)$ does not split in $A_5$.
        \item $(1~~2)$ commutes with  $(1~~2)(3~~4)$, so its conjugacy class does not split.
        \item The size of the centraliser of $(1~~2~~3~~4~~5)$ is $5$, so the centraliser is equal to the subgroup generated by it, that is, only 5-cycles commute with the 5-cycle. So we have no odd elements in the centraliser, and the conjugacy class splits into two in $A_n$.
    \end{itemize}
    We now have enough information to categorise the conjugacy classes of $A_n$.\par
    \begin{tabular}{c|c}
        Element $\gamma$ & $|\ccl_{A_n}(\gamma)$ \\
        \hline
        $e$ & $1$ \\
        $(1~~2~~3)$ & $20$ \\
        $(1~~2)(3~~4)$ & $15$ \\
        $(1~~2~~3~~4~~5)$ & $12$ \\
        $(2~~1~~3~~4~~5)$ & $12$
    \end{tabular}
\end{example}
Recall: a \underline{simple} group is one where there exist no non-trivial normal subgroups\par % WHAT?
\begin{theorem}
    $A_5$ is simple. \label{thmA5Simple}
\end{theorem}
\begin{proof}
    Suppose we have a proper normal subgroup $N$ of $A_5$.\par
    Then $N$ is a union of conjugacy classes in $A_5$.\par
    It must contain the identity, and we consider the possible orders of $N$.
    \begin{itemize}
        \item Just the identity: $|N| = 1$
        \item The 3-cycles: $|N| = 1+20=21$
        \item $(2-2)$-cycles also: $|N| = 1+20+15 = 36$
        \item Also half the 5-cycles: $|N| = 1+20+15+12=48$
        \item All elements: $|N| = 60$
        \item $(2-2)$-cycles: $|N| = 1+15 = 16$
        \item Also some 5-cycles: $|N| = 1+15+12 = 28$
        \item The rest of the 5-cycles: $|N| = 1+15+12+12 = 40$
        \item Only half the 5-cycles: $|N| = 1+12 = 13$
        \item All the 5-cycles: $|N| = 1+12+12 = 25$
        \item 3-cycles and 5-cycles: $|N| = 1+20+12 = 33$
        \item The rest of the 5-cycles: $|N| = 1+20+12+12 = 45$
    \end{itemize}
    But by \ref{thmLagrange}, none of these (except $A_5$ and $\{e\}$) can be subgroups.
\end{proof}
\end{document}