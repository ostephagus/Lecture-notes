\documentclass[../Main.tex]{subfiles}

\begin{document}
\section{An Introduction to Actions}
\begin{definition}{Group action}
    An \underline{action} of a group $G$ on a set $X$ is a map:
    \begin{align*}
        G \times X &\rightarrow X \\
        (g, x) &\rightarrow gx
    \end{align*}
    that satisfies:
    \begin{enumerate}
        \item $ex = x~\forall x \in X$;\label{actAxiomIdentity}
        \item $(g_1 g_2)x = g_1(g_2 x) \forall g_1, g_2 \in G, x \in X$.\label{actAxiomAssoc}
    \end{enumerate}
\end{definition}
$G \actson X$ means $G$ acts on $X$.\par
The trivial action is $gx=x \forall g \in G, x \in X$.\par
The symmetric group $Sym(X) \actson X$ with $fx=f(x) \forall f \in Sym(X), x \in X$.\par
For any $X \subseteq \mathbb{C}$, $Isom(X) \actson X$ as above, as it does for any subgroup. If further $X$ is the regular n-gon $X_n$, $D_{2n} \actson X_n$. It also acts on the set of vertices $\{z \in \mathbb{C} | z^n = 1\}$.\par
Every group $G$ can act on itself. Let $X = G$, then the \underline{regular action} is:
\begin{equation*}
    g \gamma = g \cdot \gamma, g \in G, \gamma \in X
\end{equation*}
\section{Actions and the Symmetry Group}
\begin{theorem}
    An action of a group $G$ on a set $X$ is equivalent to a homomorphism $G \mapsto \Sym(X)$.
    \label{thmActionEquivHism}
\end{theorem}
\begin{proof}
    \begin{proofdirection}{$\Rightarrow$}{Suppose $G \actson X$}
        Then define the set of maps:
        \begin{align*}
            t_g : X &\mapsto X \\
            x &\mapsto gx
        \end{align*}        
        For any $g \in G$.\par
        Now we have that $t_{g^{-1}} \circ t_g = Id_X$, since $t_{g^{-1}}(t_g(x)) = g^{-1}g(x) = x$ by axiom~\ref{actAxiomIdentity}.\par
        $t_g$ is a bijection from $X$ to itself, so $t_g \in \Sym(X)$.
        Therefore we define the required map:
        \begin{align*}
            \phi : G &\mapsto \Sym(x) \\
            g &\mapsto t_g
        \end{align*}
        Note that $t_g \circ t_h = g_{gh}$ for any elements $g, h \in G$ by axiom~\ref{actAxiomAssoc}.\par
        Therefore, $\phi(g) \phi(h) = t_g t_h = t_{gh} = \phi(gh)$, and $\phi$ is a homomorphism.
    \end{proofdirection}
    \begin{proofdirection}{$\Leftarrow$}{Let the homomorphism $\phi$ be defined as above}
        Then we define the action with $gx = \phi(g)(x)$.        
        We have $ex = \phi(e)(x) = x$ by identity of $\Sym(X)$.\par
        We also have $(gh)x = \phi(gh)x = \phi(g) \circ \phi(h) x = g(hx)$, by homomorphism.
        So this action satisfies the required axioms.
    \end{proofdirection}
\end{proof}
\begin{theorem}[Cayley's Theorem]
    Every group $G$ is isomorphic to a subgroup of some $\Sym(X)$.
    \label{thmCayley}
\end{theorem}
\begin{proof}
    Let $X = G$. Then use the regular action: $G \actson G$, which is equivalent to the homomorphism $\phi : G \mapsto \Sym(G)$.\par
    Then let $H = \im(\phi)$ for surjectivity, with $H$ the required subgroup of $\Sym(X)$.\par
    Now consider $\ker(\phi)$. If $g \in \ker(\phi), \phi(g) = Id_g$. That is, $g \gamma = \gamma~\forall \gamma \in G$.\par
    By uniqueness of the identity, this means $g = e$. And by proposition~\ref{propInjectivityTrivialKer}, this means $\phi$ is injective.
\end{proof}
\section{Orbits and Stabilisers}
Let $G \actson X$ and $x \in X$.
\begin{definition}{Orbit}
    The \underline{orbit} of an element $x$ is the set:
    \begin{equation}
        Gx = \subsetselect{y \in X}{\exists g \in G \text{ such that } y = gx}
    \end{equation}
    It is all the places that $x$ is sent under $G$.
\end{definition}
\begin{definition}{Stabiliser}
    The \underline{stabiliser} of $x$ is the set:
    \begin{equation}
        \Stab_G(x) = \subsetselect{g \in G}{gx = x}
    \end{equation}
    It is all the elements of $G$ that leave $x$ unchanged.
\end{definition}
\begin{definition}{Transitivity}
    If $Gx = X$, that is, an element $x$ is mapped to all elements in the set $X$ under $G$, then the action is \underline{transitive}.
\end{definition}
\begin{definition}{Faithfulness}
    If every $g \in G$ has an $x \in X$ such that $gx \neq x$, except the element $e$, then the action is \underline{faithful}.\par
    The action is also \underline{faithful} if the implication $gx = x~\forall x \in X \implies g = e$ holds.\par
    The action is also \underline{faithful} if the homomorphism $G \mapsto \Sym(X)$ defined by the action is injective.
\end{definition}
Note that $\Stab_G(x)$ is sometimes written $G_x$, and $Gx$ is sometimes written $\text{Orb}_G(x)$.\par
We would like to understand how the orbit and the stabliser exist within the group $G$ and the set $X$.
\begin{propositions}{
        Let $G \actson X$.
    }
    \item $\Stab_G(x) \leq G~\forall x \in X$ \label{propStabIsSubgroup}
    \item The set of orbits $\subsetselect{Gx}{x \in X}$ partitions $X$. \label{propOrbitsPartition}
\end{propositions}
\begin{proof}
    \begin{enumerate}
        \item We check the definition of a subgroup:
            We clearly have the identity, since $e$ is always in $\Stab_G(x)$ by axiom~\ref{actAxiomIdentity}.\par
            Associativity: if $g, h \in \Stab_G(x)$,
            \begin{align*}
                (g \cdot h)x &= g(hx) \text{ by axiom~\ref{actAxiomAssoc}} \\
                &= gx = x \text{ by definition of stabilisers}
            \end{align*}
            So $g \cdot h \in \Stab_G(x)$.\par
            Inverse: if $g \in \Stab_G(x)$,
            \begin{align*}
                g^{-1} x &= g^{-1} (gx) \\
                &= (g^{-1} \cdot g) x \\
                &= ex = x
            \end{align*}
            So $g^{-1} \in \Stab_G(x)$.\par
            Therefore $\Stab_G(x) \leq G$.
        \item Consider the 2 conditions for a partition.
            First, $x = ex \in Gx$, so orbits contain all $x \in X$.\par
            Then consider an element $y$ in the intersection of two orbits $Gx_1$ and $Gx_2$.\par
            Let $g_1 x_1 = g_2 x_2$. Then $x_1 = (g_1^{-1} g_2) x_2$. Therefore any $g x_1 = (g g_1^{-1} g_2) x_2 \in Gx_2$.\par
            $Gx_1 \subseteq Gx_2$. By the same logic, $G x_2 \subseteq G x_1$.\par
            So $G x_1 = G x_2$.
    \end{enumerate}
\end{proof}
We can also prove a key theorem allowing us to enumerate groups by considering actions.
\begin{theorem}[Orbit-Stabiliser theorem]
    Suppose $G \actson X$ and $x \in X$. The assignment:
    \begin{equation*}
        g \Stab_G(x) \mapsto gx
    \end{equation*}
    Gives a well-defined bijection:
    \begin{equation*}
        G / Stab_G(x) \mapsto Gx
    \end{equation*}
    \label{thmOrbitStab}
\end{theorem}
To clear up notation, in words this is: \textit{the set of cosets, $g$ times elements of the stabiliser of $x$ for all $g$, maps to the orbit of $x$ under $G$. This is via the map: $g$ times the stabiliser of $x$ maps to the element that $x$ maps to under $g$ in the action.}
\begin{proof}
    Let $S = \Stab_G(x)$. Let $\Phi$ be the map $\Phi(gS) = gx$.
    First, we need to show $\Phi$ is well-defined. That is, if $g_1 S = g_2 S$ then $g_1 x = g_2 x$.\par
    We have $g_1 \in g_1 S$, so choose an element $s \in S$ such that $g_1 = g_2 s$.
    \begin{align*}
        g_1 x &= (g_2 s) x \\
        &= g_2 (sx) \\
        &= g_2 x \text{ since } s \text{ is a stabiliser.}
    \end{align*}
    which is as required: $\Phi$ is a well-defined map.\par
    Now for any $gx \in Gx$, choose the corresponding coset $gS$, then every output has an input and so $\Phi$ is surjective.\par
    For injectivity, suppose that $\Phi(g_1 S) = \Phi(g_2 S)$. That is, $g_1 x = g_2 x$. Let $s = g_2^{-1} g_1$.
    \begin{align*}
        s x &= (g_2^{-1} g_1) x \\
        &= g_2^{-1} (g_1 x) \\
        &= g_2^{-1} (g_2 x) \text{ by assumption} \\
        &= (g_2^{-1} g_2) x \\
        &= x
    \end{align*}
    So $s \in S$. Therefore $g_1 = (g_2 g_2^{-1}) g_1 = g_2 (g_2^{-1} g_1) = g_2 s$.\par
    So $g_1 \in g_2 S$, which means that $g_1 S = g_2 S$ by lemma~\ref{lemCosetsPartition}, and $\Phi$ is injective.
\end{proof}
Now we have a bijection, we can say useful things about the sizes of these sets:
\begin{corollary}
    Suppose $G$ is finite. If $G \actson X$ and $x \in X$, then $|G| = |Gx| \times |\Stab_G(x)| \forall x$.
    \label{corOrbitStabSizes}
\end{corollary}
\begin{proof}
    Theorem~\ref{thmOrbitStab} gives 
    \begin{equation*}
        |G : \Stab_G(x)| = |Gx|
    \end{equation*}
    Then theorem~\ref{thmLagrange} gives
    \begin{equation*}
        |G : \Stab_G(x)| = \frac{|G|}{\Stab_G(x)}
    \end{equation*}
    So equating and multiplying up gives:
    \begin{equation}
        |G| = |Gx| \times |\Stab_G(x)|
        \label{eqnOrbitStab}
    \end{equation}
\end{proof}
\section{Counting With Actions}
\begin{example}
    We consider the symmetry group of a cube. Let $x$ be a point at the middle of a face. Then the size of the orbit must be 6, since there are 6 faces. Then the stabilisers of $x$ are the transformations that fix the face. This is isomorphic to $D_8$, the symmetry group of a square. Therefore the size of $\Stab_G(x)$ is 8. Then we can simply calculate $|G| = 6 \times 8 = 48$.
\end{example}
\begin{theorem}[Cauchy's Theorem]
    If $G$ is a finite group, and $p$ is a prime that divides $|G|$, then there is an element in $G$ with order $p$.
    \label{thmCauchy}
\end{theorem}
\begin{proof}
    Consider the set $X$ consisting of $p$-tuples whose product is the identity:
    \begin{equation}
        \subsetselect{(g_1, g_2, \cdots, g_p)}{g_i \in G, g_1 g_2 \cdots g_p = e}
    \end{equation}
    and define an action of $C_p$ on $X$:\par
    Let $t$ generate $C_p$. Then let the action be defined by:
    \begin{equation*}
        t^k(g_1, g_2, \cdots, g_p) = (g_{k+1}, g_{k+2}, \cdots, g_p, g_1, g_2, \cdots, g_k)
    \end{equation*}
    That is, $t^k$ performs a rotate-right by $k$.\par
    Consider the axioms of an action:\par
    Identity: $t^0 = e$, which shifts by 0 \tick\par
    Associativity: $t^k \cdot t^l = t^{k + l}$, both of which rotate right by $k + l$.\tick\par
    But we also need to ensure that the action respects the condition of $X$ that the product of each $p$-tuple is $e$:\par
    Let $a = g_1, \cdots g_k$ and $b = g_{k + 1}, \cdots, g_p$. Now we have $a \cdot b = e$, and by proposition~\ref{propFundamentals}, $b \cdot a = e$ \tick\par
    So this is indeed an action.\par
    Next we consider the size of $X$. Each element can be chosen from $G$ different elements, except the last since this element is fully determined due to the product condition. Therefore,
    \begin{equation}
        |X| = |G|^{p-1}
    \end{equation}
    Note that $C_p$ partitions $X$ into orbits:
    \begin{equation*}
        |X| = \sum_{i=1}^{k} |C_p x_i|
    \end{equation*}
    assuming there are $k$ different orbits.\par
    Also, by theorem~\ref{thmOrbitStab}, $|C_p| = |C_p x_i| \times |\Stab_{C_p}(x_i)|$. However, $|C_p|$ is $p$, so $|C_p x_i|$ is $p$ or 1.\par
    Let $l$ be the number of size-1 orbits. Re-label the $x_i$ such that:
    \begin{equation*}
        |C_p x_i| =
        \begin{cases}
            1 & 1 \leq i \leq l \\
            p & l < i \leq k
        \end{cases}
    \end{equation*}
    So now the size of $X$ is:
    \begin{align*}
        |X| &= \sum_{i=1}^{k} |C_p x_i| \\
        &= \sum_{i=1}^{l} |C_p x_i| + \sum_{i=l+1}^{k} |C_p x_i| \\
        &= l + p(k-l)
    \end{align*}
    And by the original assumption $p | |G|$, which means $p | |G|^{p-1} = |X|$.\par
    Therefore $p | l + p(k-l)$, and thus $p | l$.\par
    $l$ could be 0, but if not $l geq 2$.\par
    A size-1 orbit means $x$ is unaffected by shifts, which in this case implies that all the elements of the $p$-tuple are equal. Note that there are $l$ of these. We have the case $x = (e, e, \cdots, e)$, so there must be at least one such $x$, and by conditions on $l$ that means that $l \geq 2$.\par
    Therefore, there is at least one more $x$ with all elements equal. For the product condition to hold, this element $g$ must have $g^p = e$.\par
    So there is at least one non-identity element $g$ with $g^p = e$.
\end{proof}
\section{Conjugation}
\begin{definition}{Conjugation}
    Let $G$ be a group, $g \in G$. For some $g' \in G, g' \neq g$, $g'$ is a \underline{conjugate} of $g$ if there is some element $h$ such that $h \cdot g \cdot h^{-1} = g'$. Then $g'$ is the \underline{conjugate of $g$ by $h$}.
\end{definition}
\begin{example}
    If $G$ is abelian, $h \cdot g \cdot h^{-1} = h \cdot h^{-1} \cdot g = g$, so elements are only conjugate with themselves.
\end{example}
However, conjugation provides a good way to study non-abelian groups. We shall see that conjugates of an element are those that behave in a similar way. This might correspond to some notion of \textit{shape} or \textit{type}.\par
The \underline{conjugation action} is a very important action for non-abelian groups. It is defined thus:
\begin{proposition}
    $G$ acts on itself via conjugation. That is:
    \begin{align}
        G &\mapsto G \\
        (h, g) &\mapsto h g h^{-1} \nonumber
    \end{align}
\end{proposition}
\begin{proof}
    Identity: $h = e$ gives $e \cdot g \cdot e = g$ \tick\par
    Associativity: $h = h_1 h_2$ gives:
    \begin{align*}
        (h_1 h_2) g (h_1 h_2)^{-1} &= h_1 h_2 g h_2^{-1} h_1^{-1} \\
        &= h_1 (h_2 g h_2^{-1}) h_1^{-1}
    \end{align*}
    Which is the result of $h_2$ then $h_1$, as required.
\end{proof}
\begin{definition}{Conjugacy classes}
    The \underline{conjugacy class} of an element $g$ is the set:
    \begin{equation*}
        \ccl_G(g) = \subsetselect{hgh^{-1}}{h \in G}
    \end{equation*}
    They are the orbits of the conjugation action.
\end{definition}
\begin{definition}{Centraliser}
    The \underline{centraliser} of an element $g$ is the set:
    \begin{equation*}
        C_G(g) = \subsetselect{h \in G}{g = hgh^{-1}}
    \end{equation*}
    They are the stabilisers of the conjugation action, and represent all the elements that commute with $g$.
\end{definition}
\begin{definition}{Centre}
    The \underline{centre} of a group $G$ is the union of centralisers:
    \begin{equation*}
        Z_G = \subsetselect{h \in G}{g = hgh^{-1}~\forall g \in G}
    \end{equation*}
    It is the set of elements that commute with everything in the group.
\end{definition}
\begin{example}
    Consider the group $S_3$. It is the permutations of a set of 3 elements.
    \begin{equation*}
        S_3 = \{e, t_1, t_2, t_3, c, c^{-1}\}
    \end{equation*}
    Where $t_i$ is the transposition that keeps element $i$ the same and swaps the other two. $c$ is the cycle that moves each element right.\par
    This group has 3 conjugacy classes:
    \begin{enumerate}
        \item The identity, conjugate with only itself
        \item The transpositions (3 of them)
        \item The cycles (2 of them)
    \end{enumerate}
\end{example}
\end{document}