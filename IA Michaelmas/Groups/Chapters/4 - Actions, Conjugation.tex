\documentclass[../Main.tex]{subfiles}

\begin{document}
\section{An Introduction to Actions}
\begin{definition}{Group action}
    An \underline{action} of a group $G$ on a set $X$ is a map:
    \begin{align*}
        G \times X &\rightarrow X \\
        (g, x) &\rightarrow gx
    \end{align*}
    that satisfies:
    \begin{enumerate}
        \item $ex = x~\forall x \in X$;\label{actAxiomIdentity}
        \item $(g_1 g_2)x = g_1(g_2 x) \forall g_1, g_2 \in G, x \in X$.\label{actAxiomAssoc}
    \end{enumerate}
\end{definition}
$G \actson X$ means $G$ acts on $X$.\par
The trivial action is $gx=x \forall g \in G, x \in X$.\par
The symmetric group $Sym(X) \actson X$ with $fx=f(x) \forall f \in Sym(X), x \in X$.\par
For any $X \subseteq \mathbb{C}$, $Isom(X) \actson X$ as above, as it does for any subgroup. If further $X$ is the regular n-gon $X_n$, $D_{2n} \actson X_n$. It also acts on the set of vertices $\{z \in \mathbb{C} | z^n = 1\}$.\par
Every group $G$ can act on itself. Let $X = G$, then the \underline{regular action} is:
\begin{equation*}
    g \gamma = g \cdot \gamma, g \in G, \gamma \in X
\end{equation*}
\section{Actions and the Symmetry Group}
\begin{theorem}
    An action of a group $G$ on a set $X$ is equivalent to a homomorphism $G \mapsto \Sym(X)$.
    \label{thmActionEquivHism}
\end{theorem}
\begin{proof}
    \begin{proofdirection}{$\Rightarrow$}{Suppose $G \actson X$}
        Then define the set of maps:
        \begin{align*}
            t_g : X &\mapsto X \\
            x &\mapsto gx
        \end{align*}        
        For any $g \in G$.\par
        Now we have that $t_{g^{-1}} \circ t_g = Id_X$, since $t_{g^{-1}}(t_g(x)) = g^{-1}g(x) = x$ by axiom~\ref{actAxiomIdentity}.\par
        $t_g$ is a bijection from $X$ to itself, so $t_g \in \Sym(X)$.
        Therefore we define the required map:
        \begin{align*}
            \phi : G &\mapsto \Sym(x) \\
            g &\mapsto t_g
        \end{align*}
        Note that $t_g \circ t_h = g_{gh}$ for any elements $g, h \in G$ by axiom~\ref{actAxiomAssoc}.\par
        Therefore, $\phi(g) \phi(h) = t_g t_h = t_{gh} = \phi(gh)$, and $\phi$ is a homomorphism.
    \end{proofdirection}
    \begin{proofdirection}{$\Leftarrow$}{Let the homomorphism $\phi$ be defined as above}
        Then we define the action with $gx = \phi(g)(x)$.        
        We have $ex = \phi(e)(x) = x$ by identity of $\Sym(X)$.\par
        We also have $(gh)x = \phi(gh)x = \phi(g) \circ \phi(h) x = g(hx)$, by homomorphism.
        So this action satisfies the required axioms.
    \end{proofdirection}
\end{proof}
\end{document}