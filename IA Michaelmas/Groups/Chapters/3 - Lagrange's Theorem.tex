\documentclass[../Main.tex]{subfiles}

\begin{document}
\section{Preliminary Lemmas}
We would like to make sense of the size of subgroups of a group $G$. For this, we will introduce a key theorem. The proof will slice $G$ into pieces, each of which is the size of the subgroup, $H$. A key idea is the \underline{coset}.
\begin{definition}{Coset}
    Let $H \leq G, g \in G$. Then the \underline{left coset of $H$ by $g$}, $gH = \subsetselect{gh}{h \in H}$. All elements of $H$, left-multiplied by $g$.\par
    The \underline{right coset of $H$ by $g$}, $Hg = \subsetselect{hg}{h \in H}$.
\end{definition}
Note: cosets are not necessarily distinct! We must therefore be careful when using cosets to define functions.\par
The set of all left cosets in $G$ is denoted $G / H = \subsetselect{gH}{g \in G}$. The set of right cosets is $H \backslash G$.
\begin{lemma}
    Let $H \leq G$. Then the set of left cosets of $H$ partitions $G$. That is,
    \begin{enumerate}
        \item $G$ is the union of all left cosets
        \item Cosets are disjoint
    \end{enumerate}
    \label{lemCosetsPartition}
\end{lemma}
\begin{proof}
    Firstly, for any $g \in G, g = g \cdot e \in gH$, so all elements of $G$ are in some coset, and therefore $G$ is the union of all cosets as required.\par
    Secondly, suppose there exists two cosets $g_1H$ and $g_2 H$, with an element $k$ in the intersection.\par
    Then $k = g_1 h_1 = g_2 h_2$, for some $h_1, h_2 \in H$.\par
    \begin{align*}
        g_1 &= (g_2 h_2) h_1^{-1} \\
        &= g_2 (h_2 h_1^{-1})
    \end{align*}
    Therefore, for any element of $g_1 H, g_1 h$, we have also that $g_1 h = g_2 ((h_2 h_1^{-1}) h)$, which is an element of $g_2 H$.\par
    That is, $g_1 H \subseteq g_2 H$.\par
    Using the same logic, it can be shown that $g_2 H \subseteq g_1 H$. So $g_1 H = g_2 H$.\par
    If two cosets overlap, they are the same coset. Therefore all cosets are disjoint as required.
\end{proof}
\begin{lemma}
    Let $H \leq G$. There exists a bijection $H \mapsto gH~\forall g \in G$.
    \label{lemCosetsBijection}
\end{lemma}
\begin{proof}
    We have a map:
    \begin{align*}
        H &\mapsto gH \\
        h &\mapsto gh
    \end{align*}
    And we can show it has a well-defined inverse:
    \begin{align*}
        gH &\mapsto H \\
        gh &\mapsto (g^{-1} g)H = H
    \end{align*}
    since inverses are unique.
\end{proof}
\begin{corollary}
    The size of any coset is the same. That is, $|H| = |gH|~\forall g \in G$.
    \label{corCosetsSameSize}
\end{corollary}
We can also define:
\begin{definition}{Index of a subgroup}
    The \underline{index} of a subgroup $H$ in a group $G$ is $|G : H| = |G / H|$. Also written $\left[G : H\right]$.
\end{definition}
\section{Lagrange's Theorem and Corollaries}
We now have the necessary groundwork for:
\begin{theorem}[Lagrange's Theorem]
    If $G$ is finite, and there is a subgroup $H \leq G$, then $|G| = |G : H| |H|$.
    \label{thmLagrange}
\end{theorem}
\begin{proof}
    \begin{align*}
        |G| &= \sum_{gH \in G / H} |gH| \text{ by lemma~\ref{lemCosetsPartition}} \\
        &= \sum_{gH \in G / H} |H| \text{ by corollary~\ref{corCosetsSameSize}} \\
        &= |G : H| |H| \text{ by definition of } |G : H|.
    \end{align*}
\end{proof}
The theorem itself is useful, but gives rise to many useful corollaries:
\begin{corollary}
    If $G$ is finite and contains an element $g$, then the order of $g$ divides the order of $G$.
    \label{corElementOrderDividesGrpOrder}
\end{corollary}
\begin{proof}
    Let the subgroup $H$ be $\langle g \rangle$. Then result follows.
\end{proof}
\begin{corollary}
    If $G$ is finite and contains an element $G$, then $g^{|G|} = e$.
    \label{corGroupOrderExponent}
\end{corollary}
\begin{proof}
    Follows directly from corollary~\ref{corElementOrderDividesGrpOrder}.
\end{proof}
\begin{corollary}
    If $|G|$ is prime, then $G$ is cyclic. Also, any element in $G$ except $e$ generates this group.
    \label{corPrimeGroupCyclic}
\end{corollary}
\begin{proof}
    Choose any $g \in G$. By corollary~\ref{corElementOrderDividesGrpOrder}, its order is $|G|$ or 1. If it is 1, then $g = e$. If not, then $\langle g \rangle$ has the same size as $G$, and since it is a subgroup we must have $\langle g \rangle = G$. That is, $G$ is cyclic.
\end{proof}
\nonexaminablesection{An Application to Number Theory}
Recall that $\varphi(n)$ is the number of elements in $\subsetselect{x \in \Z_n}{\hcf{(x, n)} = 1}$\par
Also recall that $x \in \Z_n$ has a multiplicative inverse mod $n$ if it is coprime with $n$. Therefore, we can define a group from this resticted domain:
\begin{equation*}
    \Z_n^{\times} = \subsetselect{x \in \Z_n}{\hcf{(x, n)} = 1}
\end{equation*}
And it has group operation $\times_n$, multiplication mod $n$.
We can then provide a different proof of the Euler-Fermat Theorem:
\begin{theorem}[Euler-Fermat Theorem]
    Let $x, n \in \Z$. If $\hcf{(x, n)} = 1, x^{\varphi(n)} \equiv 1~(\text{mod }n)$.
\end{theorem}
\begin{proof}
    Note that $|\Z_n^{\times}| = \varphi(n)$. Therefore, by corollary~\ref{corGroupOrderExponent}, $x^{\varphi(n)} \equiv 1$.
\end{proof}
\end{document}