\documentclass[../Main.tex]{subfiles}

\begin{document}
In a time-independent situations, Maxwell's equations reduce to:
\begin{align}
    \nabla \cdot \vec{E} &= \frac{\rho}{\epsilon_0} \label{eqnMTED} \\
    \nabla \cdot \vec{B} &= 0 \label{eqnMTBD} \\
    \nabla \times \vec{E} &= 0 \label{eqnMTEC} \\
    \nabla \times \vec{B} &= \mu_0 \vec{J} \label{eqnMTBC}
\end{align}
Now since $\vec{E}$ and $\vec{B}$ are decoupled, we can study them separately.

\underline{Electrostatics} is the study of the electric field generated by a stationary charge distribution. We are concerned with equations~\ref{eqnMTED} and \ref{eqnMTEC}.
\section{Gauss's Law}
Consider a closed surface $S$ enclosing a volume $V$. Integrating \eqnref{eqnMTED} over $V$ gives:
\begin{equation*}
    \int_V \nabla \cdot \vec{E}~dV = \int_V\frac{\rho}{\epsilon_0}~dV
\end{equation*}
Then using the divergence theorem, and the notation $Q_V = \int_V \rho~dV$,
\begin{equation}
    \int_S \vec{E} \cdot \vec{dS} = \frac{Q_v}{\epsilon_0}
    \label{eqnGaussLaw}
\end{equation}
This is the integral version of equation~\ref{eqnMED} and is also valid in time-dependent situations. It says that the electric flux is proportional to the total charge enclosed.

In special situations, we can use \eqnref{eqnGaussLaw} together with symmetry to deduce $\vec{E}$ from $\rho$, by choosing a suitable Gaussian surface $S$.
\subsection{Spherical Symmetry}
Use spherical polar coordinates and consider the case $\rho = \rho(r)$, with total charge $Q$ contain within a radius $R$.

We then assume the electric field is spherically symmetric, so it has the form $\vec{E} = E(r) \vec{e_r}$. This automatically satisfies \eqnref{eqnMTEC}. We now apply \eqnref{eqnGaussLaw} to a sphere of radius $r$. First, for $r > R$:
\begin{align*}
    \int_S \vec{E} \cdot \vec{dS} &= E(r) \int_S \vec{e_r} \cdot \vec{dS} \\
    &= E(r) \int_s dS \\
    &= E(r) 4\pi r^2 \\
    &= \frac{Q}{\epsilon_0}
\end{align*}
Therefore for $r > R$,
\begin{equation*}
    \vec{E} = \frac{Q}{4\pi \epsilon_0 r^2} \vec{e_r}
\end{equation*}
Then considering a test particle of charge $q$ in this field, this will experience a Lorentz force:
\begin{equation*}
    \vec{F} = q\vec{E} = \frac{Qq}{4\pi \epsilon_0 r^2} \vec{e_r}
\end{equation*}
In the limit $R \to 0$, we obtain the electric field of a point charge at the origin. This corresponds to a charge density $\rho = Q \delta(\vec{x})$.

\begin{remark}
    There is a close analogy between the Coulomb force and the gravitational force between particles with mass,
    \begin{equation*}
        \vec{F} = -\frac{GMm}{r^2} \vec{e_r}
    \end{equation*}
    Both involve an inverse square law and the product of the charges/masses. However, we have two key differences:
    \begin{enumerate}
        \item while gravity is always attractive, electric forces can be attractive or repulsive depending on the sign of the charges;
        \item gravity is far weaker, for a proton-proton interaction the electric force is $10^{36}$ times stronger.
    \end{enumerate}
\end{remark}
\subsection{Cylindrical symmetry}
Consider a cylindrically symmetric charge distribution in spherical polar coordinates, $\rho = \rho(r)$ and $\vec{E} = E(r) \vec{e_r}$. Let the total charge be $\lambda$ per unit length, contained within an outer cylindrical radius $R$.

Consider a Gaussian surface of radius $R$ and length $L$.
\begin{align*}
    \int_S \vec{E} \cdot \vec{dS} &= E(r) \int_{\text{curved part}} dS \\
    &= 2\pi r L E(r) \\
    &= \frac{\lambda L}{\epsilon_0}
\end{align*}
Then for any $L$, 
\begin{equation*}
    \vec{E} = \frac{\lambda}{2\pi \epsilon_0 r} \vec{e_r}
\end{equation*}
We can consider $R \to 0$, where $\rho$ defines a line charge $\rho(r) = \lambda \delta(x)\delta(y)$
\subsection{Planar Symmetry}
Consider a planar charge distribution $\rho = \rho(z)$ in Cartesian coordinates. Let the total charge be $\sigma$ per unit area, contained within a region $-d<z<d$. Assume reflectional symmetry about $z = 0$. That is, $\rho(z)$ is an even function. Now we have $\vec{E} = E(z) \vec{e_z}$. Because $\vec{E}$ is a divergence, differentiation an even function gives an odd function, so $E(-z) = -E(z)$.

We apply Gauss's law to a ``Gaussian pillbox'' of height $2z$ and over an arbitrary area $A$. For $z > d$,
\begin{align*}
    \int_S \vec{E} \cdot \vec{dS} &= E(z) A = E(-z) A \\
    &= 2E(z) A \\
    &= \frac{\sigma A}{\epsilon_0}
\end{align*}
Then for any area $A$:
\begin{equation*}
    \vec{E}(z) =
    \begin{cases}
        \frac{\sigma}{2\epsilon_0} \vec{e_z} & z > d \\
        -\frac{\sigma}{2\epsilon_0} \vec{e_z} & z < -d
    \end{cases}
\end{equation*}
In the case $d \to 0$, we obtain the electric field of a surface charge $\sigma$ per unit area. This corresponds to $\rho = \sigma \delta(z)$, and results in a uniform electric field. However, there exists a discontinuity at $z = 0$.
\subsection{Surface Charge and Discontinuity}
Let $\vec{n}$ be a unit vector normal to the charged surface, pointing from region 1 to region 2. Then the discontinuity in the electric field is given by:
\begin{equation}
    [\vec{n} \cdot \vec{E}] = \frac{\sigma}{\epsilon_0}
    \label{eqnSurfaceChargeDcty}
\end{equation}
where $\sigma$ is the surface charge per unit area and $[X] = X_2 - X_1$ denotes a discontinuity.

The tangential components of the electric field are continuous:
\begin{equation}
    [\vec{n} \times \vec{E}] = \vec0
    \label{eqnSurfaceChargeCty}
\end{equation}
Note that equations \ref{eqnSurfaceChargeDcty} and \ref{eqnSurfaceChargeCty} hold for more general surfaces. For \eqnref{eqnSurfaceChargeDcty}, this can be seen by considering a small area $\delta A$ on the surface, and then applying the above ideas to this small area.

For \eqnref{eqnSurfaceChargeCty}, consider a rectangle with two sides parallel to the surface (at heights $h$ and $-h$ above it) and two sides intersecting the surface orthogonally.
Use \eqnref{eqnMTEC} and Stokes' Theorem:
\begin{equation*}
    \oint_C \vec{E} \cdot \vec{dx} = \int_{S_C} \nabla \times \vec{E} \cdot \vec{dS} = 0
\end{equation*}
and therefore $[\vec{E}\cdot \vec{dS}] = 0$. Apply the same idea to a rectangle turned $\frac\pi2$ radians about $\vec{n}$ to get the other component of $[\vec{E} \times \vec{n}] = 0$.
\section{Electrostatic Potential}
For general $\rho(\vec{x})$ we cannot determine $\vec{E}(\vec{x})$ using Gauss's Law alone. \Eqnref{eqnMTEC} implies that $\vec{E}$ can be written in terms of a potential:
\begin{definition}{Electrostatic potential}
    The \underline{electrostatic potential} or \underline{electric potential} is a scalar field $\Phi(\vec{x})$ defined by:
    \begin{equation}
        \vec{E}(\vec{x}) = - \nabla \Phi(\vec{x})
        \label{eqnElecPotDef}
    \end{equation}
\end{definition}
\begin{definition}{Potential difference}
    The \underline{potential difference} (or \underline{voltage}) between two points $\vec{x_1}$ and $\vec{x_2}$ is:
    \begin{align*}
        \Phi(\vec{x_2}) - \Phi(\vec{x_1}) &= \int d\Phi \\
        &= -\int_{\vec{x_1}}^{\vec{x_{2}}} \vec{E} \cdot \vec{dx}
    \end{align*}
\end{definition}
The electric force on a particle of charge $q$ is:
\begin{equation*}
    \vec{F} = q\vec{E} = -q \nabla \Phi
\end{equation*}
and therefore this is a conservative force associated with the \underline{potential energy}:
\begin{equation*}
    U(\vec{x}) = q\Phi(\vec{x})
\end{equation*}

\Eqnref{eqnMTED} implies that $\Phi$ satisfies Poisson's equation:
\begin{equation*}
    -\nabla^2 \Phi = \frac{\rho}{\epsilon_0}
\end{equation*}
Including boundary conditions $\Phi(\vec{x}) \to 0$ as $|\vec{x}| \to \infty$, we get the solution:
\begin{equation}
    \Phi(\vec{x}) = \frac{1}{4\pi \epsilon_0} \int_{\R^3} \frac{\rho(\vec{y})}{|\vec{x} - \vec{y}|} d^3 \vec{y}
    \label{eqnPoissonIntegralSoln}
\end{equation}
For the derivation, see the course IB Methods. This is the convolution of $\rho(\vec{x})$ with the potential of a unit point charge, $[4\pi \epsilon_0 |\vec{x}|]^{-1}$, which is the solution of:
\begin{equation*}
    -\nabla^2 \Phi = \frac{\delta(\vec{x})}{\epsilon_0}
\end{equation*}
\begin{remark}
    $\vec{E}$ is unaffected if we add an arbitrary constant to $\Phi$. We usually choose this such that $\Phi \to 0$ as $|\vec{x}| \to \infty$ %TODO: Clarify the remark above that says the same thing.
    If $\rho(\vec{x})$ does not decay sufficiently rapidly, this may not be possible, as in the case of a line charge where $E_r \propto r^{-1}$ and $\Phi \propto \log(r)$.
\end{remark}
\subsection{Point Charge}
The potential due to a point charge at the origin is:
\begin{equation*}
    \Phi(\vec{x}) = \frac{q}{4\pi \epsilon_0 |\vec{x}|} = \frac{q}{4 \pi \epsilon_0 r}
\end{equation*}
\subsection{Electric Dipole}
Consider two equal and opposite charges at different positions. Without loss of generality, let a charge $-q$ be at $\vec{x} = \vec0$, and a charge $q$ be at $\vec{x} = \vec{d}$.
The potential due to the dipole is then:
\begin{equation*}
    \Phi(\vec{x}) = \frac{q}{4\pi \epsilon_0} \left(-\frac{1}{|\vec{x}|} + \frac{1}{|\vec{x} - \vec{d}|}\right)
\end{equation*}
In the case $\vec{d} \to \vec0$, we apply Taylor's Theorem for a scalar field,
\begin{equation*}
    f(\vec{x} + \vec{h}) = f(\vec{x}) + (\vec{h} \cdot \nabla)f(\vec{x}) + \frac12 (\vec{h} \cdot \nabla)^2 f(\vec{x}) + O(|\vec{h}|^3)
\end{equation*}
Then this gives:
\begin{align*}
    \Phi(\vec{x}) &= \frac{q}{4 \pi \epsilon_0} \left(-\frac{1}{r} + \frac{1}{r} - (\vec{d} \cdot \nabla) \frac1r + O(|\vec{d}|^3)\right) \\
    &= \frac{q}{4\pi \epsilon_0} \frac{\vec{d} \cdot \vec{x}}{|\vec{x}|^3} + O(|\vec{d}|^2)
\end{align*}
Now take $\lim_{|\vec{d}| \to 0}$ but with $q \vec{d}$ finite. We obtain a \underline{point dipole} with\\\underline{electric dipole moment}:
\begin{equation*}
    \vec{p} = q\vec{d}.
\end{equation*}
Its potential is:
\begin{equation*}
    \Phi(\vec{x}) = \frac{\vec{p} \cdot \vec{x}}{4\pi \epsilon_0 |\vec{x}|^3}
\end{equation*}
Then the electric field generated by this is:
\begin{equation*}
    \vec{E} = - \nabla \Phi = \frac{4(\vec{p} \cdot \vec{x}) \vec{x} - |\vec{x}|^2 \vec{p}}{4\pi \epsilon_0 |\vec{x}|^5} \\
\end{equation*}
We may wish to consider spherical polar coordinates, aligned with $\vec{p} = p \vec{e_z}$.
\begin{align*}
    \Phi &= \frac{p \cos(\theta)}{4\pi \epsilon_0 r^2} \\
    E_r &= -\frac{\partial \Phi}{\partial r} = \frac{2p\cos\theta}{4\pi\epsilon_0 r^3} \\
    E_\theta &= -\frac1r\frac{\partial \Phi}{\partial \theta} = \frac{p\sin\theta}{4\pi\epsilon_0 r^3} \\
    E_\phi &= 0
\end{align*}
\begin{remarks}
    \item $\Phi$ and $\vec{E}$ are not spherically symmetric.
    \item They decrease more rapidly with $r$ than for a point charge.
\end{remarks}
The charge distribution of the point dipole can be written in terms of delta functions. For a point dipole at the origin,
\begin{align*}
    \rho(\vec{x}) &= -\vec{p} \cdot \nabla \delta(\vec{x}) \\
    \Phi(\vec{x}) &= \vec{p} \cdot \nabla \left(\frac{1}{4\pi \epsilon_0 |\vec{x}|}\right)
\end{align*}
\subsection{Field Lines and Equipotentials}
\begin{definition}{Electric field line}
    \underline{Electric field lines} are the integral curves of $\vec{E}$, the curves that are everywhere tangent to $\vec{E}$.
\end{definition}
\begin{remarks}
    \item Since $\nabla \cdot \vec{E} = \frac{\rho}{\epsilon_0}$, field lines begin at positive charges and end at negative charges.
    \item In electrostatics, $\vec{E} = -\nabla \Phi$, so the field lines are perpendicular to the \underline{equipotential surfaces} $\Phi(\vec{x}) = \text{const}$.
\end{remarks}
\subsection{Dipole in an External Field}
Consider a dipole $\vec{p}$ in an external electric field $\vec{E} = -\nabla \Phi$ generated by distant charges. With $-q$ at $\vec{x}$ and $q$ at $\vec{x} + \vec{d}$, the potential energy of the dipole due to the external field is:
\begin{equation*}
    U = -q \Phi(\vec{x}) + q \Phi(\vec{x} + \vec{d})
\end{equation*}
When $\vec{d}$ is small, we find:
\begin{align*}
    U &=\lim_{\vec{d} \to \vec0} q (\vec{d} \cdot \nabla) \Phi(\vec{x}) + O(|\vec{d}|^2) \\
    &= \vec{p} \cdot \nabla \Phi = -\vec{p} \cdot \vec{E}
\end{align*}
Then this energy is minimised when the electric dipole is aligned along $\vec{E}$.
\subsection{Multipole Expansion}
For a general charge distribution $\rho(\vec{x})$, confined to a ball $V$ of radius $R$ centred on the origin.
\begin{equation*}
    \Phi(\vec{x}) = \frac{1}{4\pi \epsilon_0} \int_{|\vec{x}| < R} \frac{\rho(\vec{y})}{|\vec{x} - \vec{y}|} d^3 \vec{y}
\end{equation*}
Then consider the external potential for when $|\vec{x}| < R$. Consider an expansion:
\begin{align*}
    \frac{1}{|\vec{x} - \vec{y}|} &= \frac1r - (\vec{y} \cdot \nabla)\frac1r + \frac12 (\vec{y} \cdot \nabla)^2 \frac1r + O(|\vec{y}|^3) \\
    &= \frac1r \left[1 + \frac{\vec{y} \cdot \vec{x}}{r^2} + \frac{3(\vec{y} \cdot \vec{x}) - |\vec{y}|^2 |\vec{x}|^2}{2r^4} + O\left(\frac{R^3}{r^3}\right)\right]
\end{align*} 
Then this leads to the \underline{multipole expansion} fo the potential, as seen in Ex1Q11.
\begin{equation*}
    \Phi(\vec{x}) = \frac{1}{4\pi \epsilon_0} \left(\frac{Q}{r} + \frac{\vec{p} \cdot \vec{x}}{r^3} + \frac12 \frac{Q_{ij} x_i x_j}{r^5} + \cdots\right)
\end{equation*}
\end{document}