\documentclass[../Main.tex]{subfiles}

\begin{document}
In a time-independent situations, Maxwell's equations reduce to:
\begin{align}
    \nabla \cdot \vec{E} &= \frac{\rho}{\epsilon_0} \label{eqnMTED} \\
    \nabla \cdot \vec{B} &= 0 \label{eqnMTBD} \\
    \nabla \times \vec{E} &= 0 \label{eqnMTEC} \\
    \nabla \times \vec{B} &= \mu_0 \vec{J} \label{eqnMTBC}
\end{align}
Now since $\vec{E}$ and $\vec{B}$ are decoupled, we can study them separately.

\underline{Electrostatics} is the study of the electric field generated by a stationary charge distribution. We are concerned with equations~\ref{eqnMTED} and \ref{eqnMTEC}.
\section{Gauss's Law}
Consider a closed surface $S$ enclosing a volume $V$. Integrating \eqnref{eqnMTED} over $V$ gives:
\begin{equation*}
    \int_V \nabla \cdot \vec{E}~dV = \int_V\frac{\rho}{\epsilon_0}~dV
\end{equation*}
Then using the divergence theorem, and the notation $Q_V = \int_V \rho~dV$,
\begin{equation}
    \int_S \vec{E} \cdot \vec{dS} = \frac{Q_v}{\epsilon_0}
    \label{eqnGaussLaw}
\end{equation}
This is the integral version of equation~\ref{eqnMED} and is also valid in time-dependent situations. It says that the electric flux is proportional to the total charge enclosed.

In special situations, we can use \eqnref{eqnGaussLaw} together with symmetry to deduce $\vec{E}$ from $\rho$, by choosing a suitable Gaussian surface $S$.
\subsection{Spherical Symmetry}
Use spherical polar coordinates and consider the case $\rho = \rho(r)$, with total charge $Q$ contain within a radius $R$.

We then assume the electric field is spherically symmetric, so it has the form $\vec{E} = E(r) \vec{e_r}$. This automatically satisfies \eqnref{eqnMTEC}. We now apply \eqnref{eqnGaussLaw} to a sphere of radius $r$. First, for $r > R$:
\begin{align*}
    \int_S \vec{E} \cdot \vec{dS} &= E(r) \int_S \vec{e_r} \cdot \vec{dS} \\
    &= E(r) \int_s dS \\
    &= E(r) 4\pi r^2 \\
    &= \frac{Q}{\epsilon_0}
\end{align*}
Therefore for $r > R$,
\begin{equation*}
    \vec{E} = \frac{Q}{4\pi \epsilon_0 r^2} \vec{e_r}
\end{equation*}
Then considering a test particle of charge $q$ in this field, this will experience a Lorentz force:
\begin{equation*}
    \vec{F} = q\vec{E} = \frac{Qq}{4\pi \epsilon_0 r^2} \vec{e_r}
\end{equation*}
In the limit $R \to 0$, we obtain the electric field of a point charge at the origin. This corresponds to a charge density $\rho = Q \delta(\vec{x})$.

\begin{remark}
    There is a close analogy between the Coulomb force and the gravitational force between particles with mass,
    \begin{equation*}
        \vec{F} = -\frac{GMm}{r^2} \vec{e_r}
    \end{equation*}
    Both involve an inverse square law and the product of the charges/masses. However, we have two key differences:
    \begin{enumerate}
        \item while gravity is always attractive, electric forces can be attractive or repulsive depending on the sign of the charges;
        \item gravity is far weaker, for a proton-proton interaction the electric force is $10^{36}$ times stronger.
    \end{enumerate}
\end{remark}
\subsection{Cylindrical symmetry}
Consider a cylindrically symmetric charge distribution in spherical polar coordinates, $\rho = \rho(r)$ and $\vec{E} = E(r) \vec{e_r}$. Let the total charge be $\lambda$ per unit length, contained within an outer cylindrical radius $R$.

Consider a Gaussian surface of radius $R$ and length $L$.
\begin{align*}
    \int_S \vec{E} \cdot \vec{dS} &= E(r) \int_{\text{curved part}} dS \\
    &= 2\pi r L E(r) \\
    &= \frac{\lambda L}{\epsilon_0}
\end{align*}
Then for any $L$, 
\begin{equation*}
    \vec{E} = \frac{\lambda}{2\pi \epsilon_0 r} \vec{e_r}
\end{equation*}
We can consider $R \to 0$, where $\rho$ defines a line charge $\rho(r) = \lambda \delta(x)\delta(y)$
\subsection{Planar Symmetry}
Consider a planar charge distribution $\rho = \rho(z)$ in Cartesian coordinates. Let the total charge be $\sigma$ per unit area, contained within a region $-d<z<d$. Assume reflectional symmetry about $z = 0$. That is, $\rho(z)$ is an even function. Now we have $\vec{E} = E(z) \vec{e_z}$. Because $\vec{E}$ is a divergence, differentiation an even function gives an odd function, so $E(-z) = -E(z)$.

We apply Gauss's law to a ``Gaussian pillbox'' of height $2z$ and over an arbitrary area $A$. For $z > d$,
\begin{align*}
    \int_S \vec{E} \cdot \vec{dS} &= E(z) A = E(-z) A \\
    &= 2E(z) A \\
    &= \frac{\sigma A}{\epsilon_0}
\end{align*}
Then for any area $A$:
\begin{equation*}
    E(z) =
    \begin{cases}
        \frac{\sigma}{2\epsilon_0} \vec{e_z} & z > d \\
        -\frac{\sigma}{2\epsilon_0} \vec{e_z} & z < -d
    \end{cases}
\end{equation*}
In the case $d \to 0$, we obtain the electric field of a surface charge $\sigma$ per unit area. This corresponds to $\rho = \sigma \delta(z)$, and results in a uniform electric field. However, there exists a discontinuity at $z = 0$.
\subsection{Surface Charge and Discontinuity}
Let $\vec{n}$ be a unit vector normal to the charged surface, pointing from region 1 to region 2. Then the discontinuity in the electric field is given by:
\begin{equation}
    [\vec{n} \cdot \vec{E}] = \frac{\sigma}{\epsilon_0}
    \label{eqnSurfaceChargeDcty}
\end{equation}
where $\sigma$ is the surface charge per unit area and $[X] = X_2 - X_1$ denotes a discontinuity.
\end{document}