\documentclass[../Main.tex]{subfiles}

\begin{document}
\underline{Magnetostatics} is the study of the magnetic field generated by a stationary current distribution.

Recall the two equations for the magnetic field in time-independent situations:
\begin{align*}
    \nabla \cdot \vec{B} &= 0 \tag{\ref{eqnMTBD}} \\
    \nabla \times \vec{B} &= \mu_0 \vec{J} \tag{\ref{eqnMTBC}}
\end{align*}

\Eqnref{eqnMTBC} implies that $\nabla \cdot \vec{J} = 0$. This is the time-independent equation of charge conservation.

\section{Ampere's Law}
Consider a closed curve $C$ that is the boundary of an open surface $S$. Integrate \eqnref{eqnMTBC} over $S$ and apply Stokes's theorem to obtain Ampere's Law:
\begin{equation}
    \int_C \vec{B} \cdot \vec{dx} = \mu_0 \int_S \vec{J} \cdot \vec{dS} = \mu_0 I.
    \label{eqnAmpere}
\end{equation}
Since $\nabla \cdot \vec{J} = 0$, the same current $I$ flows through \underline{any} open surface $S$ such that $\partial S = C$.

Ampere's Law is the integral version of \eqnref{eqnMTBC}. It is valid as long as $\frac{\partial \vec{E}}{\partial t} = 0$. In words, Ampere's Law says that the circulation of a magnetic field around a loop is proportional to the total current through that loop.

In situations with certain symmetries, we can use Ampere's Law to deduce $\vec{B}$ from $\vec{J}$.
A cylindrically symmetrical situation could involve:
\begin{itemize}
    \item an axial current distribution, $\vec{J}(\vec{x}) = J_z(r) \vec{e_z}$;
    \item an azimuthal current distribution, $\vec{J}(\vec{x}) = J_\phi(r) \vec{e_\phi}$;
    \item a combination of the above, in which case we can superpose solutions.
\end{itemize}
Since the curl of $\vec{B}$ is proportional to $\vec{J}$, in these situations we have $B_\phi \propto J_z$ and $B_z \propto J_\phi$.
\subsection{Long Straight Wire}
First consider a long straight wire. We will model this as a cylinder of radius $R$ with infinite length. Let the current through it be $I$, travelling parallel to its axis.

To find $B_\phi(r)$ generated by $J_z(r)$, apply Ampere's Law to a circle $C$ of radius $r$. Choose the simplest surface for $S$: a flat disc with boundary $C$. Consider first $r > R$:
\begin{align*}
    \int_C \vec{B} \cdot \vec{dx} &= B_\phi(r) \int_C \vec{e_\phi} \cdot \vec{dx} \\
    &= B_\phi(r) \int_C dl = 2\pi r B_\phi(r) \\
    &= \mu_0 I
\end{align*}
Therefore, outside the wire:
\begin{equation*}
    \vec{B} = \frac{\mu_0 I}{2\pi r} \vec{e_\phi}
\end{equation*}
\begin{remarks}
    \item This is independent of the way that the current flows inside the wire.
    \item This decays with $r^{-1}$ moving away from the wire.
\end{remarks}
\subsection{Solenoid}
Consider a thin wire coiled around a cylindrical tube of radius $R$. An \textit{ideal solenoid} is infinitely long and tightly wound, having cylindrical symmetry and purely azimuthal current.

The wire carries current $I$ and has $N$ turns per unit length of the tube.

\begin{figure}
    \centering
    \begin{tikzpicture}[scale=1]
        \def\h{6} % Cylinder height
        \def\r{1} % Cylinder radius
        \pgfmathsetmacro{\d}{\r * 2} % Cylinder diammeter
        \def\rSkew{0.3} % Front-back radius for 3D skew
        \def\a{1.5}
        \def\b{2.5}
        \def\cBottom{2}
        \def\cTop{4}

        \draw (0, 0) -- (0, \h);
        \draw (\d, 0) -- (\d, \h);
        \draw (0, 0) arc[x radius=\r, y radius=\rSkew, start angle=180, end angle=360];
        \draw[dashed] (\d, 0) arc[x radius=\r, y radius=\rSkew, start angle=0, end angle=180];
        \draw (\r, \h) ellipse[x radius=\r, y radius=\rSkew]
            -- (\d, \h) node[pos=0.5, label=north west:$R$] {};
        
        \foreach \y in {2.5, 2.6, ..., 3.9} {
            \draw (0, \y) arc[x radius=\r, y radius = \rSkew, start angle=180, end angle=360];
        }

        \draw[color=blue] (\a, \cBottom) -- (\a, \cTop) -- (\b, \cTop)
            node[pos=0.7, anchor=south] {$C$} -- (\b, \cBottom)
            node[pos=0.5, anchor=east] {$S$} -- cycle;
        
        \draw[<->] ($(\r, 0.5) + (0, \cTop)$) -- ($(\a, 0.5) + (0, \cTop)$)
            node[pos=0.5, anchor=south] {$a$};
        
        \draw[<->] ($(\r, 1) + (0, \cTop)$) -- ($(\b, 1) + (0, \cTop)$)
            node[pos=0.5, anchor=south] {$b$};
        
        \draw[<->] ($(\b, \cBottom) + (0.5, 0)$) -- ($(\b, \cTop) + (0.5, 0)$) 
            node[pos=0.5, anchor=west] {$L$};
    \end{tikzpicture}
    \caption{Diagram of a Solenoid}
    \label{figSolenoid}
\end{figure}

Let $C$ be rectangular, as in figure~\ref{figSolenoid}. Taking $a < b < R$ or $R < a < b$ gives:
\begin{equation*}
    L(B_z(a) - B_z(b)) = 0
\end{equation*}
And so the magnetic field is the same either side of the solenoid.

However, taking $a < R < b$ (as depicted in figure~\ref{figSolenoid}) gives:
\begin{equation*}
    L(b_z(a) - B_z(b)) = \mu_0 N L I
\end{equation*}
Assuming $B_z(r) \to 0$ as $r \to \infty$, we deduce that:
\begin{equation*}
    B_z(r) =
    \begin{cases}
        \mu_0 N I & r < R \\
        0 & r > R
    \end{cases}
\end{equation*}
The ideal solenoid is an example of a surface current, here of the form:
\begin{equation*}
    J_\phi(r) = K_\phi\delta(r - R)
\end{equation*}
with $K_\phi = NI$. Generally, a surface current density $\vec{K}$ produces a discontinuity in the tangential magnetic field:
\begin{equation*}
    [\vec{n} \times \vec{b}] = \mu_0 \vec{K}
\end{equation*}
This follows in the general case from Ampere's Law applied to an infinitesimally small rectangle perpendicular to the surface. We see that the normal component is continuous by \eqnref{eqnMTBD}.
\section{Magnetic Vector Potential}
\Eqnref{eqnMTBD} implies that the magnetic field can be written in terms of a magnetic vector potential $\vec{A}(\vec{x})$:
\begin{equation*}
    \vec{B} = \nabla \times \vec{A}
\end{equation*}
Note that $\vec{A}$ is not unique. Considering a \underline{gauge transformation} $\tilde{\vec{A}} = \vec{A} + \nabla \chi$ gives the same magnetic field $\vec{B}$. A convenient gauge for many calculations is the \underline{Coulomb gauge} in which $\nabla \cdot \vec{A} = 0$.

We can choose Coulomb gauge by taking $\chi$ in the above to solve $-\nabla^2 \chi = \nabla \cdot \vec{A}$. Then in terms of $\vec{A}$, \eqnref{eqnMTBC} is:
\begin{align}
    \nabla \times (\nabla \times \vec{A}) &= \mu_0 \vec{J} \nonumber \\
    -\nabla^2 \vec{A} &= \mu_0 \vec{J} \label{eqnMagnetPoisson}
\end{align}
using the identity $\nabla \times (\nabla \times \vec{A}) = \nabla(\nabla \cdot \vec{A}) - \nabla^2 \vec{A}$ and Coulomb gauge.
\section{Biot-Savat Law}
The solution of Poisson's Equation over all space with $\vec{J}$ decaying to zero far from $\vec0$ is:
\begin{equation}
    \vec{A}(\vec{x}) = \frac{\mu_0}{4\pi} \int_{\R^3} \frac{\vec{J}(\vec{y})}{|\vec{x} - \vec{y}|} d^3 \vec{y}
    \label{eqnMagnetSoln}
\end{equation}
Compare this with the solution for the electric potential in \eqnref{eqnPoissonIntegralSoln}. We see that the constant changes from $\frac{1}{\epsilon_0}$ to $\mu_0$, and this is a vector equation (where the solution for each individual component is a separate copy of \eqnref{eqnPoissonIntegralSoln} with the constants changed).
\end{document}