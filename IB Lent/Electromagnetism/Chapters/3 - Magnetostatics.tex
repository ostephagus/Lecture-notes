\documentclass[../Main.tex]{subfiles}

\begin{document}
\underline{Magnetostatics} is the study of the magnetic field generated by a stationary current distribution.

Recall the two equations for the magnetic field in time-independent situations:
\begin{align*}
    \nabla \cdot \vec{B} &= 0 \tag{\ref{eqnMTBD}} \\
    \nabla \times \vec{B} &= \mu_0 \vec{J} \tag{\ref{eqnMTBC}}
\end{align*}

\Eqnref{eqnMTBC} implies that $\nabla \cdot \vec{J} = 0$. This is the time-independent equation of charge conservation.

\section{Ampere's Law}
Consider a closed curve $C$ that is the boundary of an open surface $S$. Integrate \eqnref{eqnMTBC} over $S$ and apply Stokes's theorem to obtain Ampere's Law:
\begin{equation}
    \int_C \vec{B} \cdot \vec{dx} = \mu_0 \int_S \vec{J} \cdot \vec{dS} = \mu_0 I.
    \label{eqnAmpere}
\end{equation}
Since $\nabla \cdot \vec{J} = 0$, the same current $I$ flows through \underline{any} open surface $S$ such that $\partial S = C$.

Ampere's Law is the integral version of \eqnref{eqnMTBC}. It is valid as long as $\frac{\partial \vec{E}}{\partial t} = 0$. In words, Ampere's Law says that the circulation of a magnetic field around a loop is proportional to the total current through that loop.

In situations with certain symmetries, we can use Ampere's Law to deduce $\vec{B}$ from $\vec{J}$.
A cylindrically symmetrical situation could involve:
\begin{itemize}
    \item an axial current distribution, $\vec{J}(\vec{x}) = J_z(r) \vec{e_z}$;
    \item an azimuthal current distribution, $\vec{J}(\vec{x}) = J_\phi(r) \vec{e_\phi}$;
    \item a combination of the above, in which case we can superpose solutions.
\end{itemize}
Since the curl of $\vec{B}$ is proportional to $\vec{J}$, in these situations we have $B_\phi \propto J_z$ and $B_z \propto J_\phi$.
\subsection{Long Straight Wire}
First consider a long straight wire. We will model this as a cylinder of radius $R$ with infinite length. Let the current through it be $I$, travelling parallel to its axis.

To find $B_\phi(r)$ generated by $J_z(r)$, apply Ampere's Law to a circle $C$ of radius $r$. Choose the simplest surface for $S$: a flat disc with boundary $C$. Consider first $r > R$:
\begin{align*}
    \int_C \vec{B} \cdot \vec{dx} &= B_\phi(r) \int_C \vec{e_\phi} \cdot \vec{dx} \\
    &= B_\phi(r) \int_C dl = 2\pi r B_\phi(r) \\
    &= \mu_0 I
\end{align*}
Therefore, outside the wire:
\begin{equation*}
    \vec{B} = \frac{\mu_0 I}{2\pi r} \vec{e_\phi}
\end{equation*}
\begin{remarks}
    \item This is independent of the way that the current flows inside the wire.
    \item This decays with $r^{-1}$ moving away from the wire.
\end{remarks}
\subsection{Solenoid}
Consider a thin wire coiled around a cylindrical tube of radius $R$. An \textit{ideal solenoid} is infinitely long and tightly wound, having cylindrical symmetry and purely azimuthal current.

The wire carries current $I$ and has $N$ turns per unit length of the tube.

\begin{figure}
    \centering
    \begin{tikzpicture}[scale=1]
        \def\h{6} % Cylinder height
        \def\r{1} % Cylinder radius
        \pgfmathsetmacro{\d}{\r * 2} % Cylinder diammeter
        \def\rSkew{0.3} % Front-back radius for 3D skew
        \def\a{1.5}
        \def\b{2.5}
        \def\cBottom{2}
        \def\cTop{4}

        \draw (0, 0) -- (0, \h);
        \draw (\d, 0) -- (\d, \h);
        \draw (0, 0) arc[x radius=\r, y radius=\rSkew, start angle=180, end angle=360];
        \draw[dashed] (\d, 0) arc[x radius=\r, y radius=\rSkew, start angle=0, end angle=180];
        \draw (\r, \h) ellipse[x radius=\r, y radius=\rSkew]
            -- (\d, \h) node[pos=0.5, label=north west:$R$] {};
        
        \foreach \y in {2.5, 2.6, ..., 3.9} {
            \draw (0, \y) arc[x radius=\r, y radius = \rSkew, start angle=180, end angle=360];
        }

        \draw[color=blue] (\a, \cBottom) -- (\a, \cTop) -- (\b, \cTop)
            node[pos=0.7, anchor=south] {$C$} -- (\b, \cBottom)
            node[pos=0.5, anchor=east] {$S$} -- cycle;
        
        \draw[<->] ($(\r, 0.5) + (0, \cTop)$) -- ($(\a, 0.5) + (0, \cTop)$)
            node[pos=0.5, anchor=south] {$a$};
        
        \draw[<->] ($(\r, 1) + (0, \cTop)$) -- ($(\b, 1) + (0, \cTop)$)
            node[pos=0.5, anchor=south] {$b$};
        
        \draw[<->] ($(\b, \cBottom) + (0.5, 0)$) -- ($(\b, \cTop) + (0.5, 0)$) 
            node[pos=0.5, anchor=west] {$L$};
    \end{tikzpicture}
    \caption{Diagram of a Solenoid}
    \label{figSolenoid}
\end{figure}

Let $C$ be rectangular, as in figure~\ref{figSolenoid}. Taking $a < b < R$ or $R < a < b$ gives:
\begin{equation*}
    L(B_z(a) - B_z(b)) = 0
\end{equation*}
And so the magnetic field is the same either side of the solenoid.

However, taking $a < R < b$ (as depicted in figure~\ref{figSolenoid}) gives:
\begin{equation*}
    L(b_z(a) - B_z(b)) = \mu_0 N L I
\end{equation*}
Assuming $B_z(r) \to 0$ as $r \to \infty$, we deduce that:
\begin{equation*}
    B_z(r) =
    \begin{cases}
        \mu_0 N I & r < R \\
        0 & r > R
    \end{cases}
\end{equation*}
The ideal solenoid is an example of a surface current, here of the form:
\begin{equation*}
    J_\phi(r) = K_\phi\delta(r - R)
\end{equation*}
with $K_\phi = NI$. Generally, a surface current density $\vec{K}$ produces a discontinuity in the tangential magnetic field:
\begin{equation*}
    [\vec{n} \times \vec{b}] = \mu_0 \vec{K}
\end{equation*}
This follows in the general case from Ampere's Law applied to an infinitesimally small rectangle perpendicular to the surface. We see that the normal component is continuous by \eqnref{eqnMTBD}.
\section{Magnetic Vector Potential}
\Eqnref{eqnMTBD} implies that the magnetic field can be written in terms of a magnetic vector potential $\vec{A}(\vec{x})$:
\begin{equation*}
    \vec{B} = \nabla \times \vec{A}
\end{equation*}
Note that $\vec{A}$ is not unique. Considering a \underline{gauge transformation} $\tilde{\vec{A}} = \vec{A} + \nabla \chi$ gives the same magnetic field $\vec{B}$. A convenient gauge for many calculations is the \underline{Coulomb gauge} in which $\nabla \cdot \vec{A} = 0$.

We can choose Coulomb gauge by taking $\chi$ in the above to solve $-\nabla^2 \chi = \nabla \cdot \vec{A}$. Then in terms of $\vec{A}$, \eqnref{eqnMTBC} is:
\begin{align}
    \nabla \times (\nabla \times \vec{A}) &= \mu_0 \vec{J} \nonumber \\
    -\nabla^2 \vec{A} &= \mu_0 \vec{J} \label{eqnMagnetPoisson}
\end{align}
using the identity $\nabla \times (\nabla \times \vec{A}) = \nabla(\nabla \cdot \vec{A}) - \nabla^2 \vec{A}$ and Coulomb gauge.
\section{Biot-Savat Law}
The solution of Poisson's Equation over all space with $\vec{J}$ decaying to zero far from $\vec0$ is:
\begin{equation}
    \vec{A}(\vec{x}) = \frac{\mu_0}{4\pi} \int_{\R^3} \frac{\vec{J}(\vec{y})}{|\vec{x} - \vec{y}|} d^3 \vec{y}
    \label{eqnMagnetSoln}
\end{equation}
Compare this with the solution for the electric potential in \eqnref{eqnPoissonIntegralSoln}. We see that the constant changes from $\frac{1}{\epsilon_0}$ to $\mu_0$, and this is a vector equation (where the solution for each individual component is a separate copy of \eqnref{eqnPoissonIntegralSoln} with the constants changed).

We should check that the solution satisfies the assumed Coulomb gauge condition:
\begin{align*}
    \nabla \cdot \vec{A}(\vec{x}) &= \frac{\mu_0}{4\pi} \int_V \nabla_{\vec{x}} \cdot \left(\frac{\vec{J}(\vec{y})}{|\vec{x} - \vec{y}|}\right) d^3 \vec{y} \\
    &= \frac{\mu_0}{4\pi} \int_V \vec{J}(\vec{y}) \cdot \nabla_{\vec{x}} \left(\frac{1}{|\vec{x} - \vec{y}|}\right) d^3 \vec{y} \\
    &= -\frac{\mu_0}{4\pi} \int_V \vec{J}(\vec{y}) \cdot \nabla_{\vec{y}} \left(\frac{1}{|\vec{x} - \vec{y}|}\right) d^3 \vec{y} \\
    &= -\frac{\mu_0}{4\pi} \int_V \nabla_{\vec{y}} \cdot \left(\frac{\vec{J}(\vec{y})}{|\vec{x} - \vec{y}|}\right) d^3 \vec{y} \quad\text{since $\nabla \cdot \vec{J} = 0$} \\
    &= -\frac{\mu_0}{4\pi} \int_{\partial V} \frac{\vec{J}(\vec{y}) \cdot \vec{dS}}{|\vec{x} - \vec{y}|} \\
\end{align*}
Then if the current is contained within a finite volume, we take $\vec{V}$ to be larger than this and so $\nabla \cdot \vec{A} = 0$. If instead the current is not contained within a finite region, we require that it decays sufficiently fast.

To find the magnetic field, we derive $\nabla \times \vec{A}$ from \eqnref{eqnMagnetSoln}.
\begin{equation}
    \vec{B}(\vec{x}) \frac{\mu_0}{4\pi} \int_{\R^3} \frac{\vec{J}(\vec{y}) \times (\vec{x} - \vec{y})}{|\vec{x} - \vec{y}|^3} d^3 \vec{y}
    \label{eqnBiotSavat}
\end{equation}
This is the \underline{Biot-Savat law}, giving the magnetic field generated by a stationary current distribution.

A special case is when the current is restricted to a thin wire in the form of a curve $C$. Then the current element $\vec{J} d^3\vec{x}$ can be replaced by $I \vec{dx}$ where $I$ is the current in the wire and $\vec{dx}$ is the vector line element parallel to the wire.

Charge conservation requires that $I$ is constant along the wire. We then find that \eqnref{eqnBiotSavat} becomes:
\begin{equation}
    \vec{B}(\vec{x}) = \frac{\mu_0I}{4\pi} \int_C \frac{\vec{dy} \times (\vec{x} - \vec{y})}{|\vec{x} - \vec{y}|^3}
    \label{eqnBiotSavatWire}
\end{equation}
We could also think of the thin wire as having current density given by:
\begin{equation*}
    \vec{J}(\vec{x}) = I \int_C \delta(\vec{x} - \vec{y}) \vec{dy}
\end{equation*}
Charge conservation then takes the form:
\begin{align*}
    \nabla \cdot \vec{J}(\vec{x}) &= I\int_C \nabla_{\vec{x}}\delta(\vec{x} - \vec{y}) \cdot \vec{dy} \\
    &= - I \int_C \nabla_{\vec{y}} \delta(\vec{x} - \vec{y}) \cdot \vec{dy} \\
    &= -I\left[\delta(\vec{x} - \vec{x_2}) - \delta(\vec{x} - \vec{x_1})\right] \\
\end{align*}
where $\vec{x_1}$ and $\vec{x_2}$ are the endpoints of the curve $C$. If $C$ is closed then $\vec{x_1} = \vec{x_2}$ and $\nabla \cdot \vec{J} = 0$. If instead we take an infinite wire, then $\nabla \cdot \vec{J} = 0$ for any finite $\vec{x}$. We could understand the charge conservation equation as describing a source of current at one end and a sink at the other. This makes sense in terms of standard circuits including batteries or other current sources.

We will check that \eqnref{eqnBiotSavatWire} gives the same result as Ampere's Law for a long straight thin wire. Let the axis of the wire be along the $z$ axis, and use cylindrical polar coordinates.

We have $\vec{x} = r\vec{e_r}$ and we can take $z = 0$ without loss of generality. For an arbitrary point on the wire, let $\vec{y} = z \vec{e_z}$. Therefore $\vec{x} - \vec{y} = r\vec{e_r} - z\vec{e_z}$. By \eqnref{eqnBiotSavatWire},
\begin{align*}
    \vec{B}(\vec{x}) &= \frac{\mu_0 I}{4\pi} \vec{e_\phi} \int_{-\infty}^{\infty} \frac{r}{(r^2 + z^2)^{3 / 2}} dz \\
    &= \frac{\mu_0 I}{4\pi} \vec{e_{\phi}} \left[\frac{z'}{r\sqrt{r^2 + z^2}}\right]_{-\infty}^\infty \\
    &= \frac{\mu_0 I}{2\pi r} \vec{e_\phi}
\end{align*}
\section{Magnetic Dipoles}
\subsection{Magnetic Multipole Expansion}
For a general current distribution $\vec{J}(\vec{x})$ confined to a ball $V = B_R(\vec0)$, 
\begin{equation*}
    A(\vec{x}) = \frac{\mu_0}{4\pi} \int_V \frac{\vec{J}(\vec{y})}{|\vec{x} - \vec{y}|} d^3 \vec{y}
\end{equation*}
The external field (where $|\vec{x}| = r > R$) can be found by expanding:
\begin{equation*}
    \frac{1}{|\vec{x} - \vec{y}|} = \frac1r \left(1 + \frac{\vec{x} \cdot \vec{y}}{r^2} + O\left(\frac{R^2}{r^2}\right)\right)
\end{equation*}
and this gives a multipole expansion as with electrostatics. We now need to calculate the moments of the current distribution. Since $\vec{J} = \vec{0}$ on $\partial V$ and $\nabla \cdot \vec{J} = 0$, the divergence theorem implies:
\begin{align*}
    0 &= \int_{\partial V} x_i J_j dS_j \\
    &= \int_{V} \frac{\partial}{\partial x_j} (x_i J_j) d^3 \vec{x} \\
    &= \int_{V} (\delta_{ij} J_j + x_i \frac{\partial J_j}{\partial x_j}) d^3 \vec{x} \\
    &= \int_{V} J_i d^3 \vec{x} \text{ since $\vec{J}$ is divergence-free}
\end{align*}
then this is the zeroth moment in the expansion, and so it is zero. In electrostatics the zeroth moment was the total charge, and since there are no magnetic charges we see this is zero in magnetostatics.

Now we evaluate the first moment:
\begin{align*}
    0 &= \int_{\partial V} x_i x_j J_k dS_k \\
    &= \int_V \frac{\partial}{\partial x_k} \left(x_i x_j J_k\right) d^3 \vec{x} \\
    &= \int_V \left(\delta_{ij} x_j J_k + x_i \delta_{jk} J_k + x_i x_j \frac{\partial J_k}{\partial x_k}\right) d^3 \vec{x} \\
    &= \int_V x_j J_i d^3\vec{x} + \int_V x_i J_j d^3\vec{x}
\end{align*}
and so the first moment is an antisymmetric matrix. Any antisymmetric tensor can be represented using a matrix. This is the \underline{magnetic dipole moment} $\vec{m}$ defined:
\begin{equation*}
    \vec{m} = \frac12 \int_V \vec{x} \times \vec{J} d^3\vec{x}
\end{equation*}
Returning to the multipole expansion for $\vec{A}$,
\begin{align*}
    A_i(\vec{x}) &= \frac{\mu_0}{4\pi |\vec{x}|} \left(\int_V J_i(\vec{y}) d^3 \vec{y} + \frac{x_j}{|\vec{x}|^2} \int_V y_j J_i(\vec{y}) d^3\vec{y} + \cdots\right) \\
    &= \frac{\mu_0}{4\pi |\vec{x}|} \left(0 + \frac{x_j \epsilon_{jik} m_k}{|\vec{x}|^2} + \cdots\right)
\end{align*}
Then the leading approximation is from the magnetic dipole moment:
\begin{equation*}
    \vec{A}(\vec{x}) \approx \vec{A_{dipole}}(\vec{x}) = \frac{\mu_0}{4\pi} \frac{\vec{m} \times \vec{x}}{|\vec{x}|^3}
\end{equation*}
which is the vector potential due to a point dipole $\vec{m}$ at the origin.

We calculate the magnetic field by taking a curl:
\begin{align*}
    \vec{B_{dipole}} &= \nabla \times \vec{A_{dipole}}\\
    &= \frac{\mu_0}{4\pi} \left(\frac{3(\vec{m} \cdot \vec{x}) \vec{x} - |\vec{x}|^2 \vec{m}}{|\vec{x}|^5}\right)
\end{align*}
A point dipole $\vec{m}$ at the origin corresponds to:
\begin{align*}
    \vec{J} &= \nabla \times \left(\vec{m} \delta(\vec{x})\right) \\
    \vec{A} &= \nabla \times \left(\frac{\mu_0 \vec{m}}{4\pi |\vec{x}|}\right)
\end{align*}
\subsection{Examples of Dipoles}
The magnetic dipole moment of a thin wire carrying current $I$ around a closed curve $C$ is:
\begin{equation*}
    \vec{m} = \frac{I}{2} \int_C \vec{x} \times \vec{dx}
\end{equation*}
To evaluate this, let $\vec{a}$ be any constant vector. By Stokes' theorem, consider:
\begin{align*}
    \vec{m} \cdot \vec{a} &= \frac{I}{2} \int_C\vec{a} \cdot (\vec{x} \times \vec{dx}) \\
    &= \frac{I}{2} \int_C (\vec{a} \times \vec{x}) \cdot \vec{dx} \\
    &= \frac{I}{2} \int_S \nabla \times (\vec{a} \times \vec{x}) \cdot \vec{dS} \\
    &= I \int_S \vec{a} \cdot \vec{dS}
\end{align*}
where $S$ is an open surface with boundary $C$. We use the identity:
\begin{align*}
    \nabla \times (\vec{a} \times \vec{x}) &= \vec{x} \cdot \nabla \vec{a} - \vec{a} \cdot \nabla \vec{x} + (\nabla \cdot \vec{x}) \vec{a} - (\nabla \cdot \vec{a}) \vec{x} \\
    &= \vec0 - \vec{a} + 3\vec{a} - \vec0 = 2\vec{a}.
\end{align*}
and now, since $\vec{a}$ is arbitrary, we conclude:
\begin{equation*}
    \vec{m} = I \vec{S}
\end{equation*}
where $\vec{S} = \int_S \vec{n} dS$ is the vector area of $S$. It is a property only of $C = \partial S$, it is independent of the surface used to span $C$.
\begin{example}[Circular current loop]
    Consider a circle defined by $x^2 + y^2 = a^2, z = 0$.

    The magnetic dipole moment is:
    \begin{equation*}
        \vec{m} = I \pi a^2 \vec{e_z}
    \end{equation*}
    Then at a point on the $z$ axis far from the origin, the dipole approximation gives:
    \begin{align*}
        B_z &= \frac{\mu_0}{4\pi} \left(\frac{3m_z z^2 - z^2 m_z}{|z|^5}\right) \\
        &= \frac{\mu_0 I a^2}{2|z|^3}
    \end{align*}
    while the exact solution is (via Ex2Q3):
    \begin{equation*}
        B_z = \frac{\mu_0 I a^2}{2(z^2 + a^2)^{\frac32}}
    \end{equation*}
    so the dipole approximation agrees with this for $z \gg a$.
\end{example}
\subsection{Permanent Magnets}
A bar magnet has a north and south pole and a dipole moment. This comes from the superposition of aligned dipoles on the atomic scale. Atoms contain electrons, which are spinning charged particles with a magnetic dipole moment.

Though spin is a quantum effect, we can think of it in a classical sense by considering the electron as rotating about an axis. This forms an electric current loop, with a magnetic dipole moment aligned along the axis of rotation. The current is proportional to the charge of the particle and its spin. When all the electrons' spin axes are aligned, we get permanent magnets.

As far as we know, there are no magnetic charges (monopoles).

The Earth is another example of a permanent magnet. The liquid iron outer core of the earth is a conducting fluid in convective motion and supports electric currents that generate a magnetic field. At the earth's surface, this resembles a dipole magnetic field.
\section{Magnetic Forces}
The Lorentz force on a particle of charge $q_i$ is at position $\vec{x_i}(t)$ is:
\begin{equation*}
    \vec{F_i}(t) = q_i \left(\vec{E}(\vec{x}(t)) + \dvec{x_i}(t) \times \vec{B}(\vec{x}(t))\right)
\end{equation*}
In the limit of continuous charge and current densities, the Lorentz force per unit volume is then:
\begin{equation}
    \rho \vec{E} + \vec{J} \times \vec{B}
    \label{eqnLorentzPerVolume}
\end{equation}
We can recover the discrete version by substituting:
\begin{align*}
    \rho &= \sum_{i=1}^{n} q_i \delta(\vec{x} - \vec{x_i}(t)) \\
    \vec{J} &= \sum_{i=1}^{n} q_i \dvec{x_i}(t) \delta(\vec{x} - \vec{x_i}(t))
\end{align*}
\subsection{Forces Between Thin Wires}
Consider two or more thin wires with current $I_i$ along curves $C_i$. The total magnetic field is:
\begin{equation*}
    \vec{B} = \sum_{i=1}^{n} \vec{B_i} = \sum_{i=1}^{n} \frac{\mu_0I_i}{4\pi} \int_{C_i} \frac{\vec{dx_i} \times (\vec{x} - \vec{x_i})}{|\vec{x} - \vec{x_i}|^3}
\end{equation*}
The current density is:
\begin{equation*}
    \vec{J} = \sum_{i=1}^{n} J_i(\vec{x}) = \sum_{i=1}^{n} I_i \int_{C_i} \delta(\vec{x} - \vec{x_i}) \vec{dx_i}
\end{equation*}
The total magnetic force acting on a volume $V$ is:
\begin{equation*}
    \vec{F} = \int_V \vec{J} \times \vec{B}
\end{equation*}
then the force acting on wire $i$ is
\begin{align*}
    \vec{F_i} &= \int_{\R^3} \vec{J_i}(\vec{x}) \times \vec{B}(\vec{x}) d^3\vec{x} \\
    &= I_i \int_{C_i} \vec{dx_i} \times \vec{B}(\vec{x_i}) \\
    &= \sum_{j=1}^{n} \vec{F_{ij}}
\end{align*}
where $\vec{F_{ij}}$ is the force on wire $i$ due to wire $j$.
\begin{equation*}
    \vec{F_{ij}} = \frac{\mu_0 I_i I_j}{4\pi} \int_{C_i} \int_{C_j} \vec{dx_i} \times \left(\frac{\vec{dx_j} \times (\vec{x_i} - \vec{x_j})}{|\vec{x_i} - \vec{x_j}|^3}\right) \\
\end{equation*}
and we see in Ex2Q4 that $\vec{F_{ij}} = -\vec{F_{ji}}$.

We cannot use this expression to find the \underline{self-force} $\vec{F_{ii}}$ because we encounter a non-integrable singularity. We can show that in fact $\vec{F_{ii}} = \vec0$ by considering a wire that is not thin, and taking the limit as cross-sectional area tends to 0.
\subsection{Long Parallel Wires}
Consider two infinitely long, thin wires separated by a distance $r$. Consider cylindrical polar coordinates with the axis along wire 2. We know the magnetic field for wire 2:
\begin{equation*}
    \vec{B_2} = \frac{\mu_0 I_2}{2\pi r}\vec{e_\phi}
\end{equation*}
The force on wire 1 from wire 2 is, from the previous subsection,
\begin{equation*}
    \vec{F_{12}} = I_1 \int_{-\infty}^{\infty} \vec{e_z} \times \vec{B_2} dz 
\end{equation*}
however this is infinite. We must instead discuss the force per unit length:
\begin{equation*}
    I_1 \vec{e_z} \times \vec{B_2} = -\frac{\mu_0 I_1 I_2}{2\pi r} \vec{e_r}
\end{equation*}
which depends on the signs of the currents. The force is attractive if the currents have the same sign. Otherwise, the force is repulsive.
\subsection{Force and Torque on a Magnetic Dipole}
Consider a localised current distribution (current loop) confined to a ball $V = B_R(\vec0)$. Place this in an external magnetic field $\vec{B}(\vec{x})$ that varies slowly over the length scale $R$. Then the magnetic torque for this current loop around $\vec0$ is:
\begin{align*}
    \vec{\tau} &= \int_V \vec{x} \times (\vec{J}(\vec{x}) \times \vec{B}(\vec{x})) d^3 \vec{x} \\
    &= \int_V \left((\vec{x} \cdot \vec{B}(\vec{x}))\vec{J}(\vec{x}) - (\vec{x} \cdot \vec{J}(\vec{x}))\vec{B}(\vec{x})\right)
\end{align*}
Within $V$, we expand $\vec{B}$ as a Taylor series:
\begin{equation*}
    B_i(\vec{x}) = B_i(\vec0) + x_j \frac{\partial B_i}{\partial x_j} (\vec{0}) + \cdots
\end{equation*}
Retaining only the zeroth order term (uniform field),
\begin{align*}
    \tau_i &\approx B_j(\vec0) \int_V x_j J_i d^3\vec{x} - B_i(\vec0) \int_V x_j J_j d^3\vec{x} \\
    &= B_j(\vec0) \epsilon_{ijk} m_k
\end{align*}
where we have used here the magnetic dipole moment vector $\vec{m}$.

In general,
\begin{equation*}
    \vec{\tau} = \vec{m} \times \vec{B}
\end{equation*}
where $\vec{B}$ is evaluated at the location of the dipole, and $\vec{\tau}$ is measured about this point.

For the force, we need more terms in the expansion.
\begin{align*}
    \vec{F} &= \int_V \vec{J}(\vec{x}) \times \vec{B}(\vec{x}) d^3 \vec{x} \\
    F_i &\approx \int_V \epsilon_{ijk} J_j(\vec{x}) \left(B_k(\vec0) + x_l \frac{\partial}{\partial x_l} B_k(\vec0)\right) d^3\vec{x} \\
    &= \epsilon_{ijk} B_k(\vec0) \int_V J_j d^3\vec{x} + \epsilon_{ijk} \frac{\partial B_k}{\partial x_l}(\vec0) \int_V x_l J_j d^3\vec{x} \\
    &= 0 + \epsilon_{ijk} \frac{\partial B_k}{\partial x_l}(\vec0) \epsilon_{ljn} m_n \\
    &= \frac{\partial B_k}{\partial x_i} m_k - \frac{\partial B_k}{\partial x_k}(\vec0) m_i \\
    &= \frac{\partial}{\partial x_i} \left(m_k B_k\right)(\vec0) \text{ because $\nabla \cdot \vec{B} = 0$}.
\end{align*}
Then we can bring the components together, $\vec{F} \approx \nabla(\vec{m} \cdot \vec{B}(\vec{x}))$.

We note that this is a conservative force, and can be written as $\vec{F} = -\nabla U$ where $U = -\vec{m} \cdot \vec{B}$. This is the potential energy of a magnetic dipole in an external field. As in the electric case, this is minimised when $\vec{m}$ is aligned with $\vec{B}$.
\end{document}