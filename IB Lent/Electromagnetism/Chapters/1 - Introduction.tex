\documentclass[../Main.tex]{subfiles}

\begin{document}
\section{Charges and Currents}
\underline{Electric charge} is a physical property of elementary particles.
\begin{itemize}
    \item It is a signed quantity, so can be positive, negative, or zero.
    \item It is quantised, always an integer multiple of the elementary charge $e$.
    \item It is conserved, even if the particles carrying the charge are created or destroyed.
\end{itemize}
By convention, $e$ is a positive quantity and the electron has charge $-e$ and the proton has charge $e$. The neutron has charge $0$.

This quantisation is small enough that on macroscopic scales the charge can be considered to have a continuous \underline{electric charge density} $\rho$.

The total charge in a volume $V$ is:
\begin{equation*}
    Q = \int_V \rho dV
\end{equation*}
The \underline{electric current density} $\vec{J}(\vec{x}, t)$ is the flux of electric charge per unit area. The current flowing through a surface $S$ is:
\begin{equation*}
    I = \int_S \vec{J} \cdot d\vec{S}
\end{equation*}
Consider a time-independent volume $V$ with boundary $S = \partial V$. Since charge is conserved,
\begin{gather*}
    \frac{dQ}{dt} = -I \\
    \frac{d}{dt} \int_V \rho dV + \int_S \vec{J} \cdot d\vec{S} = 0 \\
    \int_V \left(\frac{\partial\rho}{\partial t} + \nabla \cdot \vec{J}\right) dV = 0
\end{gather*}
Then since this is for any time-independent volume $V$, in fact the integrand must be zero:
\begin{equation}
    \frac{\partial \rho}{\partial t} + \nabla \cdot \vec{J} = 0
    \label{eqnChargeConserve}
\end{equation}
This is the \underline{equation of charge conservation}. It has the standard form of a conservation law, the time derivative of a conserved quantity's density added to the flux of the charge is zero.

The discrete charge distribution of a single particle of charge $q_i$ and position vector $\vec{x_i}(t)$ is:
\begin{align*}
    \rho(\vec{x}, t) &= q_i \delta(\vec{x} - \vec{x_i}(t)) \\
    \vec{J}(\vec{x}, t) &= q_i \dvec{x_i} \delta(\vec{x} - \vec{x_i}(t))
\end{align*}
Therefore, for $N$ particles it is:
\begin{align*}
    \rho(\vec{x}, t) &= \sum_{i = 1}^N q_i \delta(\vec{x} - \vec{x_i}(t)) \\
    \vec{J}(\vec{x}, t) &= \sum_{i = 1}^N q_i \dvec{x_i} \delta(\vec{x} - \vec{x_i}(t))
\end{align*}
\section{Fields and Forces}
Electromagnetism is a \textit{field theory} - charged particles interact by generating fields around them which are experienced by other charged particles.

In general we have two time-dependent vector fields, the electric field $\vec{E}(\vec{x}, t)$ and the magnetic field $\vec{B}(\vec{x}, t)$.

\begin{definition}{Lorentz Force}
    The \underline{Lorentz Force} on a particle of charge $q$ and velocity $\vec{v}$ is:
    \begin{equation*}
        \vec{F} = q(\vec{E} + \vec{v} \times \vec{B})
    \end{equation*}
\end{definition}
\section{Maxwell's Equations}
In this course we will explore the many consequences of \underline{Maxwell's Equations}:
\begin{align}
    \nabla \cdot \vec{E} &= \frac{\rho}{\epsilon_0} \label{eqnMED}\\
    \nabla \cdot \vec{B} &= 0 \label{eqnMBD} \\
    \nabla \times \vec{E} &= -\frac{\partial \vec{B}}{\partial t} \label{eqnMEC} \\
    \nabla \times \vec{B} &= \mu_0 \left(\vec{J} + \epsilon_0 \frac{\partial \vec{E}}{\partial t}\right) \label{eqnMBC}
\end{align}
These are coupled linear PDEs in space and time. They involve two positive constants:
\begin{align*}
    \epsilon_0&\text{ (vacuum permittivity)} \\
    \mu_0&\text{ (vacuum permeability)} \\
\end{align*}
Charges ($\rho$) and currents ($\vec{J}$) are the sources of the EM fields.

Each equation has an equivalent integral form (see later), related via the divergence theorem or Stokes's theorem.

These are the vacuum equations, that apply on microscopic scales or in a vacuum. There are related macroscopic versions that apply in media. For this, see the course II Electrodynamics.

The equations are consistent, both with each other and with charge conservation. We can see this by taking the divergence of \eqnref{eqnMEC} and the time derivative of \eqnref{eqnMBD}, both give that $\frac{\partial}{\partial t} (\nabla \cdot \vec{B}) = 0$. We can also evaluate the left-hand side of \eqnref{eqnChargeConserve}:
\begin{align*}
    \frac{\partial \rho}{\partial t} + \nabla \cdot \vec{J} &= \frac{\partial}{\partial t} \left(\epsilon_0 \nabla \cdot \vec{E}\right) + \nabla \cdot \left(-\epsilon_0 \frac{\partial E}{\partial t} + \frac{1}{\mu_0} \nabla \times \vec{B}\right) \\
    &= \frac{\partial}{\partial t} \left(\epsilon_0 \nabla \cdot \vec{E}\right) + \nabla \cdot \left(-\epsilon_0 \frac{\partial E}{\partial t}\right) \\
    &= 0
\end{align*}
\section{Units}
The SI unit of electric charge is the Coulomb ($C$). Since 2019, this has been defined using the elementary charge, so the elementary charge is an exact number of Coulombs:
\begin{equation*}
    e = 1.602176634\times10^{-19}
\end{equation*}
The SI unit of electric current is the Ampere or amp ($A$), equal to $1 C s^{-1}$. This is an SI base unit. The other SI base units needed for electromagnetism are the second ($s$), metre ($m$), and kilogram ($kg$).

From the Lorentz force law, we see that the units of $\vec{E}$ and $\vec{B}$ must be:
\begin{align*}
    [\vec{E}] &= kg~m~s^{-3}A^{-1} & [\vec{B}] &= kg~s^{-2}~A^{-1}
\end{align*}
The unit for $\vec{B}$ is also known as the Tesla, $T$.

From Maxwell's equations we can find the units of $\epsilon_0$ and $\mu_0$. These are, along with their experimentally-determined values:
\begin{align*}
    \epsilon_0&= 8.854\dots\times10^{-12}~kg^{-1}~m^{-3}~s^4~A^2 \\
    \mu_0&= 1.256\dots\times10^{-6}~kg~m~s^{-2}~A^{-2} \\
    &\approx 4\pi \times10^{-7}
\end{align*}
The fact that $\mu_0 \approx 4\pi\times10^{-7}$ is that previously this defined the units, but this is no longer the case so this is now an approximation.

The speed of light is an exact integer quantity in SI units,
\begin{equation*}
    c = \frac{1}{\sqrt{\mu_0 \epsilon_0}} = 299792458~m~s^{-1}
\end{equation*}
\end{document}