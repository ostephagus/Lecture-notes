\documentclass[../Main.tex]{subfiles}

\begin{document}
\section{Streamlines and Pathlines}
There are two natural ways to think of flow:
\begin{enumerate}
    \item A stationay observer watching flow go past. This is the Eulerian perspective.
    \item A moving observer travelling along with the flow. This is the Lagrangian perspective.
\end{enumerate}
The first approach will be used throughout the course, [because the second approach is very difficult to study]. We consider a continuous velocity field $\vec{u}(x, t)$, fixed on the lab frame.
\begin{definition}{Streamline}
    A \underline{streamline} is a curve that is everywhere parallel to the flow of the fluid at a given instant in time.
\end{definition}
A streamline through a point $\vec{x_0}$ at a time $t_0$ is given in parametric form:
\begin{equation*}
    \vec{x} = \vec{x}(s, \vec{x_0}, t_0)
\end{equation*}
It solves:
\begin{equation*}
    \quad\frac{d\vec{x}}{ds} = \vec{u}(\vec{x}, t_0),\quad \vec{x}(0, \vec{x_0}, t_0) = \vec{x_0}
\end{equation*}
The set of streamlines shows the direction of flow at a given instant in time.
\begin{example}
    Consider a velocity field in two dimensions, $\vec{u}(\vec{x}, t) = (1, t)$. Then the streamlines are horizontal at $t = 0$, and diagonal at $t = 1$.
    \label{expStreamlines}
\end{example}
\begin{definition}{Pathline}
    A \underline{pathline} is the trajectory of the fluid particle $\vec{x} = \vec{x}(\vec{x_0}, t)$ which is at $\vec{x_0}$ at $t = 0$.

    Pathlines are also known as Lagrangian trajectories.
\end{definition}
Here we say that a ``fluid particle'' is a very small bit of fluid. Experiments have shows that around 10 molecules is appropriate.

The pathline solves:
\begin{equation*}
    \frac{d\vec{x}}{dt} = \vec{u}(\vec{x}, t);\quad x(\vec{x_0}, t_0) = \vec{x_0}
\end{equation*}
\begin{example}
    Return to example~\ref{expStreamlines}. We solve:
    \begin{align*}
        \frac{dx}{dt} &= 1 &\implies x &= x_0 + t \\
        \frac{dy}{dt} &= t &\implies y &= y_0 + \frac{t^2}{2}
    \end{align*}
    Then this is a parabola, $y - y_0 = \frac12 (x - x_0)^2$.
\end{example}
\begin{definition}{Steady flow}
    A fluid flow is \underline{steady} if the velocity field $\vec{u}$ does not depend on time.
\end{definition}
In the case of steady flow, pathlines and streamlines are the same.
\section{The Material Derivative}
we would like to characterise the rate of change of quantities moving with a fluid.

Consider a quantity (scalar field) $F(\vec{x}, t)$ in a fluid flow. We compute the rate of change of $F$ (in time) seen by a moving observer. We denote this $\frac{DF}{Dt}$.

Over a small time interval $\delta t$:
\begin{align*}
    \delta F &= F(\vec{x} + \vec{\delta x}, t + \delta t) - F(\vec{x}, t) \\
    &= \delta t \frac{\partial F}{\partial t} + (\vec{\delta x} \cdot \nabla) F + O((\delta t)^2)
\end{align*}
Now divide by $\delta t$ and take the limit $\delta t \to 0$. Note that $\vec{\delta x} = \vec{u} \delta t$.
\begin{equation}
    \frac{DF}{Dt} = \underbrace{\frac{\partial F}{\partial t}}_{\text{local derivative}} + \underbrace{(\vec{u} \cdot \nabla) F}_{\text{advected derivative}}
    \label{eqnMatDerivative}
\end{equation}
The local derivative is simply the change due to $F$. However, we must also consider the change in $F$ \textit{in the direction of the flow}, which gives rise to the advected derivative term.

Later in the course we will see how to apply this to vector fields.
\section{Conservation of Mass}
Let $\rho(\vec{x}, t)$ be the mass density, with units $\frac{M}{L^3}$. The relationship between $\vec{u}$ and $\rho$ can be considered by thinking of a fixed volume $V$. Then the mass in this volume can only change due to fluid flowing across the boundary $\partial V$. The volume of fluid that flows out of a small surface area $\delta A$ in a small time $\delta t$ is $(\vec{u} \cdot \vec{n}) \delta A~\delta t$. Then the mass out is $\rho$ times this.
\begin{equation*}
    \frac{dM}{dt} = -\int_{\partial V} \rho \vec{u} \cdot \vec{dS}
\end{equation*}
We use the divergence theorem to re-write the right-hand side of this:
\begin{equation*}
    \frac{dM}{dt} = -\int_{V} \nabla \cdot (\rho \vec{u})
\end{equation*}
Then because this is valid for any fixed volume $V$,
\begin{equation}
    \frac{\partial \rho}{\partial t} + \nabla \cdot (\rho \vec{u}) = 0
    \label{eqnMassConserve}
\end{equation}
\begin{remark}
    The quantity $\rho \vec{u}$ is also known as the \textit{mass flux}.
\end{remark}
By expanding out the mass flux term, we rewrite as:
\begin{equation*}
    \frac{D \rho}{D t} = -\rho \nabla \cdot \vec{u}
\end{equation*}
\begin{definition}{Incompressible flow}
    A fluid flow is \underline{incompressible} if $\frac{D \rho}{D t} = 0$. Since density is non-zero, this is equivalent to $\nabla \cdot \vec{u} = \vec{0}$.
\end{definition}
In this course, we will take $\rho(\vec{x}, t) = \rho$ constant. This implies that $\nabla \cdot \vec{u} = 0$.
\section{Kinematic Boundary Conditions (KBC)}
We use boundary conditions to enforce mass conservation at the boundary of a fluid.

Consider the material boundary of a body of fluid having velocity $\vec{U}(\vec{x}, t)$ (where $\vec{U}$ is defined only on the boundary). At a point $\vec{x}$ on the boundary, the fluid velocity relative to the surface is $\vec{u} - \vec{U}$. The mass crossing the interface over a small surface element $\delta A$ in time $\delta t$ is given by:
\begin{equation*}
    \rho (\vec{u} - \vec{U}) \cdot \vec{n}~\delta A~\delta t = 0
\end{equation*}
Therefore at the interface, we require:
\begin{equation}
    \vec{u} \cdot \vec{n} = \vec{U} \cdot \vec{n}
    \label{eqnKBC}
\end{equation}
This is the kinematic boundary condition.
\begin{remark}
    In equation~\ref{eqnKBC}, $\vec{n}$ appears on both sides, so there is no need to normalise it (as long as it is the same on both sides).
\end{remark}
\begin{examples}{We consider some consequences of this.}
    \item If the boundary is fixed, $\vec{U} = \vec{0}$, then $\vec{u} \cdot \vec{n} = 0$.
    \item Consider an air-water interface (an example of a free surface) defined by $z = \zeta(x, y, t)$. Equivalently, the surface could be defined by $F(x, y, z, t) = 0$ with:
        \begin{equation*}
            F = z - \zeta(x, y, t)
        \end{equation*}
        Then $\vec{n}$ is parallel to $\nabla F$ given by:
        \begin{equation*}
            n \parallel \nabla F = (-\zeta_x, -\zeta_y, 1)
        \end{equation*}
        Then given the velocities:
        \begin{equation*}
            \vec{u} = (u, v, w)^T,\qquad \vec{U} = (0, 0, \zeta_t)^T
        \end{equation*}
        we find a KBC:
        \begin{equation*}
            -u \zeta_x - v\zeta_y + w = \zeta_t.
        \end{equation*}
        Equivalently,
        \begin{equation*}
            w = \zeta_t + u \zeta_x + v \zeta_y = \frac{D \zeta}{D t}
        \end{equation*}
        Or, $\frac{D F}{D t} = 0$.
\end{examples}
\section{Streamfunction for 2D Incompressible Flow}
Recall that incompressible flow means $\nabla \cdot \vec{u} = 0$. Therefore, there exists a vector potential such that $u = \nabla \times \vec{A}$.

\subsection{2D Cartesian Coordinates}
Consider the flow in 2 dimensions:
\begin{align*}
    \vec{u} &= (u, v, 0)^T & \vec{A} &= (0, 0, \psi(x, y)) \\
    \therefore u &= \frac{\partial \psi}{\partial y} & v &= -\frac{\partial \psi}{\partial x}
\end{align*}
Then the dimension of the streamfunction is $[\psi] = \frac{L^2}{T}$.
\begin{example}
    Consider the flow $\vec{u} = (y, x, 0)$. Then indeed we have $\nabla \cdot \vec{u} = 0$.

    We find $\psi$ by integrating:
    \begin{align*}
        \frac{\partial \psi}{\partial y} &= u = y \\
        \therefore \psi &= \frac12 y^2 + f(x) \\
        \frac{\partial \psi}{\partial x} &= -x \\
        \therefore \psi &= \frac12 (y^2 - x^2) + C
    \end{align*}
\end{example}
\begin{propositions}{
        Consider a streamfunction $\psi$.
        \label{propsStreamfunction}
    }
    \item Streamlines are given by $\psi = \text{constant}$.
    \item $|\vec{u}| = |\nabla \psi|$. That is, the flow is faster if the streamlines are closer.
    \item For two points $\vec{x_0}, \vec{x_1}$, then the flow rate between these two points is:
        \begin{equation*}
            \int_{\vec{x_0}}^{\vec{x_1}} \vec{u} \cdot \vec{n}~d\ell = \psi(\vec{x_1}) - \psi(\vec{x_0})
        \end{equation*}
        Note that this is independent of the path from $\vec{x_0}$ to $\vec{x_1}$.
    \item $\psi$ is constant at a rigid boundary.
\end{propositions}
\subsection{2D Polar Coordinates}
Now consider the components of $\vec{u}$ and $\vec{A}$.
\begin{align*}
    \vec{u} &= u_r(r, \theta)\vec{e_r} + u_\theta(r, \theta) \vec{e_\theta} \\
    \vec{A} &= \phi(r, \theta) \vec{e_z}
\end{align*}
Then:
\begin{align*}
    \frac1r \frac{\partial \phi}{\partial \theta} = u_r(r, \theta),\qquad -\frac{\partial \phi}{\partial r} = u_\theta(r, \theta)
\end{align*}
\nonexaminablesubsection{Stokes Streamfunction}
Consider cylindrical coordinates ($r, \theta, z)$, and consider a flow that is symmetric about the $z$ axis. Then we can find a Stokes Streamfunction defined by:
\begin{align*}
    u_r &= -\frac1r \frac{\partial \Psi}{\partial z} \\
    u_z &= \frac1r \frac{\partial \Psi}{\partial r} \\
\end{align*}
Where $\Psi = \Psi(r, z)$ is independent of $\phi$.
\end{document}