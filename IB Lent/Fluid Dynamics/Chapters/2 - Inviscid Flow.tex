\documentclass[../Main.tex]{subfiles}

\begin{document}
\section{Surface and Volume Forces}
There exist two types of forces that are exerted on a fluid:
\begin{enumerate}
    \item forces proportional to the volume, such as gravity;
    \item forces proportional to surface area, such as pressure and viscous stresses (discussed later).
\end{enumerate}
Volume forces, or body forces, will be denoted with $\bdforce$. Defined $\bdforce~\delta V$ to be the force acting on a small volume element $\delta V$. This has dimensions $[\bdforce] = \frac{F}{L^3}$. For example, gravity gives $\bdforce = \rho \vec{g}$.

Often we have conservative forces where $\bdforce = - \nabla \chi$. In the case of gravity, $\chi = \rho g z$.
\end{document}