\documentclass[../Main.tex]{subfiles}

\begin{document}
Up to now, we have neglected any friction in the fluid flow. We assumed that the stress tensor was $\vec{\tau} = -p\vec{n}$. The Euler Momentum equation was:
\begin{equation*}
    \rho \frac{D \vec{U}}{D t} = - \nabla p + \bdforce
\end{equation*}
In this chapter, we will include \textit{viscosity}. We will need to include a component of the stress perpendicular to the normal, and a new term in the Euler momentum equation.

In this course, we will focus on 2D parallel viscous flows. In Cartesian coordinates,
\begin{equation*}
    \vec{u} = (u(y, t), 0, 0)^T
\end{equation*}
Then this has horizontal streamlines, and we see $\nabla \cdot \vec{u} = 0$ immediately. Full treatment of 3D flows is in the course II Fluids.
\section{Plane Couette (Shear) Flow}
\begin{figure}
    \centering
    \begin{tikzpicture}[scale=1]
        \pgfmathsetmacro{\h}{2}
        \pgfmathsetmacro{\flowlinespacing}{0.2}

        \draw (0, 0) -- (5, 0);
        \draw[->] (0, \h) -- (5, \h) node[right] {$U$};
        \draw[|-|] (-0.5, 0) -- (-0.5, \h) node[anchor=east, pos=0.5] {$h$};

        \pgfmathsetmacro{\numflowlines}{\h / \flowlinespacing - 1}

        \foreach \n [evaluate=\n as \y using \n*\flowlinespacing] in {1, 2, ..., \numflowlines} {
            \draw[->, blue] (2, \y) -- +(\y, 0);
        }
    \end{tikzpicture}
    \caption{Diagram of the Couette Cell}
    \label{figCouetteCell}
\end{figure}
Consider a simple thought experiment: steady flow between two parallel plates, driven only by the motion of the top plate (see figure~\ref{figCouetteCell}). This is called Newton's Experiment or the Couette Cell. Experimentally, we observe that for a wide variety of Newtonian fluids (e.g. air, water, honey, silicone oil, glycerol):
\begin{enumerate}
    \item fluid velocity near the upper plate is $U$;
    \item fluid velocity near the bottom plate is $0$;
    \item fluid flow velocity varies linearly between $0$ and $U$, that is,
        \begin{equation*}
            u(y) = U \frac{y}{h};
        \end{equation*}
    \item the tangential stress $\tau$ to move the top plate at speed $U$ is linear in $U$ and inversely proportional to $h$:
    \begin{equation*}
        \frac{F}{A} = \tau \propto \frac{U}{h}
    \end{equation*}
    and we call $U / h$ the \underline{shear rate}.
\end{enumerate}
Therefore, we write $\tau = \mu \frac{U}{h}$ and define $\mu$ to be the \underline{dynamic viscosity}. This is Newton's Empirical Law of Viscosity. We generalise this empirical law to include a derivative:
\begin{equation*}
    \tau = \mu \frac{\partial u}{\partial \vec{n}}
\end{equation*}
for cases where $u$ does not vary linearly with $h$.
\section{2D Parallel Viscous Flow}
We can give the values of viscosity and density for different fluids.

\begin{tabular}{|c|c|c|c|}
    \hline
     & Density & Dynamic viscosity & Kinematic viscosity \\
     & $kg~m^{-3}$ & $Pa~s$ & $m^2 s^{-1}$ \\
    \hline
    Water & $1000$ & $10^{-3}$ & $10^{-6}$ \\
    Air & $1$ & $2\times10^{-5}$ & $2 \times 10^{-5}$ \\
    Golden syrup & $1400$ & $200$ & $0.14$ \\
    \hline
\end{tabular}
\subsection{Steady Flow with No Body Force}
Consider the Euler Momentum Equation on an infinitesimal volume of fluid $\delta x~\delta y$. Let the flow be $u(y) \vec{e_x}$. The viscous stress at $y$ is $-\mu \frac{\partial u}{\partial y}(y)$, and at $y + \delta y$ is $\mu \frac{\partial u}{\partial y} (y + \delta y)$.
This is a steady flow, so we must have that the forces sum to zero.

Consider the forces along $\vec{e_x}$:
\begin{equation*}
    \left[p(x) - p(x + \delta x)\right] \delta y + \left[\mu \frac{\partial u}{\partial y} (y + \delta y) - \mu \frac{\partial u}{\partial y}(y)\right]\delta x = 0
\end{equation*}
We now consider a Taylor expansion up to $\delta x~\delta y$:
\begin{align*}
    0 &= \delta x~\delta y \left[-\frac{\partial p}{\partial x} + \mu \frac{\partial^{2}u}{\partial y^{2}}\right] \\
    \implies& \mu \frac{\partial^{2}u}{\partial y^{2}} - \frac{\partial p}{\partial x} = 0
\end{align*}
In the $y$ direction, we do the same and find $\frac{\partial p}{\partial y} = 0$.
\subsection{Unsteady Flow with Body Forces}
We apply the same process (see Ex2Q1):
\begin{equation}
    \begin{split}
    \rho \frac{\partial u}{\partial t} = \mu \frac{\partial^{2}u}{\partial y^{2}} - \frac{\partial p}{\partial x} + f_x \\
    0 = -\frac{\partial p}{\partial y} + f_y
    \end{split}
    \label{eqnViscousFlowForces}
\end{equation}
Note that the ``acceleration'' term does not have the familiar $(\vec{u} \cdot \nabla)\vec{u}$ because $u$ is a flow along $\vec{e_x}$ and is a function only of $y$.
\subsection{Boundary Conditions}
So far, at boundaries, we have considered only the no penetration boundary condition $\vec{u} \cdot \vec{n} = \vec{U} \cdot \vec{n}$. We now have higher-order terms due to viscosity, and so we need another boundary condition.

Experiments show that, for viscous flow at a rigid boundary, $\vec{u} = \vec{U}$. This is the \textit{no slip} boundary condition. This extends the no penetration boundary condition to include that the tangential components of $\vec{u}$ and $\vec{U}$ must match, not just the normal components.
\subsection{Examples}
\begin{example}[Poiseuille flow in a channel]
    Pronunciation guide: pw\"osoi.

    We are interested in flow between two stationary plates at $y = 0$ and $y = h$, filled with a viscous fluid. Let the pressure on the left be $p_1$ and on the right $p_0 < p_1$. Let the length of the channel being considered be $L$.

    Ignoring body forces, the equations to solve are:
    \begin{equation*}
        \begin{cases}
            \frac{\partial p}{\partial y} = 0, \frac{\partial p}{\partial x} = \mu \frac{\partial^{2}u}{\partial y^{2}} & (x, y) \in (0, L) \times (0, h) \\
            u(0) = u(h) = 0 &
        \end{cases}
    \end{equation*}
    Note that in the second equation (involving the $x$ derivative of pressure), we have a function in $x$ on the LHS, and a function of $y$ on the RHS. Therefore, both must be constant:
    \begin{equation*}
        \frac{\partial p}{\partial x} = -G, \mu \frac{\partial^{2}u}{\partial y^{2}} = -G
    \end{equation*}
    for a constant $G > 0$. This is interesting, it is not immediately obvious that the pressure would be linear over this range.
    
    We solve the second-order equation:
    \begin{equation*}
        \frac{\partial^{2}u}{\partial y^{2}} = -\frac{G}{\mu} \implies u(y) = \frac{G}{2\mu} y(h-y)
    \end{equation*}
    by applying boundary conditions. We can now calculate the flow rate (volume flux per unit length perpendicular to flow):
    \begin{equation*}
        q = \int_{0}^{h} u(y) dy = \frac{(p_1 - p_0)h^3}{12\mu L}
    \end{equation*}

    We can also consider the force balance on the whole fluid. The equations we just solved are for an infinitesimal region of fluid, so we check these apply to the whole fluid:
    \begin{align*}
        p_0 h - p_1 h &= L\tau_{\text{top}} - L\tau_{\text{bottom}} \\
        &= L \frac{\partial u}{\partial y}(h) - L \frac{\partial u}{\partial y}(0) \\
        &= L\frac{G}{2}h - -L\frac{G}{2}h \\
        &= LGh = -(p_0 - p_1)h
    \end{align*}
    as required.
\end{example}
\begin{figure}
    \centering
    \begin{tikzpicture}[>=To, line cap=round, line join=round]

        % Define parameters
        \def\ang{20} % Angle of the slope
        \def\L{8}    % Length of the slope
        \def\h{0.6}  % Thickness of the fluid layer

        % Define colors
        \definecolor{customblue}{RGB}{0, 85, 145} 

        % Coordinates
        \coordinate (O) at (0,0); % Vertex
        \coordinate (BaseLeft) at (-\L, 0);
        \coordinate (SlopeTop) at ({-\L*cos(\ang)}, {\L*sin(\ang)});

        % Draw the ground (Black)
        \draw[very thick] (BaseLeft) -- (O);

        % Draw the slope base (Black)
        \draw[very thick] (SlopeTop) -- (O);

        % Calculate top surface coordinates
        % Normal vector to slope (pointing up-left) is (-sin(ang), cos(ang))
        \coordinate (Shift) at ({-sin(\ang)*\h}, {cos(\ang)*\h});
        \coordinate (TopStart) at ($(SlopeTop) + (Shift)$);
        \coordinate (TopEnd) at ($(O) + (Shift)$);

        % Draw the fluid layer hatching
        \begin{scope}
            \clip (SlopeTop) -- (O) -- (TopEnd) -- (TopStart) -- cycle;
            % Draw diagonal lines manually
            \foreach \x in {-10,-9.5,...,2} {
                \draw[customblue, thick] (\x, -2) -- ({\x+4}, 4);
            }
        \end{scope}

        % Draw the top surface line (Blue)
        \draw[very thick, customblue] (TopStart) -- (TopEnd);

        % Draw the angle alpha
        \draw[thick] (-1.8,0) arc (180:{180-\ang}:1.8);
        \node at (-2.3, 0.35) {\large $\alpha$};

        % Coordinate System
        % Position it above the midpoint
        \coordinate (CSOrigin) at ($(SlopeTop)!0.45!(O) + (0.5, 2.5)$);
        \begin{scope}[shift={(CSOrigin)}, rotate={-\ang}]
            \draw[->, very thick] (0,0) -- (1.0,0) node[right] {\large $\vec{e_x}$};
            \draw[->, very thick] (0,0) -- (0,1.0) node[above] {\large $\vec{e_y}$};
        \end{scope}

        % Velocity Vector u
        % Position it slightly above the layer
        \coordinate (UPos) at ($(TopStart)!0.55!(TopEnd) + (0, 0.4)$);
        \begin{scope}[shift={(UPos)}, rotate={-\ang}]
            \draw[->, very thick, customblue] (-0.6,0) -- (0.6,0) node[midway, above] {\large $u$};
        \end{scope}

    \end{tikzpicture}
    \caption{Digram of fluid flow down an incline}
    \label{figFluidIncline}
\end{figure}
\begin{example}[Viscous flow down an incline]
    Consider viscous fluid flowing down an incline under the influence of gravity (figure~\ref{figFluidIncline}). Assume steady flow, $\vec{u} = u(y) \vec{e_x}$, and that the air does not exert shear stresses on the fluid. This is valid because $\mu_{\text{air}} << \mu_{\text{water}}$.
    
    Let the height of the fluid be $h$ (constant).
    \begin{equation*}
        \bdforce = \rho \vec{g} = \left(\rho g \sin(\alpha) - \rho g \cos(\alpha)\right) %TODO: Check
    \end{equation*}
    Balancing in the $y$ direction:
    \begin{align*}
        0&= -\frac{\partial p}{\partial y} - \rho g \cos(\alpha), p(h) = p_{atm} \\
        \implies& p = p_{atm} + \rho g \cos(\alpha) (h - y)
    \end{align*}
    In the $x$ direction.
    \begin{align*}
        0 &= -\underbrace{\frac{\partial p}{\partial x}}_{0} + \mu \frac{\partial^{2}u}{\partial y^{2}} + \rho g \sin(\alpha) \\
        \implies& \frac{\partial^{2}u}{\partial y^{2}} = -\frac{\rho g}{\mu} \sin(\alpha)
    \end{align*}
    But what are our boundary conditions? The no-slip on the plane gives $u(0) = 0$, and since there is no shear stress at the top we find $\mu \frac{\partial u}{\partial y}(h) = 0$. This gives the solution for $u$:
    \begin{equation*}
        u(y) = \frac{\rho g \sin(\alpha)}{2\mu}y(2h - y)
    \end{equation*}
    which is half of a parabola (such that the turning point is along $y = h$).
\end{example}
\subsection{Boundary Conditions at an Interface}
Consider an interface between two fluids. We require:
\begin{enumerate}
    \item no slip condition: equal velocities at the interface ($u_1 = u_2$);
    \item continuity of stress:
        \begin{enumerate}
            \item in the normal direction: pressures must match ($p_1 = p_2$)
            \item in the tangential direction: stresses must match
                \begin{equation*}
                    \mu_1 \frac{\partial u_1}{\partial y} = \mu_2 \frac{\partial u_2}{\partial y}
                \end{equation*}
                this tells us that we will have a discontinuity in the gradient $\frac{\partial u}{\partial y}$.
        \end{enumerate}
\end{enumerate}
\section{Unsteady Parallel Viscous Flow}
Consider the following illustrative problem:
\begin{example}[Rayleigh's Problem / Stokes's First Problem]
    Consider the ``impulsively started plate'', and a semi-infinite fluid in $y > 0$, initially at rest. Let there be no pressure gradient nor force ($\bdforce = \vec0$).

    At $t > 0$, let the plate (at $y = 0$) move at constant velocity $(U, 0)$.

    Consider the equations of motion:
    \begin{align*}
        x:& \frac{\partial p}{\partial y} = 0 \implies p = p_{atm} \\
        y:& \rho \frac{\partial u}{\partial t} = - \frac{\partial p}{\partial x} + \mu \frac{\partial^{2}u}{\partial y^{2}} + f_x\\
        \therefore& \rho \frac{\partial u}{\partial t} = \mu \frac{\partial^{2}u}{\partial y^{2}} 
    \end{align*}
    and impose boundary conditions:
    \begin{align*}
        u(0, t) &= U \text{ (no slip)} \\
        u(\infty, t) &= 0
    \end{align*}

    Now define $\nu = \frac{\mu}{\rho}$ as the \underline{kinematic viscosity}. This finally gives:
    \begin{equation}
        \frac{\partial u}{\partial t} = \nu \frac{\partial^{2}u}{\partial y^{2}}
        \label{eqnFluidDiffusion}
    \end{equation}
    which is a diffusion equation (compare with the Heat Equation from the course IB Methods). If $\nu = 0$ then we get no flow. We can understand $\nu$ as a diffusivity coefficient for momentum. It has dimensions $L^2 T^{-1}$.

    We could solve this equation using the ideas of IB Methods. However, we will consider dimensional analysis to first reduce to an ODE, then solve.
    \begin{align*}
        u(t, t) &= f(y, t, U, \nu) \\
        &= Uf(y, t, \nu) \text{ because the PDE is linear in $U$} \\
        &= U f\left(\frac{y}{\sqrt{\nu t}}\right) \text{ by dimensional analysis}.
    \end{align*}
    Now define $\eta = \frac{y}{\sqrt{\nu t}}$, the \underline{similarity variable}. We want to substitute this into \eqnref{eqnFluidDiffusion}, so we need partial derivatives:
    \begin{equation*}
        \frac{\partial \eta}{\partial t} = -\frac12 \frac{\eta}{t}
    \end{equation*}
    Then substituting this in:
    \begin{equation*}
        -\frac12 \frac{\eta}{t} f'(\eta) U = \frac{\eta}{Ut} U f''(\eta) %TODO: Check
    \end{equation*}
    and the boundary conditions are $f(0) = 1, f(\infty) = 0$. The solution is:
    \begin{equation}
        u(y, t) = U \operatorname{ercf}\left(\frac{y}{2\sqrt{\nu t}}\right)
        \label{eqnFluidDiffusionSoln}
    \end{equation}
    where the complementary error function is:
    \begin{equation*}
        \operatorname{erfc}(x) = \frac{2}{\pi} \int_x^\infty e^{-t^2} dt
    \end{equation*}
    For a diagram of the fluid flow, see figure~\ref{figImpulsePlate}.

    Then the value of viscous stress at $y = 0$ is:
    \begin{equation*}
        \tau = \mu \left.\frac{\partial u}{\partial y}\right|_{y = 0} = -\frac{\mu U}{\sqrt{\nu t}} f'(0) = -\frac{\mu U}{\sqrt{\pi \nu t}}
    \end{equation*}
    and so we need to consider $\frac{\mu}{\sqrt{\nu}}$ for each fluid to understand how the solution will be different for different fluids. This is quantity is $200$ times larger for water than air.
    \label{expImpulsePlate}
\end{example}
\begin{figure}
    \centering
    \begin{tikzpicture}[
            >={Straight Barb[length=2mm, width=2mm]},
            wall/.style={very thick},
            hatch/.style={thick},
            profile/.style={very thick, blue},
            vector/.style={->, thick, blue}
        ]

        % Diagram 1: t = 0^+
        \draw[wall] (0,0) -- (3.6,0);
        \foreach \x in {0.2, 0.8, 1.4, 2.0, 2.6, 3.2} {
            \draw[hatch] (\x, 0) -- (\x-0.3, -0.3);
        }
        \draw[profile] (0.2, 2.5) -- (0.2, 0.3) .. controls (0.2, 0.05) and (0.4, 0.05) .. (0.6, 0.05) -- (3.4, 0.05);
        \draw[vector] (0.2, 0.15) -- (0.4, 0.15);
        \node at (1.0, 3.0) {\Large $t=0^+$};

        % Diagram 2: Intermediate t
        \draw[wall] (4.5,0) -- (8.2,0);
        \foreach \x in {4.7, 5.3, 5.9, 6.5, 7.1, 7.7} {
            \draw[hatch] (\x, 0) -- (\x-0.3, -0.3);
        }
        \draw[profile] (4.7, 0) -- (4.7, 2.5);
        \draw[profile] plot [smooth] coordinates {(4.7, 2.5) (4.9, 1.8) (5.2, 1.3) (5.7, 0.8) (6.5, 0.3) (7.8, 0.05)};
        \draw[vector] (4.7, 1.8) -- (4.9, 1.8);
        \draw[vector] (4.7, 1.3) -- (5.2, 1.3);
        \draw[vector] (4.7, 0.8) -- (5.7, 0.8);
        \draw[vector] (4.7, 0.3) -- (6.5, 0.3);

        % Diagram 3: t large
        \draw[wall] (9.0,0) -- (13.0,0);
        \foreach \x in {9.2, 9.8, 10.4, 11.0, 11.6, 12.2, 12.8} {
            \draw[hatch] (\x, 0) -- (\x-0.3, -0.3);
        }
        \draw[profile] (9.2, 0) -- (9.2, 2.5);
        \draw[profile] plot [smooth] coordinates {(10.0, 2.2) (10.2, 2.0) (10.8, 1.5) (11.5, 1.0) (12.2, 0.5) (12.8, 0.05)};
        \draw[vector] (9.2, 2.0) -- (10.2, 2.0);
        \draw[vector] (9.2, 1.5) -- (10.8, 1.5);
        \draw[vector] (9.2, 1.0) -- (11.5, 1.0);
        \draw[vector] (9.2, 0.5) -- (12.2, 0.5);
        \node at (11.0, 3.0) {\Large $t$ large};

    \end{tikzpicture}
    \caption{Diagram of the Rayleigh Problem}
    \label{figImpulsePlate}
\end{figure}
\begin{example}
    Consider example~\ref{expImpulsePlate} with another boundary at $y = h$.

    For small times, when $\sqrt{\nu t} \ll h$, the fluid acts like the previous example.
    For long times, we have Couette flow and the flow velocity is linear in $y$.

    The characteristic timescale here is $t \sim \frac{h^2}{\nu}$ for diffusion of momentum. For golden syrup with $h = 0.1m$ this is $0.07s$, and for water it is $3$ hours.
    \label{expImpulsePlate2}
\end{example}
\end{document}