\documentclass[../Main.tex]{subfiles}

\begin{document}
In this chapter we would like to generalise the concept of a Taylor Series. Recall that if $f \in C^\infty$, we have:
\begin{equation*}
    f(x) = \sum_{m=0}^{\infty} \frac{f(x_0)^{(m)}}{m!}(x - x_0)^m
\end{equation*}
In the complex plane, we will see that we do not even need the function to be differentiable at $x_0$!
\section{Defining the Laurent Series}
\begin{proposition}
    Let a function $f(z)$ be analytic in an annulus:
    \begin{equation*}
        \subsetselect{z \in \C}{R_1 < |z - z_0| < R_2}.
    \end{equation*}
    Then it has a \underline{Laurent series},
    \begin{equation*}
        f(z) = \sum_{m=-\infty}^{\infty} a_m (z - z_0)^m
    \end{equation*}
    which is convergent within the annulus and uniformly convergent within compact subsets of the annulus. %TODO: What is a compact subset? And does this matter?

    Further, if $f(z)$ is analytic at $z = z_0$ (so $R_1 = 0$), then it has a Taylor series:
    \begin{equation*}
        f(z) = \sum_{m=0}^{\infty} a_m (z - z_0)^m \text{ with } a_m = \frac{f^{(m)}(z_0)}{m!}
    \end{equation*}
    \label{propLaurent}
\end{proposition}
\begin{proof}
    Let $z_0 = 0$ after possibly translating the argument. Let $z \in \C$ with $R_1 < r_1 < |z| < r_2 < R_2$. Let $\gamma_1, \gamma_2$ be a anticlockwise, circular contours of radii $r_1, r_2$.

    As we have previously seen, we can bridge these two contours with radius lines in the opposite direction, to create $\gamma = \gamma_2 - \gamma_1$ in the limit as the bridge gets infinitely thin. %TODO: Diagram?
    Now consider \thmref{thmCauchyIntegral} (setting $\xi$ to be $z$ and $z$ to be $z_0$):
    \begin{align*}
        f(z) &= \frac{1}{2\pi i} \oint_{\gamma} \frac{f(\xi)}{\xi - z} d\xi \\
        &= \frac{1}{2\pi i} \oint_{\gamma_2} \frac{f(\xi)}{\xi - z}d\xi - \frac{1}{2\pi i} \oint_{\gamma_1} \frac{f(\xi)}{\xi - z} d\xi
    \end{align*}
    We now have two cases. For $\gamma_1$, we have that $\left|\frac{\xi}{z}\right| < 1$, so we can use a geometric series as follows:
    \begin{align*}
        I_1 &= -\frac{1}{2\pi i} \oint_{\gamma_1} \frac{f(\xi)}{\xi - z} d\xi \\
        &= \frac{1}{2\pi i z} \oint_{\gamma_1} \frac{f(\xi)}{1 - \xi / z} d\xi \\
        &= \frac{1}{2\pi i z} \oint_{\gamma_1} f(\xi) \sum_{n=0}^{\infty} \left(\frac{\xi}{z}\right)^n d\xi \\
        &= \frac{1}{2\pi i z} \sum_{n=0}^{\infty} \left(\oint_{\gamma_1} f(\xi) \xi^n d\xi\right)z^{-n-1} \\
        &=\sum_{m=-\infty}^{-1} \left(\frac{1}{2\pi i z} \oint_{\gamma_1} f(\xi) \xi^{-m-1} d\xi\right)z^m \\
    \end{align*}
    Note also that in the case that $f$ is analytic all the way to $z_0$, we find $I_1 = 0$.
    
    For $\gamma_2$, $\left|\frac{z}{\xi}\right|<1$ and so:
    \begin{align*}
        I_2 &= \frac{1}{2\pi i} \oint_{\gamma_2} \frac{f(\xi)}{\xi - z}d\xi \\
        &= \frac{1}{2\pi i} \oint_{\gamma_2} \frac{f(\xi)}{\xi}\sum_{m=0}^{\infty} \left(\frac{z}{\xi}\right)^m d\xi \\
        &= \sum_{m=0}^{\infty} \left(\frac{1}{2\pi i} \oint_{\gamma_2} f(\xi) \xi^{-m-1} d\xi\right)z^m
    \end{align*}
    Therefore putting these together gives the Laurent series. Note that in the case of $f$ analytic at $z_0$, $I_1 = 0$ and we recover the Taylor series.
\end{proof}
\begin{examples}{
        Consider the Laurent series of the following series:
    }
    \item $f(z) = \frac{e^z}{z^3}$. Then we find:
        \begin{equation*}
            f(z) = z^{-3}\sum_{m=0}^{\infty} \frac{z^n}{m!} = \sum_{m=-3}^{\infty} \frac{z^n}{(m+3)!}
        \end{equation*}
    \item $f(z) = e^{\frac1z}$. We get:
        \begin{equation*}
            f(z) = \sum_{m=0}^{\infty} \frac{1}{z^n} \frac{1}{m!} = \sum_{m=-\infty}^{0} \frac{z^n}{(-n)!}
        \end{equation*}
    \item $f(z) = \frac1{z-a}$. If $|z| < |a|$,
        \begin{equation*}
            f(z) = -\frac{1}{a} \frac{1}{1 - z / a} = -\frac{1}{a} \sum_{m=0}^{\infty} \frac{z^n}{a^n}
        \end{equation*}
    \item An example to show that Laurent series get difficult to compute very fast.
        Consider $f(z) = \frac{e^z}{z^2 - 1}$ around $z = 1$. Define $\xi = z - 1$ so that:
        \begin{align*}
            f(z) &= \frac{e^\xi e}{\xi(\xi +2)} \\
            &= \frac{e e^\xi}{2\xi} \frac{1}{1 + \xi / 2} \\
            &= \frac{e}{2\xi} \left(1 + \xi - \frac12 \xi^2 + \cdots\right)\left(1 - \frac{\xi}{2} + \left(\frac{\xi}{2}\right)^2 + \cdots\right)
        \end{align*}
        which is a multiplication of two infinite series, and is difficult to compute.
\end{examples}
\begin{remark}
    Often, as with Taylor series for real functions, we are only interested in the first few terms of a series like this.
\end{remark}
\section{Multiplicity of Zeroes}
\begin{theorem}[Fundamental Theorem of Algebra]
    For a polynomial $p(z)$ of degree $m \geq 1$, there are exactly $m$ roots counting multiplicity. That is, $p$ can be written uniquely as:
    \begin{equation*}
        p(z) = A\sum_{j=1}^{k} (z - z_j)^{m_j}
    \end{equation*}
    where the $m_j$ sum to $m$.
    \label{thmFundamentalAlgebra}
\end{theorem}
\begin{definition}{Zeroes}
    The \underline{zeroes} of an analytic function are the points $z_0$ where $f(z_0) = 0$.
\end{definition}
\begin{definition}{Order of a zero}
    A zero $z_0$ of an analytic function $f(z)$ has \underline{order} $m$ if in its Taylor expansion,
    \begin{equation*}
        f(z) = \sum_{k=m}^{\infty} a_k (z - z_0)^k,
    \end{equation*}
    the first non-zero coefficient is $a_m$.
\end{definition}
\begin{examples}{
        We will consider the order of the zeroes of the following functions:
    }
    \item A polynomial. We verify that our definition of order matches with that for polynomials for the function $f(z) = z^3 + iz + i$.
        \begin{equation*}
            f(z) = (z + i)^2 (z - i)
        \end{equation*}
        then this has a simple zero at $z = i$ and a zero of order 2 at $z = -i$.
    \item Consider $f(z) = \sinh(z) = \frac12(e^z - e^{-z})$. $f(z) = 0$ at $z = im\pi$. We consider the derivative of $f$, $\cosh(z)$ and we see that $\cosh(im\pi) = (-1)^m \neq 0$. That means that the zeroes of $\sinh$ are simple.
    \item Consider $f(z) = \sinh^3(z)$.
        We know from the previous examples that this has a zero at $i\pi$. We consider a Taylor expansion around this point by considering $z = \xi + i\pi$:
        \begin{align*}
            \sinh^3(\pi) &= \left[\sinh(\xi + i \pi)\right]^3 \\
            &= \left[\sinh(\xi) \underbrace{\cosh(i\pi)}_{-1} + \cosh(\xi) \underbrace{\sinh(i\pi)}_{0} \right]^3 \\
            &= (-\sinh(\xi))^3 = -\left(\xi + \frac{1}{3!}\xi^3 + \cdots\right)^3 \\
            &= -\xi^3 - \frac12 \xi^5 + \cdots \\
            &= i(z - i\pi)^3 -\frac12(x - i\pi)^5
        \end{align*}
        and so the root is of order 3.
\end{examples}
\section{Singularities}
\begin{definition}{Singularity}
    A \underline{singularity} of a function $f(z)$ is a point $z = z_0$ where $f$ is not analytic.
\end{definition}
\begin{definition}{Isolated singularity}
    An \underline{isolated singularity} of a function $f$ is a singularity $z = z_0$ where the function is analytic in a neighbourhood around $z_0$. If no such neighbourhood exists, $z_0$ is a \underline{non-isolated singularity}
\end{definition}
\begin{examples}{}
    \item $f(z) = log(z)$ has a singularity at $z = 0$ and $z = \infty$. These are both non-isolated singularities, because there must be a branch cut between them which prevents a neighbourhood for which the function is analytic.
    \item $f(z) = \frac{1}{\sinh(z)}$ has singularities at $z = in\pi$ for $n \in \Z$. These are isolated, because $f$ is analytic around these points.
    \item $f(z) = \frac{1}{\sinh\left(\frac{1}{z}\right)}$. This has singularities at $z = \frac{1}{in\pi}$ for $n \in \Z$. Then for $n \neq 0$, these singularities are isolated. However, at $n = 0$ we cannot find a neighbourhood free from singularities. We can intuitively understand this as all the singularities clustering at $0$ as $n \to \infty$.
\end{examples}
\begin{remarks}
    \item So far we have seen that non-isolated singularities can be caused by branch cuts or by ``clustering'' of singularities
    \item Isolated singularities are useful because we can use Laurent series with the inner radius tending to $0$ around the singularity to approximate the function.
\end{remarks}
\subsection{Categorising Singularities}
Given a function $f$ and a singularity $z_0$, we can do the following to classify it:
\begin{enumerate}
    \item check whether $z_0$ is a branch point;
    \item if not, check whether $z_0$ is a different non-isolated singularity;
    \item if neither, find the Laurent series and check:
    \begin{enumerate}
        \item if $a_m$ = 0 for all negative $m$, then $f(z)$ is equal to its Taylor series and we have a removable singularity. $f(z_0) = a_0$;
        \item if there exists $N > 0$ such that for all $m < -N-1$ then $a_m = 0$, and $a_{-N}$ (that is, the Laurent series terminates for negative $N$), then $f$ has a pole of order $N$ at $z = z_0$;
        \item if none of the above is true, $f$ has an \underline{essential singularity} at $z = z_0$.
    \end{enumerate}
\end{enumerate}
\begin{examples}{
        We can now classify singularities of different functions.
    }
    \item The function $f(z) = \frac{1}{z-i}$ has a simple pole (order 1) at $z = i$.
    \item $f(z) = \frac{\cos(z)}{z}$ has a simple pole at $0$.
        \begin{equation*}
            f(z) = \sum_{m=0}^{\infty} (-1)^m \frac{z^{2m-1}}{m!}
        \end{equation*}
    \item $g(z) = \frac{z^3}{(z-1)^3(z-i)^2}$. We can simply read off a triple pole (order 3) at $z = 1$ and a double pole at $z = i$. We find here that the order of the poles are exactly the multiplicities of the zeroes of the denominator.
    \item $f(z) = z^2$. This has no singularities on $\C$, but on $\Cstar$ it has a double pole at $\infty$. $f(\frac{1}{\xi})$ has a double pole at $0$.
    \item $f(z) = e^{1 / z}$. The Laurent series is:
        \begin{align*}
            f(z) &= \sum_{m=0}^{\infty} \frac{1}{z^n} \frac{1}{m!} \\
            &= \sum_{m=-\infty}^{0} \frac{1}{m!}z^m
        \end{align*}
        which does not terminate as $m \to -\infty$. Therefore there is an essential singularity at $z = 0$.
    \item $f(z) = \sin(1 / z)$ also has an essential singularity at $z = 0$.
    \item $f(z) = \frac{e^z - 1}{z}$ has a removable singularity at $z = 0$. We find the Laurent series:
        \begin{align*}
            f(z) &= \frac1z \left(\sum_{m=0}^{\infty} \frac{1}{m!} z^m - 1\right) \\
            &= \frac{1}{z} \sum_{m=1}^{\infty} \frac{z^m}{m!} \\
            &= \sum_{m=0}^{\infty} \frac{z^m}{(m+1)!}
        \end{align*}
        and we see that $a_m = 0$.
    \item $f(z) = \frac{\sin(z)}{z}$ is a famous example of a removable singularity.
    \item Consider the rational function:
        \begin{equation*}
            f(z) = \frac{P(z)}{Q(z)}
        \end{equation*}
        where $P$ and $Q$ are polynomials. If $Q(z)$ has a zero of order $n > 0$ and if $P(z)$ has a zero of order $m \geq n$ at $z = z_0$, then $f$ has a removable singularity at $z = z_0$,
        \begin{equation*}
            f(z_0) = \frac{P^{(n)}(z_0)}{Q^{(n)}(z_0)}
        \end{equation*}
\end{examples}
\begin{proposition}
    Let $f : \C \mapsto \C$ and suppose that $f$ has a zero of order $m$ at $z_0$. Then the reciprocal function $g(z) = \frac{1}{f(z)}$ has a pole of order $m$ there.
    \label{propZeroesSingsInverse}
\end{proposition}
\begin{proof}
    TODO %TODO
\end{proof}
\begin{proposition}
    Let $f(z)$ have an essential singularity at $z = z_0$. Then in the neighbourhood excluding $z_0$, $f$ takes all values, except at most one.
    \label{propGrandPicard}
\end{proposition}
\begin{proof}
    Proof is far beyond the scope of this course.
\end{proof}
\begin{remark}
    The function $f(z) = e^{1 / z}$ takes every value except $0$.
\end{remark}
\section{Residues}
\begin{definition}{Residue}
    The \underline{residue}, denoted $\res_{z = z_0} f(z)$ of a function $f(z)$ with an isolated singularity at $z_0$ is the coefficient of $a_{-1}$ in the Laurent series about $z_0$.
\end{definition}
\begin{proposition}
    Let $f$ have a pole of order $n$ at $z = z_0$. Then the residue is given by:
    \begin{equation*}
        \res_{z = z_0} f(z) = \lim_{z \to z_0} \frac{1}{(n-1)!} \frac{d^{n-1}}{dz^{n-1}} \left[(z - z_0)^n f(z)\right]
    \end{equation*}
    \label{propResDerivatives}
\end{proposition}
\begin{proof}[$n = 1$]
    We have:
    \begin{align*}
        f(z) &= \frac{a_{-1}}{z - z_0} + a_0 + a_1(z - z_0) + \cdots \\
        (z - z_0) f(z) &= a_{-1} + a_0 (z - z_0) + a_1 (z - z_0)^2 \cdots
    \end{align*}
    and when taking the limit this gives $a_{-1}$.
\end{proof}
\begin{examples}{
        Consider the following functions. We calculate their residues:
    }
    \item $f(z) = \frac{e^z}{z^3} = \frac1{z^3} \left(1 + z + \frac{z^2}{2} + \cdots\right)$ and we read off the residue is $\frac12$.
    \item $f(z) = \frac{e^z}{z^2 - 1}$. Then the residue is:
        \begin{align*}
            \res_{z = 1} f(z) &= \lim_{z \to 1} (z - 1_ \frac{e^z}{(z + 1)(z - 1)}) \\
            &= \frac{e}{2}.
        \end{align*}
    \item $f(z) = \frac{1}{z^8 - w^8}$. Then we can identify the poles $z = we^{ik\pi / 4}$ for $k = 0$ to $7$. However, trying to factorise and use \thmref{propResDerivatives} would be a bad idea. The better idea is to use L'H\^opital's Rule:
        \begin{align*}
            \res_{z = w} f(z) &= \lim_{z \to w} \frac{z-w}{z^8 - w^8} \\
            &= \lim_{z \to w} \frac{1}{8z^7} = \frac{1}{8w^7}
        \end{align*}
    \item $f(z) = \sinh(\pi z)$ has simple zeroes at $z = i n$ where $n \in \Z$. Therefore the reciprocal function $g(z) = \frac{1}{\sinh(\pi z)}$ has simple poles there. The residue is:
        \begin{align*}
            \res_{z = i n} f(z) &= \lim_{z \to i n} \frac{z - n i}{\sinh(\pi z)} \\
            &= \lim_{z \to i n} \frac{1}{\pi \cosh(\pi z)} \\
            &= \frac{(-1)^n}{\pi}
        \end{align*}
    \item $f(z) = \frac{1}{\sinh^3(z)}$. We have already seen:
        \begin{equation*}
            \frac{1}{\sinh^3(z)} = -(z - i \pi)^{-3} + \frac12 (z - i \pi)^{-1} + \cdots
        \end{equation*}
        and so the residue is $\frac12$.
\end{examples}
\begin{theorem}
    Let $\gamma$ be a simple closed contour tranversed anticlockwise. Let $f(z)$ be analytic within $\gamma$ except for an isolated singularity at $z_0$.
    \begin{equation}
        \int_{\gamma} f(z) dz = 2\pi i a_{-1} = 2\pi i \res_{z = z_0} f(z)
        \label{eqnIntOverSing}
    \end{equation}
    \label{thmIntOverSing}
\end{theorem}
\begin{proof}
    By deforming $\gamma$ into a circle $\gamma_r$ around $z_0$ of radius $r$ (valid by \thmref{propContourShrink}), we can evaluate the integral more easily.
    \begin{align*}
        \oint_\gamma f(z) dz &= \oint_{\gamma_r} f(z) dz \\
        &= \oint_{\gamma_r} \sum_{m=-\infty}^{\infty} a_m (z - z_0)^m dz \\
        &= \sum_{m = -\infty}^{\infty} a_m \oint_{\gamma_r} (z - z_0)^m dz \\
        &= \sum_{m = -\infty}^{\infty} a_m \int_0^2\pi r^m e^{im\theta} i r e^{i\theta} dz \\
        &= \sum_{m = -\infty}^{\infty} a_m \int_0^2\pi r^{m+1} e^{i(m+1)\theta} dz \\
    \end{align*}
    and this integral is only nonzero at the $m = -1$ term, where it is $2\pi i$.
\end{proof}
\end{document}