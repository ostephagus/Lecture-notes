\documentclass[../Main.tex]{subfiles}

\begin{document}
\section{Contours and Integrals}
Given two points $a$ and $b$, we can find infinitely many paths from $a$ to $b$, and so we need to choose which path to integrate along (unless, as we will see later, the function is such that the integral is independent of the path).
\begin{definition}{Curve}
    A \underline{curve} $\gamma$ is a continuous map
    \begin{equation*}
        \gamma : [0, 1] \mapsto \C
    \end{equation*}
\end{definition}
\begin{remarks}
    \item We notationally take the interval $I$ over which $\gamma$ is defined to be $[0, 1]$. This is done without loss of generality, since we can map any interval to $[0, 1]$ by changing the parameter of the curve.
    \item We often refer to the image of $\gamma$ as the curve itself.
    \item Curves include a notion of direction from $\gamma(0)$ to $\gamma(1)$.
\end{remarks}
\begin{definition}{Closed curve}
    A \underline{closed curve} is a curve $\gamma$ with $\gamma(0) = \gamma(1)$. It has the same point for the start and end
\end{definition}
\begin{definition}{Simple curve}
    A \underline{simple curve} is a curve that does not intersect itself except at the endpoints (so a closed curve is still simple).
\end{definition}
\begin{definition}{Contour}
    A \underline{contour} is a piecewise-differentiable curve.
\end{definition}
We introduce some notation:
\begin{enumerate}
    \item $-\gamma$: this defines the curve that traverses $\gamma$ in the opposite direction. $(-\gamma)(t) = \gamma(1-t)$;
    \item $\gamma_1 + \gamma_2$: in the case that $\gamma_1(1) = \gamma_2(0)$, we can join curves
        \begin{equation*}
            (\gamma_1 + \gamma_2)(t) =
            \begin{cases}
                \gamma_1(2t) & t < \frac12 \\
                \gamma_2(2t-1) & t \geq \frac12
            \end{cases}
        \end{equation*}
\end{enumerate}
\begin{definition}{Contour integral}
    The \underline{contour integral} of a function $f$ over a contour $\gamma$ is:
    \begin{equation*}
        \int_{\gamma} f(z) dx = \int_0^1 f(\gamma(t)) \gamma'(t) dt
    \end{equation*}
\end{definition}
\begin{remark}
    This is very similar to the integral in $\R^n$:
    \begin{equation*}
        \int_{C} \vec{F} \cdot \vec{dx} = \int_{A}^{B} \vec{F}(\vec{x}(t)) \cdot \vec{x}'(t) dt
    \end{equation*}
\end{remark}
We also have the contour integral in terms of a Riemann sum. Dissecting the curve into:
\begin{equation*}
    0 = t_0 < t_1 < \cdot < t_m = 1
\end{equation*}
and let $\delta t_k = t_{k+1} - t_k$. Let $z_k = \gamma(t_k)$, and define $\delta z_k = z_{k+1} - z_k$.
\begin{equation*}
    \int_\gamma f(z) dz = \lim_{\Delta \to 0} \sum_{k=0}^{m-1} f(z_k) \delta z_m
\end{equation*}
where $\Delta = \max_{k = 0, \cdots, m-1} \delta t_k$.
\begin{example}
    Let $f(z) = \frac1z$ and $\gamma_1, \gamma_2$ be half circles from $-1$ to $1$, with $\gamma_1$ with positive imaginary part, $\gamma_2$ with negative imaginary part. Let both have the form $\gamma(\theta) = e^{-i\theta}$.
    \begin{align*}
        I_1 &= \int_{\gamma_1} f(z) dz = \int_{\pi}^0 \frac{ie^{i\theta}}{e^{i\theta}} d\theta = -i\pi \\
        I_2 &= \int_{\gamma_2} f(z) dz = \int_{-\pi}^0 \frac{ie^{i\theta}}{e^{i\theta}} d\theta = i\pi \\
    \end{align*}
\end{example}
\subsection{Rules of Integration}
All the following results can be proved via standard properties of integration:
\begin{enumerate}
    \item Joining contours: suppose that $\gamma_1(1) = \gamma_2(0)$. Then:
        \begin{equation*}
            \int_{\gamma_1 + \gamma_2} f(z) dz = \int_{\gamma_1} f(z) dz + \int_{\gamma_2} f(z) dz.
        \end{equation*}
    \item Reversed contours:
        \begin{equation*}
            \int_{-\gamma} f(z) dz = -\int_\gamma f(z) dz.
        \end{equation*}
    \item Independence of contours: if $f$ is differentiable then:
        \begin{equation*}
            \int_\gamma f(z) dz = f(\gamma(1)) - f(\gamma(0))
        \end{equation*}
        that is, the integral depends only on the endpoints.
    \item Integration by parts works.
    \item We can find the length of a curve by integration:
        \begin{equation*}
            L_\gamma = \int_\gamma |dz| = \int_0^1 |\gamma'(t)| dt
        \end{equation*}
        from which it follows that if $|f(z)| < f_0$,
        \begin{equation*}
            \left|\int_\gamma f(z) dz\right| \leq f_0 L_\gamma
        \end{equation*}
\end{enumerate}
As a convention, $\gamma$ moves anticlockwise, so the interior of $\gamma$ is on the left (with respect to the tangent of $\gamma$).
\begin{definition}{Connected domain}
    An open set $D$ is a \underline{connected domain} in $\C$ if every $z_1, z_2 \in D$ can be connected by a curve whose image is in $D$.
\end{definition}
\begin{definition}{Simply connected domain}
    A connected domain $D$ is \underline{simply connected} if every curve in $D$ encloses curves $D$.
\end{definition}
\begin{remark}
    This is the statement of ``no holes''. Removing a single point makes a domain not simply connected.
\end{remark}
\section{Cauchy's Theorem}
We will need the following from IA Vector Calculus:
\begin{theorem}[Green's Theorem in the Plane]
    Let $P$ and $Q$ be continuous with continuous partial derivatives on $D \supseteq M$ where $M$ is the interior of a simple closed contour $\gamma$ in $\R^2$. Then:
    \begin{equation}
        \oint_\gamma (Pdx + Qdy) = \iint_M\left(\frac{\partial Q}{\partial x} - \frac{\partial P}{\partial y}\right)dx~dy
        \label{eqnGreen}
    \end{equation}
    \label{thmGreen}
\end{theorem}
\begin{theorem}[Cauchy's Theorem]
    If $f(z)$ is analytic in a simply connected domain $D$, and $\gamma$ is a closed contour inside $D$,
    \begin{equation}
        \oint_\gamma f(z) dz = 0
        \label{eqnCauchy}
    \end{equation}
    \label{thmCauchy}
\end{theorem}
\begin{proof}
    Write $f(z) = u(x, y) + i v(x, y)$ where $z = x + iy$. By \thmref{thmGreen},
    \begin{align*}
        \oint_{\gamma} f(z) dz &= \oint_\gamma (u + i v)(dx + i dy) \\
        &= \oint_\gamma (u cd - v dy) + i \oint_\gamma(v dx + u dy) \\
        &= \iint_M \left(-\frac{\partial v}{\partial x} - \frac{\partial u}{\partial y}\right) dx~dy + i \iint_M \left(\frac{\partial u}{\partial x} - \frac{\partial y}{\partial y}\right) dx~dy \\
        &= 0 \text{ by Cauchy-Riemann relations}.
    \end{align*}
\end{proof}
\section{Deforming Contours}
\begin{proposition}
    Let $\gamma_1, \gamma_2$ be contours from $a \in \C$ to $b \in \C$. Let $f(z)$ be analytic on both contours and in the region enclosed by the contours. Then:
    \begin{equation}
        \int_{\gamma_1} f(z) dz = \int_{\gamma_2} f(z) dz
        \label{eqnPathIndep}
    \end{equation}
    \label{propPathIndep}
\end{proposition}
\begin{proof}
    Let $\gamma_1, \gamma_2$ not intersect each other except at the endpoints. Then $\gamma_1 - \gamma_2$ (the contour formed by $\gamma_1$ and then $-\gamma_2$) is a simple, closed contour.
    
    We have by \thmref{thmCauchy} that:
    \begin{align*}
        0 &= \oint_{\gamma_1 - \gamma_2} f(z) dz \\
        &= \int_{\gamma_1} f(z) dz + \int_{-\gamma_2} f(z) dz \\
        &= \int_{\gamma_1} f(z) dz - \int_{\gamma_2} f(z) dz \\
    \end{align*}
    and so the integrals are the same.

    Now suppose that the curves intersect at $c_1, \cdots, c_n \in \C$. Then we consider the above for $a$ to $c_1$, $c_1$ to $c_2$ and so on to get the same result.
\end{proof}
\begin{remark}
    We should compare this with exact differentials in $\R^2$. This is because:
        \begin{align*}
            df &=f(z) dz \\
            &= (u + i v)(dx + i~dy) \\
            &= \underbrace{(u + iv)}_{P}dx + \underbrace{(-v + i u)}_{Q} dy
        \end{align*}
        then Cauchy-Riemann implies that $\frac{\partial P}{\partial y} = \frac{\partial Q}{\partial x}$.
\end{remark}
Let $\gamma_1$ and $\gamma_2$ be closed contours that can be continuously deformed into each other. Let $f(z)$ be analytic on and in between $\gamma_1$ and $\gamma_2$.
\begin{proposition}[Contour Shrinking]
    Let $\gamma_1$ and $\gamma_2$ be closed contours that can be continuously deformed into each other. Let $f(z)$ be analytic on and in between $\gamma_1$ and $\gamma_2$. Then:
    \begin{equation*}
        \int_{\gamma_1} f(z) dz = \int_{\gamma_2} f(z) dz
    \end{equation*}
    \label{propContourShrink}
\end{proposition}
\begin{proof}
    Consider figure~\ref{figContourShrink}. %TODO.
\end{proof}
\begin{theorem}[Cauchy Integral Formula]
    Let $f$ be analytic on an open set $D \subseteq \C$. Let $z \in D$ and $\gamma$ be a simple, closed contour in $D$ that encloses $z_0 \in D$ (anticlockwise). Then:
    \begin{equation*}
        f(z_0) = \frac{1}{2\pi i} \oint_\gamma \frac{f(z)}{z - z_0} dz
    \end{equation*}
    and:
    \begin{equation*}
        f^{(m)}(z_0) = \frac{m!}{2\pi i} \oint_\gamma \frac{f(z)}{(z - z_0)^{m+1}} dz
    \end{equation*}
    \label{thmCauchyIntegral}
\end{theorem}
\begin{proof}[non-examinable]
    Consider $z_0$ and a contour $\gamma$ that encloses it. 
    The function $\frac{f(z)}{z - z_0}$ is analytic on and inside $\gamma$ except at $z_0$. We use \thmref{propContourShrink} to shrink $\gamma$ to a circle $\gamma_\epsilon$ of radius $\epsilon$ around $z_0$.
    \begin{equation*}
        \oint_\gamma \frac{f(z)}{z - z_0} dz = \oint_{\gamma_{\epsilon}} \frac{f(z)}{z - z_0} dz
    \end{equation*}
    We now evaluate the integral over $\gamma_\epsilon$. Let $z = z_0 + \epsilon e^{i\theta}$ and $dz = i \epsilon e^{i\theta}$.
    \begin{align*}
        \oint_{\gamma_\epsilon} \frac{f(z)}{z - z_0} dz &= \int_{0}^{2\pi} \frac{f(z_0 + i \epsilon e^{i\theta})}{\epsilon e^{i\theta}} i \epsilon e^{i\theta} d\theta \\
        &= \int_{0}^{2\pi} if(z_0 + e\epsilon^{i\theta}) d\theta \\
        \intertext{Take $\lim_{\epsilon \to 0}$:}
        &= 2\pi i f(z_0)
    \end{align*}
    Then we have the required result. For the derivatives, we simply differentiate this expression $m$ times.
\end{proof}
\begin{remarks}
    \item Knowing $f$ on $\gamma$ immediately gives us $f(z)$ at every point inside $\gamma$. This also gives an existence proof for the Laplace equation, because given Dirichlet boundary conditions for the Laplace equation (specifying $f$ on a contour), we can find $f$ at every point inside the contour, equivalent to solving the Laplace equation on this domain.
    \item We simply imposed the condition that $f$ is analytic, and we get that all higher derivatives exist! This is remarkable, and does not exist for $\R$.
\end{remarks}
\begin{theorem}
    If $f(z)$ is an entire function (analytic on $\C$) and bounded, then $f$ must be constant.
    \label{thmEntireBddConst}
\end{theorem}
\begin{proof}[non-examinable]
    Since $f$ is bounded, there exists some $c_0 \in \R$ such that $\abs{f(z)} \leq c_0$ for all $z \in \C$. Let $z_0 \in \C$ be arbitrary. Let $\gamma_r$ be the contour along a circle of radius $r$ around $z_0$ traversed anticlockwise. By \thmref{thmCauchyIntegral} with $m = 1$,
    \begin{align*}
        f'(z_0) &= \frac{1}{2\pi i} \oint_{\gamma_r}\frac{f(z)}{(z - z_0)^2} dz \\
        \abs{f'(z_0)} &\leq \frac{1}{2\pi} \oint_{\gamma_r} \frac{c_0}{r^2} \abs{dz} \\
        &= \frac{c_0}{r}
    \end{align*}
    Then taking $r \to \infty$, we find that $f'(z_0) = 0$. Since $z_0$ was any point in $\C$, we must have $f'(z) = 0$ for all $z$. That is, $f$ is constant.
\end{proof}
\end{document}